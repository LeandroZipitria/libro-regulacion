% Options for packages loaded elsewhere
\PassOptionsToPackage{unicode}{hyperref}
\PassOptionsToPackage{hyphens}{url}
%
\documentclass[
  12pt,
  spanish,
]{book}
\usepackage{lmodern}
\usepackage{amssymb,amsmath}
\usepackage{ifxetex,ifluatex}
\ifnum 0\ifxetex 1\fi\ifluatex 1\fi=0 % if pdftex
  \usepackage[T1]{fontenc}
  \usepackage[utf8]{inputenc}
  \usepackage{textcomp} % provide euro and other symbols
\else % if luatex or xetex
  \usepackage{unicode-math}
  \defaultfontfeatures{Scale=MatchLowercase}
  \defaultfontfeatures[\rmfamily]{Ligatures=TeX,Scale=1}
\fi
% Use upquote if available, for straight quotes in verbatim environments
\IfFileExists{upquote.sty}{\usepackage{upquote}}{}
\IfFileExists{microtype.sty}{% use microtype if available
  \usepackage[]{microtype}
  \UseMicrotypeSet[protrusion]{basicmath} % disable protrusion for tt fonts
}{}
\makeatletter
\@ifundefined{KOMAClassName}{% if non-KOMA class
  \IfFileExists{parskip.sty}{%
    \usepackage{parskip}
  }{% else
    \setlength{\parindent}{0pt}
    \setlength{\parskip}{6pt plus 2pt minus 1pt}}
}{% if KOMA class
  \KOMAoptions{parskip=half}}
\makeatother
\usepackage{xcolor}
\IfFileExists{xurl.sty}{\usepackage{xurl}}{} % add URL line breaks if available
\IfFileExists{bookmark.sty}{\usepackage{bookmark}}{\usepackage{hyperref}}
\hypersetup{
  pdftitle={Regulación económica para economías en desarrollo},
  pdfauthor={Rosario Domingo, Jorge Ponce, Leandro Zipitría},
  pdflang={es},
  hidelinks,
  pdfcreator={LaTeX via pandoc}}
\urlstyle{same} % disable monospaced font for URLs
\usepackage{longtable,booktabs}
% Correct order of tables after \paragraph or \subparagraph
\usepackage{etoolbox}
\makeatletter
\patchcmd\longtable{\par}{\if@noskipsec\mbox{}\fi\par}{}{}
\makeatother
% Allow footnotes in longtable head/foot
\IfFileExists{footnotehyper.sty}{\usepackage{footnotehyper}}{\usepackage{footnote}}
\makesavenoteenv{longtable}
\usepackage{graphicx,grffile}
\makeatletter
\def\maxwidth{\ifdim\Gin@nat@width>\linewidth\linewidth\else\Gin@nat@width\fi}
\def\maxheight{\ifdim\Gin@nat@height>\textheight\textheight\else\Gin@nat@height\fi}
\makeatother
% Scale images if necessary, so that they will not overflow the page
% margins by default, and it is still possible to overwrite the defaults
% using explicit options in \includegraphics[width, height, ...]{}
\setkeys{Gin}{width=\maxwidth,height=\maxheight,keepaspectratio}
% Set default figure placement to htbp
\makeatletter
\def\fps@figure{htbp}
\makeatother
\setlength{\emergencystretch}{3em} % prevent overfull lines
\providecommand{\tightlist}{%
  \setlength{\itemsep}{0pt}\setlength{\parskip}{0pt}}
\setcounter{secnumdepth}{5}
\usepackage{booktabs}
\usepackage{lmodern}
\usepackage{hyperref}
\usepackage{setspace}\onehalfspacing
\usepackage{titling}
\posttitle{\end{center}}
\usepackage[nottoc]{tocbibind}
\ifxetex
  % Load polyglossia as late as possible: uses bidi with RTL langages (e.g. Hebrew, Arabic)
  \usepackage{polyglossia}
  \setmainlanguage[]{spanish}
\else
  \usepackage[shorthands=off,main=spanish]{babel}
\fi
\usepackage[]{natbib}
\bibliographystyle{econometrica-es.bst}

\title{Regulación económica para economías en desarrollo}
\author{Rosario Domingo, Jorge Ponce, Leandro Zipitría}
\date{Esta version: 2020-07-20}

\begin{document}
\maketitle

{
\setcounter{tocdepth}{1}
\tableofcontents
}
\newpage

\hypertarget{prefacio}{%
\chapter*{Prefacio}\label{prefacio}}
\addcontentsline{toc}{chapter}{Prefacio}

``Regulación Económica para Economías en Desarrollo'' por Rosario Domingo, Jorge Ponce y Leandro Zipitría se distribuye bajo una Licencia Creative Commons Atribución 4.0 Internacional \href{http://creativecommons.org/licenses/by-nc-sa/4.0/}{Creative Commons Attribution-NonCommercial-ShareAlike 4.0 International License}.

\textbf{Atribución} - Cite el trabajo de la siguiente forma: Domingo, Rosario, Jorge Ponce y Leandro Zipitría (2016). Regulación económica para economías en desarrollo. Departamento de Economía - FCS, Universidad de la República, Montevideo.

ISBN: 978-9974-0-1325-4

Edición y corrección: Rosario Domingo.

Diseño de tapa: Rodolfo Fuentes / NAO

Los puntos de vista y las opiniones expresadas en esta publicación, son responsabilidad de los autores y no necesariamente reflejan una posición del Departamento de Economía de la Facultad de Ciencias Sociales de la Universidad de la República.

\newpage

\hypertarget{intro}{%
\chapter*{Presentación}\label{intro}}
\addcontentsline{toc}{chapter}{Presentación}

La regulación por parte de los gobiernos es una de las características principales de las economías modernas. Los mercados están sometidos a distintas reglas que afectan su funcionamiento y el de las empresas que en ellos actúan. A medida que las economías se desarrollan, las fallas de mercado imponen restricciones al funcionamiento de la economía que requieren ser atendidas, de forma de que puedan explotar todo su potencial. Por tanto, conocer los instrumentos regulatorios disponibles y entender cuáles son sus efectos sobre los mercados es clave para el adecuado diseño de las políticas públicas. Asimismo, no sólo hay que regular cuando es requerido, sino que hay que hacerlo de forma adecuada.

Este libro presenta los principales conceptos de la regulación económica en forma simple, concisa y a la vez rigurosa. Su objetivo es que el lector pueda familiarizarse con las herramientas regulatorias rápidamente evitando, en la medida de lo posible, las discusiones técnicas que subyacen los conceptos. Está dirigido a técnicos del sector público con independencia de su formación profesional. Asimismo, está escrito y diseñado con el objetivo de atender los problemas relevantes de los países en vías de desarrollo y, en particular, los países latinoamericanos. En él se describen los problemas que afectan a estos países, cuyas instituciones regulatorias son muchas veces débiles para enfrentar, de forma adecuada, la tarea de fomentar el funcionamiento de los mercados.

El texto tiene su origen en el Diploma en Regulación Económica que se dicta en la Universidad de la Habana, con el objetivo de formar técnicos en las principales herramientas del análisis económico regulatorio. El mismo se realiza y publica en el marco del programa de colaboración que el Departamento de Economía de la UdelaR desarrolla con la Facultad de Economía de la Universidad de la Habana, con el financiamiento de la Agencia Sueca Internacional de Cooperación al Desarrollo (ASDI). Asimismo, los dos últimos capítulos del libro están basados fuertemente en un trabajo realizado por Domingo y Zipitría para la Unidad Reguladora de los Servicios de Energía y Agua de Uruguay y que contó con financiamiento de la Cooperación Andina de Fomento (CAF).

El capítulo \ref{reg-ec}, escrito por Zipitría, introduce la temática de la regulación económica. Discute la racionalidad económica de la regulación y las distintas teorías que la explican. Se ilustra con distintas regulaciones de la realidad de Uruguay y cómo éstas se interpretan a la luz de las teorías descritas. Por último, desarrolla las características de los mercados de monopolio natural en los cuales se observa no sólo una importante regulación, sino también la presencia de empresas públicas en distintos países en vías de desarrollo. Los sectores de monopolio natural, en donde por razones tecnológicas participa una única empresa del mercado, son sensibles desde el punto de vista político debido a que sus productos son consumidos de forma masiva y existen impor tantes costos hundidos. A la vez, en ellos no es posible confiar en la competencia para que resuelva el funcionamiento eficiente del mercado.

En el capítulo \ref{inst-reg}, Zipitría introduce los principales instrumentos regulatorios y sus efectos sobre el mercado. Si bien el principal instrumento de evaluación del funcionamiento de los mercados es la eficiencia, el capítulo no cuestiona otros fines que pueda tener la regulación. Ello permite investigar los efectos, buscado y no buscados, que la aplicación de los instrumentos regulatorios tiene sobre el mercado. Muchas veces, en el diseño regulatorio, el énfasis se pone en el efecto directo que el instrumento tiene sobre el comportamiento que se busca modificar, cuando los efectos indirectos pueden alterar la conducta de los agentes en los mercados en sentido no previsto y empeorar el funcionamiento general del mismo. El análisis se realiza identificando la forma de regulación óptima en cada estructura de mercado ---competencia perfecta, oligopolio y monopolio natural--- y los efectos que la misma tiene sobre su eficiencia.

El capítulo \ref{def-comp}, del mismo autor, desarrolla los principales conceptos del análisis de defensa de la competencia. En aquellos mercados donde la competencia es factible, muchas veces las empresas privadas buscan evitar que esta opere y reduzca las rentas que obtienen. Por ello, se ha desarrollado un campo específico que intenta comprender el funcionamiento de los mercados y el efecto de las conductas que los agentes desarrollan. En la medida en que los países menos desarrollados tienen dificultades para recopilar y acceder a información detallada de la forma en la que operan los mercados, se presenta una discusión de los instrumentos que permiten comprender su funcionamiento y tomar decisiones fundamentadas sobre las acciones de los privados. Se desarrolla el concepto de mercado relevante, que es el primer paso de toda investigación de competencia, pero que es también una herramienta general que puede ser útil para entender el funcionamiento de los mercados. Posteriormente se analizan el abuso de posición dominante, la colusión, las restricciones verticales y las fusiones de empresas que son las conductas investigadas en esta materia. En cada caso, se expone las razones económicas de los agentes para adoptar cada una de ellas, así como sus efectos positivos y negativos sobre la eficiencia.

En el capítulo \ref{instituciones-reg}, Ponce desarrolla una serie de herramientas para el diseño de las instituciones regulatorias. La regulación de los agentes privados en general, y en las economías en desarrollo en particular, está sometida a una serie de problemas de información, restricciones técnicas y presupuestales que hay que tomar en consideración a la hora del diseño de dichas instituciones. En particular, existe el riesgo de la captura del regulador por parte del sector privado, un compromiso limitado y falta de credibilidad de los reguladores, poca rendición de cuenta de sus acciones, baja eficiencia para recaudar recursos fiscales que permita financiar las actividades reguladas, corrupción y baja capacidad de hacer cumplir las normas. Estos problemas alteran el diseño institucional óptimo de las agencias encargadas de actuar sobre el sector privado y, si no son tenidos en cuenta, pueden determinar que estos organismos se alejen en forma relevante de los fines perseguidos.

En el capítulo \ref{reg-eepp}, Domingo y Zipitría introducen la regulación de las empresas públicas. A diferencia de los países desarrollados, en los países en vías de desarrollo los Estados participan en la actividad productiva en muchos sectores de la economía. Las empresas públicas tienen características distintivas de las privadas, que requieren considerarse a la hora de analizar su eficiencia o su regulación. En particular, están fuertemente influenciadas por el sistema político el cual les adjudica múltiples roles. Esta multiplicidad de objetivos tiene impacto sobre le eficiencia del funcionamiento de estas empresas, así como sobre la posibilidad de evaluar el comportamiento y gestión de sus directores. Asimismo, algunas veces las empresas públicas están sometidas a restricciones presupuestales blandas, lo que afecta también en forma negativa su eficiencia. Por otra parte, las restricciones que enfrentan sus directores se traducen en mecanismos particulares de control de su actuación, de forma de evitar que sean utilizadas con fines espúreos.

Por último, el capítulo \ref{eepp-uy} de los mismos autores presenta una descripción de las empresas públicas en América Latina en general, y un detalle de estas empresas y su regulación en Uruguay. El objetivo es ilustrar con ejemplos concretos las múltiples dimensiones del análisis de este tipo de empresas, así como la dinámica política a la cual están sometidas. Asimismo, sirve para comprender los reveses que han enfrentado los reguladores cuando tienen que regular a este tipo de empresas.

Esperamos que este texto, por la presentación llana de los conceptos, sirva de guía y de manual de referencia tanto para los participantes del Diploma en Regulación Económica, como para los formuladores de políticas.

Los autores.

\hypertarget{reg-ec}{%
\chapter{Regulación económica}\label{reg-ec}}

Leandro Zipitría

La esencia del funcionamiento libre del mercado es que a los agentes les está permitido tomar sus propias decisiones \citep[357]{Viscusi2005}. En términos generales, el mercado
\footnote{Un mercado es el lugar donde se intercambian derechos de propiedad. Los derechos de propiedad incluyen cuatro derechos: (i) al uso; (ii) a obtener los retornos del bien; (iii) a transferir el bien; y (iv) a hacer cumplir los derechos de propiedad.}
asigna en forma eficiente los recursos en mercados perfectamente competitivos. Por ello, no existe motivo para intervenir en los mercados, dado que las decisiones privadas de los agentes coinciden con el bienestar social. Los mercados por sí solos resuelven el problema de la asignación de recursos.

En algunos casos, sin embargo, el mercado no asigna en forma eficiente los recursos y se producen \textbf{fallas de mercado}, las que pueden ser de distinto tipo:

\begin{enumerate}
\def\labelenumi{\arabic{enumi}.}
\item
  \textbf{Externalidades}: refieren a costos o ingresos que los agentes no internalizan cuando toman una decisión. Una externalidad negativa se da, por ejemplo, cuando una empresa contamina un río y no considera los costos que acarrea limpiarlo o mejorar su proceso productivo para reducir la contaminación al elegir su nivel de producción. Una externalidad positiva ocurre cuando se crea conocimiento básico, ya que el científico o la institución financiadora no puede apropiarse completamente de los ingresos que este conocimiento genera. En ambos casos, los mercados no asignan en forma eficiente los recursos y tenderán a sobre (sub) producir si la externalidad es negativa (positiva).
\item
  \textbf{Bienes públicos}: son aquellos no excluibles (no se puede excluir a los agentes de su consumo) y no rivales (el consumo de una unidad del bien no disminuye la cantidad disponible para los demás consumidores).
  \footnote{Los bienes económicos tradicionales son excluibles y rivales.}
  Ejemplo de este tipo de bienes es el conocimiento, el aire o la defensa de una nación.
\item
  \textbf{Poder de mercado}: define la capacidad de una empresa o conjunto de empresas de fijar el precio de forma rentable por encima del nivel competitivo, conducta que conduce a pérdidas del bienestar social \citep[p.~8]{Carlton2004}. La existencia de poder de mercado determina mercados no competitivos. Este texto analiza la regulación de este tipo de falla de mercado.
\item
  \textbf{Asimetrías de información}: se da cuando las partes que intervienen en las
  transacciones no cuentan con la misma información sobre el producto, el servicio o el activo objeto de la transacción. Los problemas de asimetría de información impregnan las relaciones económicas entre agentes y puede hacer colapsar a los mercados. La idea clásica refiere a una situación en donde existen distintas calidades de bienes pero los demandantes no conocen la calidad específica que ofrece cada agente. En estas circunstancias, para determinadas valoraciones de compradores y vendedores, se puede demostrar que sólo serán transados los de peor calidad \citep[pp.~244-245]{Wolfstetter1999}. En el sector financiero y en la salud la asimetría de información es un problema relevante que determina que se necesiten instrumentos regulatorios específicos para mitigarla.
\end{enumerate}

En las economías modernas, los gobiernos en forma rutinaria toman decisiones que buscan afectar la forma en la que los agentes económicos se comportan y modificar tanto el bienestar de los mismos como el de la sociedad. En particular, en su rol de reguladores, los gobiernos restringen el conjunto de acciones disponibles para los agentes. Por regulación se entiende las limitaciones impuestas a los agentes respecto de la discreción de las decisiones que pueden tomar, materializadas a través del uso de instrumentos legales y bajo la amenaza de alguna sanción. Los gobiernos, a diferencia de los demás agentes sociales, tienen el monopolio del poder de coerción,
\footnote{Esta idea está basada en \citep{Weber1921}.}
en particular, de hacer cumplir las normas que ellos mismos dictan.
\footnote{El lenguaje dice mucho de los pueblos; el idioma inglés tiene una palabra para ``hacer cumplir las normas'', enforcement, que el idioma español carece.}

Cuando se regula un mercado su resultado en términos de eficiencia asignativa y productiva está co-determinado por las fuerzas del mercado y el proceso administrativo.
\footnote{No se consideran en este análisis elementos que refieren a si las normas son justas o éticas como justificación de la regulación, sólo si son eficientes desde el punto de vista económico. Para una discusión y crítica de esta visión, véase \citep[capítulo 3]{Baldwin2011}.}
Las regulaciones pueden ser de distinto tipo, y a los efectos de este texto se dividen en:

\begin{enumerate}
\def\labelenumi{\arabic{enumi}.}
\item
  \textbf{Económicas}: aquellas que afectan la eficiencia asignativa ---y productiva---
  de los mercados, como ser la fijación de precios o cantidades, el número de
  competidores, la entrada o salida del mercado.
\item
  \textbf{Sociales}: incluyen las regulaciones medioambientales, las condiciones laborales, la protección del consumidor, y la seguridad de productos, entre otras. Entre ellas podemos encontrar las regulaciones que establecen requerimientos sobre la forma en la que se eliminan sustancias tóxicas en el ambiente, la información que deben tener los productos alimenticios o medicamentos sobre sus componentes, o los controles que se establecen sobre los lugares donde se venden los productos.
\end{enumerate}

No por ser las más importantes, sino porque es el eje central del texto, nos vamos a concentrar en las regulaciones económicas. Sin embargo, las regulaciones sociales también tienen justificación económica y deben ser consideradas a la hora de formular las distintas políticas, ya que no sólo son relevantes en sí mismas, sino que también tienen impactos sobre las variables económicas. Las regulaciones económicas pueden, a su vez, dividirse entre:

\begin{enumerate}
\def\labelenumi{\arabic{enumi}.}
\item
  \textbf{Regulaciones de estructura}: incluyen normas que alteran la estructura del mercado, como ser las que afectan la libre entrada y salida de empresas o agentes del mercado.
\item
  \textbf{Regulaciones de conducta o comportamiento}: afectan variables como el precio, la calidad de los productos, la cantidad que se transa en los mercados, la publicidad que pueden llevar a cabo las empresas, entre otros.
\end{enumerate}

En los últimos años ha surgido una importante literatura denominada economía del comportamiento, que se encarga de analizar los determinantes psicológicos de la conducta de los agentes, en particular consumidores. La misma enfatiza que los individuos no son lo racionales que la teoría económica sugiere, e impulsa regulaciones de comportamiento que afectan la forma en la que los agentes procesan la información que reciben y toman las decisiones, llamados ``empujones'' o nudges en inglés. En la medida en que esta literatura está enfocada a modificar la conducta de los demandantes, este texto no abordará la regulación de comportamiento, la que se puede encontrar en textos como \citep[ o \citet{Kanheman2015}]{Sunstein2009}.

La regulación es una las características distintivas de los Estados. Hay normas generales relativas a: defensa de la competencia; defensa del consumidor; quiebra de empresas; sociedades comerciales; habilitaciones para locales comerciales e industriales; normas bromatológicas sobre la calidad de los productos o procesos de producción de alimentos; entre otras. También se dictan normas específicas o sectoriales, como las vinculadas a: las telecomunicaciones; la banca; la energía eléctrica; el transporte aéreo, terrestre y marítimo; los productos lácteos, cárnicos y semillas; y un muy largo etcétera.

En Uruguay, los instrumentos regulatorios utilizados son, básicamente, de dos tipos:

\begin{enumerate}
\def\labelenumi{\arabic{enumi}.}
\item
  Regulación de precios: se fija el precio de la leche fresca; el valor del taxi y del transporte urbano; las cuotas mutuales y el valor de los tiques moderadores en el sector de la salud; el precio de la energía eléctrica;\footnote{En este caso es tenue la línea que separa la regulación de la fijación de precios por parte de la empresa monopólica.} los precios máximos en el mercado del gas licuado de petróleo (GLP) y el de los combustibles, entre otros.
  \footnote{En América Latina en general, y en Uruguay en particular, las regulaciones de precio se basan en normas asociadas a las llamadas leyes de abastecimiento, que buscan regular directamente la oferta y el precio de los productos, sancionando penalmente el acaparamiento, entre otras prácticas. En Uruguay, se encuentra vigente la Ley No.~10.940 de setiembre de 1947 (sobre la base de la Ley No.~10.075 de 1941 que regulaba los artículos de primera necesidad) que creó el Consejo Nacional de Subsistencias y Contralor de Precios. Esta norma permite, bajo ciertos parámetros, que el Poder Ejecutivo intervenga y fije el precio de los productos e intervenir empresas y secuestrar sus productos, bajo ciertos supuestos (artículo 13). En la actualidad, Venezuela cuenta con un control similar, a través de la Ley de Precios Justos, que fija el precio para los artículos de primera necesidad. En este caso los controles los realiza la Superintendencia de Precios Justos. También Argentina cuenta con leyes de este tipo desde 1974, con modificaciones introducidas en 2010, véase la \href{http://infoleg.mecon.gov.ar/infolegInternet/anexos/55000-59999/58603/texact.htm}{ley No.~20.680}.}
\item
  Controles a la entrada y salida de los mercados: en el mercado bancario, operan importantes regulaciones ---prudenciales--- a la entrada y salida de bancos, que incluyen autorización previa por parte del Banco Central del Uruguay (BCU); en la telefonía celular, para entrar al mercado hay que acceder a una frecuencia que la otorga la Unidad Reguladora de los Servicios de Comunicaciones (URSEC); en la telefonía fija está prohibida la entrada a operadores del sector privado; en la habilitación a las farmacias, existe una ley que prohíbe que las mismas se instalen a menos de 300 metros de otra ya existente; en el establecimiento de cualquier emprendimiento de venta de productos alimenticios y de uso doméstico (supermercado) de más de 200 metros cuadrados, la Ley de Protección de la Micro, Pequeña y Mediana Empresa Comercial y Artesanal, establece que se requiere consultar a una comisión asesora del Intendente municipal quien expide la autorización final; en la habilitación a carnicerías se requiere, entre otros, aprobación expresa del Instituto Nacional de Carne (INAC).
\end{enumerate}

Pero ¿cuál es la justificación de la regulación económica?
\footnote{En esta temática se sigue fundamentalmente a \citet{Viscusi2005}, capítulo 10; \citet{Hertog1999}; y \citet{Noll1989}.}
En este capítulo se presentan distintas explicaciones desde la ciencia económica, aunque en la actualidad este campo también lo trata la ciencia política. La primera justificación proviene de la teoría del interés público, que establece que el gobierno actúa en forma benevolente para resolver fallas de mercado.

La segunda se basa en la teoría de la captura regulatoria, en sus distintas versiones, para la cual la regulación es un instrumento utilizado por los grupos de poder para redistribuir renta a su favor. En tercer lugar, se introduce una mirada que va más allá de las particularidades del mercado en sí mismo y la regulación económica se vincula a la interrelación entre el Estado y el sector privado y, en particular, a la forma que tiene el Estado de controlar la convivencia social. Por último, se realiza un detallado análisis de las características de la regulación de monopolios naturales.

\hypertarget{la-teoruxeda-del-interuxe9s-puxfablico}{%
\section{La teoría del interés público}\label{la-teoruxeda-del-interuxe9s-puxfablico}}

El principal fundamento para la intervención del gobierno en los mercados se basa en la idea de que existen situaciones en las cuales los mercados no conducen al máximo bienestar social, en particular, cuando se producen fallas de mercado. Según esta teoría, la regulación persigue el bienestar general cuando existen fallas de mercado y su objetivo es corregirlas. El gobierno que diseña y ejecuta estas regulaciones lo hace con espíritu benevolente, y no existen costos de transacción para implementarlas. Antes de continuar, conviene discutir qué se entiende por costos de transacción.

Este concepto tiene distintas definiciones, la más amplia de las cuales establece que son los costos que se incurren para realizar un intercambio económico. Los costos de transacción incluyen: (i) los costos de búsqueda e información, es decir, determinar si el producto está disponible en el mercado, cuál es su precio, quién lo tiene disponible, etc.;(ii) los costos de negociación entre las partes para alcanzar un acuerdo; y (iii) los costos de vigilar y hacer cumplir el contrato refrendado entre las partes. Como puede apreciarse, en la realidad es muy difícil encontrar situaciones en las cuales los costos de transacción entre las partes sean cero.

Esta teoría recibe, al menos, dos críticas. La primera refiere a la propia existencia de una potencial ganancia de eficiencia ---condición necesaria para la regulación---, puesto que no explica cómo se logra alcanzar esta ganancia a través de la intervención del gobierno. En otros términos, cómo se pasa de una demanda pública a un resultado concreto normativo con su consiguiente beneficio social.

En efecto, como señala \citet{Noll1989}, existen mecanismos alternativos a la regulación como, por ejemplo, otorgar dinero a las familias de menores recursos en lugar de fijar precios máximos a los bienes. Si ello no es posible, es porque existen otras restricciones o los costos de transacción no son cero. Sin embargo, no hay razón teórica para que los problemas de información que enfrentan los privados, asociados a estos costos de transacción, los puede superar el gobierno.

Asimismo, regular tiene costos. Estos están vinculados a la necesidad de contratar una burocracia que analice el mercado y monitoree su desempeño, así como a las ineficiencias que pueda generar la propia regulación en el mercado (por ejemplo, menores incentivos al ingreso de empresas si se limita el precio). Otras veces la regulación tiene una inercia propia que dificulta el ajuste de la misma a los cambios en las condiciones de mercado.

A vía de ejemplo, en Uruguay existe históricamente un precio máximo a la leche fresca, producto del desabastecimiento que sufrió la ciudad de Montevideo en la década del 30 \citep{Marti2013}. Aun cuando las condiciones de mercado cambiaron sustancialmente, esta regulación se ha mantenido casi por un siglo.
\footnote{Dos problemas adicionales surgen en este análisis. En primer lugar, si por alguna consideración política esta regulación fuera necesaria, ¿qué bienes integran la canasta de productos de primera necesidad, cuyo monitoreo es necesario? En segundo lugar, ¿por qué se regula el precio de la leche, como artículo de primera necesidad, y no se regulan los otros productos que integran esta canasta como la harina o el pan? Las fallas de mercado no explican el porqué de esta regulación.}
Asimismo, no queda claro que el instrumento elegido (precio máximo) sirva como solución al problema original (desabastecimiento).

La segunda crítica refiere a que se debería observar la regulación sólo en aquellos sectores con fallas de mercado, lo que no siempre se cumple. En muchos países se regulan mercados potencialmente competitivos como el de transporte. En Estados Unidos (EUA) las críticas a este enfoque fueron más fuertes. En la década de los 60, una serie de estudios demostraron que la regulación, más que subsanar las fallas de mercado, tendía a aumentar los beneficios de las industrias.

El problema radica en que el gobierno no está exento de la falta de información, ni de los costos de transacción antes mencionados, por lo que nada asegura que se puedan alcanzar los resultados óptimos a través de la intervención estatal. Los procesos técnico-políticos que llevan a la conformación de las decisiones gubernamentales son complejos y en ellos, generalmente, intervienen muchos actores con intereses diversos.

Uno de los problemas fundamentales es que regular implica redistribuir riqueza entre los agentes ---ya sea productores o consumidores--- y ello lo convierte en un mecanismo muy atractivo para que los diversos actores busquen participar en su diseño o implementación.
\footnote{A vía de ejemplo, la Ley de Grandes Superficies que mencionáramos anteriormente tiene como objetivo explícito defender a la micro, pequeña y mediana empresa comercial y artesanal, uno de los miembros de la comisión asesora del Intendente es un representante de estos emprendimientos.}

\hypertarget{la-teoruxeda-de-la-captura-regulatoria}{%
\section{La teoría de la captura regulatoria}\label{la-teoruxeda-de-la-captura-regulatoria}}

La evidencia empírica en EUA mostró que, hasta la década de los 60, la regulación no estaba necesariamente correlacionada con las fallas de mercado, sino todo lo contrario. Es decir, muchas regulaciones más que resolver fallas de mercado estaban dirigidas a la protección de las empresas instaladas. Ello llevó al desarrollo de la Teoría de la Captura Regulatoria, la que sostiene que o bien la regulación se realiza en respuesta a las demandas de la industria (los legisladores están capturados por la industria) o las agencias regulatorias, con el paso del tiempo, terminan siendo controladas por las industrias que deberían regular.

\citet{Stigler1971} señala que la regulación es uno de los caminos que tienen los grupos de interés para incrementar su bienestar, utilizando al Estado para que redistribuya riqueza a su favor a costa de otros grupos sociales. El Estado es un agente atractivo para redistribuir riqueza debido a que dispone del poder de coerción. Por tanto, las regulaciones a la entrada de los mercados, o las que imponen costos excesivos o innecesarios a las nuevas empresas, son, según esta teoría, el resultado de la captura del proceso económico por parte de los instalados.

En Uruguay existen distintos ejemplos de normas que se ajustarían a esta teoría. En primer lugar, la \href{http://www.parlamento.gub.uy/leyes/AccesoTextoLey.asp?Ley=17715\&Anchor=}{Ley No.~17.715} estableció una distancia mínima parael establecimiento de nuevas farmacias respecto a las ya instaladas y, como resultado, disminuye la competencia potencial de nuevos entrantes sobre los precios de los instalados. En segundo lugar, las leyes \href{http://www.parlamento.gub.uy/leyes/AccesoTextoLey.asp?Ley=17188\&Anchor=}{No.~17.188} y \href{http://www.parlamento.gub.uy/leyes/AccesoTextoLey.asp?Ley=17657\&Anchor=}{No.~17.657} instauraron restricciones para el establecimiento de grandes superficies (mayores a 200 metros cuadrados). Las mismas crearon una comisión asesora al Intendente de cada departamento, que debe elaborar un informe preceptivo ---pero no vinculante--- respecto de la conveniencia de la instalación de nuevos supermercados, atendiendo a la demanda excedente, la creación neta de empleos y la desaparición de establecimientos tradicionales. Por último, en el sector de las telecomunicaciones se observa en los últimos años un involucramiento relativamente laxo por parte del órgano regulador (URSEC) a investigar y sancionar posibles prácticas anticompetitivas que realice la empresa estatal ANTEL.

Sin embargo, tampoco es claro que toda la regulación sea el producto de las acciones de los agentes ya instalados en los mercados, como lo demuestra la existencia de subsidios cruzados entre consumidores. En Uruguay, por ejemplo, la regulación fija precios únicos para algunos bienes o servicios, lo que limita la capacidad de las empresas para fijar precios. El ejemplo más claro es el del servicio de taxímetro en el cual la tarifa está regulada (fija), lo que impide que las empresas suban los precios en los momentos de mayor demanda.

Otras veces, este sesgo surge, más que en las normas en sí mismas, en su aplicación. A las empresas grandes, a veces, se las controla de forma más fuerte y efectiva que a las pequeñas, en el cumplimiento de las normas. A vía de ejemplo, la empresa papelera UPM (ex BOTNIA) ha tenido que cumplir con exigentes normativas para poder construir su planta y existen distintos puntos de monitoreo, a lo largo del Río Uruguay, para estudiar la calidad del agua. En el otro extremo, el control de la contaminación de los ríos ha sido históricamente un objetivo de segundo orden por parte de las autoridades departamentales \citep{Caffera2004}. En efecto, la Intendencia Municipal de Montevideo ---al menos en el pasado--- ha evitado sancionar a empresas que contaminan los ríos, evitando su posible cierre, con el objetivo de proteger el empleo.

Entonces la pregunta parece ser ¿qué mercados serán los regulados? \citet{Peltzman1976} desarrolla un enfoque teórico que presenta tres componentes: (i) la regulación redistribuye riqueza; (ii) los legisladores actúan guiados por el interés de mantener su puesto, lo que implica que las normas que aprueban están diseñadas para maximizar el apoyo político que reciben; (iii) los grupos de interés
\footnote{Entendidos en sentido amplio: consumidores, empresas establecidas, competidoras, proveedores, trabajadores, etc.}
compiten para ofrecer apoyo político a cambio de una legislación favorable.

El resultado al que llega es que la regulación estará sesgada a favor de aquellos grupos que estén mejor organizados o ganen más con una legislación favorable, los que estarían más dispuestos a invertir recursos para lograr el apoyo político a sus propuestas. Más específicamente, la regulación beneficiará a pequeños grupos con preferencias más homogéneas a costa de grandes grupos con preferencias heterogéneas, en la medida en que las primeras tienen mayor facilidad para alcanzar consensos relativamente estables entre sus componentes.

Tras estos argumentos está el efecto del parasitismo: grupos pequeños en los cuales cada miembro tiene mucho para ganar con la legislación, tienen más incentivos para apoyar políticamente la redacción, discusión y aprobación de una ley. Por su parte, los grupos grandes ---en general de consumidores--- están menos informados sobre el alcance de la normativa e individualmente pierden poco, en relación con lo que gana cada integrante del grupo que se beneficia con la regulación. A la vez, los costos de oponerse a la iniciativa son altos y terminan beneficiando a terceros que no incurren en dichos costos. En este sentido, los costos de organizar grupos numerosos con intereses difusos, donde la acción que pueda desplegar alguno de sus integrantes genera beneficios al resto de los componentes del grupo, hace que sean más vulnerables a la hora de oponerse a iniciativas de los grandes grupos empresariales.

\citet{Peltzman1976} presenta un modelo de fijación de precios por parte de un legislador/regulador que busca maximizar el apoyo político. La función de apoyo político es una función decreciente del precio (porque los consumidores reducen el apoyo si el precio sube) y creciente en los beneficios de la empresa. Por otro lado, los beneficios de las empresas son crecientes en el precio, hasta que alcanzan el valor de monopolio. En este modelo, el precio de equilibrio que elegirá el regulador es aquél que maximice el apoyo político y, a la vez, esté sobre la frontera de beneficios de las empresas. Por tanto, se cumple que el precio de equilibrio estará entre el competitivo y el de monopolio. Como el precio está entre los ``extremos'' se puede concluir que las industrias que tenderán a estar reguladas son aquellas relativamente competitivas (a demanda de las empresas) o aquellas relativamente monopolísticas (a demanda de los consumidores).Según \citet{Viscusi2005} esta situación es la que se observa en EUA, donde se regulan mercados monopólicos como la telefonía local y de larga distancia, la transmisión de gas y electricidad y los ferrocarriles; o sectores relativamente competitivos como el transporte carretero, los taxis y la producción de energía o gas.

En esta línea, \citet{Noll1989} presenta una vinculación interesante entre la regulación y su retroalimentación con los grupos de interés. En efecto, los sectores regulados, que generalmente obtienen beneficios extraordinarios, construyen a su vez, dentro de las empresas, espacios de negociación de las cuasi rentas obtenidas entre dueños y empleados. Esto no sólo genera distorsiones en la asignación de recursos e ineficiencias productivas (concepto que se desarrollará más adelante), sino que genera resistencia por parte de los propios empleados a modificaciones de las normas regulatorias. Esto es particularmente interesante para Uruguay, donde la oposición a la desregulación de los mercados de combustibles, telecomunicaciones o agua, provino principalmente de los propios empleados de las empresas monopólicas, que posteriormente volcaron a los consumidores a su favor.

Por último, \citet{Becker1983} sostiene que la regulación se da en un contexto de competencia entre grupos de interés. La regulación, según esta visión, es utilizada por aquellos grupos más poderosos para incrementar su bienestar. En última instancia, todos los grupos de interés generan alguna presión para intentar influir sobre las políticas y, a la vez, tienen que vencer el problema del parasitismo. El modelo que desarrolla Becker tiene algunos resultados de política.

En primer lugar, las políticas regulatorias que se implementarán serán aquellas que mejoran la eficiencia, debido a que las ganancias que obtienen los grupos de interés los inducirán a producir una mayor presión para que se regule. En segundo lugar, y vinculado a lo anterior, los sectores que tenderán a estar regulados son aquellos con más fallas de mercado. Por último, no sólo se regularán los sectores con fallas de mercado, sino también aquellos donde los grupos de interés sean más eficientes para ejercer su presión. En otros términos, no sólo la ganancia de eficiencia determina el resultado, sino la tecnología que utilizan los grupos de presión para obtener su cometido.

\hypertarget{impuestos-a-travuxe9s-de-la-regulaciuxf3n}{%
\section{Impuestos a través de la regulación}\label{impuestos-a-travuxe9s-de-la-regulaciuxf3n}}

La regulación no sólo es un instrumento para corregir fallas de mercado o para que grupos de interés obtengan renta a su favor. \citet{Posner1971} sostiene que los gobiernos utilizan la regulación para redistribuir rentas entre consumidores, vía subsidios cruzados en la venta de productos. Ello puede obedecer a distintos factores. Un servicio donde es clásico el subsidio entre sectores es el transporte urbano de pasajeros. Bajo este servicio, que en muchos países tiene características de servicio público, se fija un único precio en todo el transporte, lo que se transforma en un subsidio implícito desde los recorridos de mayor demanda a los de menor demanda.

Como se verá en el capítulo \ref{inst-reg}, ello tiene efectos sobre la calidad del servicio, en la medida que, para que el mecanismo de subsidio cruzado se pueda sostener, se requiere impedir la entrada a los mercados más rentables, lo que en economía se conoce como el descreme de los mercados. Al existir dos demandas, el precio debería ser menor en el mercado donde la demanda es mayor, debido a que las posibilidades de oferta serán también mayores y los consumidores tienen más oportunidades de arbitraje, mientras que en los mercados con menor demanda habrá menos oferentes y el precio será mayor.

En telefonía, este tipo de subsidio cruzado se le conoce como servicio universal. Por ejemplo, en la telefonía fija, existe una tarifa única en ciudades de alta y baja demanda, cuando el costo unitario de la red es mucho más alto en las ciudades con menor densidad. En este caso, la regulación de único precio ---unida a la restricción de entrada al mercado, para evitar el descreme de los consumidores--- se utiliza como un instrumento que sustituye a los impuestos, como mecanismo que permite financiar las inversiones. En efecto, los altos costos asociados a la telefonía serán difíciles de recaudar en mercados donde la densidad o los ingresos son menores y, por tanto, un subsidio cruzado vía precio permite que el gobierno realice inversiones en aquellos mercados donde no tiene capacidad de recaudación, y que los consumidores con mayor capacidad los subsidien.

Fijar impuestos a través de la regulación es una práctica factible cuando la tecnología o las instituciones impositivas son débiles, ya sea porque la base tributaria es pequeña, o porque son altos los costos de recaudar, las distorsiones que conlleva la tributación o la evasión. En cualquiera de estos casos, en lugar de fijar un impuesto, recaudarlo, separar el dinero y pagar a la empresa, es más fácil y sencillo que la propia empresa cobre el dinero directamente a través del precio. Ello genera los incentivos adecuados a la empresa para cobrar por el producto. Sin embargo, por otro lado, esta regulación genera barreras a la entrada al mercado que luego son difíciles de alterar.

\hypertarget{regulaciuxf3n-y-cumplimiento-de-las-normas}{%
\section{Regulación y cumplimiento de las normas}\label{regulaciuxf3n-y-cumplimiento-de-las-normas}}

Las teorías anteriores presentan una visión de la regulación como un instrumento en sí mismo, aislado de otros disponibles por los gobiernos para controlar el accionar de las empresas en los mercados. Sin embargo, \citet{Djankov2003} y \citet{Shleifer2005} presenta una visión diferente de la regulación. Sostienen que si la sociedad quiere, por alguna razón, controlar a las empresas privadas dispone de cuatro estrategias diferentes, las que presentan una tensión entre los poderes del Estado y las libertades individuales. Esta tensión, en última instancia, se refleja en que a menor intervención del Estado, mayores son las pérdidas sociales debidas a la expropiación privada (la denominan desorden), mientras que a mayor intervención del Estado mayores son las pérdidas debido a la expropiación del Estado (la denominan dictadura). Si el Estado se retira de la vida privada, los agentes quedan a merced del desorden, pero se minimizan los riesgos de dictadura, mientras que al aumentar la participación del Estado se reduce el desorden, pero aumentan los riesgos de dictadura. En otros términos, los agentes privados pueden expropiarse unos a otros, pero el Estado también puede expropiar a los privados.
\footnote{Esta visión está muy vinculada a las ideas de los derechos de propiedad. Para que estos puedan ejercerse se requiere un Estado que los haga cumplir. Sin embargo, si el Estado no tiene contrapesos, también puede apropiarse de los derechos de los privados.}

Los cuatro instrumentos de que dispone una sociedad para controlar al sector privado son: (i) la disciplina de mercado; (ii) los litigios judiciales entre privados; (iii) la regulación; y (iv) la propiedad pública. Estas estrategias no son mutuamente excluyentes, dado que en muchas sociedades conviven entre sí. En Uruguay, por ejemplo, existe propiedad pública en el sector de telecomunicaciones, el que a su vez está regulado por un órgano regulador específico, los privados pueden llevar a juicio civil a la empresa pública y, en última instancia, pueden dejar de comprarle. El uso creciente de los instrumentos regulatorios balancea la forma en que la sociedad pondera los riesgos asociados al desorden y la dictadura. La clave es que, a medida que se hace más complejo para la sociedad controlar al sector privado (desorden), se comienza a recorrer un arco de posibilidades que implican una mayor intervención del Estado.

¿Por qué distintas sociedades resuelven un mismo problema de forma diferente? Esta pregunta no tiene una única respuesta. Las capacidades institucionales para regular de los países difieren entre sí. Se observan diferencias en relación con lo que las normas permiten hacer a los reguladores, la capacidad sancionatoria, el grado de tecnificación de la burocracia y la permeabilidad del sistema político a las presiones del sector privado. Asimismo, lo que algunas sociedades toleran que se resuelva a través del Sistema Judicial, otras requieren de una intervención directa del Estado \citep{Alesina2005}. Lo interesante de este enfoque es que permite comprender en forma amplia la existencia de regulación y empresas públicas. Hay teorías que abordan el tema de las empresas públicas como un asunto político \citep{Shleifer1994}, donde los políticos utilizan a las empresas públicas para mantenerse en el poder. Sin embargo, \citet{Shleifer2005} presenta una visión que fundamenta la existencia de empresas públicas como una solución eficiente desde el punto de vista institucional: dadas las restricciones que enfrentan los Estados, el avance en una dirección regulatoria representa la mejor solución posible para controlar el desorden.

\hypertarget{la-regulaciuxf3n-de-monopolios-naturales}{%
\section{La regulación de monopolios naturales}\label{la-regulaciuxf3n-de-monopolios-naturales}}

Esta sección se centra fundamentalmente en estudiar la regulación de aquellos sectores que no están sometidos a competencia debido a que es eficiente que la producción la lleve a cabo una única empresa. En algunos sectores de la economía ---en particular en transmisión de energía, agua potable, saneamiento, vías férreas, o transmisión de datos--- la tecnología de producción de bienes y servicios se caracteriza por la existencia de importantes inversiones fijas y hundidas.
\footnote{Las inversiones hundidas son aquellas que una vez realizadas no pueden asignarse a otros usos más allá de los previstos originalmente.}

Ello determina que la provisión de ciertos servicios públicos sea eficiente si la realiza un número limitado de oferentes y, en algunos casos, sólo si lo hace una única empresa. La necesidad de contar con activos fijos que tienen un costo muy importante y que en sí mismos permiten atender a toda la demanda, implica que la duplicación de esta infraestructura sea ineficiente desde el punto de vista económico. Uno de los supuestos que explica la existencia de mercados competitivos es que las empresas que en ellos operan tienen un tamaño tal que con una infraestructura dada no pueden atender a toda la demanda.
\footnote{Los supuestos se completan con la libre entrada y salida de los mercados, la perfecta homogeneidad de los bienes, la ausencia de costos de transacción, información perfecta por parte de los agentes y, en el largo plazo, igual acceso a la tecnología de producción.}
Ello permite que varias empresas puedan coexistir para atender el mercado. Por su parte, cuando es eficiente que una única empresa atienda a todo el mercado se entiende que existe un monopolio natural.
\footnote{Para un desarrollo sencillo de la teoría de monopolio natural ver \citet{Viscusi2005}, capítulo 11.}

Para avanzar en este concepto, hay que considerar que la eficiencia económica es un fenómeno que tiene múltiples dimensiones.
\footnote{En \citet{Viscusi2005}, capítulo 4, se presenta una exposición del concepto de eficiencia. Para un desarrollo en profundidad ver \citet{Motta2004}, capítulo 2.}
Eficiencia refiere a la forma en la que se utilizan los recursos para obtener un determinado nivel de producción. Sin embargo, este concepto tiene dos dimensiones según si la mirada es de corto o largo plazo. En el corto plazo, esto es cuando se considera un mercado o tecnología de producción dada, se define la \textbf{eficiencia estática}.

Por su parte, \textbf{la eficiencia dinámica} es un concepto de largo plazo que refiere a la forma en la que se introducen nuevos productos o nuevos procesos de producción. A vía de ejemplo, cuando se habla de eficiencia en el mercado de la telefonía fija se está analizando la eficiencia estática del sector, el producto está dado y la tecnología de producción también. Por su parte, cuando se estudia la introducción de fibra óptica al mercado de transmisión de datos, dado que éste es un producto nuevo que crea un mercado con características diferentes a los existentes ---permite brindar más servicios que la tecnología tradicional---, se analiza la eficiencia dinámica.

Este último tipo de eficiencia, implica estudiar los incentivos que tienen los agentes para llevar a cabo actividades de investigación y desarrollo (I+D) que permitan introducir nuevos procesos productivos o desarrollar nuevos productos. La clave está en el proceso por el cual se generan nuevos mercados, sea de productos o procesos de producción. La eficiencia dinámica es esencial en aquellos mercados, como el de las telecomunicaciones, donde la norma es la introducción permanente de nuevas tecnologías. Estas nuevas tecnologías pueden provocar cambios en los mercados existentes y hacer que muchos de ellos desaparezcan o se transformen en forma radical en períodos breves.

A su vez, el énfasis en cada tipo de eficiencia tiene asociada una visión diferente de la competencia. Cuando se analiza la eficiencia estática se está estudiando la competencia \textbf{en el mercado}, mientras que si se analiza la eficiencia dinámica el énfasis está en la competencia \textbf{por el mercado}. Las competencias en y por el mercado tienen lógicas diferentes aunque en ambas subyace la misma noción de empresas compitiendo entre sí.
\footnote{La importancia de la competencia en el desarrollo económico ha sido desarrollada por \citet{Aghion2005}.}

La eficiencia estática se centra en comprender cómo compiten las empresas y cómo utilizan sus recursos productivos en un mercado dado con una tecnología de producción específica; y tiene dos miradas:

\begin{enumerate}
\def\labelenumi{\arabic{enumi}.}
\item
  \textbf{eficiencia asignativa} que refiere al excedente o bienestar económico que genera una determinada estructura de mercado. Un mercado competitivo genera un excedente mayor que una estructura monopólica debido a que el monopolio tiene incentivos a restringir la oferta, de forma de aumentar el precio de mercado y sus beneficios. Por tanto, el monopolio genera una ineficiencia asignativa o pérdida de eficiencia asignativa.
\item
  \textbf{eficiencia productiva} que analiza la forma en la que se utilizan los insumos para generar un nivel de producto dado. La clave es determinar si existe derroche de recursos. Este análisis tiene a su vez dos niveles. Por un lado, si existen diversas tecnologías disponibles, cuál es la que permite obtener un nivel dado de producto al menor costo posible. Por otro lado, para un nivel dado de tecnología, si la misma se utiliza en todo su potencial. En el análisis de la eficiencia productiva surgen diversos factores que determinan que una empresa sea ineficiente. Puede ser el resultado de una mala elección de la tecnología, por ejemplo vender servicios de transmisión de datos utilizando
  un par trenzado (cable de cobre) en vez de fibra óptica;
  \footnote{Véase \url{http://es.wikipedia.org/wiki/Medio_de_transmisi\%C3\%B3n\#Medios_de_transmisi.C3.B3n_guiados}}
  o puede ser el resultado de utilizar fibra óptica pero llegar a pocos clientes y, por tanto, que el servicio sea muy costoso.
\end{enumerate}

La clave en el análisis de la eficiencia productiva está en que la elección de la tecnología y su uso dependen de los incentivos que enfrenten las empresas \citep{Nickell1997}. En general, la competencia en el mercado se transforma en un incentivo suficiente para que las empresas utilicen la mejor tecnología disponible y la usen al máximo de su potencial, ya que de otra forma corren el riesgo de quebrar.
\footnote{Véase \citet{Motta2004}, sección 2.3.2.}
En estos casos la competencia se transforma en un mecanismo de selección de tipo darwinista en el cual sólo sobreviven las empresas más eficientes. Otras veces, la forma de propiedad sirve como incentivo al uso eficiente de los recursos. Si la empresa cotiza sus acciones en la Bolsa y es ineficiente, entonces obtendrá menores beneficios de los potenciales y, por ello, corre el riesgo de perder valor para sus accionistas. Por lo tanto, es más probable que sus gerentes sean despedidos, mecanismo que tienen los accionistas para proteger su inversión.
\footnote{\citet{Laffont1993} señalan que las empresas públicas pueden también sufrir el relevo de sus directivos, o del propio gobierno, pero las causas son generales -no dependen del funcionamiento específico de la empresa- y los incentivos de los políticos a mirar el resultado de la empresa son menores a los de los inversores.}

La existencia de un monopolio natural en determinados mercados, se explica por la eficiencia productiva. Cuando resulta eficiente que la producción del bien o servicio se realice en una única empresa ---los costos de producción son menores respecto a fragmentar la producción en varias firmas---, entonces existe un monopolio natural. Es claro que esta característica de la función de costos depende, también, de la forma de la demanda. Cuanto menor sea la demanda más factible es que la inversión realizada por una única empresa alcance para atenderla. En Uruguay, una única planta de refinamiento de petróleo es suficiente para abastecer a todo el mercado interno, lo que resultaría imposible en el caso de Argentina, que cuenta con unas diez refinerías de petróleo.

Cuando es eficiente que una única empresa sirva al mercado (eficiencia productiva), esta única empresa, si maximiza beneficios, se comportará como un monopolista. A menor competencia en el mercado, donde el monopolio es la expresión extrema, más restringirán las empresas su producción de forma tal que los productos resulten escasos y los consumidores paguen precios mayores por los mismos.

Por tanto, estos mercados conllevan un compromiso entre eficiencias que tienen direcciones opuestas: por un lado, la eficiencia productiva ---costos--- favorece la existencia de una única empresa en el mercado; por otro, la eficiencia asignativa ---poder de mercado--- implica que surja el riesgo de abuso de posición de dominio por parte del único oferente. La forma en la que los Estados han resuelto este compromiso es permitir la existencia de una única empresa, pero regular su precio para evitar abusos. Este balance explica el surgimiento de la regulación por parte del Estado.

Los servicios públicos, generalmente, se encuentra en mercados donde al menos un componente de la actividad constituye un monopolio natural. Estos servicios se caracterizan por: (i) un alto monto y especificidad de los activos involucrados; (ii) la existencia de economías de escala o de variedad; y (iii) el consumo masivo de muchos de los bienes y servicios que producen. Las \textbf{economías de escala} implican que los costos unitarios de producción decrecen a medida que la producción se expande, mientras que las \textbf{economías de variedad} indican que la producción conjunta de determinados bienes presenta costos menores que si se producen en forma separada \citep{Bergara2003}. Ambas economías explican la existencia de monopolios naturales.

Asimismo, en la provisión de muchos servicios públicos se observa sectores de onopolio natural que conviven con segmentos potencialmente competitivos. Por ejemplo, el servicio de electricidad tiene distintos componentes: generación, transmisión y distribución. Mientras la transmisión eléctrica es un ejemplo típico de monopolio natural, tanto la generación como la distribución son segmentos donde puede operar la competencia.
\footnote{Una descripción detallada del sector eléctrico se puede encontrar en \citet{Viscusi2005},capítulo 12.}
Por su parte, la fibra óptica ---que tiene características de monopolio natural--- es un insumo necesario para actividades que se desarrollan en mercados que pueden ser competitivos como la transmisión de datos, ya sea telefonía, internet o televisión para abonados.
\footnote{No hay estudio para Uruguay que concluya que la fibra óptica sea un monopolio natural, aunque existen pistas de que si podría serlo en muchos mercados geográficos.}

A modo de resumen, en el Tabla \ref{tab:cuadro1} se presentan los principales tipos de eficiencia y su impacto sobre la estructura de mercado.

\begin{table}

\caption{\label{tab:cuadro1}Tipos de eficiencia, características e impacto sobre estructuras de mercado.}
\centering
\begin{tabular}[t]{l|l|l|l|l}
\hline
Tipo de eficiencia & Tipo de competencia & Categorías & Características & Efecto de la competencia\\
\hline
Estática & En el mercado & Asignativa & Asociada a poder de mercado & Mejora la eficiencia\\
\hline
Estática & En el mercado & Productiva & Asociada a la tecnología & Resultado ambiguo\\
\hline
Dinámica & Por el mercado & Dinámica & Asociada a la innovación & Ex ante, induce la innovación\\
\hline
\end{tabular}
\end{table}

Fuente: elaboración propia.

La regulación surge como un mecanismo para controlar los incentivos que tienen las empresas monopólicas a cobrar precios abusivos a los consumidores, es decir resolver una falla de mercado. Como resultado de la misma, el mercado se suplanta por un mecanismo burocrático que define varias de las variables relevantes: el precio, la calidad de los productos, o las inversiones que puede o no realizar la empresa. La regulación no es una solución perfecta, en la medida en que los reguladores tienen en general menos información que la empresa regulada y ello puede impedir alcanzar resultados de primer óptimo.
\footnote{El resultado de primer óptimo es el que se obtendría si el regulador tuviera el mismo nivel de información que la empresa. Si ello no es así, entonces ese resultado no se puede implementar, y se deben establecer valores de las variables que induzcan a las empresas a revelar su información y a reducir, en la medida de lo posible, las distorsiones de eficiencia.}
Sin embargo, es aceptado que en el caso de los monopolios naturales es preferible un mecanismo que concentre la producción en una empresa ---eficiencia productiva--- y que regule las demás variables relevantes para evitar el abuso vía precio ---eficiencia asignativa.

La regulación tiene en sí misma complejos problemas de implementación: ¿cuál es la discrecionalidad óptima del regulador?; ¿cómo se debe fijar y revisar las tarifas?; ¿qué costos deben considerarse?; ¿cuál es la tasa de retorno adecuada de los proyectos?; ¿qué mínimos de calidad se deben exigir a las empresas?; ¿qué justificaciones se aceptan para los apartamientos?; ¿se prevé la expropiación de la empresa o no?; ¿se puede introducir nueva competencia en mercados relacionados? Todos estos elementos determinan la forma en la que se diseñan las instituciones regulatorias e involucran un fino balance entre flexibilidad y expropiación \citep{Bergara2003, Spiller2013}.
\footnote{La regulación de servicios públicos, por su naturaleza, presenta dificultades a la hora de fijar reglas claras y hacerlas cumplir, siendo el marco institucional que delimita estas actividades un ingrediente clave para alcanzar los objetivos regulatorios \citep{Bergara2003}.}

Por otra parte, el órgano regulador dispone de diversos instrumentos a los efectos de evaluar y sancionar a las empresas que no cumplan con las regulaciones establecidas. En primer lugar, partiendo del paradigma tradicional regulatorio, donde las empresas tienen como objetivo claramente establecido maximizar los beneficios que obtienen por realizar la actividad productiva \citep{Armstrong2007}, un instrumento sencillo, a los efectos de alinear los objetivos de las empresas a los del regulador en las variables relevantes, es la aplicación de sanciones pecuniarias. Toda vez que la empresa incumpla con los términos del servicio, el regulador impone una sanción que reduce los beneficios de la empresa. Si la sanción es mayor al beneficio que obtiene la empresa por no cumplir con los objetivos impuestos por el regulador, entonces, evitará incurrir en estos comportamientos.
\footnote{Véase \citep{Becker1968}}

Un segundo elemento que incide en el comportamiento de las empresas privadas es su reputación. Si la empresa cotiza en Bolsa, esto es, obtiene su financiamiento a través del mercado de capitales, cualquier sanción que efectúe el regulador tendrá no sólo el efecto directo sobre los beneficios de la empresa, sino también sobre el valor de las acciones para sus titulares, lo que impacta en las decisiones que adoptan los gerentes.

Un tercer elemento, que recoge la historia de las empresas públicas en Uruguay, refiere a la amenaza de estatizar los servicios públicos. En efecto, si la calidad de los servicios prestados es deficiente, o si la empresa incumple lo establecidoen el contrato de concesión, entonces existe la posibilidad de que el Estado asuma la responsabilidad de la prestación del servicio, desplazando a la empresa privada. En la medida en que los servicios públicos tienen características de monopolio natural y el Estado es el responsable de que los mismos se brinden en forma adecuada a sus ciudadanos, si fallan los instrumentos regulatorios para alcanzar este objetivo, entonces el Estado debe hacerse cargo del servicio en cuestión. En estos tres casos, la regulación puede incidir sobre el accionar de la empresa privada a través de las sanciones económicas.

Asimismo, la provisión de servicios públicos por parte de empresas privadas implica alguna relación contractual entre el gobierno y la empresa proveedora. Esta relación puede establecerse a través de una concesión, una licencia, una autorización, u otra figura jurídica. En general, estos contratos establecen los objetivos generales de la tarea que debe realizar la empresa como operador, y el rol regulador y planificador del gobierno.

Sin embargo, la lógica regulatoria es diferente en el caso de empresas públicas. Estas empresas pueden perseguir objetivos distintos a la eficiencia y, por tanto, la regulación de empresas privadas y públicas encuentra límites diferentes. Sin embargo, la regulación de empresas públicas se encuentra con políticas cuya aplicación puede ser discutible y, por tanto, los instrumentos regulatorios tradicionales no pueden aplicarse en esos casos.

\hypertarget{conclusiones}{%
\section{Conclusiones}\label{conclusiones}}

La regulación es uno de los instrumentos con los que cuentan los Estados para intervenir en los mercados. Las regulaciones pueden obedecer a factores económicos o sociales. Si bien su origen ha estado vinculado a las fallas de mercado, también puede deberse a problemas redistributivos.
\footnote{Es decir, las fallas de mercado no son condición necesaria ni suficiente para la regulación.}
Las teorías queexplican la regulación van desde una visión optimista del Estado, en términos de su accionar desinteresado para resolver los problemas del mercado, hasta hipótesis que explican la regulación como el resultado de un proceso de captura por parte de la industria para obtener rentas.

En cualquier caso, la regulación es un proceso complejo \citep{Dixit1998}. Involucra el diseño de una institución reguladora, dotarla con fondos necesarios para llevar a cabo su tarea y con funcionarios capacitados para cumplir sus objetivos. Sus cometidos requieren un adecuado balance para que las instituciones sean efectivas. Por un lado, tienen que estar diseñados de forma lo suficientemente amplia para atender los problemas que motivaron su creación, evitando el arbitraje regulatorio por el cual las empresas eluden controles si pueden realizar acciones que no están reguladas, pero que generan los mismos efectos sobre el mercado que las que sí lo están . Por otro, estos controles deben ser lo suficientemente específicos para evitar que los reguladores se extralimiten en sus tareas o persigan otros intereses a los previstos.

No debe olvidarse que los reguladores, por más bien intencionados que puedan estar y aunque dispongan de recursos adecuados, contarán siempre con menos información que las empresas que deben regular. Ello determina que siempre estén un paso atrás de las empresas. Asimismo, como se verá en el capítulo \ref{inst-reg}, la decisión de regular un mercado afecta a un conjunto de variables y ello debe ser considerado a la hora del diseño regulatorio. Muchas veces fijar una variable altera los incentivos de los agentes y tiene impacto sobre otras de ellas, lo que puede más que compensar el efecto positivo original de la regulación.
\footnote{Un típico ejemplo son los precios máximos en mercados competitivos, que generan escasez.}
Por último, cuando el Estado regula será responsable, tanto de los buenos como de los malos resultados del mercado. La regulación genera siempre efectos sobre la eficiencia de los mercados.

\hypertarget{inst-reg}{%
\chapter{Instrumentos regulatorios}\label{inst-reg}}

Leandro Zipitría

Este capítulo presenta los principales instrumentos regulatorios y sus efectos en los mercados. Si el objetivo es estudiar los problemas regulatorios que surgen con motivo del poder de mercado de las empresas, el principal instrumento regulatorio que considera la literatura es el precio. Sin embargo, la efectividad de este instrumento para corregir los problemas detectados en el mercado dependerá de las características del primero, de la estructura del segundo, y del objetivo planteado por la regulación.

En general, los problemas que busca resolver la regulación son de eficiencia económica, ya que se presupone que para los asuntos distributivos existen otros instrumentos disponibles. No obstante, muchas veces los Estados intervienen sobre los mercados con el objetivo de resolver problemas distributivos, y es importante conocer los efectos que estos instrumentos tendrán sobre el mercado. Asimismo, en ciertas situaciones el precio no es suficiente para resolver el problema bajo análisis y debe ser complementado con otros instrumentos. Este capítulo presenta los principales instrumentos regulatorios y sus efectos en los mercados.

Si bien en gran parte de este capítulo se trabajará como si el regulador pudiera conocer las variables relevantes de la o las empresas que pretende regular, en la gran mayoría de los casos este desconoce el funcionamiento del mercado y de las empresas que en él operan. Esta asimetría de información es muy difícil de resolver, ya que implica diseñar instrumentos que permitan a las empresas revelar las variables que el regulador desconoce.

Sin embargo, este acto de revelación nunca es desinteresado o altruista, sino que se inscribe en un marco estratégico de las empresas con vistas a influir en la regulación. Para que las empresas revelen la información que disponen, deben tener incentivos, lo que implica costos para el regulador. Por ello, el gobierno en general, y los reguladores en particular, conocerán en forma relativamente indirectael funcionamiento de los mercados y siempre estarán menos informados que las empresas. En otros términos, a veces se señala que la regulación viene rezagada respecto a las acciones del mercado, producto de estas asimetrías de información.

Los instrumentos regulatorios son diversos.
\footnote{En el capítulo \ref{def-comp} se realiza una descripción de un conjunto de instrumentos que se utilizan para analizar los mercados y fomentar la competencia.}
Unos instrumentos fijan o limitan algunas de las variables del mercado y, por tanto, restringen las acciones de las empresas. El precio es la principal variable regulatoria y su determinación, por parte del regulador, influye sobre los beneficios de las empresas y, con ello, sobre otras acciones que estas pueden llevar a cabo. Precio no sólo refiere al valor de un bien o servicio, sino también al momento o valor por el que se incrementa.

El regulador, también, puede establecer restricciones para el ingreso al mercado, limitando o simplemente impidiendo el ingreso de nuevas empresas, o puede establecer restricciones a la salida de empresas del mercado, aun cuando estas tengan pérdidas por vender sus bienes en el mercado. Estas dos restricciones, al ingreso y a la salida, se utilizan, por ejemplo, cuando se regula el transporte público, e implica impedir que una empresa abandone una determinada frecuencia aun cuando esta sea deficitaria (restricción a la salida), o en otros casos impedir el ingreso a frecuencias con mayor demanda (restricción de entrada). Otro instrumento que puede utilizar el regulador es someter a su control o aprobación la realización de determinadas inversiones por parte de las empresas, como la incorporación de determinadas tecnologías, y, en algunos casos, puede obligar a las empresas a invertir en ciertos activos específicos.

Existen diversas formas de presentar los instrumentos regulatorios. Se puede analizar cada instrumento (precio, entrada y salida, calidad, etc.) y determinar el efecto que tiene en los distintos mercados. Alternativamente, se puede analizar cada estructura de mercado y estudiar el efecto que los distintos tipos de intervenciones tienen sobre su desempeño. Esta última alternativa es la que se seguirá en este capítulo, lo que permitirá tener una visión integral del funcionamiento de los instrumentos regulatorios y de los efectos de la regulación, tanto deseados como no deseados. Asimismo, se vincularán los instrumentos regulatorios con otros que se analizarán en capítulos siguientes, como la defensa de la competencia.

Aunque parezca evidente, lo primero que hay que tener claro cuando se piensa en regular un mercado es el por qué. En principio, los economistas sostieneque la regulación es necesaria para resolver algún tipo de falla de mercado. Sin embargo, muchas veces la regulación existe por inercia o tradición;
\footnote{El caso más claro en Uruguay es el de la leche fresca, cuyo precio máximo se encuentra regulado desde hace décadas.}
o porque se persiguen otros fines, como distributivos o asegurar la prestación de un servicio;
\footnote{El transporte público, a vía de ejemplo.}
o porque se busca atender otras fallas de mercado, como se analizó en el capítulo \ref{reg-ec}. La regulación, también, surge por demanda de los actores en los mercados.
Una vez que se ha resuelto regular ---y se sabe por qué---, el siguiente paso es determinar el instrumento adecuado que resuelve el problema identificado. Luego se requiere establecer una infraestructura que permita fijar el instrumento elegido y monitorear el desempeño del mercado. Idealmente, habría que realizar evaluaciones para analizar el funcionamiento del instrumento aplicado, aunque, en la práctica, esto se hace con poca frecuencia.
\footnote{Además de los propios regulados, que muchas veces defienden la regulación, también los reguladores estarán interesados en mantenerla. Una vez que el mecanismo ha sido puesto en marcha, es difícil desactivarlo.}

En términos ideales, cualquier regulación que impacte sobre el mercado debería contar con un análisis costo-beneficio, y sólo deberían ser aprobadas aquellas cuyos beneficios superen a los costos. En Europa o Inglaterra, existen lineamientos de evaluación de impacto regulatorio, que apuntan a la \textbf{mejor regulación}, en oposición a menor regulación.
\footnote{Véase \citet{UE2015} y \citet{BRE2015}}
Sin embargo, en los últimos años la evaluación costo-beneficio ha cambiado su foco hacia una evaluación de políticas, que es un campo mucho más amplio que el que se analiza aquí.
\footnote{El lector interesado puede ver una aproximación amigable a estas técnicas en \citet{Gertler2011}}
Mientras la evaluación costo-beneficio se centra en las regulaciones que afectan la competencia en los mercados, la evaluación de políticas es un instrumento mucho más amplio para determinar la efectividad de las políticas públicas, incluidas las regulaciones de mercado.

En la siguiente sección se analiza la regulación de mercados competitivos, mientras que en la segunda sección se analizan los mercados oligopólicos. En la tercera sección se analizan los instrumentos regulatorios en los mercados de monopolios naturales. Al final, se presentan algunas reflexiones sobre el impacto del uso de los instrumentos regulatorios en los mercados.

\hypertarget{regulaciuxf3n-de-mercados-competitivos}{%
\section{Regulación de mercados competitivos}\label{regulaciuxf3n-de-mercados-competitivos}}

El juego de mercado ---la interacción entre oferta y demanda--- arroja un resultado de equilibrio en términos de precios y cantidades transadas. Si el mercado es perfectamente competitivo, se obtienen algunos resultados deseables desde el punto de vista económico.
\footnote{El adjetivo deseable hace referencia a la eficiencia del resultado, no a si el mismo es justo o equitativo, cualquiera sea la forma en la que se definan.}
Para avanzar en los efectos de la regulación sobre estos mercados, primero hay que entender cuáles son sus características y evaluar su desempeño.
\footnote{Esta sección y la siguiente están basadas en \citet{Viscusi2005}, capítulo 16.}

¿Qué caracteriza a un mercado perfectamente competitivo? Los supuestos tradicionales incluyen: (i) atomicidad de oferentes y demandantes; (ii) perfecta homogeneidad del producto; (iii) información perfecta de todos los participantes respecto al precio, ubicación y stock de los productores y productos; (iv) igual acceso a la tecnología de producción, que impide que alguna empresa tenga una ventaja sobre las restantes; y (v) libre entrada y salida del mercado. Básicamente, estos supuestos permiten que los consumidores arbitren a los productores, estos últimos estén en igualdad de condiciones unos con otros y puedan aprovechar las oportunidades que el mercado les provee.

En estos mercados, las empresas actúan como tomadores de precio ya que cualquier intento por aumentarlo inducirá a los consumidores a arbitrar a los vendedores. Salvo en los mercados donde hay rendimientos crecientes a escala en la producción ---un elemento que incide en la existencia de un monopolio natural--- en el resto de los mercados es factible encontrar equilibrios competitivos. Los rendimientos crecientes a escala son una propiedad de la función de producción ---aquella que permite transformar insumos en productos--- y que determina que los costos promedio de producción disminuyan al aumentar el volumen producido.

La atomicidad es un supuesto de importancia menor, con relación a los otros supuestos. La Teoría de los Mercados Disputables, pone el énfasis en la competencia potencial de los posibles entrantes, respecto a la competencia efectiva en los mercados. Si bien esta teoría requiere de supuestos relativamente fuertes ---como la ausencia de costos hundidos en el mercado y la homogeneidad del producto--- pone sobre el tapete la importancia de la amenaza a la entrada como mecanismo que disciplina a las empresas establecidas.
\footnote{Los costos hundidos son aquellos que no se recuperan si la empresa quiere salir del mercado.}
Si la competencia potencial es lo suficientemente fuerte, no es necesario que el mercado esté atomizado para obtener los resultados de competencia perfecta.

El equilibrio de competencia perfecta tiene algunas propiedades deseables. En primer lugar, es la estructura con el menor precio relativo, ya que cualquier estructura alternativa ---oligopolio o monopolio--- implica que las empresas tendrán un cierto grado de poder de mercado. En segundo lugar, y como corolario del anterior, la cantidad transada en ese mercado será la mayor posible. En tercer lugar, los consumidores obtendrán el mayor excedente posible, en la medida en que se cumplen los dos primeros puntos. Desde el punto de vista del consumidor, no hay situación de mercado más beneficiosa que la competencia.

En economía, el \textbf{excedente del consumidor} es la diferencia entre lo que este está dispuesto a pagar por el producto y lo que efectivamente paga en el mercado. En competencia perfecta, como el precio de todos los bienes es el menor posible, la cantidad de bienes será la mayor posible y el excedente del consumidor será máximo. En cuarto lugar, las empresas producen al menor costo posible, ya que la producción se realiza en el punto donde los costos medios son mínimos. Esta condición técnica se alcanza debido a que el ingreso de nuevas empresas al mercado llevará la producción individual al punto en el cual se minimizan los costos. Por último, en competencia perfecta las empresas realizan un beneficio económico cero en el largo plazo. Desde el punto de vista de la eficiencia, el resultado de competencia perfecta es deseable.

Si la competencia en el mercado no es suficiente, lo que provoca precios altos, entonces existe una serie de pasos previos a la regulación. En primer lugar, actuar sobre las posibles barreras o regulaciones estatales que dificulten el ingreso de empresas al mercado e impidan el ajuste competitivo. En segundo lugar, si estas barreras son producto del accionar de las empresas, impulsar acciones de defensa de la competencia, siguiendo el análisis que se realiza en el capítulo \ref{def-comp}.

Sin embargo, puede suceder que el Estado quiera regular un mercado competitivo, a pesar de sus beneficios, fijando directamente el precio del producto. Ello tendrá un impacto negativo sobre el bienestar, ya sea porque los precios aumentan, o porque se genera escasez. A continuación se presentan distintos escenarios donde un regulador actúa sobre alguna variable en un mercado competitivo.

En un primer escenario, supongamos que el regulador fija un precio por debajo del precio competitivo. Ello tiene diversos impactos sobre el mercado. En primer lugar, dado que el precio es menor al de equilibrio, la oferta a ese precio será menor a la demanda y, por tanto, habrá escasez de productos. Un ejemplo de esta situación, lo observamos hoy en Venezuela, donde los precios de los productos están fijados administrativamente considerando márgenes de ganancia ``adecuados''. Como estos precios no reflejan la escasez relativa de los productos, se genera una escasez de productos básicos que es creciente con la distancia entre el precio fijado y el precio de equilibrio.
\footnote{Sobre los márgenes en Venezuela, véase el artículo 5o de la \href{http://www.superintendenciadepreciosjustos.gob.ve/sites/default/files/PROVIDENCIA\%20070-2015.pdf}{Providencia Administrativa No.~070/2015}. Sobre la escasez en Venezuela, véase la información en Wikipedia:\href{https://es.wikipedia.org/wiki/Escasez_en_Venezuela}{``Escasez en Venezuela''}}

En segundo lugar, y asociado a este efecto, los consumidores tenderán a sobre utilizar o sobre demandar el producto, ya que la baja en el precio expande la demanda. Esta sobre demanda surge porque el precio no refleja el equilibrio de mercado ni las posibilidades de los oferentes. Ello tiene un efecto que debe ser considerado cuando se regulan algunos servicios públicos como el transporte, ya que si el precio es muy bajo, no habrá capacidad que soporte la demanda. Si, por otra parte, el precio se fija por encima del competitivo, se produce el efecto inverso, ya que la demanda cae y la oferta es excesiva.

En cualquier escenario, regular un mercado competitivo lleva a una pérdida de eficiencia con relación a no regularlo. Pero si la decisión es la de fijar un precio ---máximo o mínimo---, entonces es mejor complementarlo con restricciones a la salida o entrada al mercado.

En un segundo escenario, supongamos que se regula fijando un precio mínimo (máximo) por encima (debajo) del competitivo y operan restricciones a la entrada (salida) al mercado. En este caso, nuevas empresas querrán entrar (salir) del mercado, debido a que existe una renta positiva (negativa). Sin embargo, como la demanda cae (aumenta) las empresas producirán por debajo (encima) del punto donde los costos medios son mínimos, lo que en cualquier caso aumenta su costo de producción. En definitiva, restringir la entrada (salida) del mercado, permite que las empresas mejoren la eficiencia productiva, al disminuir sus costos de producción respecto a una situación en la cual existe libre entrada y salida.

Sin embargo, si bien es fácil exponer estos mecanismos, es mucho más complejo en la realidad poder encontrar los puntos donde oferta es igual a demanda y manejar un número de oferentes en el mercado. No debe olvidarse que mientras las empresas reciben directamente los efectos de sus acciones, los reguladores en cambio sólo perciben estos efectos en forma indirecta.

Ambas regulaciones (fijación de precio y restricción de entrada/salida) tienen impactos sobre los incentivos de los agentes, que alteran otras variables competitivas. En el caso en que se fija un precio máximo y se limita la entrada, se producen al menos dos efectos. En primer lugar, al establecerse el precio del producto las empresas recurrirán a otros mecanismos para competir. En particular, buscarán diferenciarse por calidad u otros atributos del producto, de forma de atraer a la demanda. Ello a la vez genera un problema dinámico, ya que las empresas tenderán a aumentar la calidad más allá de lo que prevalecería en un mercado sin regulación de precio, lo que, a su vez, determina que estas empresas no podrán sobrevivir si se elimina la regulación, ya que deberán competir con empresas cuya calidad será menor.

En otros términos, la sobre inversión en calidad es una respuesta de los agentes a la regulación de precio. Esta posibilidad de que las empresas inviertan en calidad surge siempre que las empresas no puedan coludir entre sí. Si lo hacen, entonces pueden evitar realizar estas inversiones y simplemente repartirse la renta extra. Esta posibilidad lleva al segundo efecto, ya que la restricción de entrada afecta los incentivos de las empresas a controlar la ineficiencia productiva. En efecto, cuando la amenaza de la competencia es laxa, las empresas tienden a relajar sus controles sobre los costos, lo que incrementa la ineficiencia productiva.

Muchas veces, como se vio en el capítulo \ref{reg-ec}, las empresas buscan generar una demanda por regulación y reparten renta con sus empleados, de forma de generar grupos de interés más fuertes para sostener las restricciones (y la renta). A la vez, este aislamiento relativo que enfrentan las empresas determina que el mecanismo darwinista de selección falla, y las empresas ineficientes no salen del mercado ya que están protegidas por el precio alto y la regulación de entrada.

En el caso en que la regulación establece precios por debajo de los competitivos, y se incluye una restricción a la salida (o una obligatoriedad de brindar el producto), nuevamente encontramos dos efectos. Por un lado, si el precio no reconoce el retorno al empresario, o si directamente no cubre los costos, pero existe una obligación de vender el producto, entonces se verá afectada su calidad.

En Uruguay, ello pasaba con el pan tarifado, que era un pan que las panaderías estaban obligadas a ofertar a un precio determinado. En caso de no contar con este producto, debían sustituirlo por otro de mayor calidad. En los hechos, el precio que se fijaba por el producto apenas cubría el costo de la harina para producirlo, con el consiguiente impacto sobre la calidad del mismo. Por otra parte, esta regulación afecta la inversión en formación de capital de la empresa. Si el precio de mercado no cubre los costos de reposición del capital, entonces las empresas dejarán de invertir y, en algunos casos, hasta de mantener las inversiones realizadas, con el consiguiente deterioro sobre la eficiencia.

Otro escenario posible es cuando el regulador establece subsidios cruzados, afectando a más de un mercado al mismo tiempo. Si existen dos mercados con demandas y costos diferentes, pero se fija un único precio que es a la vez mínimo en un mercado y máximo en el otro, entonces el regulador estará fijando esquemas de subsidios entre mercados. Un resultado paradójico de este tipo de políticas es que ningún consumidor estará satisfecho. Los consumidores que pagan el producto más caro, no estarán satisfechos sabiendo que la entrada de más empresas implicaría mejores precios, mientras que aquellos que lo pagan más barato, en general, encuentran restricciones de oferta dado el precio que enfrentan. Una práctica más eficiente de fijación de precio, que se verá en la sección \ref{monop-nat}, es establecer una política de precios de Ramsey. En los hechos esta política implica fijar precios en cada mercado atendiendo a la demanda en cada uno de ellos.

En Uruguay, el transporte terrestre de pasajeros tiene este tipo de políticas. El Poder Ejecutivo fija una tarifa por kilómetro recorrido, independientemente del destino. Asimismo, se asigna a dos empresas las frecuencias entre dos ciudades y también se reparten los destinos entre las empresas, de forma que tengan destinos rentables y otros poco rentables. Por tanto, existe un subsidio cruzado entre los destinos con un precio fijado independientemente de la demanda y restricciones a la entrada y salida de los mercados.

\hypertarget{mercados-oligopuxf3licos}{%
\section{Mercados oligopólicos}\label{mercados-oligopuxf3licos}}

Si bien los mercados perfectamente competitivos tienen características económicas deseables, los mismos son la excepción, principalmente en economías pequeñas. Pero aun cuando los mercados no sean perfectamente competitivos, puede haber competencia entre oferentes aunque su número sea reducido. Si esto ocurre, entonces el mercado es un oligopolio.

A diferencia de los mercados de competencia perfecta, las empresas en oligopolio enfrentan una demanda con pendiente negativa. En los mercados perfectamente competitivos se suponía que cada empresa tenía un tamaño tal que no podía atender a todo el mercado y, por tanto, las empresas eran tomadoras de precio. Por ello, se enfrenten a una demanda perfectamente elástica al precio de mercado: precios mayores sacarán a la empresa del mercado, precios menores arrojarán beneficios menores a mantener el precio, o aún negativos. Sin embargo, cuando hay pocas empresas en el mercado cada una tiene cierto grado de poder de mercado y, por tanto, cualquier aumento que fije en el precio provocará una caída de su demanda que pasará a sus competidores. La demanda que enfrenta cada empresa tiene una relación inversa con el precio que fija.

Existen múltiples razones por las cuales los mercados no funcionan en forma perfectamente competitiva. En primer lugar, los bienes difícilmente son homogéneos. Todas las empresas buscan diferenciar sus productos a través de la publicidad, presentación o estrategias de marketing. Asimismo, aun cuando los bienes sean homogéneos, como la gasolina, los puntos de venta pueden estar ubicados a distancias que hacen que los consumidores los perciban como productos diferentes. Los mismos productos vendidos en dos tiendas en dos barrios diferentes serán productos distintos para el consumidor, toda vez que el costo de trasladarse de un lugar a otro haga inútil el esfuerzo de arbitrar a los productores. Este concepto está vinculado al de mercado relevante geográfico que se analiza en el capítulo \ref{def-comp}.

En segundo lugar, en muchos mercados existen barreras a la entrada y, en particular, costos hundidos que desalientan el ingreso de nuevas empresas. Como se señala en el capítulo \ref{def-comp}, los costos hundidos son muy importantes a la hora de considerar el ingreso a un mercado. Ejemplos de estos costos son la capacitación de personal, la publicidad, la inversión en la construcción de una marca o la instalación de activos específicos ---maquinaria--- para realizar una determinada actividad. Cuanto más específicos sean los activos, menos líquidos serán en caso de que el emprendimiento fracase, lo que incrementa los riesgos de entrar al mercado.

En tercer lugar, los consumidores no están perfectamente informados ni sobre las tiendas que venden los productos, ni sobre los precios que ellas fijan. Esta asimetría de información permite a las tiendas cobrar precios superiores a aquellos consumidores que tienen un costo de búsqueda mayor, ya sea porque tienen menos tiempo para informarse o una menor utilidad por hacerlo. Este desconocimiento, además, incrementa los costos de entrar al mercado ya que las empresas tienen que hacer conocer a los consumidores ---y, a veces, hasta a los propios comercios--- sus productos.

Un cuarto elemento es tecnológico. Las empresas tienen un tamaño de escala (escala mínima eficiente), que es la que define el nivel mínimo de producción eficiente para la empresa. Si existen economías de escala, definida como aquella situación en la cual los costos medios disminuyen a medida que aumenta la producción, entonces un mercado soportará un número acotado de productores. Si, además, la demanda del producto es relativamente acotada, entonces un mercado no podrá soportar un gran número de empresas, a menos que esté dispuesto a tolerar altas ineficiencias productivas, en términos de duplicación de activos fijos. Por tanto, el tamaño del mercado pone un límite al número de empresas que pueden operar en él.

Si hay un número limitado de oferentes, los beneficios que obtendrán serán el decisiones de las firmas. Toda vez que una empresa fije una variable (precio, calidad, ubicación geográfica, publicidad, capacidad, etc.) ello tendrá impacto sobre los restantes oferentes, que a su vez reaccionarán. A diferencia de los mercados competitivos, donde la única variable de decisión es la cantidad producida y ella no genera reacción de los demás oferentes, en un oligopolio las decisiones de las empresas estarán vinculadas entre sí.

El oligopolio presenta diferencias importantes respecto a la competencia perfecta. A los efectos de la regulación, la diferencia principal es que en competencia perfecta la libre entrada a los mercados conduce a un resultado eficiente (menores precios, mayor cantidad, menores costos), mientras que en oligopolio la libre entrada produce sobre o sub ingreso al mercado \citep{Mankiw1986}.

El exceso de ingreso al mercado se produce por el efecto \textbf{robo de negocio}, en el cual la empresa entrante observa el beneficio que podría obtener al ingresar, que está compuesto por beneficios genuinos que se generan en el mercado pero también por aquellos que quita a las empresas instaladas. Como a la sociedad sólo le importa los primeros (nuevos beneficios que se generan por el ingreso de la firma), puesto que los segundos (los que se apropia de empresas instaladas) constituyen un tema redistributivo, los incentivos a entrar al mercado son más fuertes de lo óptimo. Por tanto, entran demasiadas empresas al mercado. Este exceso de ingreso se produce, fundamentalmente, en mercados donde los bienes son relativamente homogéneos. En el caso de que los bienes sean diferenciados, la entrada de nuevas empresas al mercado genera un nuevo beneficio para los consumidores que se suma a la reducción en el precio, y es que aumenta la variedad de productos disponibles.

En este caso, como los productores no internalizan este efecto que producen sobre los consumidores, se puede producir una entrada menor a la óptima en el mercado. En realidad los productores tienen incentivos al ingreso, producto del efecto robo de negocio, pero aun así la entrada puede ser ineficiente.

Por tanto, según el paradigma con el que se evalúan las regulaciones en este texto ---la eficiencia--- existirían razones para regular la entrada en mercados oligopólicos. Sin embargo, regular mercados oligopólicos presenta múltiples problemas. El primero, es saber cuál es el número óptimo de empresas en el mercado, tarea que es muy compleja debido a las fuertes asimetrías de información entre el regulador y las empresas que participan o podrían participar en el mercado. Además, hay que saber si el problema es de ingreso excesivo o insuficiente.

En segundo lugar, el anuncio de una política regulatoria en estos mercados dispararía acciones por parte de los interesados de forma de manipular el proceso regulatorio a su favor, en los términos discutidos en el capítulo \ref{reg-ec}, que fomentaría la captura regulatoria. Ello es muy factible si se consideran las asimetrías de información señaladas anteriormente.

En tercer lugar, la libre entrada es un buen antídoto contra la ineficiencia productiva. Las empresas, si no están sometidas a la competencia, tienden a relajarse, disminuir la calidad de sus productos o reducir la innovación, en la medida en que no enfrentan la amenaza de dejar el mercado. Por todos estos problemas, es mejor soportar cierto grado de ineficiencia en la entrada de empresas, que crear una oficina para regular el número óptimo de ellas.

\hypertarget{monop-nat}{%
\section{Monopolios naturales}\label{monop-nat}}

En principio, todos los monopolios naturales deberían estar regulados.
\footnote{Esta sección se basa en \citet{Viscusi2005} y \citet{Joskow2007}.}
Desde el punto de vista de la eficiencia, el balance entre eficiencia productiva y asignativa establece que la prioridad sea la primera, mientras la segunda se controle a través de la regulación. Sin embargo, existen monopolios naturales con distintas características. Por ejemplo, carreteras, aeropuertos, puertos u otras obras de infraestructura tienen características de monopolio natural. Un \textbf{monopolio natural} opera cuando es más barato ---menos costoso--- producir un conjunto de bienes dentro de una empresa que dividir la producción entre distintas empresas. Es decir, es más barato atender la demanda del mercado a través de una única empresa. Esta situación se da por una condición técnica de la función de costos, la subaditividad.

El monopolio natural puede ocurrir en la producción de uno o más productos. Cuando se analiza un único producto, una de las fuentes principales que influyen en el surgimiento de un monopolio natural, es la existencia de \textbf{economías de escala}. Muchos mercados utilizan tecnologías en las que operan economías de escala, como el gas por cañería, el agua potable o el saneamiento. Todos ellos tienen un elemento en común: existen importantes inversiones fijas y hundidas. Por ello, es eficiente que una única empresa abastezca al mercado, ya que duplicar la inversión fija aumenta los costos de llevar el servicio a cada uno de los hogares.

Las economías de escala son una condición suficiente, pero no necesaria para la existencia de un monopolio natural. En los hechos, puede ser que los costos medios sean crecientes para una demanda dada, pero permitir el ingreso de una nueva empresa y duplicar los activos fijos provoca que los costos medios se disparen por encima del incremento que se observaría con una única empresa, de forma que conviene mantener la producción en una única empresa.

Cuando las empresas producen más de un producto, entonces el análisis es similar, pero hay que agregar una nueva dimensión al análisis: los costos de producción pueden disminuir cuando se produce conjuntamente más de un bien. Esta característica técnica se conoce como \textbf{economías de alcance o variedad}, las que surgen por la presencia de un activo o insumo común a la producción de los distintos bienes.

Por ejemplo, el petróleo es un insumo que permite obtener gas, gasolinas y asfalto, lo que determina que existan economías de alcance en la producción conjunta de los bienes. La generación de energía eléctrica utiliza un único activo ya sea para generar electricidad para consumo residencial o industrial, que son dos demandas ---productos--- diferentes. Un único aeropuerto permite brindar servicios de pasajeros y carga, mientras que algo similar ocurre con las carreteras. Sin embargo, cuando las empresas producen más de un bien, para que exista un monopolio natural se requiere la existencia simultánea de economías de alcance y de escala.
\footnote{En realidad, se cumple cuando existen economías de alcance y costos medios decrecientes a lo largo de un rayo, que es una forma de economía de escala. La referencia clásica del análisis de monopolios naturales es \citet{Panzar1989}}

\hypertarget{regulaciuxf3n-ex-ante}{%
\subsection{\texorpdfstring{Regulación \emph{ex ante}}{Regulación ex ante}}\label{regulaciuxf3n-ex-ante}}

Un primer mecanismo para regular los monopolios naturales es sustituir la competencia ex post por la competencia ex ante. Esta teoría fue desarrollada por \citet{Demsetz1968} como una reacción a la regulación de precio de los servicios públicos. La idea es sencilla e implica fomentar la competencia entre oferentes en la etapa previa a la concesión de un bien o servicio. Si existe un número suficiente de empresas interesadas en desarrollar una actividad, construir un puerto o una carretera, entonces se puede licitar para adjudicar el monopolio a la empresa que oferte el precio más bajo por el servicio, dado un nivel de calidad mínimo.

Este tipo de procedimiento se puede utilizar para algunos servicios. La licitación resulta un instrumento adecuado, por ejemplo, para la construcción y mantenimiento de una carretera, un puerto o un aeropuerto. Este instrumento requiere establecer de antemano todas las variables relevantes (calidad del servicio e inversiones y su mantenimiento) mientras dure el período de concesión, lo que es factible sólo en algunos casos. En última instancia, se debe restringir el número de variables de decisión, de forma que se pueda asignar la licitación sobre la base de criterios objetivos establecidos \emph{ex ante}.

Una forma de hacerlo es incluyendo un proceso en dos etapas en la licitación, en la primera se seleccionan las ofertas en términos de la calidad del proyecto, y una vez seleccionados los que pasan el nivel mínimo establecido se procede a la elección del ganador por precio. Si no es posible establecer de antemano todas las variables relevantes para llevar a cabo el proyecto, debido, entre otros, a que existe incertidumbre sobre algunas variables del mercado, utilizar el procedimiento licitatorio puede llevar a renegociar el contrato durante la vigencia de la concesión. Si la empresa tiene una costosa inversión hundida, renegociar un contrato la pone en una posición débil para hacer valer sus intereses.
\footnote{Ver \citet{Bergara2003} capítulo 2, o \citet{Williamson1998}.}
En estos casos la credibilidad del gobierno para atarse de manos, esto es evitar expropiar a la empresa, pasa a ser el elemento determinante en el juego regulatorio.

Por otra parte, para que las inversiones en estos mercados se realicen, los plazos por los que se concesiona el servicio tienen que ser lo suficientemente extensos para que los inversores puedan recuperar la inversión a un precio razonable para el consumidor. Durante la vigencia de la concesión, tienen que existir reglas claras y creíbles para que el inversor tenga incentivos a invertir en los activos necesarios. Cuanta mayor seguridad exista respecto al procedimiento licitatorio y a la ejecución del posterior contrato de concesión, mayor número de oferentes habrá inicialmente, y menores tasas de retorno exigirán al proyecto. La clave es, por tanto, el diseño de los pliegos de la licitación.

\hypertarget{regulaciuxf3n-de-precio}{%
\subsection{Regulación de precio}\label{regulaciuxf3n-de-precio}}

El problema del monopolio es que fija precios excesivos respecto a los que prevalecerían en mercados más competitivos. Por tanto, se puede fijar un precio máximo en los mercados donde opera un monopolio natural. En principio supondremos que el regulador tiene a disposición toda la información relevante sobre el mercado y la empresa. En este marco, puede elegir los valores de precio que sean adecuados para minimizar la pérdida de eficiencia asignativa.

Si la empresa produce un único producto, el regulador podría replicar el resultado de competencia perfecta y fijar el precio igual al costo marginal, o sea al costo de producir la última unidad. Sin embargo, en los mercados de monopolio natural este tipo de estrategia impedirá que la empresa pueda cubrir sus costos fijos, ya que debido a la existencia de economías de escala, los costos medios son menores a los marginales en el tramo relevante donde se fijaría el precio.
\footnote{Esto no ocurre en un mercado competitivo, donde la regla de fijar el precio igual al costo de la última unidad producida permite recuperar los costos fijos.}

En general, este problema se puede resolver si se subsidia a la empresa los costos fijos. Sin embargo, ello es bastante difícil de implementar en términos políticos (subsidiar a un monopolio no parece una idea fácilmente defendible) y además los gobiernos, muchas veces, no cuentan con los recursos para pagar estos costos directamente. Por tanto, una alternativa es fijar el precio igual al costo medio y con ello generar recursos para sostener los costos fijos.

Otra alternativa posible es establecer una tarifa en dos partes, en vez de fijar un precio único. \textbf{Las tarifas en dos partes} involucran un pago fijo y luego un pago variable por unidad consumida. El pago fijo puede ser una tarifa de acceso, o un pago fijo mensual por el servicio. Algunas veces permite consumir un número fijo de unidades, otras veces es sólo un pago sin contrapartida.

Por ejemplo, la empresa de telefonía fija en Uruguay tiene un cargo fijo de cómputos y luego un cargo variable, a partir de un determinado número de unidades, mientras que la empresa de gas tiene un cargo fijo de metros cúbicos de gas y luego un cargo variable. En ambos casos, el esquema fijo se paga aún si no se consume o si se está por debajo del umbral establecido, mientras que si se supera el umbral se paga por unidad adicional. Lo interesante de este esquema es que se puede utilizar el pago fijo para recuperar los costos fijos y el precio para cubrir los costos variables. Esta alternativa es más eficiente, ya que fijar un precio igual al costo medio tiene implícita una ineficiencia asignativa, en la medida en que el precio que regiría es mayor al costo marginal. Con la tarifa en dos partes esta ineficiencia desaparece.

Las tarifas en dos partes son un tipo de tarifa no lineal, a través de la cual el precio varía según el consumo del agente. Otro tipo de tarifa no lineal es cuando se cobra distintas tarifas según el consumo del agente, en donde se fija un precio para las primeras \(X\) unidades consumidas, otro precio \(Y\) a partir de la unidad \(X+1\), y así sucesivamente. Este tipo de tarifa no lineal es la que se utiliza en Uruguay en el sector de agua potable. El precio del metro cúbico de agua consumida varía según el tramo de consumo y, en particular, es creciente con el consumo mensual.
\footnote{Ver \href{http://www.ose.com.uy/descargas/clientes/tarifas/ose_decreto_tarifario_2019.pdf}{tarifas de OSE}.}
Este tipo de tarifa busca desincentivar el uso excesivo de agua potable, penalizando a los consumidores que la utilizan para finesrecreativos.

Otro elemento a considerar en el caso de los servicios públicos ---dada la sensibilidad política de estas actividades--- es que muchas veces los precios fijados no reflejan el valor de los servicios sino que se implementan subsidios cruzados entre distintos sectores. Asimismo, el hecho de que opere una única empresa en el mercado genera incentivos a relajar el esfuerzo, dado que los consumidores están cautivos.

Por ello, no basta con determinar el precio en un marco de monopolio. El regulador debe establecer también mínimos de la calidad del servicio, dado que el monopolista tiene incentivos a utilizar la calidad del servicio para discriminar a los consumidores y aumentar sus beneficios. En efecto, dado un nivel de precio, si el monopolista puede elegir la calidad de los servicios entonces tenderá a ampliar la brecha entre la menor y mayor calidad. De esta forma, se discrimina a los consumidores con más disposición a pagar vía la compra del servicio de mejor calidad pero más caro.
\footnote{Ver \citet{Belleflamme2015}, sección 9.2.1.}

En la práctica, los órganos reguladores disponen de un conjunto mayor de instrumentos \citep{Berg2013}. Los mismos pueden agruparse en aquellos que son sustantivos para el cumplimiento de sus cometidos, y aquellos que son auxiliares y sirven para cumplir en forma adecuada con los primeros. Entre los instrumentos sustantivos que disponen los reguladores, además de fijar el precio del servicio, también se encuentran: (i) la emisión de las licencias que permiten operar en los mercados regulados (regulación de la entrada); y (ii) el establecimiento de los estándares de calidad del servicio (regulación de la calidad).

Por otra parte, es necesario que el regulador disponga de herramientas de apoyo para cumplir con sus cometidos fundamentales, en forma adecuada. Entre los instrumentos de apoyo se encuentran: (i) el control y auditoría del desempeño de las empresas; (ii) la posibilidad de imponer sanciones; (iii) el establecimiento de estándares contables; (iv) arbitrar disputas entre agentes; y (v) asesorar al Estado en la materia. El cuadro \ref{tab:cuadro2} resume los instrumentos regulatorios disponibles.

\begin{table}

\caption{\label{tab:cuadro2}Instrumentos regulatorios.}
\centering
\begin{tabular}[t]{l|l}
\hline
Cometidos sustantivos & Cometidos de apoyo\\
\hline
Determinacion del precio (servicio y acceso) & Auditoría y control\\
\hline
Fijación de la calidad & Establecer sanciones\\
\hline
Emitir licencias & Determinación de estándares contables\\
\hline
 & Arbitrar disputas\\
\hline
 & Asesorar al Estado\\
\hline
\end{tabular}
\end{table}

Fuente: elaborado en base a \citet{Berg2013}.

Existen algunos casos especiales donde la fijación de precios únicos o tarifas en dos partes no resultan útiles para generar los incentivos adecuados a los agentes. Ello ocurre cuando la demanda sufre fluctuaciones importantes en distintos momentos de tiempo. Una estrategia de único precio conducirá a sobre o sub demanda, según el caso. Si el precio que se fija es el promedio del que regiría en los períodos de alta y baja demanda, entonces será muy bajo (alto) cuando la demanda es alta (baja) y habrá sobre (sub) utilización del servicio.

Este tipo de regulación tiene efectos como los que vimos cuando se analizó el servicio de transporte, en el cual siempre habrá consumidores insatisfechos por el servicio, producto de la escasez o los altos precios. En el caso de la energía eléctrica este problema es particularmente importante, en la medida en que esta no puede almacenarse,
\footnote{Lo mismo pasa con el transporte, donde la capacidad excedente de los buses que van vacíos en un horario no se pueden ``almacenar'' para ser llenados en horarios pico.}
y en los momentos de pico de demanda hay que generar a toda capacidad. Asimismo, la demanda de energía fluctúa según la estación del año y también durante el día.
\footnote{En Uruguay, para el 19 de enero de 2016 se preveía una fluctuación de la demanda de energía con un mínimo de 990 kW a las 6 horas y un máximo de 1.580 a las 15 horas, es decir un incremento del 60\% de la demanda en el transcurso del día.}
La generación de energía eléctrica es muy costosa en términos de los activos fijos necesarios para producirla. Si la demanda es muy superior a la oferta, la escasez de energía eléctrica se transforma en cortes de energía (apagones).

Por tanto, la política regulatoria de precio incide sobre la oferta del sector. Si se establece un precio único, existirá sobre demanda en los períodos de demanda alta, es decir la política regulatoria tenderá a exacerbar el problema.

El instrumento óptimo en estos casos son los \textbf{precios pico-valle}, a través de los cuales los consumidores internalizan el efecto que tienen sobre la capacidad. Esta política implica fijar tarifas diferenciadas en el pico y en el valle que reflejen el costo de la capacidad, en cada momento de tiempo. Esta forma de regulación permite hacer un uso eficiente de la capacidad en todo momento.

En Uruguay, la empresa de energía eléctrica (UTE) tiene un sistema de tarifa pico-valle para los grandes consumidores. El horario valle es entre las 0 y las 7 horas, mientras que el horario pico va de las 18 a las 22 horas, el resto de las horas se ubican en el horario llano.

Por otra parte, algunas empresas que operan en sectores de monopolio natural son multiproducto, lo que implica que existe cierta complementariedad en la producción de estos bienes o servicios. Como se mencionara, esta complementariedad puede surgir por el uso de un activo o un insumo común a la producción de los distintos bienes. Ello determina que el proceso de fijación de precios sea más complejo, ya que es necesario imputar una proporción del uso del activo o del insumo a cada uno de los bienes, lo que es muy difícil debido a la mencionada complementariedad de las actividades.

En definitiva, cualquier mecanismo de imputación será arbitrario ya que el proceso productivo no permite separar el uso de los activos o insumos en la producción de cada uno de los bienes. La fijación de precios sobre la base de los costos determina un importante abanico de precios posibles, dependiendo de cómo se imputen estos costos.

El instrumento regulatorio óptimo en estos casos es la fijación mediante \textbf{precios de Ramsey}, los que se obtienen haciendo máximo el excedente total de la sociedad, sujeto a la restricción de que la empresa obtenga beneficios
económicos nulos.
\footnote{Los beneficios económicos son distintos de los beneficios contables. En particular, los costos de producción se valoran en términos del costo de oportunidad, es decir su valor en la mejor alternativa. Por tanto, si los beneficios económicos son cero, ello implica que el empresario está ganando lo mismo que obtendría en la mejor actividad alternativa. Beneficios económicos cero no implican que la empresa no obtenga ganancias.}
El resultado lleva a fijar precios de forma tal que los consumidores que tienen una mayor valoración por el producto ---su demanda es más inelástica--- paguen más por él. Por otra parte, en la medida en que un precio mayor al costo marginal implica una pérdida de eficiencia asignativa, esta se minimiza si se carga con un precio mayor a aquellos consumidores que más requieren del bien o tienen menos alternativas disponibles para sustituirlo.

A vía de ejemplo, en la determinación de los precios de los distintos derivados de los combustibles (gasolinas, GLP, asfalto, etc.) las demandas que tengan menos sustitutos deberían ser las que carguen con una mayor proporción de los costos fijos de la refinación. Este tipo de regla, si bien es eficiente, puede ser políticamente difícil de implementar dado que quienes tienen una demanda más inelástica no necesariamente son aquellos que tienen mayores recursos económicos.

Independientemente de qué tarifa o precio fije el regulador, la misma deberá cumplir varios objetivos económicos. Por un lado, no debe ser abusiva para los consumidores, al mismo tiempo que debe asegurar un retorno adecuado a la inversión realizada por la empresa. Por otro, la tarifa debe generar los incentivos necesarios para que la calidad de los productos sea conveniente y las inversiones permitan sustentar el servicio en el tiempo. Aún más, deben estar diseñadas para que las empresas tengan incentivos a hacer el mejor uso de los insumos y la tecnología disponible, evitando ineficiencias productivas.

Para lograr estos objetivos en la fijación de la tarifa, el regulador necesita determinar qué insumos forman parte de los costos de producción y evaluar qué inversiones deben ser consideradas en este precio. Ello representa una compleja tarea, que requiere información muy detallada.

El problema de información resulta clave en el diseño de la política regulatoria. En el análisis realizado hasta este punto, se supuso en forma explícita que el regulador conoce el mercado y a la empresa y, por tanto, puede elegir el precio que cumple los requisitos anteriores utilizando instrumentos como: precio único, tarifa en dos partes, tarifa pico-valle.

Sin embargo, en la realidad los reguladores están menos informados que las empresas sobre el funcionamiento específico del mercado y sobre las decisiones que estas toman en relación a sus costos o al esfuerzo que realizan para producir. La regulación en contextos de información asimétrica implica que la empresa debe revelar información a través de distintos mecanismos, los cuales determinan que el regulador entregue parte de la renta monopólica a la empresa regulada como forma de obtener la información necesaria para poder regular.
\footnote{Véase \citet{Baron1982}.}
La dinámica regulatoria entre el regulador y la empresa incorpora elementos más complejos al análisis, como la
posibilidad de renegociación de los contratos o, directamente, la expropiación de la empresa.
\footnote{Al no ser un elemento central del análisis no se desarrollarán estas problemáticas. El lector interesado puede ver la sección 2.5 de \citet{Armstrong2007}. Con relación a los compromisos creíbles y los mecanismos para evitar la expropiación, existen múltiples referencias, entre ellas \citet{Bergara2003} y \citet{Spiller2013}.}

En un marco de información asimétrica, distintos mecanismos para establecer precios se traduce en distintos incentivos para las empresas, y los instrumentos de regulación de precio, en casos de monopolios naturales, se reducen a dos:
\footnote{Esta sección se basa en \citet{Armstrong2007}, capítulo 3.}

\begin{enumerate}
\def\labelenumi{\arabic{enumi}.}
\item
  \textbf{Regulación de la tasa de retorno}: fija un precio que asegura a la empresa un retorno justo por las inversiones realizadas. Se determinan las inversiones y los gastos necesarios para el desarrollo de la actividad y se fija una tasa de retorno a esta inversión a los efectos de determinar el precio.
  \footnote{Este tipo de regulación es la más utilizada en Uruguay, donde se analiza la evolución de las paramétricas de costos para la determinación de la variación de los precios.}
  Este instrumento genera dos problemas de incentivos. Por un lado, al tener cubiertos los costos no genera incentivos a la eficiencia, esto es, no hay incentivos a controlar los gastos operativos. Por otro lado, si la tasa de retorno que garantiza el regulador es distinta a la tasa de retorno del inversor, se pueden dar procesos de subinversión ---si es menor--- o de sobre inversión en capital ---si es mayor---. El precio fijado tendrá un impacto sobre la elección de la tecnología de producción, ya que una tasa de interés menor a la esperada por el inversor, inducirá a utilizar mano de obra en vez de capital y a la vez inducirá una menor reposición de capital fijo.
  \footnote{Este efecto se conoce en economía como Efecto Averch y Johnson ya que fueron los primeros que estudiaron el impacto de este tipo de regulación sobre los incentivos. Ver \citet{Averch1962}.}
  Por otra parte, este instrumento es muy útil cuando existen problemas de credibilidad por parte de los reguladores. Al garantizar el retorno de la inversión, el inversor ve acotados los riesgos que debe enfrentar. Obviamente que este razonamiento presupone que el regulador no cambiará las reglas de juego en el camino, lo que no siempre se cumple.
\item
  \textbf{Precio techo}: establece un máximo al incremento de precios promedio que puede fijar la empresa a lo largo de un período. En particular, el incremento de precios para un período dado, por ejemplo cinco años, se basa en el valor de la evolución del Índice de Precios al Consumo (IPC) menos un factor de productividad específico a la tecnología de la industria. En la medida en que este mecanismo de fijación de precios está desatado de los costos, genera fuertes incentivos a la eficiencia, ya que cualquier eficiencia que surja será apropiada por la empresa. Asimismo, la empresa tiene discreción para fijar la canasta de precios, dado que el regulador sólo fija la evolución del precio promedio. Sin embargo, en contextos macroeconómicos inestables, por ejemplo si existen grandes
  devaluaciones o recesiones, este instrumento puede llevar al quiebre de las empresas.
\end{enumerate}

El cuadro \ref{tab:cuadro3}, basado en \citet{Armstrong2007}, resume las características de la regulación por tasa de retorno y por precio techo.

\begin{table}

\caption{\label{tab:cuadro3}Características de la regulación de Tasa de retorno y Precio techo.}
\centering
\begin{tabular}[t]{l|l|l}
\hline
Característica / Instrumento & Precio techo & Tasa de retorno\\
\hline
Flexibilidad de la empresa para fijar precios relativos & Sí & No\\
\hline
Plazo de ajuste regulatorio & Largo & Corto\\
\hline
Sensibilidad de precios a costos & Bajo & Alto\\
\hline
Discreción regulatoria & Sustancial & Limitada\\
\hline
Incentivos a reducción de costos & Fuerte & Limitada\\
\hline
Incentivos a la inversión en activos hundidos & Limitada & Fuerte\\
\hline
\end{tabular}
\end{table}

Fuente: tabla 27.1 de \citet{Armstrong2007}.

Un último instrumento regulatorio, en el caso de monopolios naturales, es la regulación por comparación (\emph{yardstick} \emph{competition}). Este prevé comparar el desempeño de empresas con características similares o que ofrecen el mismo servicio en distintas localizaciones, de forma de obtener una referencia que permita determinar el precio de la actividad desarrollada por el monopolista. Ello requiere que el entorno en el que operan las empresas sea realmente comparable, a los efectos de poder evaluar el desempeño y fijar el precio.

Por ejemplo, se podría comparar el precio de la energía eléctrica en Uruguay y Argentina, de forma de determinar cuál debería ser el precio óptimo del servicio. Sin embargo, es complejo encontrar dos empresas que operen de forma similar para poder fijar el precio, ya que en caso de no ser similares, las diferencias en el precio pueden obedecer a distintas demandas, diferentes fuentes de financiamiento, distintos regímenes salariales, etc.

Este tipo de instrumento lo utiliza la Unidad Reguladora de los Servicios de Energía y Agua (URSEA) de Uruguay para comparar el precio de venta de combustibles, que son monopolio de la empresa estatal ANCAP, con el precio de paridad de importación. Ello permite conocer el grado de eficiencia relativa de la empresa monopólica en relación a la mejor alternativa disponible, que es la importación de los combustibles refinados.
\footnote{La información con la metodología está disponible en la siguiente página \href{https://www.gub.uy/unidad-reguladora-servicios-energia-agua/sites/unidad-reguladora-servicios-energia-agua/files/2019-12/Metodologia_PPI_diciembre_2017_0\%281\%29.pdf}{web}.}

\hypertarget{regulaciuxf3n-de-acceso}{%
\subsection{Regulación de acceso}\label{regulaciuxf3n-de-acceso}}

En algunos mercados conviven segmentos monopólicos y ---potencialmente--- competitivos. A vía de ejemplo, en la generación de energía eléctrica existen múltiples oferentes, pero las líneas de transmisión desde los generadores a los consumidores tienen características de monopolio natural. En la telefonía, existen múltiples operadores de larga distancia internacional, mientras que la telefonía local es en general monopólica. Este tipo de plataformas se conoce con el nombre de facilidades esenciales.
\footnote{El capítulo \ref{def-comp} trata la forma en la que se regula este tipo de facilidades desde el punto de vista de la competencia.}

El problema que se presenta al regulador es cómo permitir el acceso a los segmentos monopólicos, cuando en otros segmentos compiten varias empresas, entre ellas la operadora del monopolio. Por ejemplo, sin acceso a la transmisión de energía eléctrica es imposible que los generadores oferten el servicio. Sin embargo, si el dueño de la plataforma compite además con otros generadores, no tendrá incentivos a darle acceso pues con ello afectará su posición competitiva.

Si el acceso no está regulado, y la empresa puede determinar quién accede a su plataforma, se pueden generar situaciones no deseadas desde el punto de vista social. En primer lugar, si la empresa no da acceso a la plataforma y los competidores pueden replicarla, entonces el competidor terminará construyendo su propia plataforma, lo que desde el punto de vista social es ineficiente ya que se duplican activos costosos.

En segundo lugar, si la plataforma no puede ser replicada, entonces la propietaria puede utilizar el acceso a su favor para incidir sobre la competencia en el mercado competitivo. Ello puede determinar que el mercado competitivo no se desarrolle como tal, ya sea porque la empresa niegue el acceso a la plataforma o lo permita pero a una tarifa discriminatoria que impida a otros competir con ella.

En la actualidad los servicios de telefonía, transmisión de datos y televisión para abonados podrían ---tecnológicamente--- utilizar una única plataforma: la fibra óptica. Sin embargo, en Uruguay, la empresa de telefonía y transmisión de datos tiene su plataforma, y las de televisión para abonados la suya. Asimismo, se ha impedido a otras empresas de transmisión de datos tanto utilizar la red instalada, así como construir su propia red. Ello determina una situación paradójica donde conviven varias redes ---que es ineficiente--- con la negativa a duplicar redes ---que es eficiente--- pero con una negativa de acceso ---que restringe la competencia en el mercado de transmisión de datos.

En los casos en que la empresa propietaria del insumo monopólico también compite en la provisión de servicios finales a los consumidores, la alternativa a la regulación de acceso es la separación de industrias por componentes.
\footnote{Un breve resumen de los costos y beneficios de la separación por componentes se presenta en \citet{Viscusi2005}, capítulo 15.}
La separación por componentes refleja otra vez la tensión entre los distintos tipos de eficiencia (productiva vs.~asignativa). La separación es deseable para impedir comportamientos anticompetitivos, fundamentalmente, cuando la empresa que tiene el insumo monopólico utiliza el acceso para excluir a sus competidores e incidir a su favor en la competencia en el mercado final (eficiencia asignativa).

Sin embargo, esta alternativa no es deseable si existe algún tipo de sinergia o complementariedad entre los segmentos que permita que la producción sea más eficiente (eficiencia productiva). Lo mismo ocurre cuando la integración de actividades permite reducir los costos de transacción o de coordinación entre ellas. En cualquier caso, separar la plataforma del resto de las actividades productivas es complejo.

Una alternativa consiste en fijar una tarifa de acceso a la plataforma. Sin embargo, esta tarifa debe balancear incentivos contrapuestos. Si la tarifa es muy alta, entonces puede ser utilizada por la empresa para generar una barrera en el mercado competitivo. Por otra parte, si la tarifa es muy baja entrarán muchas empresas al mercado competitivo, dado que sus costos serán muy bajos. Asimismo, estas empresas estarán parasitando la inversión realizada por un tercero, ya que el precio que pagarán inducirá a esperar que otro realice la inversión, para poder aprovecharla a un costo menor.

Por tanto, el problema de determinar los costos es clave para fijar adecuadamente la tarifa de acceso. Pero si los costos son comunes a varias actividades, es decir, si existen economías de variedad, entonces la empresa que opera en el mercado monopólico tendrá incentivos a manipular los costos de forma de pasar gran parte de ellos al segmento monopólico, reduciendo los costos en el segmento competitivo. Ello le permite mejorar su posición competitiva en el mercado, reduciendo su precio y, a la vez, empeorando la posición competitiva de sus competidores.

Por los elementos presentados, determinar el precio eficiente en estos mercados es complejo. Veamos el caso de mercados donde el acceso a la red es de un único lado (\emph{one-way network access}). Mercados de este tipo son las redes de telefonía local cuando se conectan a una línea internacional; o la transmisión de energía eléctrica que conecta a los distribuidores con consumidores; o la red de gas que conecta a los productores de gas.

Una primer regla para fijar tarifa de acceso en este tipo de mercados es establecerla como la diferencia entre el precio de mercado del bien competitivo (por ejemplo llamadas internacionales o electricidad al público) y el costo marginal de llevar a cabo estas llamadas.
\footnote{Romado de \citet{Joskow2007}, capítulo 10.1.}
Esta regla se conoce como la Regla de Precio de Componente Eficiente (RPCE). Si bien esta regla tiene algunas propiedades deseables, entre las que se encuentra permitir el acceso al mercado competitivo sólo a las empresas eficientes y no interferir con los subsidios cruzados que existan, tiene también algunos problemas.

El primero es que no dice nada respecto a cómo se fija el precio final en el mercado competitivo, lo que resulta fundamental para determinar la tarifa de acceso. El segundo problema es que la regla no funciona si el entrante en el mercado competitivo tiene poder de mercado, ya que habría que descontar ese margen del precio final para el cálculo de la tarifa. El tercero es que puede generar incentivos a que el entrante en el mercado competitivo se saltee la red, si el costo que se cobra es mayor al de instalar una red propia.

La complejidad de elementos que deben considerarse a la hora de fijar esta tarifa, ha llevado a algunos autores a plantear un precio techo global a la empresa monopólica que incluya la tarifa de acceso. Sin embargo, como esta estrategia da libertad a la empresa para fijar los precios individuales ---siempre que se cumpla con el precio techo global--- puede inducir a la empresa a fijar precios predatorios en el mercado competitivo como forma de inducir a las empresas a salir del mercado. Establecer el precio de acceso al segmento competitivo sigue siendo un tema complejo e intensivo en información.

\hypertarget{conclusiones}{%
\section{Conclusiones}\label{conclusiones}}

Regular es complejo. Hay que tomar en consideración las características del mercado sobre el que se busca intervenir y su estructura. Algunas veces, la regulación resulta una elección entre males con distinto peso relativo. Es difícil encontrar un mecanismo que resuelva una situación problemática sin que ello no genere un problema adicional sobre alguna otra variable. Al regular se altera los incentivos de las empresas. No tomar en cuenta estos efectos puede llevar a que la situación bajo regulación sea peor que si esta no existiera.

Otras veces, puede llevar a que ningún consumidor esté satisfecho con la situación a la que conduce la regulación, y ello puede implicar costos políticos para la autoridad que diseña la regulación. En mercados competitivos, toda regulación de precio que establezca un techo a esta variable disminuye los incentivos al ingreso de nuevas empresas que puedan disputar las rentas del mercado. Limitar el ingreso al mercado, ya sea por menores márgenes o una barrera legal, lleva a que las empresas instaladas pierdan incentivos a ser costo eficiente, a atender la calidad de sus productos, a realizar las inversiones necesarias, entre otros.

A veces este efecto es imposible de evitar, como en el caso de los monopolios naturales. Ello lleva a construir una serie de regulaciones adicionales a las de precio, que hace complejo ---aunque necesario--- el proceso regulatorio. Estos balances no buscan llevar a concluir que no es necesario regular, ya que en muchos mercados sí lo es. Por el contrario, buscan reflexionar sobre el alcance de la regulación, los instrumentos óptimos para alcanzar los fines propuestos y los costos que se están dispuestos a asumir cuando se regula. Los instrumentos dependerán de la característica de los mercados, de la capacidad del regulador, de la información que disponga y de la credibilidad que tenga el Estado para hacer cumplir las reglas que fija.

\hypertarget{def-comp}{%
\chapter[Defensa de la competencia ]{\texorpdfstring{Defensa de la competencia \footnote{Este capítulo fue escrito sobre la base de diversos cursos que el autor dictó en universidades (Facultad de Ciencias Sociales -- UdelaR, Universidad de la Habana, Universidad de Montevideo) y organismos (Ministerio de Economía y Finanzas -- Uruguay). En general está construido basándose en los textos de Buccirossi (2008), \citet{Motta2004} y \citet{Viscusi2005}. Sin embargo, la referencia concreta de cada cita se ha perdido con el tiempo, por lo que el lector debe considerar que en su mayoría este no es un texto original y que el material corresponde a los autores citados.}}{Defensa de la competencia }}\label{def-comp}}

Leandro Zipitría

Las economías modernas asignan un rol fundamental a los mercados en la asignación de recursos. En general, se considera que estos asignan los recursos en forma eficiente. Sin embargo, en otros casos el accionar de los agentes impide que los mercados alcancen por sí mismos la máxima eficiencia posible. Las bondades de la competencia, que permiten a los consumidores obtener productos a los mejores precios y de la mayor calidad posible, requiere de políticas públicas que la incentiven y la protejan.

Mientras que la regulación busca suplantar al mercado en la determinación de alguna de las variables relevantes, defensa de la competencia busca impedir que las empresas eviten o impidan el accionar del mercado. En otros términos, a través del instrumento de defensa de la competencia no se establecen las variables de mercado, las que serán determinadas en última instancia por la oferta y la demanda. Esta herramienta busca desalentar aquellas acciones que realizan las empresas que impidan el ajuste de mercado.

En competencia perfecta el mercado ajusta ---principalmente porque los bienes son homogéneos, no hay barreras a la entrada y la salida, y los agentes están atomizados--- y permite alcanzar el mejor resultado desde el punto de vista social. Sin embargo, en la realidad las empresas enfrentan barreras para entrar a los mercados, los bienes no son homogéneos, y las empresas no están atomizadas. Ello genera mercados oligopólicos en los cuales algunas de las empresas tienen la capacidad de manejar el mercado para favorecerlas.

Esas acciones son las que defensa de la competencia busca corregir. Desde el punto de vista metodológico, defensa de la competencia es un instrumento que permite analizar los mercados y determinar cuáles son los principales cuellos de botella que enfrenta la competencia. Por tanto, es una herramienta cuya utilidad trasciende los límites de las investigaciones puntuales y sirve para comprender cómo funcionan los mercados. Es un instrumento válido, aún si los países no cuentan con normas de defensa de la competencia.

Los orígenes de la defensa de la competencia se remontan al Siglo XVI en Inglaterra. Previamente, durante la Edad Media, la regla era el otorgamiento de monopolios o licencias para el desarrollo de la actividad económica por parte de los gobernantes. Esto les permitía a los empresarios obtener parte de las rentas monopólicas de la actividad que desarrollaban. En 1599 en Inglaterra, un Tribunal declara nula una licencia monopólica otorgada por la Corona Británica a un empresario, en el fallo del ``caso de los monopolios''.
\footnote{Véase el caso \href{https://en.wikipedia.org/wiki/Darcy_v_Allein}{Darcy vs.~Allein}.}
Posteriormente, en 1890 en Estados Unidos de América (EUA) se dicta el Acta \emph{Sherman}, la que supone una de las primeras normativas que buscan prevenir el accionar de las empresas monopólicas.
\footnote{En realidad la primer norma de defensa de la competencia fue la \emph{\href{https://www.competitionbureau.gc.ca/eic/site/cb-bc.nsf/eng/04427.html}{Act for Prevention and Supression of Combinations Formed in Retraint of Trade}}, de Canadá del año 1889.}
A partir de allí, las normas de defensa de la competencia se han expandido en el mundo y, en la actualidad, la mayoría de los países cuentan con este tipo de legislación.

Defensa de la competencia se aplica caso a caso, para conductas concretas. Mientras que la regulación fija una variable en un sector de actividad, como por ejemplo el precio del transporte colectivo o el de la energía eléctrica, en defensa de la competencia se analiza si una conducta concreta en un mercado determinado constituye o no una limitación a la competencia. Cualquier decisión que se tome será contingente a esta situación particular y, si bien será indicativa de otras similares, su extensión a otros sectores o empresas requiere de un nuevo análisis.

Antes de desarrollar los conceptos, corresponde realizar una advertencia. La temática de defensa de la competencia ha tenido un desarrollo explosivo en las últimas décadas. Están disponibles en la web cientos de casos tratados por decenas de organismos de defensa de la competencia de múltiples países. Cuando se comienza una investigación concreta, revisar los antecedentes internacionales siempre es valioso ya que permite construir una composición de lugar sobre una situación particular, en general desconocida. Sin embargo, estos antecedentes deben inscribirse en su justo contexto. Las legislaciones de diferentes países varían en el énfasis que ponen en la aplicación de las normas, las que dependen no sólo de lo que la norma establezca sobre prácticas prohibidas o permitidas, sino también del contexto político particular que puede pautar la forma en que la misma se aplica.

En la medida en que las normas de defensa de la competencia son ambiguas y generales, los procesos de aplicación concreta están sujetos a grandes variaciones entre países. Asimismo, cada caso concreto se puede alejar, a veces en forma importante, de lo que la teoría establece respecto a una conducta en particular (por ejemplo, en el caso de la fijación de precios de reventa en la Unión Europea). En otros casos, existe discrecionalidad respecto a los umbrales de dominancia (¿a partir de qué punto una empresa es dominante?). Aún más, las legislaciones difieren en la forma en la que se analiza una conducta, como por ejemplo en el caso de precios predatorios. Todos estos elementos hacen que los antecedentes internacionales sean útiles, siempre y cuando se analicen en contexto.

¿Cuál es el objetivo de la defensa de la competencia? Existen diferencias entre lo que postula la teoría y lo que cada legislación establece. En teoría, la defensa de la competencia busca proteger el funcionamiento del mercado, entendido como un instrumento para asignar recursos en la economía. Este decide, en teoría, cuantos bienes se producen y cuál debe ser su precio. En los mercados, el precio se transforma en la señal que los agentes ---demandantes y oferentes--- miran para comprar los bienes o asignar recursos productivos. Pero con defensa de la competencia ¿qué es lo que específicamente se protege? En particular, los mercados son un mecanismo eficiente de asignación de recursos, por tanto el objetivo de la defensa de la competencia es proteger la eficiencia. En particular, se busca sancionar conductas que reducen el bienestar de los agentes, donde el bienestar está medido en términos de excedente del consumidor y del productor. El análisis de estas variables económicas permite hacer una evaluación del impacto de cada conducta.

Sin embargo, algunas conductas tienen distintos impactos según si se miden en términos de excedente del consumidor o excedente total. Esta diferencia entre las medidas de bienestar ha provocado discrepancia en la aplicación de las normas de competencia, aunque parece existir una convergencia a nivel mundial a utilizar el excedente del consumidor
como regla general \citep[capítulo 2]{Motta2004}.

El impacto de la conducta de los agentes sobre la eficiencia se refleja en el mercado de forma bastante simple. Las prácticas que se analizan ---anticompetitivas--- son aquellas que tienen impacto sobre los precios o sobre la variedad de productos y tienen un impacto negativo sobre el bienestar económico. A vía de ejemplo, supongamos que se produce un incremento en el precio de un producto. Si este incremento obedece a un aumento en la demanda, entonces se verifica el ajuste natural del mercado. Sin embargo, si el incremento se produce debido al acuerdo entre los oferentes para mejorar sus márgenes ---un cartel---, entonces el mecanismo competitivo ha dejado de actuar y las empresas deben ser sancionadas. Por tanto, una buena regla para entender el funcionamiento del mercado es investigar las razones económicas que están detrás de los fenómenos observados.

Las normas de competencia se aplican de distintas formas. Por un lado, los órganos de defensa de la competencia pueden realizar estudios de mercados y sugerir criterios que atiendan a un funcionamiento más eficiente. Otras veces, estudian las regulaciones del gobierno y sugieren cambios que reduzcan barreras innecesarias a la entrada o restricciones al normal funcionamiento de los mercados. Por último, la aplicación a casos concretos, en general, se realiza a través de un procedimiento que puede ser judicial o administrativo, y que es similar a un juicio. En estos casos, se inicia un expediente y se presentan pruebas respecto a la conducta investigada, se analiza el mercado y con posterioridad se falla. Sobre este fallo surgen luego recursos administrativos o judiciales hasta que la sentencia final queda firme. Es decir, es un proceso largo y complejo.

A su vez, las normas de defensa de la competencia pueden ser interpretadas de dos formas. Por un lado, las normas pueden tener una interpretación estricta, conocida como per se, a través de las cuáles las conductas se sancionan en sí mismas por el sólo hecho de ser llevadas a cabo y con independencia de sus efectos. Por otro, las normas pueden estar sujetas a la regla de la razón, 4 por el cual se establece que las normas de defensa de la competencia no pueden ser interpretadas en forma literal, sino que debe analizarse si las conductas realizadas restringen el comercio, es decir se analiza sus efectos.

La metodología de defensa de la competencia busca comprender el funcionamiento de los mercados y determinar los cuellos de botella que permiten a las empresas obtener rentas extraordinarias. En principio, todas aquellas conductas que realizan las empresas y que les permite obtener renta sin trasladar beneficios a los consumidores son susceptibles de perjudicar el bienestar y pueden considerarse conductas anticompetitivas.

El primer paso en el análisis es tratar de determinar si las empresas pueden, a través de sus conductas, ejercer un poder de mercado. Este concepto económico es difícil de determinar en la práctica, y por ello se aproxima a través de mecanismos indirectos. El mercado relevante es un instrumento útil para comprender el funcionamiento del mercado, entender sus componentes en términos de productos y áreas geográficas y comprender la cadena vertical que permite que los consumidores accedan a los productos. En última instancia, comprender la competencia en el mercado.
Diversas son las conductas anticompetitivas que se investigan en defensa de la competencia. La más importante es el abuso de posición dominante que permite a las empresas comportamientos monopólicos. Por otra parte, las canastas o la discriminación de precios son conductas que en ciertos contextos facultan a las empresas a aumentar sus precios a los consumidores. La depredación, por su parte, busca directamente eliminar la competencia y aumentar los precios a los consumidores. Estas acciones consolidan una posición innecesaria para las empresas desde el punto de vista social.

Por otra parte, aun cuando las empresas no sean lo suficientemente grandes como para imponer sus condiciones al mercado, pueden lograr un cometido similar si logran restringir la competencia entre sí. Cuando dos empresas compe-
tidoras acuerdan no competir, están coludiendo en el mercado. Las licitaciones se acuerdan para perjudicar a gobiernos y empresas, los clientes se reparten para restringir sus alternativas, las mejoras se evitan ya que la competencia no es necesaria. Estos acuerdos destruyen la propia esencia del mercado y son de las conductas más perjudiciales.

En las economías modernas los procesos productivos que llevan a que los consumidores disfruten de ciertos bienes y servicios son muy complejos. En muchos de estos procesos intervienen un conjunto de empresas en distintos niveles de la cadena de producción y distribución. La interrelación entre estas empresas lleva a un complejo entramado contractual que puede afectar el precio o la calidad de los productos a que accede el consumidor. A veces, las empresas pueden tener dificultades para acceder a insumos para fabricar sus productos, o a los minoristas para exhibirlos. Estas restricciones verticales pueden afectar la competencia en los mercados, con el consiguiente perjuicio a los consumidores.

Las empresas pueden crecer ---vertical u horizontalmente--- introduciendo mejoras en sus productos, expandiendo la capacidad o expandiendo sus puntos de venta. Otras veces, para alcanzar los mismos fines se fusionan con empresas relacionadas. Sin embargo, ello produce distintos resultados que pueden afectar la competencia en el mercado. Una fusión horizontal puede mejorar y hacer más eficiente la capacidad de producción, pero también elimina un competidor y puede permitir cobrar precios mayores. Una fusión vertical puede permitir internalizar procesos y mejorar su control, pero también puede impedir que empresas rivales accedan a insumos estratégicos. Otras fusiones, sin embargo, se realizan entre empresas que no están en capacidad de afectar a los mercados.

En las secciones siguientes se desarrolla la temática de defensa de la competencia. En primer lugar, se analiza la importancia que tiene investigar el funcionamiento del mercado y la participación de los agentes. En las siguientes tres secciones se desmenuzan distintas conductas que realizan las empresas en los mercados: abuso de posición dominante, colusión e integración vertical. En estas secciones no se presentarán los modelos teóricos que explican los efectos anticompetitivos de las conductas, que se pueden encontrar desarrollados en \citet{Motta2004}. En cada caso, el énfasis estará en los aspectos prácticos de las conductas y sus efectos, pero inscriptos en su racionalidad económica. Por último, se analizan las fusiones y adquisiciones, conductas que afectan la estructura del mercado.

\hypertarget{mercado-relevante-y-poder-de-mercado}{%
\section{Mercado relevante y poder de mercado}\label{mercado-relevante-y-poder-de-mercado}}

El principal objetivo de defensa de la competencia es evitar conductas de empresas que les permita ejercer o incrementar su poder de mercado, que es la capacidad que tienen de fijar precios por encima del precio de competencia. Para entender este concepto hay que comparar la situación actual con una alternativa. El menor precio que se puede fijar en un mercado se observa cuando el mercado es perfectamente competitivo. Toda vez que el mercado se aleja de esta situación, los precios tienden a ser mayores, siendo el monopolio la estructura de mercado con el precio más alto posible. Es decir, cuanto menor es la competencia más fácil es para las empresas aumentar el precio. O a la inversa, la competencia permite que los precios de los productos se reduzcan.

Por tanto, la pregunta que sobrevuela el análisis en defensa de la competencia es: ¿permite la conducta de la empresa aumentar o mantener los precios de sus productos? Evaluar el impacto de una conducta sobre el mercado, en general, y los precios, en particular, puede ser complejo. Hay que establecer un vínculo causal entre un fenómeno y un determinado resultado en el mercado. Por un lado, puede ser que la conducta anticompetitiva de la empresa no se haya materializado y, por tanto, no hay evidencia concreta del impacto de la misma. Por otro, muchas veces confluyen múltiples factores en forma simultánea que pueden explicar un mismo fenómeno: hay un aumento de salarios, la demanda externa se resiente, los costos aumentan, entre otros.

Aislar estos fenómenos de aquel bajo estudio ---la supuesta conducta anticompetitiva--- es una tarea compleja e intensiva en información. Sin embargo, algunas veces sí puede identificarse con relativa certidumbre la relación entre dos fenómenos, como cuando aumentan los precios de un producto luego de que los productores se han puesto de acuerdo.

La evaluación directa del efecto de una conducta sobre el precio es más un hecho excepcional que lo normal en defensa de la competencia. La sofisticación técnica de estos análisis ---no sólo para su realización, sino también para la adecuada interpretación de sus resultados--- determina que el más utilizado, para determinar la ilicitud de una determinada conducta, sea establecer el mercado relevante. La determinación del mercado relevante es parte de una alternativa metodológica que intenta analizar los efectos de las conductas de los agentes en los mercados. Básicamente, consiste en construir un discurso que, a través del análisis del mercado (empresas, características del producto, barreras a la entrada) y el tipo de conducta, permita concluir sobre los efectos perjudiciales de esta última.

Implica, también, considerar explicaciones alternativas a la conducta y los efectos que ella podría tener sobre el mercado, así como los posibles beneficios de la conducta, ya que son pocas las conductas que son siempre perjudiciales. En resumen, el mecanismo alternativo para determinar el poder de mercado de la empresa involucra: (i) determinar el mercado relevante; (ii) determinar las cuotas de mercado de las empresas en el mercado
definido previamente; y (iii) analizar la importancia de las barreras a la entrada en ese mercado.

\hypertarget{mercado-relevante}{%
\subsection{Mercado relevante}\label{mercado-relevante}}

El mercado relevante es el mercado en el cual es vendido un producto particular. El origen de este concepto está en el modelo de empresa dominante, uno de los modelos específicos de la organización industrial.
\footnote{Véase \citet{Landes1981}}
Este modelo es un monopolio atenuado, donde existe una empresa dominante y un conjunto de empresas pequeñas que se comportan como precio aceptantes (franja competitiva). La empresa dominante fija el precio en el mercado, pero toma en cuenta a la franja competitiva en la medida en que sabe que si aumenta los precios inducirá a las empresas pequeñas a aumentar la oferta. El poder de mercado en este modelo ---es decir, la capacidad de aumentar los precios sobre los competitivos--- depende de varios factores. En primer lugar, cuanto más elástica sea la demanda ---mayor la respuesta de la cantidad demandada ante cambios en los precios--- menor será el poder de mercado de cualquier empresa.

En segundo lugar, cuanto mayor sea la cuota de mercado de la empresa dominante, más fácilmente podrá subir los precios. Por último, cuanto menor sea la respuesta de la franja competitiva, también mayor será la posibilidad de que la empresa dominante pueda aumentar sus precios. Este último resultado puede deberse a: (i) la franja competitiva está operando sobre el máximo de su capacidad instalada; (ii) existen barreras que impiden a nuevas empresas entrar al mercado; u (iii) operan restricciones a la importación del mismo producto de otras
regiones.

Por tanto, este modelo sienta las bases para el análisis del mercado relevante del bien o servicio bajo estudio.
Antes de formalizar la determinación del mercado relevante, debe reiterarse que es una herramienta que permite deducir el poder de mercado de la o las empresas investigadas. No es una finalidad de la investigación determinar el mercado relevante, sino que sirve para determinar la posición de mercado de las empresas.

El mercado relevante tiene dos componentes: (i) el mercado relevante de producto, y (ii) el mercado relevante geográfico. En ambos, la clave para su determinación es considerar los productos sustitutos, tanto desde el
punto de vista de la oferta como de la demanda.

La Unión Europea establece que el mercado relevante de producto ``se compone de todos aquellos bienes o servicios que los consumidores ven como intercambiables o sustituibles ya sea por las características del producto, su precio
o uso'' \citep{UE1997}. Esta definición está sesgada a considerar factores desde el lado de la demanda del producto, aunque en la realidad también se considera la posibilidad de sustituir oferentes (recordar el modelo de empresa dominante anteriormente señalado).

Hay productos donde la demanda es relativamente clara: muchos tratamientos médicos cuentan con un número limitado de alternativas (diálisis para los pacientes con insuficiencia renal crónica; insulina para los pacientes diabéticos); en la construcción se puede utilizar algunos insumos pero no otros (hormigón armado en vez de madera, o hierro para los pilotes de la estructura); las computadoras requieren un sistema operativo para poder funcionar. En otros casos la línea es más tenue, pero igualmente puede afinarse el análisis para determinar el mercado específico objeto de estudio. Un ejemplo clásico es el de los automóviles, donde existen múltiples calidades y precios. En estos casos, corresponde analizar un mercado más pequeño, como ser los de alta gama o los de bajo precio, dado que en general atienden a consumidores distintos y no se solapan entre sí. Sin embargo, en otros casos la sustituibilidad puede ser asimétrica: la cerveza premium en Uruguay puede tener como sustituta al vino de calidad preferente (VCP), pero el vino VCP puede ser sustituido con cerveza común o con otras bebidas alcohólicas.

No hay regla en este análisis, dado que dependerá de las preferencias ---gustos--- de los consumidores, las que son relativamente estables pero pueden variar en el mediano plazo. Además, varían también entre países dependiendo de las costumbres y las culturas.

Comenzando el análisis de la sustituibilidad del producto por el lado de la demanda, corresponde luego determinar la misma por el lado de la oferta. En particular, ¿las empresas en el mercado tienen capacidad ociosa?; ¿pueden entrar otras empresas al mercado, ya sea porque se pueden dedican a producir otros bienes o porque son entrantes nuevos, con relativa facilidad? La interesante es establecer cómo opera este mecanismo para determinar el mercado relevante de producto.

Supongamos que el mercado que se analiza es el del papel para producir hojas para utilizar en máquinas impresoras. Desde el punto de vista de la demanda es un mercado relativamente determinado: para imprimir se requiere determinado tipo de hojas, con una dimensión y espesor dado para que cumpla su función. Sin embargo, desde el punto de vista de la oferta, las empresas pueden sustituir fácilmente papel de distintos tamaños y gramajes a la hora de producir, dado que la tecnología de producción lo permite. Supongamos, también, que los productores de cartón pueden pasar en forma rápida a producir papel para impresión, si a alguna empresa se le ocurre subir el precio. Entonces: (i) por el lado de la demanda la sustituibilidad está acotada; (ii) por el lado de la oferta, se puede incorporar al mercado de producto los demás papeles de la industria (agrandar el mercado); y (iii) también se puede incorporar a los productores de cartón, dado que ellos pueden pasar a producir el bien de forma rápida. En este ejemplo, el mercado de producto estará constituido por los productores de papel y cartón, es decir, el mercado relevante será el más grande de los mercados definidos por la oferta y la demanda.

Una vez determinado un mercado relevante de producto, corresponde restringir el área geográfica donde operan las empresas. La clave en este caso es la posibilidad de sustituir o no oferta de otros lugares geográficos, ya sea por parte de oferentes o demandantes. Hay productos que son locales, como por ejemplo el corte de cabello; otros son globales, como los aviones. La diferencia entre ambos es que en el primer caso, los costos de transporte son muy altos con relación al producto (no tiene mucho sentido viajar sólo para cortarse el cabello), mientras que en el segundo es a la inversa (transportar un avión es relativamente barato en comparación a construirlo).

La \citet{UE1997} define el mercado relevante geográfico como ``el área en el cual las empresas están involucradas en la oferta y demanda de productos o servicios, y en los cuales las condiciones de competencia son lo suficientemente homogéneas que pueden distinguirse de áreas vecinas''. En estos casos, el rol de las importaciones, los aranceles y los costos de transporte son clave para determinar la extensión del mercado geográfico. Asimismo, también dependerá de las características del producto: si el producto es el servicio de diálisis no parece correcto tomar un espacio geográfico mayor a una ciudad (y en algunos casos esto sería excesivo); si el producto es hormigón armado, este tiene un tiempo de desplazamiento antes de comenzar a fraguar; si el producto es cerveza, si la distancia que debe trasladarse es muy importante, entonces el producto se encarece fuertemente respecto a su costo de producción y deja de ser competitivo.

Es difícil establecer una regla general para determinar el mercado geográfico, ya que dependerá de las características del producto y del país.

En conclusión, definir el mercado relevante no es una tarea sencilla. Implica conocer cómo actúan los consumidores, las características intrínsecas del producto, su proceso productivo, los costos de almacenamiento y transporte así como los aranceles, la forma en la que se distribuye y llega al público, y un largo etcétera. Sin embargo, estos elementos pueden ser construidos a través de evidencia cualitativa, como entrevistas a informantes calificados o encuestas a consumidores.

También se puede analizar evidencia cuantitativa si se dispone de alguna información básica. Para recolectar evidencia respecto al mercado relevante, una primera aproximación es la que puedan aportar informantes calificados. Todos los mercados cuentan con agentes que tienen información sobre su funcionamiento, ya sean públicos o privados, que pueden ayudar a entender cómo opera. El único problema en estos casos es determinar si estas declaraciones son producto de un agente imparcial o si este tiene intereses en el asunto en cuestión. Estos informantes permiten también clarificar donde se encuentra el nudo del problema, o qué información puede estar disponible para complementar los estudios.

Como alternativa a esta aproximación, se pueden realizar encuestas a los consumidores o experimentos. Ambas estrategias deben estar bien diseñadas para que sus conclusiones sean válidas y, en general, son costosas. Los experimentos son mejores a las encuestas, ya que permiten estudiar el comportamiento de los consumidores en situaciones reales, mientras que las últimas refieren a situaciones hipotéticas.
\footnote{Un experimento interesante es el de \citet{Conlon2013}.}

Es importante tener una guía respecto a qué información se quiere obtener,pero el procedimiento para alcanzarla debe ser flexible, de forma de poder lidiar con los múltiples obstáculos que surgen en una investigación de defensa de la competencia. No toda la información necesaria estará disponible, lo importante es analizar la consistencia de la información útil para la investigación con la que sí está disponible. En muchos casos, esa consistencia estará dada por los distintos modelos económicos que explican el comportamiento de los agentes en el mercado. Estos modelos tienen una lógica que permite predecir cómo se comportarán los agentes, y a ellos se recurre para interpretar la realidad.

El análisis cualitativo puede completarse con un análisis cuantitativo. Existen algunos métodos sencillos que permiten aportar información sobre el mercado relevante con relativamente poca información.
\footnote{Véase \citet{Haldrup2003} como referencia general a estos modelos.}
Por ejemplo, si dos bienes están en el mismo mercado relevante, los precios de los bienes deberían evolucionar de forma similar. Supongamos que la cerveza y el vino forman parte del mismo mercado, es decir son sustitutos cercanos.
\footnote{Basado en \citet{Zipitria2010}.}
Entonces el precio de la cerveza y el vino deberían tener comportamientos similares: si los productores de cerveza suben el precio (o a la inversa), los consumidores se pasan al vino y el precio del vino sube.

Las series de precio estarán correlacionadas y el coeficiente de correlación mide el grado de relación lineal entre las dos variables. Este estadístico varía entre --1 y 1. Si toma el valor 0, entonces las series son independientes; si toma el valor 1 las series son perfectamente idénticas. El signo positivo o negativo refiere al tipo de relación: si es positivo, las dos series varían en la misma dirección, si es negativo varían en dirección contraria. Si dos bienes pertenecen al mismo mercado, se debería esperar una fuerte correlación positiva. Por el contrario, si la correlación es negativa, los bienes son complementarios antes que sustitutos: si aumenta el precio de un bien cae su demanda y, por tanto, la demanda del bien complementario, lo que reduce el precio de este bien, o sea, los precios tienen trayectorias opuestas.

Esta correlación negativa debe ser fuerte porque en las series de precio agregadas existe una fuerte inercia entre los precios de los productos, como resultado de la inflación o de la propia tendencia general de precios de la economía. Este test, en función de los coeficientes de correlación, se utilizan fundamentalmente para descartar que los bienes pertenezcan al mismo mercado, en el caso de que la correlación sea negativa o baja (menor a 0,8). En cambio, si la correlación es alta, no se puede sacar ninguna conclusión ya que esta relación puede obedecer a otros factores, como insumos comunes o a la tendencia general de precios de la economía. Es decir, en caso de que el test de correlación sea positivo la relación puede resultar espúrea.

Un test más sofisticado es el de cointegración. Si las series son no estacionarias, lo que en estadística implica que tienen una tendencia aleatoria, entonces se puede investigar si dos o más series de precio tienen un sendero aleatorio común.
\footnote{Véase \citet{Haldrup2003}.}
A diferencia del test anterior, este sí permite concluir si dos productos pertenecen al mismo mercado, a lo que se llega si se puede determinar que los precios de los productos tienen un sendero aleatorio común. Estos test, el de correlación y el de cointegración, sirven para aportar información o reforzar la construcción realizada con el análisis cualitativo. Buscan sumar evidencia respecto a si dos o más productos pertenecen o no a un determinado mercado. Por tanto, deben leerse en su conjunto y en forma coherente.

El análisis reseñado puede utilizarse para determinar mercados tanto en el análisis de conductas de los agentes, donde está implícita que alguna de las empresas tiene posición dominante, como en el análisis de fusiones, donde el principal peligro es que, como resultado, las empresas fusionadas puedan aumentar los precios. En el caso de fusiones, también se utiliza el test SSNIP (\emph{Small but Significative Non-Transitory Increase in Price}, por sus siglas en inglés) o test del monopolista hipotético, uno de los más usados en defensa de la competencia.
\footnote{Este surge con los \href{https://www.justice.gov/archives/atr/1982-merger-guidelines}{lineamientos de fusiones} del Departamento de Justicia y la Federal Trade Commission de EUA, en 1982}
En primer término, se presentará el test y luego se discutirán las limitaciones que el mismo presenta para su utilización en casos que no involucren fusiones.

El test es bastante sencillo. Se basa en la siguiente pregunta: ¿puede una empresa, hipotética monopolista, subir los precios de sus productos de forma permanente y de modo sustancial (por ejemplo 5-10\% real)? Si la respuesta es afirmativa, entonces el mercado relevante está definido por esa empresa. Si la respuesta es negativa, entonces existe alguna otra empresa o producto que impide este incremento y, en consecuencia, se los debe incorporar al análisis para determinar el mercado.

Este es un proceso iterativo en el que se siguen incorporando bienes hasta que la respuesta sea afirmativa. Este proceso no está exento de problemas ya que muchas veces el orden en el que se incorporan productos cambia el resultado y, con ello, el mercado relevante. El objetivo del test SSNIP es establecer el menor grupo de productos y área geográfica en el cual un hipotético monopolista que lo controle, pueda mantener precios superiores a los competitivos.

La literatura es unánime respecto a que el test SSNIP es el correcto para determinar mercados. Sin embargo, es muy intensivo en información, ya que se requiere información de precios de muchos mercados para poder construir un modelo que pueda determinar este efecto. En muchos casos se utiliza una pregunta a los informantes calificados, ``¿puede la empresa X subir sus precios un 5-10\% en forma no transitoria si ocurre tal conducta?'' Es un test que sirve fundamentalmente para fusiones, y su uso en casos de abuso de posición dominante no permite obtener resultados confiables por dos razones.

En primer lugar, muchas veces las empresas realizan acciones no para aumentar su poder de mercado sino para mantenerlo. Por ejemplo, si una empresa dominante impide que otra entre al mercado, el statu quo se mantendrá y, por lo tanto, el precio del producto no se altera. El problema en este caso, es que con una mayor competencia el precio bajaría, lo que no ocurre porque el potencial entrante no ingresa al mercado. En segundo lugar, y vinculado a lo anterior, cuando las empresas tienen posición dominante no tienen incentivos a seguir subiendo el precio, dado que ya lo han establecido en el mayor valor posible. Por tanto, preguntarse si el precio seguiría subiendo no tiene sentido porque ya llegó al límite.

Además, hay otro elemento interesante que proviene del análisis de la demanda que ayuda a entender este fenómeno. Cuando el precio de un producto es bajo, en general, los sustitutos son pocos. Sin embargo, cuando el precio comienza a aumentar, también surgen productos sustitutos que antes no existían. Por ejemplo, cuando un mensaje de texto por celular cuesta U\$S 0,2, los sustitutos de este mensaje serán pocos, quizá un correo electrónico o una llamada breve. Pero si mandar un mensaje de texto costara U\$S 1.500, se podría viajar casi a cualquier parte del mundo y entregar el mensaje personalmente.

Por tanto, la primer defensa de cualquier empresa con posición dominante es aducir que existen múltiples sustitutos a sus productos, aunque esto sea consecuencia de tener tal posición.

A los efectos del test SSNIP este problema implica que, si no se está alerta, se incorporarán productos al
mercado cuando ello no corresponde, porque la empresa ya tiene posición dominante y, por tanto, siempre pasa el test, ya que nunca querrá aumentar los precios más de lo que ya lo ha hecho. Esta situación se conoce como ``Falacia
del Celofán'',
\footnote{El caso es \href{https://supreme.justia.com/cases/federal/us/351/377/}{United States vs.~E.L du Pont de Nemours \& Co.}, resuelto por la Corte Suprema de EUA en 1956.}
debido a un caso de defensa de la competencia en EUA donde se cometió el error de agrandar el mercado relevante a una empresa con posición dominante en el mercado del celofán, en el entendido de que existían múltiples sustitutos para sus productos.

Tan importante es el test SSNIP que en EUA algunos autores abogan por abandonar directamente la definición del mercado relevante. Kaplow (2010) pone sobre el tapete algunas de las limitaciones que tiene utilizar este test para definir mercados relevantes. Sin embargo, Werden (2012) señala que el mercado relevante es una herramienta que permite consolidar una narrativa consistente, así como distinguir los elementos centrales de los secundarios.

Después de esta discusión, quizá el lector esté desorientado respecto a cómo proceder para determinar un mercado relevante. Por ello, a continuación se presentan algunas pistas:

\begin{itemize}
\item
  En primer lugar, hay que mantener la perspectiva en el producto que adquiere el consumidor. Muchas veces los problemas de competencia acontecen en algún lugar de la cadena de valor, por ejemplo a nivel minorista. Pero el producto no cambia por ello. Es decir, no se puede confundir el mercado donde ocurre la conducta, con la conducta en sí misma.
\item
  En segundo lugar, y aunque parezca contradictorio, la conducta aporta elementos hacia donde enfocar el problema. Por ejemplo, si el problema está en la distribución hay que analizar cómo afecta ese problema puntual el acceso del consumidor al producto. No tiene sentido hacer el análisis en el eslabón de la producción, dado que allí no está actuando la restricción competitiva. O si la restricción se observa en un determinado mercado geográfico, no tiene sentido estudiar lo que pasa en los mercados de otras regiones, salvo que sean sustitutos geográficos y, por tanto, integran el mismo mercado geográfico relevante.
\item
  En tercer lugar, si bien el objetivo es encontrar el mínimo mercado donde ocurre una determinada conducta, muchas veces el mercado puede agrandarse si las condiciones competitivas son similares. Por ejemplo, en un caso de distribución de helados en Alemania se presentaban restricciones al acceso de unos productores a las heladeras de las tiendas. Las tiendas eran relativamente pequeñas, por lo que el mercado geográfico relevante debería alcanzar a un pequeño círculo con centro en la tienda, quizá no mayor a medio kilómetro. Sin embargo, como la conducta se manifestaba igual en todo el territorio y este tipo de tienda era la regla en todo el país, el mercado se definió como todo el país. Ello no cambiaba la dominancia relativa de la empresa y hace más sencillo el análisis.
\end{itemize}

\hypertarget{evaluaciuxf3n-del-poder-de-mercado}{%
\subsection{Evaluación del poder de mercado}\label{evaluaciuxf3n-del-poder-de-mercado}}

Una vez definido el mercado relevante ---de producto y geográfico--- hay que determinar la posición de mercado de las empresas que en él operan. Recuérdese que la definición del mercado relevante es un instrumento para deducir la capacidad que tienen las empresas de fijar precios en el mercado. En esta etapa hay que describir el mercado y contextualizar la competencia. Para ello, en primer lugar, se debe medir la concentración en el mercado relevante, para lo que se utilizan dos indicadores: (i) el Indicador de Concentración y (ii) el Indicador de Herfindahl-Hirschman (HHI).

\textbf{El Indicador de Concentración}, calcula cuanto representa en el total del mercado la participación de las empresas más importantes. Un típico indicador de concentración es el \(C_4\) que mide la cuota de mercado de las cuatro empresas más grandes de ese mercado. Sin embargo, este indicador no dice mucho si hay pocas empresas, lo que ocurre a menudo en economías pequeñas.

El indicador HHI es la suma del cuadrado de las cuotas de mercado de cada empresa: \(HHI = \sum s_i^2\), donde \(\sum\) indica la suma de todas las cuotas, y \(s_i^2\) es la cuota de mercado de la empresa i al cuadrado. El HHI varía entre 0, cuando la competencia es perfecta y hay infinidad de empresas en el mercado, y 10.000 cuando hay una única empresa ---monopolio--- con el 100\% del mismo. Este indicador tiene algunas propiedades deseables: (i) penaliza la concentración, en la medida que si una empresa tiene una cuota muy importante, entonces el indicador ---al estar elevado al cuadrado--- presentará valores mayores; (ii) permite comparar entre mercado, si en un mercado el HHI es 3.000 y en otro es 5.000, podemos decir que el segundo mercado está más concentrado que el primero.

Tanto el índice de concentración como el HHI se pueden construir para distintas variables, que pueden arrojar distinta información relevante. Tres variables son las utilizadas para comparar la concentración del mercado: (i) la capacidad de producción, (ii) el valor de la producción, o (iii) la cantidad producida. Analizar la capacidad de producción permite conocer si hay empresas que tienen capacidad ociosa que pueden usar para satisfacer la demanda.

Si el HHI de la capacidad de producción es mayor al HHI del valor de la producción, entonces las empresas más importantes del mercado tienen capacidad ociosa, o sea tienen una mayor capacidad de la que utilizan para producir. Esto puede ser un indicador de capacidad de crecimiento de estas empresas en el mercado, pero también puede implicar la capacidad de represalia de estas empresas o la posibilidad de inundar al mercado y, en consecuencia, reducir los precios en forma importante.

Por otra parte, si se compara el HHI en valores respecto al de cantidades, se tiene una fotografía de cómo juegan los precios. Si el indicador en valor es mayor al de cantidades, las empresas más grandes están cobrando mayores precios por sus productos, lo que es otro indicador de poder de mercado. El indicador de cantidades producidas debe construirse para productos homogéneos, por ejemplo si se calcula el volumen de agua vendido en el mercado o las toneladas de papel. Si los productos no son homogéneos, entonces las cantidades no se podrán agregar, a menos que se utilice el precio y se calculen los valores.

Una vez determinado el mercado relevante y la concentración del mismo, hay que evaluar las posibles barreras a la entrada. Dos mercados idénticos, con igual concentración, pueden determinar niveles de competencia diferentes si
las barreras a la entrada difieren. Las barreras a la entrada son uno de los conceptos más complejos de definir en economía, ya que existen diferentes definiciones \citep{McAfee2004}. La definición más aceptada es la que establece que una barrera a la entrada es un costo que tiene que pagar la empresa que entra al mercado, pero que no paga la que ya está instalada.

Sin embargo, existen múltiples costos a incurrir para entrar en el mercado, ¿cuáles son los relevantes? Los costos hundidos son los más importantes a la hora de entrar a un mercado. Estos costos son aquellos que, una vez incurridos, no pueden ser recuperados. Ejemplos de estos costos son la capacitación del personal, la publicidad o la instalación de activos específicos ---maquinaria--- para realizar una determinada actividad.

Cuanto más específicos sean los activos, menos líquidos serán en caso de que el emprendimiento fracase, lo que incrementa los riesgos de entrar al mercado. Por ejemplo, en el mercado de las bebidas gaseosas, el ingreso es relativamente fácil para una empresa que ya produce agua embotellada. Sin embargo, para competir en ese mercado se requiere construir una marca, que a su vez requiere cuantiosas inversiones en publicidad. Por tanto, ingresar a ese mercado puede ser sencillo, pero no va a permitir constituir una alternativa importante a las empresas dominantes del mercado.

Cuanto mayores sean las barreras a la entrada, más difícil será que nuevas empresas limiten el poder de mercado de la o las empresas instaladas. Por tanto, las barreras a la entrada pueden ser un síntoma del poder de mercado.
Cuando se analizan las barreras a la entrada no hay que perder de vista que sólo se deben considerar aquellos que pagan los entrantes, pero no los instalados, quienes, muchas veces, ya los pagaron.

Este es el tema central en este análisis. En el ejemplo anterior de las bebidas gaseosas, las empresas que ya construyeron una marca pagaron los costos que ello implica. Cuando una empresa instalada invierte en un sistema de distribución específico para sus productos también está invirtiendo en un activo específico. Cuando una empresa instalada realiza cursos de capacitación a su personal también está incurriendo en costos específicos. Muchas veces, estas inversiones permiten mejorar la calidad o precio de los productos, pero otras veces buscan dificultar el acceso al mercado de potenciales rivales, que deben ponerse a la altura de las instaladas para poder competir.

Atacar estos comportamientos no sólo es complejo, sino que puede ser contraproducente. Tomemos el caso de la publicidad. Las empresas la realizan para generar demanda para sus productos. Los gobiernos tienden a regularla para que no se engañe a los consumidores. Pero determinar cuál es el volumen de publicidad adecuado es complejo. Un
gobierno podrá decir si debe haber o no publicidad, pero poner un límite respecto a cuánta realizar es impensable.

\hypertarget{aplicaciuxf3n}{%
\subsection{Aplicación}\label{aplicaciuxf3n}}

El objetivo de la defensa de la competencia es impedir que las empresas ejerzan su poder de mercado. A veces, medir el poder de mercado es sencillo: si una empresa saca a otra del mercado y luego sube los precios, producto de una menor competencia, el poder de mercado se observa en forma sencilla. Sin embargo, muchas veces es difícil medir directamente el impacto de una determinada conducta sobre la capacidad de las empresas de fijar el precio. Si la evaluación directa no es posible, hay que recurrir a métodos indirectos.

El mercado relevante es un instrumento que permite inducir, a partir de evidencia sistemática, el poder de mercado de una empresa. Como todo instrumento indirecto es falible. Además, es difícil de contrastar con la realidad, dado que si la evidencia del poder de mercado fuera fácilmente obtenible, entonces la delimitación del mercado relevante sería innecesaria. Determinar el mercado relevante es un ejercicio que implica analizar en profundidad el mercado donde actúan las empresas investigadas y comprender su funcionamiento.

Una vez delimitados los mercados de producto y geográfico, se analiza la concentración en el mercado y las barreras a la entrada. Con estos elementos se puede tener una composición de lugar de la importancia relativa de cada empresa en el mercado.A continuación se presenta el detalle de la determinación de un mercado relevante en el mercado del pan lactal en Argentina.
\footnote{Fusión entre Bimbo y Fargo analizada por la Comisión Nacional de Defensa de la Competencia. Dictamen 395 de 10 de setiembre de 2004. Disponible \href{http://cndc.produccion.gob.ar/sites/default/files/cndcfiles/395_0.pdf}{aquí}}

Esta presentación no pretende fijar los estándares mínimos respecto a cómo debe delimitarse un mercado, ya que este proceso es bastante artesanal, ni implica estar de acuerdo con el procedimiento seguido en este caso o con sus conclusiones. Se presenta solamente a efectos ilustrativos, teniendo en cuenta que el dictamen presentado plantea la información que resulta coherente con una determinada decisión y que puede haber otra información relevante no incorporada que pudiera llevar a un dictamen opuesto.

Este caso surge a raíz de que dos empresas de pan lactal anuncian, en 2004, su fusión en Argentina: Bimbo y Fargo. Como en muchos otros países del mundo, cuando dos empresas importantes del mercado quieren fusionarse tienen que solicitar autorización a los organismos de defensa de la competencia. El mercado de pan en Argentina tiene dos segmentos: el mercado de pan artesanal, con un consumo de 70 kilos por persona al año, y el pan industrial, con un consumo de 3,8 kilos por persona al año. El pan industrial incluye el pan de molde y la bollería (pan para panchos ---salchicha o hotdog--- o hamburguesas). Mientras que el pan artesanal se vende directamente al público por las empresas que las producen ---en general pequeñas---, el pan industrial se distribuye a través de minoristas. En esta categoría, el 49\% de la distribución corresponde a hipermercados o supermercados, el 28\% a almacenes y el 22\% a autoservicios.

La pregunta clave es si el pan industrial y el artesanal forman parte del mismo mercado relevante de producto.
Si se analiza la demanda, existen diferencias en las características del producto que corresponde señalar: el pan industrial tiene conservantes; su duración es de 12 días, frente a los 4 días del pan artesanal; el pan industrial se vende envasado, el pan artesanal se vende al peso; el pan industrial se diferencia por marca; sólo un 2\% de los consumidores compran pan artesanal en forma diaria, mientras que el 75\% de los consumidores compra pan artesanal todos los días. Por otra parte, el pan artesanal se consume principalmente en las comidas (almuerzo y cena, 67\%), mientras que el pan industrial se consume en las colaciones (desayuno y merienda, 62\%). Por tanto, por sus características y la costumbre, ambos panes tienen distinto uso.

Otros elementos que indicarían diferencias entre estos productos son: (i) el precio del pan industrial es el doble del artesanal; (ii) la correlación de precios entre ambos productos es baja; (iii) el consumidor de pan industrial tiene un nivel socio económico mediano a alto; y (iv) la declaración de varias empresas de que el pan industrial no compite con el pan artesanal. La Comisión concluye que ``dadas las significativas diferencias entre precios, características, durabilidad, presentación, momentos de consumo y percepción de los actores del mercado del pan industrial y artesanal, no se puede afirmar que son sustitutos por el lado de la demanda''.

En principio, el pan artesanal y el pan industrial están en dos mercados de producto diferentes. Sin embargo, ¿puede el mercado ser más acotado? El pan industrial tiene tres componentes: el pan blanco (38\%), el pan negro (33\%) y la bollería (29\%). El pan negro contiene ingredientes propios (salvado, harina integral, cereales y centeno) y se utiliza para hacer dieta. La bollería se utiliza para combinar con salchichas o hamburguesas. Asimismo, la correlación de precios entre los tres productos es no significativa (cero) o tiene
signo negativo. Por tanto, desde el punto de vista de la demanda habría tres mercados de producto diferentes: pan blanco, pan negro y bollería.

Por el lado de la oferta, el expediente incluye una descripción detallada del proceso productivo que ``comienza con la preparación de una esponja líquida que consiste en una mezcla de harina, levadura, agua, alimento para levadura, oxidantes y enzimas. Esta esponja luego es fermentada durante 2 horas y media o 3 y enfriada. Como paso posterior se le agrega harina fresca, emulsionantes, agua y conservantes en una mezcladora. Luego la masa es dividida y depositada en moldes que son sometidos a vapor para una nueva fermentación durante aproximadamente una hora. La masa ya fermentada en los moldes es horneada, desmoldada y enfriada para luego ser embolsada''.

En cuanto al proceso de producción, Bimbo produce cada producto (pan blanco, negro y bollería) en distintas líneas de producción en una única planta. Las líneas de producción no comparten equipo entre ellas. Fargo, por su parte, tiene cinco plantas de producción: en una produce pan y bollería; en otra sólo pan y en una tercera sólo bollería; en otras dos plantas tiene líneas de producción dual (pan y bollería).

Desde el punto de vista de la oferta, las diferencias entre los panes (negro y blanco) son únicamente los ingredientes; mientras que entre el pan y la bollería están el armado de los bollos, los moldes y el embolsado. Concluyen que, mientras desde el punto de vista de la demanda hay tres mercados de producto (pan blanco, pan negro y bollería), desde el punto de vista de la oferta hay sólo dos: pan industrial y bollería. Por tanto, hay dos mercados de producto.

La definición del mercado geográfico, dados los mercados de producto, es más sencilla. Las empresas operan en todo el país, aunque sus principales plantas de producción están en la Provincia de Buenos Aires. Por tanto, las empresas enfrentan los mismos costos de transporte para distribuir sus productos al resto del país. Asimismo, el 63\% de las ventas se da en la capital y el gran Buenos Aires. Concluyen que, dadas las similitudes, el mercado geográfico es Argentina.

\hypertarget{abuso-de-posiciuxf3n-dominante}{%
\section{Abuso de posición dominante}\label{abuso-de-posiciuxf3n-dominante}}

La primera conducta que se analiza, es el abuso de posición dominante. Esta terminología es la que se utiliza en la Unión Europea, mientras que en EUA se la conoce como monopolización. A pesar de la diferencia en los nombres, la jurisprudencia ha ido convergiendo \citep[sección 1.2]{Motta2004}. Las prácticas de abuso de posición dominante se pueden dividir en dos.

Las prácticas \textbf{explotativas} son aquellas en las que una empresa cobra precios excesivos a los consumidores. Sin embargo, surgen múltiples problemas con este tipo de práctica. En primer lugar, existe un problema de medida: ¿excesivos respecto a qué? Las comparaciones nunca están exentas de problemas. En segundo lugar, si efectivamente se concluye que los precios son excesivos, este resultado puede ser motivo de la elección de los propios consumidores. Por ejemplo, Microsoft vende un sistema operativo (Windows) mientras que existen otros sistemas gratuitos (Linux). Sin embargo, la gente prefiere pagar por un producto que obtener el otro sin costo. Si bien existen muchas explicaciones para este resultado, hay una que resulta sencilla: Microsoft comercializa un producto de calidad que para muchos clientes es mejor que el producto de acceso gratuito. Por lo tanto, los precios elevados son simplemente el pago a la calidad del producto Microsoft.

En tercer lugar, el precio es la señal de mercado que miran las empresas para entrar en ellos. Si nada más pasa, entonces precios altos son una señal para que otras empresas entren al mercado y disputen las rentas de la empresa instalada. Por ello, las prácticas explotativas tienden a no ser sancionadas, ya que afectan las señales de mercado y, lo más importante, transforman las agencias de competencia en agencias reguladoras, lo que no es su función. Las agencias de competencia establecen que conductas son lícitas o no, pero no fijan las variables del mercado.

Por su parte, las prácticas \textbf{exclusorias} buscan impedir la entrada o forzar la salida de rivales y, en última instancia, tienen un efecto positivo sobre el precio de las empresas.

Las acciones de defensa de la competencia no están dirigidas hacia el precio de los bienes o servicios, sino contra las acciones que tienen un impacto sobre el mismo. Una pregunta clave en el análisis de esta conducta: ¿cuándo es dominante una empresa dominante? ¿Cuál es el umbral que debe pasar una empresa para que sus comportamientos se miren con lupa? Las legislaciones difieren en los umbrales, aunque existen ciertas reglas generales.

En la Unión Europea, el caso United Brands estableció dominancia con cuotas de mercado de 40-45\%; mientras que en el caso Akzo se estableció que una cuota de mercado mayor al 50\% es indicador de dominancia, salvo prueba en contrario.
\footnote{AKZO Chemie BV vs.~Commission, Caso C62/86 de 1993 (\href{https://eur-lex.europa.eu/resource.html?uri=cellar:4905ac67-5a02-44a0-ae93-7724be6073b0.0008.02/DOC_1\&format=PDF}{Akzo}) y United Brands Company and United Brands Continental BV vs.~EC Commission, caso 27/76 de 1978 (\href{https://eur-lex.europa.eu/legal-content/ES/TXT/PDF/?uri=CELEX:61976CJ0027\&from=EN}{United Brands})}

Los Lineamientos de la Comisión Europea (Unión Europea, 2009) establecen que ``Según la experiencia de la Comisión, no es probable que haya dominación si la cuota de mercado de la empresa en el mercado de referencia es inferior al 40\%.'' Por su parte, en EUA cuotas menores al 40\% tampoco se consideran dominantes, pero mayores al 70\% son evidencia de dominancia. Debe recordarse que además de las cuotas de mercado hay que considerar otros factores para establecer la dominancia, como por ejemplo las barreras a la entrada.

Las normativas de la Unión Europea y EUA difieren en la forma en la que tipifican la conducta. Mientras en la Unión Europea se persigue el abuso de la posición dominante, en EUA se prohíben los intentos de monopolización. El artículo 82 del Tratado de la Comunidad Europea establece que ``será incompatible con el mercado común y quedará prohibida, en la medida en que pueda afectar a los Estados miembros, la explotación abusiva, por parte de una o más empresas, de una posición dominante en el mercado común o en una parte sustancial del mismo''. Es interesante notar que lo que la legislación no prohíbe tener una posición dominante, sino ejercer un abuso de tal posición. En Europa existe la idea de que las empresas dominantes tienen una especial responsabilidad en mantener las condiciones de competencia sin distorsiones.

Sin embargo, estas categorías requieren de una definición, ya que son muy amplias. Sucesivas decisiones de la Comisión de la Comunidad Europea han dado forma a cada uno de los conceptos. La posición dominante es una posición de poder económico que ostenta una empresa y que le permite la facultad de obstaculizar una competencia efectiva en el mercado de referencia, proporcionándole la posibilidad de comportamientos, en medida apreciable, independientes respecto de sus competidores, sus clientes y, en definitiva, de los consumidores.
\footnote{Establecido en las Decisiones sobre Europemballage Corporation and Continental Can Company Inc.~vs.~Commission of the European Communities, Caso 6-72 de 1973 (\href{https://eur-lex.europa.eu/legal-content/ES/TXT/PDF/?uri=CELEX:61972CJ0006\&from=EN}{Continental Can}) y Hoffman-La Roche \& Co.~AG vs.~EC Commission, Caso 85/76 de 1979 (\href{https://eur-lex.europa.eu/LexUriServ/LexUriServ.do?uri=CELEX:61976CJ0085:EN:PDF}{Hoffman-La Roche})}

Por su parte, la Comisión señala que la posición dominante es un concepto objetivo que refiere a las actividades de una empresa en situación de posición dominante que pueden influir en la estructura de un mercado en el que, debido justamente a la presencia de la empresa en cuestión, la intensidad de la competencia se encuentra ya debilitada, y que producen el efecto de obstaculizar, recurriendo a medios diferentes de los que rigen una competencia normal de productos o servicios basada en las prestaciones de los agentes económicos, el mantenimiento del grado de competencia que aún exista en el mercado o el desarrollo de dicha competencia.
\footnote{Decisión sobre Hoffman La Roche antes mencionada.}

Por su parte, la legislación de EUA es el \href{https://www.law.cornell.edu/uscode/text/15/1}{Acta Sherman} de 1890 que, en su sección 2, establece que ``toda persona que monopolice o intente monopolizar, o se combine, o conspire con cualquier otra persona, o personas, para monopolizar cualquier parte del comercio o de la industria entre los diversos Estados, o con naciones extranjeras, será considerado culpable de falta grave (\ldots)''. Sin embargo, la interpretación de la doctrina rápidamente estableció que dicha norma no puede interpretarse en forma literal ya que ``el criterio que debe seguirse en cualquier caso para verificar si se ha producido una violación a la sección (2) es la regla de la razón (\ldots)'' 16 Es decir, la sanción aplica si se utilizan métodos ilícitos para monopolizar o intentar monopolizar un mercado.

Entre las posibles conductas de abuso de posición dominante, se presentan las tres principales: depredación, canastas y discriminación de precios.

\hypertarget{depredaciuxf3n}{%
\subsection{Depredación}\label{depredaciuxf3n}}

De las prácticas anticompetitivas, la depredación de precios debe ser una de las más conocidas, aun cuando en la realidad se haya verificado pocas veces. Ocurre cuando una empresa dominante busca eliminar a la competencia fijando precios por debajo de su costo de producción, sacrificando beneficios en el corto plazo, de forma de obtener beneficios en el largo plazo. Debe aclararse que si una empresa es más eficiente y vende por debajo del costo de otra establecida ---ineficiente---, esta es una práctica válida. Uno de los objetivos de la competencia es la eficiencia, lo que equivale a que los productores que sean ineficientes terminarán saliendo del mercado.

Esta práctica anticompetitiva tiene tres ingredientes fundamentales: (i) pérdidas que enfrenta la empresa que la realiza; (ii) poder de mercado para recuperar sus beneficios en el largo plazo; y (iii) barreras a la entrada que permitan consolidar la renta de la empresa depredadora. El énfasis en cada uno dependerá de las circunstancias, ya que la propia acción de depredación se transforma en una barrera, si los potenciales entrantes esperan una actitud hostil de la empresa instalada.

La depredación es una conducta específica, que se verifica en ciertas circunstancias. No es idéntico a vender por debajo del costo. Es relevante aclarar que la conducta de vender por debajo del costo refiere al costo que incurre la empresa en cuestión. Sin embargo, algunas veces, la venta por debajo de su propio costo puede ser una actividad legítima, como cuando se realizan ventas de final de temporada o para liquidar stock. También es una estrategia que utilizan muchas empresas que entran a los mercados como forma de atraer clientes. Toda empresa que quiera ingresar a un mercado se va a encontrar en desventaja frente a las instaladas, ya que estas tienen la clientela y una reputación en el mercado. Ingresar con precios por debajo del costo puede ser una estrategia interesante para estas empresas, de forma de atraer a los clientes que son sensibles al precio. Es una alternativa a la realización de campañas publicitarias.

El objetivo, cuando existe una denuncia por precios predatorios, es intentar desentrañar si es una conducta que puede afectar al mercado o beneficiarlo. Ello no es trivial, ya que la sanción implica que la empresa que lleva a cabo la conducta deba subir los precios, lo que parece un sinsentido para la política de competencia. Para llegar a este extremo, el riesgo de que la conducta permita que la empresa cobre precios monopólicos tiene que ser muy grande. Un reporte de la International Competition Network (ICN), señala que la metodología utilizada para analizar este tipo de conductas difiere entre los países en el énfasis que se pone en la recuperación de las pérdidas por parte de la empresa predatoria, y en la determinación del costo respecto del cual el precio es menor, entre otros. Asimismo, establece que ---excluyendo a Corea del Sur que tiene un número inusual de casos--- 33 países reportan 126 denuncias de precios predatorios, de los cuales sólo 24 fueron sancionados, es decir menos del 20\% \citep{ICN2008}. Ello refleja, por un lado, la dificultad de aplicar este tipo de procedimiento y, por otro, el problema que genera a las agencias de competencia tener que impedir que las empresas rebajen sus precios.

Desde el punto de vista teórico, esta conducta fue muy resistida. Varios artículos trataron el tema en la década de los 50, y \citet{Selten1978} presentó la paradoja de la irracionalidad de la predación en un modelo con información completa.
\footnote{Este trabajo le valió el Premio Nobel en Economía en 1994.}
Posteriormente, en la década de los 80, surgieron distintos modelos que incorporaron información asimétrica al análisis y demostraron que la depredación puede ser una conducta racional.

Si bien no existen metodologías aceptadas en todas las legislaciones, una de las referencias en el análisis de la predación es la desarrollada por \citet{Joskow1979}. La misma se basa en realizar un test en dos etapas: en la primera, se analiza el mercado para determinar si en él se podría verificar la predación; si se pasa esta etapa, se analizan los costos.

La primera etapa implica una evaluación de las características del mercado de forma de deducir si es racional una conducta predatoria. Por ejemplo, se analiza las barreras a la entrada, el número de empresas en el mercado, la diferenciación de productos, etc. También se analiza si la empresa que realiza la conducta tiene posición dominante. Si el mercado es disputable fácilmente por empresas entrantes, o si la empresa que realiza la conducta no es dominante, entonces el caso se cierra sin pasar a la siguiente etapa. Si por el contrario, existe riesgo de monopolización por parte de la empresa que realiza la conducta, se pasa a analizar los costos de la misma.

Aquí surge un nuevo problema, en la medida en que no es claro por debajo de qué costo debe estar el precio. Los autores establecen la siguiente regla: (i) si el precio es mayor al costo medio total (costo total dividido la cantidad producida), entonces nunca puede ser predatorio; (ii) si el precio está entre el costo variable medio (costo total excluidos los costos fijos, sobre el total de la producción) y el costo medio total, entonces puede ser predatorio sólo en circunstancias excepcionales; y (iii) si el precio es menor al costo variable medio, entonces es predatorio, salvo justificación económica en contrario.

La Unión Europea establece, por su parte, que el elemento sustancial a probar en este tipo de conducta es el sacrificio, es decir la pérdida que incurre la empresa que realiza la conducta. En particular ``La Comisión considerará que la conducta entraña un sacrificio si la empresa dominante, al aplicar un precio más bajo a toda o a una determinada parte de su producción, durante el período de referencia, o al aumentar su producción durante el período de referencia, incurrió o está incurriendo en pérdidas que podrían haberse evitado'' (\citet{UE2009}, p.~C 45/16).

El sacrificio o la pérdida se calcula utilizando el Costo Medio Evitable, que es ``la media de los costos que se podrían haber evitado si la empresa hubiera fabricado una modesta cantidad de producción (adicional), en este caso, a la cantidad presuntamente objeto de la conducta abusiva'' (\citet{UE2009}, nota al pie 2, p.~C 45/11). En particular, si el precio es menor al Costo Medio Evitable, entonces hay evidencia de sacrificio. Otro elemento que aporta es que no es necesario demostrar que como consecuencia de la conducta predatoria, los competidores salen del mercado.

\hypertarget{canastas}{%
\subsection{Canastas}\label{canastas}}

Las canastas son una práctica comercial más común de lo que parece. Al igual que en el caso anterior, y en todos los casos de abuso de posición dominante, tiene efectos positivos y negativos sobre el mercado. Existen diversos tipos de canastas: (i) puras, cuando dos productos se venden en proporciones fijas; (ii) línea completa forzosa, cuando los productos se venden en proporciones variables; y (iii) mixta, cuando los productos se pueden vender tanto juntos como separados.

Un automóvil es una canasta pura, ya que es un bien que es una canasta de sus componentes en una proporción que no admite alterar las proporciones. Otra canasta pura es la computadora que incluye el sistema operativo. Una tercera es el producto que ofrece las empresas de televisión por cable, en donde el conjunto de señales básicas no pueden ser elegidas por el consumidor. La línea completa forzosa se da cuando se compra un teléfono celular y este viene con minutos de alguna compañía. Por otra parte, la canasta mixta es utilizada en las ventas de temporadas de espectáculos o en los abonos deportivos, donde el cliente puede comprar tanto la temporada entera del mismo, o alguna función particular.

Las canastas se pueden utilizar como un mecanismo de discriminación de consumidores. Por ejemplo, a los consumidores que son fanáticos del ballet se les puede vender toda la temporada con un pequeño descuento, mientras que aquellos que quieren ver alguna función en particular tienen que pagar un sobre precio por esa función. Las empresas de televisión por cable agrupan distintas señales de forma de llegar a diferentes públicos, a los cuáles les cobran proporcionalmente más por el conjunto de señales de lo que estarían dispuestos a pagar si se vendieran por separado.

Otra forma es atar el producto al mercado posventa (\emph{aftermarket}) como en el caso de las impresoras y los cartuchos. En este caso, supongamos que hay dos tipos de consumidores, los que usan mucho la impresora y los que la usan poco. Si el productor no puede identificar cuál es cuál, como ocurre generalmente, entonces puede atar la impresora ---a menor precio--- a los cartuchos ---a mayor precio--- de forma de que los consumidores que usen poco la impresora elijan esta canasta.

Asímismo, la estrategia de atar productos puede ser utilizada por una empresa para excluir rivales del mercado. Por ejemplo, supongamos una empresa que tiene el monopolio de un producto en el mercado A, mientras que enfrenta competencia en el mercado B. Los productos de los mercados A y B son complementarios, por lo que se necesitan ambos para que funcionen. La empresa que tiene el monopolio en A puede atar los productos A y B como forma de extender su posición dominante del mercado A al B.

La Unión Europea considera que este tipo de conductas puede tener efectos anticompetitivos, si se aplican con el objetivo de cerrar el mercado a competidores. En particular, considera si la empresa es dominante en el mercado vinculante, y se cumple además que: (i) los productos vinculantes y vinculado son distintos y, (ii) es probable que la vinculación dé lugar a un cierre anticompetitivo del mercado \citep[p.~C 45/15]{UE2009}. Establece que ``dos productos son distintos sí, no existiendo una vinculación o venta por paquetes, un número sustancial de clientes compraría o habría comprado el producto vinculante sin comprar también al mismo proveedor el producto vinculado, lo que haría posible producir de forma independiente tanto el producto vinculante como el vinculado'' (ibid). Es decir, son productos distintos si se puede utilizar uno sin necesidad de comprar el otro. El riesgo de cierre anticompetitivo de mercado está vinculado a la duración de la conducta o a la posibilidad de que el mercado vinculado esté regulado.

Una variación relevante es cuando las empresas ofrecen descuentos multi-producto para canastas. En estos casos, las empresas pueden utilizar alguna ventaja en su cartera de productos para limitar la expansión de sus rivales, si estos no pueden imitar los descuentos, aun siendo igual de eficientes, porque no cuentan con los mismos bienes en la cartera. Sobre este tema se volverá cuando se analice las restricciones verticales, ya que las canastas pueden ofrecerse tanto a consumidores finales como a empresas intermediarias.

\hypertarget{discriminaciuxf3n-de-precios}{%
\subsection{Discriminación de precios}\label{discriminaciuxf3n-de-precios}}

Toda vez que un productor cobra precios diferentes por productos que tienen el mismo costo, se dice que hay discriminación de precios. Una definición alternativa establece que existe discriminación cuando el cociente entre el precio y el costo de los productos difiere. Este es un instrumento utilizado para incrementar los beneficios de las empresas, cobrando precios mayores a los consumidores que tienen mayor disposición a pagar por el producto. La clave es impedir que los consumidores puedan arbitrar entre sí el precio diferencial que establece el productor. En principio, este tipo de prácticas se puede asociar tanto a las explotativas como a las exclusorias.

La discriminación perfecta se conoce como de primer grado, e implica que el productor puede cobrar a cada consumidor el máximo que está dispuesto a pagar por el producto. Es un escenario ideal para el productor, ya que implica que conoce perfectamente al consumidor. En la realidad, el consumidor tiene incentivos a ocultar su disposición a pagar, por lo que el productor necesita recurrir a mecanismos alternativos. Sin embargo, permite extraer algunas conclusiones generales sobre el fenómeno.

La primera es que el productor está, en general, mejor bajo discriminación que sin ella, ya que obtiene beneficios mayores. La discriminación implica que el productor, en vez de cobrar un único precio por sus productos, cobra distintos precios que pueden ser mayores o menores al precio único. Por tanto, la discriminación genera un efecto ambi- guo sobre los consumidores. Mientras algunos pagan más por el producto que ya compraban, otros acceden al mismo porque pagan menos y antes no podían comprarlo. Por este motivo, impedir la discriminación puede resultar en que un número menor de consumidores accedan al producto y empeorar el bienestar, en vez de mejorarlo.

Si no es posible conocer con exactitud las preferencias de cada uno de los consumidores, falla la discriminación de primer grado y el productor puede utilizar dos mecanismos alternativos para obtener mayores beneficios. El primero es la discriminación por autoselección, donde la empresa ofrece al consumidor un menú de alternativas de forma de que, si están bien diseñadas, este revele su disposición a pagar. Son múltiples las expresiones de este tipo de discriminación.

Están, por un lado, las tarifas no lineales, como los descuentos por cantidad, en las cuales los consumidores pagan precios diferentes según la cantidad del bien que compren. Ello puede expresarse a través de precios diferentes según distintas unidades (un precio para la primer unidad y otro si se compran a partir de tres, la segunda unidad a mitad de precio, etc.) o a través del envase en el que se presenta (las bebidas gaseosas tienen precios diferentes según si el envase es de 600 cc, litro, litro y medio, etc.). Los cines ofrecen precios diferenciados según el día y la hora de la función. Algunos libros aparecen en edición de tapa dura y blanda con precios distintos. Hay productos que se venden dañados como forma de discriminar, como por ejemplo el sistema Windows en distintas versiones (licencia para una unidad no transferible, licencia para una unidad transferible, etc.) o con limitaciones (las antiguas netbook contenían una versión de Windows que no permitía cambiar el fondo de escritorio, si se quería cambiar había que comprar una nueva versión del sistema). El Ipad tiene distintos precios según la cantidad de memoria que tenga incorporada, donde la diferencia en el precio es muy superior al costo de instalar tal memoria. En todos los casos, el productor ofrece distintas opciones sabiendo que los consumidores valoran diferente el producto, y cuando este elije le está revelando su disposición a pagar.

El segundo es la discriminación por indicadores donde la empresa puede segmentar a los consumidores según alguna característica observable. Hay poblaciones que tienen características similares, como los estudiantes o los jubilados y, por tanto, se puede ofrecer descuentos a este tipo de grupos para incentivar su demanda. La característica de estos grupos es que su demanda es más elástica, es decir, responde en forma más intensa a cambios en el precio. Sin embargo, para impedir el arbitraje, el productor tiene que incorporar restricciones, como requerir documentos que prueben la situación del consumidor o vender servicios con nombre para que sólo estos puedan utilizarlo.

La discriminación también puede ser utilizada para excluir a otros productores del mercado, tal el caso cuando operan acuerdos exclusivos, que también se observan en el contexto de las relaciones verticales entre empresas. La compra exclusiva obliga a un cliente a comprar únicamente a una empresa, en general dominante. Otra forma de discriminación es otorgar descuentos condicionales, donde se recompensa a los clientes por una determinada conducta de compra. Este premio se puede establecer en forma retroactiva, es decir sobre todas las compras del consumidor actuales y previas; si las compras del consumidor pasan determinados umbrales; o sobre las compras incrementales. En ambos casos, la empresa discrimina a determinados consumidores con el objetivo de excluir a competidores del mercado.

Una conducta discriminatoria relacionada es la negativa al suministro. En este caso, la discriminación es total, en la medida en que una empresa se niega a tratar con otra. Este problema surge, fundamentalmente, cuando las empresas están vinculadas entre mercados verticales. Esta práctica incluye la denegación de suministro a clientes; la negativa a conceder licencias de derecho de propiedad intelectual; o la negativa al acceso a una red o a una facilidad esencial. Estas conductas discriminatorias aumentan el poder de mercado de la empresa que la realiza, a través del desplazamiento de los posibles competidores y, por tanto, pueden ser anticompetitivas.

Un último comentario refiere al análisis de las facilidades esenciales, que representan activos indispensables para competir en un mercado, como un puerto, aeropuerto o las líneas de transmisión de electricidad. Si la empresa transporta pasajeros o se dedica a la pesca, necesita acceso al puerto para amarrar sus barcos; si transporta pasajeros en aviones, tiene que tener acceso a un aeropuerto para que el avión aterrice; si genera energía eléctrica, tiene que tener acceso a la línea de transmisión eléctrica para llegar a los consumidores.

Impedir el acceso a estas plataformas puede resultar en la exclusión de competidores del mercado. Sin embargo, entregar el acceso en forma irrestricta genera incentivos al parasitismo
\footnote{Traducción del término free rider en inglés.}
de las inversiones, es decir esperar a que otro la haga y pague el costo para poder utilizarla. Por ello, la jurisprudencia de EUA señala que las facilidades esenciales deben cumplir con cuatro requisitos:
\footnote{Véase caso \href{https://law.justia.com/cases/federal/appellate-courts/F2/708/1081/330445/}{MCI Communications Corp.~vs.~American Tel. \& Tel. Co.}, 708 F.2d 1081, 1132-33 (7th Cir. 1982), 464 U.S. 891 (1983).}
(i) el control de la misma debe estar en manos de un monopolista; (ii) el competidor debe ser incapaz de duplicarla en forma razonable; (iii) se debe verificar la negativa al uso de la instalación al competidor por parte del monopolista; y (iv) debe ser factible para el monopolista dar acceso a la instalación al competidor. Si estos cuatro elementos se cumplen, entonces se puede obligar a dar acceso a competidores a esta facilidad esencial. El problema en estos casos es establecer la tarifa de acceso a la misma, ya que no existe un mercado para el servicio y el monopolista tiene incentivos a excluir al competidor cobrando un precio excesivo.

\hypertarget{colusiuxf3n}{%
\section{Colusión}\label{colusiuxf3n}}

El abuso de posición dominante es una conducta que lleva a cabo una única empresa. Sin embargo, a veces son varias las empresas que realizan acciones anticompetitivas. En esta sección se analiza una de estas acciones colectivas que ocurre cuando empresas que deberían competir entre sí, no lo hacen. \citet{Cabral2017} señala que la colusión son acuerdos entre empresas con el objetivo de aumentar su poder de mercado. Por su parte, \citet{Motta2004} las define como prácticas que permiten a las empresas ejercer un poder de mercado que de otra forma no tendrían, restringiendo la competencia y el bienestar. De todas las interacciones que pueden tener empresas competidoras ---por ejemplo, cooperando en investigación y desarrollo---, el énfasis estará en aquellas relaciones que explícitamente buscan reducir la competencia.

Hasta la década de los 70 la economía no había encontrado una teoría que pudiera explicar por qué las empresas querrían formar un cartel. Los principales modelos que explican la colusión suponen que las empresas que participan del mercado fijan el precio ---o la cantidad--- con independencia unas de otras. En estos modelos no hay un acuerdo tácito, sino que este es implícito. Sin embargo, en la realidad, las empresas se comunican entre sí para lograr coordinar acciones.

Estos modelos económicos explican por qué ---y bajo qué condiciones--- se alcanza la colusión, para lo que requieren de un entorno dinámico, en el cuál las empresas interactúan a lo largo del tiempo. En cada momento, tienen que decidir si participan del cartel o traicionan a sus compañeros. La situación más extrema del mercado, el monopolio, se presenta cuando hay una única empresa que controla toda la demanda y no enfrenta competencia y, por lo tanto, tiene la posibilidad de generar escasez de forma de aumentar el precio de sus productos y obtener beneficios extraordinarios. Cuando en el mercado hay más de una empresa, la competencia entre ellas las incentivará a aumentar la producción respecto a la situación de monopolio.

La colusión implica que las empresas que participan del cartel se comportarán como un monopolista, es decir producirán menos de lo que querrían, de forma de aumentar sus beneficios. Ello se logra si y sólo sí todas las empresas cumplen el acuerdo, de forma de que el precio suba debido a la escasez. Ahora bien, ¿por qué la traición sería una acción posible? El aumento de precio que resulta si opera el cartel, es la tentación de cada participante para traicionar a sus compañeros y, dado que todos los demás están produciendo menos, expandir unilateralmente la producción y ganar beneficios.
\footnote{Una alternativa es que las empresas acuerden fijar un precio de monopolio, en vez de fijar la cantidad. Pero la tentación se presenta de nuevo, bajo la forma de arbitrar el precio de los rivales. Mientras ellas están fijando un precio alto, un participante del cartel puede rebajar el precio de los demás y expandir en forma importante sus ventas.}

Si la decisión fuera por única vez, entonces no se podría sostener la colusión: todos querrían decir que participan del cartel y, mientras esperan que los demás reduzcan la producción, traicionarlos aumentándola. Lo que sucede siempre que hay traiciones es la represalia. En estos modelos, la represalia se presenta como la pérdida de beneficios que enfrentan las empresas, producto de una intensificación de la competencia una vez que descubren que han sido traicionadas.

Cuando se introduce el tiempo en este análisis se impone un costo sobre los beneficios futuros de las empresas, ya que estas no valoran igual los beneficios actuales que los futuros. En general, un mismo valor hoy y dentro de un
año pesan distinto en las decisiones de los agentes, que tienden a valorar más el presente. Cuanto más pacientes sean los agentes, más valorarán el futuro y más fácil será sostener acuerdo colusorios. Por su parte, cuanto más impacientes sean, mayor será la tentación inmediata de obtener beneficios ahora aunque cueste perder beneficios después. En otros términos, la colusión surge porque los agentes son lo suficientemente pacientes para vencer la tentación del momento en pos de una renta futura relevante.

Estos modelos presentan además una aparente paradoja: cuanto más fuerte sea el castigo, más fácil será sostener la colusión y, por tanto, menor será la necesidad de su aplicación. Esta idea ha permitido representar múltiples situaciones para predecir en qué circunstancias la colusión es más factible. \citet{Ivaldi2003} presentan un conjunto de situaciones que facilitan la colusión, las que se resumen en el cuadro \ref{tab:cuadro4}.

\begin{table}

\caption{\label{tab:cuadro4}Variables que facilitan la colusión.}
\centering
\begin{tabular}[t]{l|l|l|l}
\hline
Estructura & Oferta & Demanda & Otros\\
\hline
Menor número de competidores & Tecnologías estables & Demanda menos elástica & Menor poder de compra\\
\hline
Contacto frecuente entre las empresas & Empresas similares & Demanda creciente & Existencia de otros acuerdos cooperativos entre las empresas (ej. I+D)\\
\hline
Altas barreras a la entrada & Contacto multimercado entre empresas & Mercados estables & \\
\hline
Mercados más transparentes &  &  & \\
\hline
\end{tabular}
\end{table}

Fuente: elaborado sobre la base de Ivaldi et al.~(2003).

Cuando existe información asimétrica entre las empresas respecto a alguna variable relevante de mercado, mantener los precios rígidos puede ser la única forma de sostener los acuerdos colusivos. \citet{Rotemberg1990} encuentran este resultado en un modelo que explica, además, el surgimiento de una empresa líder que fija el precio en el mercado. \citet{Athey2004}, por su parte, demuestran en términos más generales que la rigidez de precio es un mecanismo muy efectivo para sostener acuerdos de precio entre empresas competidoras.

Si no existiera prueba concreta de acuerdo entre empresas, ¿qué evidencia podría sugerir que existe colusión? Harrington (capítulo 6 de \citet{Buccirossi2008}) propone algunos patrones de conducta que son indicadores de acuerdos colusivos. Bajo ciertas condiciones, la dispersión de los precios de las empresas ---varianza--- es menor, y existe una relativa estabilidad de las cuotas de mercado bajo un acuerdo colusivo que bajo competencia. En el caso de licitaciones o llamados de precio, si hay colusión de los oferentes se observa que los precios de las empresas en las pujas deberían estar positivamente correlacionados entre sí a lo largo del tiempo, mientras que las cuotas de mercado ---ganar o perder una licitación--- deberían estar negativamente correlacionadas.

La evidencia empírica respecto a la operación de los carteles muestra que las empresas tienen contacto entre sí, y que no sólo fijan el precio o la cantidad sino también otras variables \citep{Levenstein2006}. La fijación del precio va de la mano del reparto de clientes, la división de zonas geográficas, el reparto de licitaciones y otras medidas adicionales que permitan a los participantes controlar las acciones de los miembros y, eventualmente, implementar sanciones. También pueden incluir pagos laterales que realizan las empresas que incumplen algunos de los puntos del acuerdo, como forma de evitar represalias. Otras veces, utilizan a terceras empresas para sostener los acuerdos.

En Uruguay, en el caso de un cartel en el mercado del tomate procesado, las empresas tenían un acuerdo de cuotas que se materializaba utilizando una única imprenta para fabricar las cajas para el envasado. Esta imprenta, a su vez, entregaba las cajas a otro de los participantes en el acuerdo, para que este controlara la cuota fijada a cada empresa participante.

Los carteles, salvo el de la OPEP, son secretos por lo que son difíciles de detectar. Ello ha llevado a los gobiernos a buscar instrumentos que permitan desalentar su conformación y desestabilizar los ya formados. El principal instrumento que se utiliza son las políticas de clemencia, que reducen o exoneran las multas a las empresas o individuos que colaboran aportando información sobre los carteles en los que participan. Sin estas normas, no hay elementos jurídicos para evitar sancionar a las empresas que colaboran. Cuando en 1994 EUA introdujo
normas de clemencia, en los siguientes cinco años se detectaron y sancionaron el mismo número de carteles que en los 90 años anteriores. En la Unión Europea se introdujeron normas de clemencia en 2002 y los casos se multiplicaron por diez.

\citet{Miller2009} analiza en forma sistemática el efecto de estas políticas y concluye que desalientan la formación de carteles y aumentan las capacidades de detección por parte de las agencias de competencia. Algunos autores señalan que la clemencia no es suficiente y que se requiere recompensas para los delatores. En muchos casos, principalmente en la Unión Europea y en EUA, las personas que llevan adelante las delaciones luego tienen problemas para encontrar trabajo, por lo que se requiere recompensarlas para que la sanción social sea tolerable.

Las normas sobre carteles difieren en su contenido, la forma en la que se aplican y si contienen programas de clemencia. Si bien todas las normas de competencia tienen algún capítulo que prohíbe los carteles, las legislaciones difieren si esta prohibición es en sí misma (per se) o hay que demostrar el daño (regla de la razón). En general, existe un relativo consenso respecto a que los carte les llamados duros (hard core cartels) deberían estar prohibidos en sí mismos.

Estos carteles son acuerdos entre competidores para fijar precios, restringir la producción, repartir mercados o arreglar licitaciones. Una prohibición en sí misma sólo requiere probar que la conducta existe, ya sea por su objeto o sus efectos. Ello hace más fácil la aplicación de las normas y da mayor transparencia, ya que si la aplicación es bajo la regla de la razón hay que probar que la conducta está afectando el mercado, lo que puede ser complejo y costoso. En Uruguay, hasta el 2020, sólo se puede actuar sobre carteles que tengan una posición dominante en el mercado, lo que hacía complejo la aplicación de la norma.

En EUA los carteles duros están prohibidos en sí mismo, lo mismo que en la Unión Europea. En EUA se aplica la sección 1 del Acta Sherman (1890) que establece que ``se declara ilegal cualquier contrato, combinación bajo la forma de trust u cualquier otra, o conspiración, que restrinja el comercio entre los Estados o con otros países''.
\footnote{En el original ``Every contract, combination in the form of trust or otherwise, or conspiracy, in restraint of trade or commerce among the several States, or with foreign nations, is declared to be illegal''.}
Esta norma considera prohibidos en sí mismos a los carteles duros, o las llamadas restricciones a simple vista (\emph{naked restrains}), mientras que se aplica la regla de la razón a las restricciones auxiliares (\emph{auxiliary restrains}). En este país, como en muchos otros, la participación en carteles es un delito penal que implica pena de cárcel.

En la Unión Europea se aplica el artículo 81 del Tratado de la Comunidad Europea que establece ``1) Serán incompatibles con el mercado común y quedarán prohibidos todos los acuerdos entre empresas, las decisiones de asociaciones de empresas y las prácticas concertadas que puedan afectar al comercio entre los Estados miembros y que tengan por objeto o efecto impedir, restringir o falsear el juego de la competencia dentro del mercado común y, en particular, los que consistan en: a) fijar directa o indirectamente los precios de compra o de venta u otras condiciones de transacción; b) limitar o controlar la producción, el mercado, el desarrollo técnico o las inversiones; c) repartirse los mercados o las fuentes de abastecimiento; d) aplicar a terceros contratantes condiciones desiguales para prestaciones equivalentes, que ocasionen a éstos una desventaja competitiva; e) subordinar la celebración de contratos a la aceptación, por los otros contratantes, de prestaciones suplementarias que, por su naturaleza o según los usos mercantiles, no guarden relación alguna con el objeto de dichos contratos.2) Los acuerdos o decisiones prohibidos por el presente artículo serán nulos de pleno derecho.''

En este caso, las restricciones son también prohibidas en sí mismas y no aplica ningún tipo de excepción según el tamaño de los participantes. Sin embargo, otro tipo de acuerdos como la investigación y desarrollo; la producción conjunta; los acuerdos de compra; o de normalización técnica, se estudian bajo la regla de la razón. Un segundo elemento tiene que ver con la forma en la que se aplica la normativa. En principio la persecución de carteles es un procedimiento bastante similar al de cualquier otro delito, en el que hay que obtener evidencia de su existencia, demostrar la participación de las empresas, entre otros. Por tanto, muchos países recogen la información a través de allanamientos policiales, tanto a empresas como a casas particulares. Existen técnicas bastante desarrolladas para diseñar e implementar este tipo de procedimientos, que es relativamente complejo, a los efectos de evitar filtraciones que permitan a los infractores destruir evidencia.

Cuando esta evidencia no está disponible, ¿cómo es posible identificar la colusión con la información de mercado? En estos casos el objetivo es demostrar que la conducta que se observa en un mercado sólo puede ser explicable si existe un acuerdo entre las empresas. Es decir, hay que descartar explicaciones alternativas. Ello suele ser complejo porque una misma conducta puede ser resultado de la competencia o de un cartel. Si se observa que las empresas cobran todas el mismo precio, ello puede ser el resultado de la competencia, si el bien es homogéneo, pero también de un cartel. Lo que diferencia a uno de otro es el nivel de precios en uno y otro caso, pero para poder discriminar los escenarios se requiere obtener información que permita evaluar el poder de mercado, es decir, los costos de las empresas. Ello, en general, está fuera del alcance de las agencias de competencia.

Otra forma para identificar la colusión es ver la evolución de los precios a lo largo del tiempo, lo que se conoce como la doctrina del \textbf{paralelismo consciente}. Sin embargo, pueden existir múltiples razones competitivas que expliquen un comportamiento paralelo de las empresas. Por ello muchas veces se requiere de restricciones adicionales que descarten un comportamiento competitivo, como anunciar cambios de precio, o el mantenimiento de precios de reventa (ver sección \ref{rest-vert}, restricciones verticales), lo que se conoce como la doctrina del \textbf{paralelismo plus}.

Las conductas colusorias pueden ser tres. Por un lado están los acuerdos, definidos como todo pacto, verbal o escrito, mediante el cual varios operadores económicos se comprometen a una conducta que tiene como finalidad o efecto restringir la competencia. Por otro, las decisiones o recomendaciones colectivas, que son acuerdos adoptados en el seno de asociaciones empresariales o corporaciones, bien de carácter vinculantes (decisiones) u orientativo (recomendaciones), que tienden a uniformizar el comportamiento de los asociados y a limitar la competencia entre ellos. Por último, se encuentran las prácticas concertadas que es cualquier forma de cooperación informal entre empresas independientes que, sin haber llegado a concluir en un acuerdo propiamente dicho, conscientemente sustituyen la competencia por la cooperación práctica entre ellas.

Por otra parte, como vimos, algunos países tienen en sus legislaciones políticas de clemencia. Para que esta política sea efectiva debe dar garantías de que la empresa que aporte información quedará totalmente exonerada de multa, ya que de otra forma no tendrá incentivos a colaborar. Muchas veces, la clemencia se otorga cuando el caso ya se está investigando ya que la misma puede ser mucho más sencilla si alguno de los integrantes aporta las pruebas. En última instancia, las agencias de competencia no son recaudadoras, lo que buscan es que el mercado funcione y las empresas infractoras sean descubiertas y abandonen sus prácticas anticompetitivas. Otras legislaciones también otorgan inmunidad a empresas que aporten información sobre carteles que estén conformados en otros mercados. En cualquier caso, la clave del éxito de los programas de clemencia es que deben estar claramente diseñados, ser objetivos y evitar la incertidumbre y discrecionalidad.

\hypertarget{rest-vert}{%
\section{Restricciones verticales}\label{rest-vert}}

Los consumidores acceden a los bienes y servicios luego de un extenso proceso productivo. Los insumos son convertidos en productos, los productos distribuidos a tiendas que luego los ofrecen a los consumidores. Muchas veces, con posterioridad a la venta, existe otra etapa de asesoramiento o servicio posventa, o reparación del producto. En estas distintas etapas puede actuar una misma empresa. Sin embargo, lo más común es que sean empresas distintas las que lleven a cabo cada una de las etapas del proceso productivo. Si bien el producto final es el mismo, suponer configuraciones distintas a lo largo de la cadena de valor lleva a resultados diferentes de precio y características de los productos.

Mientras que en la sección anterior se analizó la vinculación horizontal entre mpresas ---aquellas en el mismo eslabón de la cadena de producción--- en esta sección se analizan los casos donde esta vinculación es de tipo vertical. Es decir, las empresas no son competidoras entre sí, sino que operan en mercados diferentes con lógicas distintas. Por tanto, lo que es óptimo para una empresa en un eslabón de la cadena de valor, puede no serlo para otra empresa en un eslabón diferente. La relación más fuerte entre dos unidades productivas es la integración, que es cuando forman parte de la misma empresa. En este caso,los incentivos de ambas unidades están perfectamente alineados.

Sin embargo, integrar procesos es costoso, tanto en términos de costos productivos ---economías de variedad--- como de costos de transacción. Como los incentivos en cada uno de los eslabones son distintos, la máxima libertad a cada unidad productiva puede perjudicar a una o ambas partes. Las restricciones verticales surgen como una respuesta alternativa a la integración ---fusión--- de dos unidades productivas, con el objetivo de alinear los incentivos entre sí. Por otra parte, la diferencia entre la realidad de mercado que enfrenta cada una de las empresas que acuerdan las restricciones verticales les permite, a algunas de ellas, usarla para balancear la competencia a su favor. Es decir, las empresas pueden usar las restricciones verticales no para resolver un problema de incentivos entre ellas, sino para afectar la competencia en alguno de los mercados. Esta interpretación polarizada de las restricciones verticales se explica más como un problema de balance ya que las restricciones verticales pueden resolver temas de eficiencia, a la vez que generar problemas de competencia.

Las relaciones verticales entre empresas son complejas. Involucran múltiples dimensiones diferentes ---eficiencia, incentivos, externalidades--- según si esta es hacia adelante (productor--distribuidores) o hacia atrás (proveedor--productor). Asimismo, los problemas también varían cuando el análisis es intra marca, es decir cuando se estudia las relaciones entre empresas que distribuyen una misma marca, o si se analiza la competencia inter marcas. En cualquier caso, restringir el análisis a las conductas anticompetitivas sin presentar los beneficios de las restricciones verticales puede ser peligroso, ya que nunca hay una interpretación en una única dirección. Por este motivo se presentan las dos visiones asociadas a las restricciones verticales: la eficiencia y las conductas anticompetitivas.

\hypertarget{eficiencia}{%
\subsection{Eficiencia}\label{eficiencia}}

En las relaciones verticales entre empresas se encuentran tres tipos de problemas. En primer término, muchas veces la empresa necesita incorporar ciertos activos específicos para poder producir los productos que la otra parte requiere. Según la teoría de los costos de transacción, esto la expone a la expropiación, si no se instrumentan salvaguardas. En segundo lugar, cuando se analiza la cadena de valor aguas abajo ---proveedores a minoristas--- el poder de mercado en cada uno de los eslabones genera externalidades al productor que ve como su demanda disminuye, producto de los aumentos de precios de las demás etapas. Por último, las inversiones específicas también generan incentivos al parasitismo entre las partes con resultados no óptimos en los mercados. En esta sección se desarrollan cada uno de estos problemas.

El análisis de defensa de la competencia en EUA, hasta la década de los 70, suscribía una visión negativa de las restricciones verticales. Como reacción, surge la teoría de los costos de transacción, la más influyente con relación a la eficiencia de las restricciones verticales. La misma requiere de un mínimo desarrollo para comprender sus conclusiones (un desarrollo extenso se puede encontrar en \citet{Williamson1998}). Esta teoría se enfoca en las transacciones, en un entorno en el cual los agentes no pueden establecer contratos completos entre sí y, además, tienen comportamientos oportunistas. Esta conjunción ---contratos incompletos y oportunismo--- puede generar problemas \emph{ex post} entre los agentes. Por lo tanto, el diseño de instrumentos para resolver las controversias, cuando la relación está en marcha, es fundamental.

\citet{Williamson1998} define a los costos de transacción como los costos \emph{ex ante} de diseñar, negociar y resguardar los acuerdos contractuales y los \emph{ex post} asociados a los desajustes y a la adaptación que surge cuando la ejecución del contrato sufre desvíos producto de errores, omisiones o imprevistos. Es decir, esta teoría toma en cuenta que muchos costos para las empresas no están relacionados con el propio proceso productivo, sino con el entorno contractual en el que se mueven, con proveedores o minoritas.

En este marco, llevar a cabo una transacción dentro de una empresa o en el mercado tiene costos de transacción distintos. El mercado y la empresa, que \citet{Williamson1998} llama formas organizacionales, tienen diferentes características. Mientras las empresas tienen una visión unitaria que facilita la adaptación cooperativa, el mercado resuelve la adaptación de forma autónoma.

Por otra parte, mientras los mercados generan fuertes incentivos a los participantes, ya que su supervivencia depende completamente de su esfuerzo, las empresas tienden a mitigar los incentivos a cada unidad en pos del bien común. Por último, los problemas que surgen dentro de una empresa se resuelven a través de las jerarquías, mientras que los problemas en el mercado se resuelven en la justicia.

Por tanto, mientras los mercados generan mayores incentivos que las empresas a la búsqueda de beneficio, tienen menos flexibilidad para adaptarse y requieren de instituciones externas para resolver los conflictos. Por su parte, las empresas se adaptan en forma más rápida y flexible, pero establecen incentivos más débiles a cada unidad en la búsqueda de beneficios. Estas características implican que tanto los mercados como las empresas tengan
costos de transacción distintos.

Partiendo de la base de que los contratos entre las empresas son incompletos y que los individuos que las integran tendrán comportamientos oportunistas, los riesgos de usar el mercado para las transacciones aumentan cuanto mayor sean los activos específicos para desarrollar la actividad. Un activo específico es aquel que no tiene usos alternativos y, por tanto, invertir en ellos expone a la parte que hace la inversión, al oportunismo de la otra. Si la relación entre las empresas alcanza un punto no previsto en el contrato, la empresa que tiene hundidos activos específicos está en peor posición para renegociar el acuerdo, ya que la otra parte tiene incentivos a utilizar esa ventaja a su favor y expropiarle parte de sus activos.

Por lo tanto, actuar sobre las restricciones verticales aumenta los costos de transacción de utilizar el mercado e induce la integración. Por ello, hay que tener mucho cuidado con impedir restricciones entre empresas cuando hay activos específicos en juego.
\footnote{Un análisis de la evidencia empírica sobre los límites de la integración vertical de las empresas, que también incluye una revisión sencilla de la literatura teórica, se puede encontrar en el trabajo de \citet{Lafontaine2007}.}
Gran parte de las restricciones que se analizarán están diseñadas como forma de proteger los activos específicos de alguna de las partes.

Un segundo problema en las relaciones verticales entre empresas es el del doble margen, o monopolios sucesivos. Este surge cuando existe poder de mercado en las etapas sucesivas al proceso de producción, y el problema del monopolio se magnifica. Lo interesante es que las partes tienen los incentivos a resolver el problema sin necesidad de intervención externa. En particular, si un productor monopolista le vende a un distribuidor monopolista, el margen que agregará este último determina que el precio final al público esté muy por encima del precio que regiría si ambos estuvieran integrados. Debido a la forma de la demanda, el productor pierde más beneficios de los que gana el distribuidor como resultado del incremento en el precio, ya que al fijar precio en el tramo elástico de la demanda se traduce en una caída más que proporcional en la cantidad vendida. Ello crea las condiciones para que productor y distribuidor negocien, ya que ambos pueden ganar si lo hacen. El resultado final de esta negociación entre empresas monopólicas, dependerá del poder relativo de cada una y, a prori, no es posible decir qué parte saldrá más favorecida.

Existen tres soluciones al problema del doble margen, las que se presentan suponiendo que todo el poder de negociación está en manos del productor. La primera es que el productor imponga un precio máximo de reventa al distribuidor, conocido como mantenimiento de precio de reventa (o \emph{resale price maintenance}, RPM por sus siglas en inglés). Cualquier precio menor al que fijaría el distribuidor sin este requerimiento, mejora tanto a los consumidores como al productor.

La segunda es que el productor ofrezca un contrato al distribuidor que imponga una cantidad mínima de compra. Si esta cantidad es igual a la que vendería el productor al público, en ausencia del distribuidor, entonces el distribuidor deberá fijar el precio asociado si quiere vender todas las unidades que compró. Esta solución es la inversa de la primera: en vez de fijar el precio, se fija la cantidad.

Una tercera solución es implementar una tarifa en dos partes que tiene un componente fijo y otra variable que depende de la cantidad que compre el distribuidor. Esta solución también se conoce como franquicia, ya que el pago fijo se asemeja a la compra de la licencia de la franquicia. Esta solución implica que el productor vende los productos al costo y obtiene beneficios a través del componente fijo. Como el componente fijo no distorsiona la cantidad que compra la empresa distribuidora ---aunque sí influye en su decisión de comerciar---, al comprar al costo toma la misma decisión que haría la empresa productora; es decir, el precio que fija es el que establecería aquella como monopolista. Luego, la empresa productora puede obtener la renta monopólica de la distribuidora a través del componente fijo.

El tercer tipo de problema que surge en la relación vertical entre empresas productoras y distribuidoras es más complejo, ya que depende de la forma específica que adopte la misma. El elemento común a todos ellos es el parasitismo de las inversiones ajenas. En esta situación pueden observarse tanto casos donde no hay restricción vertical entre las partes, como otros donde sí operan restricciones verticales. Entre los primeros, una lista no taxativa de ejemplos son los siguientes: (i) dos distribuidores tienen que realizar inversiones para publicitar una marca que benefician a ambos; (ii) un productor realiza inversiones específicas en un distribuidor (por ejemplo, capacitación) que otros productores pueden utilizar; (iii) un proveedor puede utilizar la marca del productor (ejemplo, McDonalds) para obtener tráfico y vender productos que le reportan ingresos, pero que el dueño de la marca no aceptaría; (iv) un productor utiliza a un distribuidor para colocar sus productos (ejemplo, Starbucks) aprovechando su imagen.

En todos los casos, hay una parte que aprovecha inversiones realizadas por otra a su favor. En la realidad estos efectos no se observan dado que las partes instrumentan restricciones entre sí para impedir el parasitismo. Sin embargo, la solución de este problema pasa por crear algún tipo de exclusividad para alinear los incentivos de las partes, lo que a su vez genera el problema del doble margen. Por tanto, la solución al problema del parasitismo está atada y requiere, también, resolver el problema del doble margen.

Entre los casos en que operan restricciones verticales, los siguientes son ejemplos de problemas específicos y sus soluciones. La distribución de productos requiere muchas veces que los distribuidores realicen actividades específicas para potenciar el uso de la marca del productor. Si los distribuidores no pueden apropiarse de los retornos de estas inversiones, no tendrán incentivos al esfuerzo. Una solución es establecer restricciones territoriales ---por ejemplo, la distribución en determinadas zonas geográficas, o la franquicia de una marca en un país--- o distribución exclusiva. En ambos casos se establece el monopolio de un distribuidor sobre un área geográfica para que internalice los esfuerzos o inversiones que realice. Sin embargo, este acuerdo de distribución territorial o exclusividad tiene que incluir una tarifa en dos partes o un RPM de forma de que el distribuidor no fije precios monopólicos. Estos acuerdos de distribución exclusiva con un instrumento muy utilizado por los productores, dado que tienen otros efectos positivos. Al asignar zonas exclusivas, los productores
pueden controlar en forma más eficiente la forma en la que se atiende cada mercado, además de generar subsidios cruzados al restringir la competencia de otros distribuidores en zonas que tienen picos de demanda.

En otros casos la imposición de un precio mínimo basta para incentivar a los distribuidores, como medida alternativa a la distribución exclusiva. Por ejemplo, si aquellos tienen que realizar acciones específicas para promocionar el producto, como por ejemplo exhibirlo o tener personal capacitado para atender a los consumidores, entonces algunos distribuidores pueden optar por no hacer las inversiones y vender el producto más barato, dado que sus costos son menores a los que sí ofrecen el servicio. Un precio mínimo basta para evitar que los distribuidores se parasiten entre sí.

Un problema común que se presenta en la relación entre productores y distribuidores es el de la reputación. Como fuera mencionado, tanto productores como distribuidores proveen reputación de sus productos o servicios. Por ejemplo, los relojes Rolex o Mont Blanc son marcas de muy alta calidad y reputación. Para poder distribuir sus productos se requiere cumplir con una serie de condiciones, es decir la empresa selecciona a sus distribuidores. Esta distribución selectiva tiene que estar acompañada por RPM de forma de evitar el problema del doble margen. En general, las agencias de competencia miran que las condiciones establecidas para la distribución selectiva sean generales y no discriminatorias.

Si el problema de reputación es a la inversa ---el distribuidor tiene la marca--- entonces el productor tiene que pagar una tarifa de acceso, de forma de evitar el parasitismo y la sobre demanda. Un caso típico es el de los supermercados, que cobran por el acceso a la góndola, peaje que regula los incentivos de los productores a utilizar la marca de los distribuidores.

\hypertarget{pruxe1cticas-anticompetitivas}{%
\subsection{Prácticas anticompetitivas}\label{pruxe1cticas-anticompetitivas}}

Si bien las relaciones verticales entre empresas presentan complejidades, las soluciones, sin embargo, están al alcance de las partes. Los mismos incentivos que se analizaron como parte de la eficiencia en las relaciones verticales, son los que generan los comportamientos anticompetitivos que pueden facilitar la colusión entre productores; utilizar a los distribuidores como mecanismo para aumentar los precios a los consumidores; y utilizar a los distribuidores para cerrar el acceso al mercado a otros productores.

Si dos o más productores quieren formar un cartel y venden sus productos a través de distribuidores, entonces otorgarles libertad de acción dificulta el control de los posibles desvíos del acuerdo. Si un productor observa un precio en un distribuidor, menor al acordado, no puede saber si este precio es producto de un desvío del productor o lo ha fijado el distribuidor. Por tanto, si los productores quieren que el acuerdo colusivo sea operativo, tienen que fijar un precio mínimo de reventa a los distribuidores de forma de obligarlos a no sabotear el acuerdo. Este precio mínimo se analizó previamente en el caso de inversiones específicas por parte de los distribuidores, lo que demuestra que un mismo instrumento puede utilizarse con fines opuestos según el contexto.

Un mecanismo alternativo, pero que permite obtener el mismo resultado, es que los productores elijan un distribuidor único para sus productos. En este caso, el distribuidor puede actuar como un monopolista que opera mirando el mercado de ambas empresas, lo que les permite a los productores fijar precios monopólicos al
consumidor.

El segundo problema anticompetitivo surge cuando se analiza la competencia intra marca y refiere a la falta de compromiso del productor. Si un productor tiene un producto relativamente valioso, distribuirlo a través de varias empresas le puede generar problemas. Si el producto se vende a través de un distribuidor, este estará dispuesto a pagar un alto precio por el producto, debido a las perspectivas de obtener una renta importante al venderlo a los consumidores. Pero una vez vendido el bien, con el plus que otorga ser el único distribuidor del mismo, el productor puede ir a otro distribuidor e intentar venderle el producto a un precio menor, lo que le permite tener una ventaja sobre el otro distribuidor, y así sucesivamente. Los distribuidores prevén que, sin una garantía de exclusividad, el productor no querrá mantener su promesa y se negarán a pagar un precio alto por el producto de la empresa productora.

La forma de recuperar la credibilidad es instrumentar algún tipo de restricción vertical, como territorios exclusivos para los distribuidores o un precio mínimo de reventa (RPM). Con el primer instrumento, el productor restringe la competencia entre distribuidores y, con ello, la posibilidad de renegociar el acuerdo. Es decir, garantiza el monopolio a los distribuidores en las zonas establecidas. El segundo señala a los distribuidores que no intentará rebajar el precio al que les vende y que, si lo hace, ello no significará una ventaja para el distribuidor (ya que no pueden arbitrarse entre sí). En última instancia, la existencia de múltiples distribuidores es una tentación que el productor busca resistir imponiéndose limitaciones. Ello le permite recuperar la credibilidad y aumentar el precio final a los consumidores y los beneficios.

Por último, las restricciones verticales permiten que los productores utilicen a los distribuidores o a los proveedores de insumo para cerrar el acceso al mercado de competidores. En el primer caso, un productor puede cerrar el mercado a otro simplemente ofreciendo contratos de exclusividad a los distribuidores. En general, si la empresa tiene posición dominante, ello dificulta el ingreso de nuevos productores al canal de distribución, ya que estos necesitan disponer de recursos para romper los acuerdos entre la empresa instalada y sus distribuidores. En el segundo, el contrato puede realizarse con un proveedor de insumos al que se puede ofrecer un contrato de exclusividad de compra que le permita a la empresa enfrentar en forma más beneficiosa la competencia en su mercado, dejando en desventaja a otras firmas que no tienen igual acceso a insumos específicos. En cualquier caso, para que estas estrategias sean exitosas, las empresas que las realizan tienen que tener una posición dominante en el mercado.

Las restricciones verticales, en resumen, pueden tener tanto efectos beneficiosos que permiten aumentar el bienestar de los agentes, como pueden utilizarse como mecanismos para ejercer prácticas anticompetitivas. Las conductas de las empresas tienen que analizarse caso a caso y determinar, en cada uno, su impacto sobre la competencia. Establecer reglas per se para restricciones verticales es un riesgo, ya que en algunos casos aumentarán los costos de transacción de utilizar el mercado y los riesgos a los que se exponen las empresas por transar entre sí. Penalizar una conducta más que otra, cuando tienen el mismo efecto, lo único que hará es incentivar a las empresas a realizar la conducta permitida. A vía de ejemplo, el RPM tiene efectos anticompetitivos similares a la fijación de cantidades ---o imposición de un surtido completo.

\hypertarget{aplicaciuxf3n-1}{%
\subsection{Aplicación}\label{aplicaciuxf3n-1}}

En la Unión Europea, las prácticas verticales se analizan igual que las conductas de abuso de posición dominante, de la cual son expresión. Sin embargo, hay lineamientos específicos para este tipo de práctica que, a pesar de haber sufrido modificaciones, son controvertidos.
\footnote{Véase \citet{UE2010}.}
Estos lineamientos, presentan una particular aversión a los acuerdos de RPM que, como se analizó, no son el único instrumento para alcanzar una conducta anticompetitiva. Se aplican a empresas con cuotas de mercado mayores al 30\%, con excepción de conductas especialmente graves (como el RPM) donde no operan excepciones, y siempre que las restricciones no impliquen plazos mayores de 5 años.

La metodología analiza nueve puntos: (i) la naturaleza del acuerdo; (ii) la posición de mercado de cada una de las partes; (iii) la posición de mercado de los competidores; iv) la posición de los compradores del producto; (v) las barreras a la entrada; (vi) la madurez del mercado; (vii) si el bien es intermedio o final; (viii) la naturaleza del producto; y (ix) otros elementos. Asimismo, se analizan una serie de conductas y se listan los beneficios y riesgos para la competencia de cada una de ellas.

Algunas conductas consideradas por la Comisión de la Unión Europea son:

\begin{enumerate}
\def\labelenumi{\arabic{enumi}.}
\item
  \textbf{Acuerdos de marca única}: son aquellos que consisten en inducir al comprador a concretar sus pedidos de un tipo de producto concreto a un único proveedor. Entre estas conductas se incluye la imposición de compras mínimas y las cláusulas de no competencia ---es decir, compras mayores al 80\% de los productos a un único proveedor---. Los riesgos que señala la Comisión en estos acuerdos, son la exclusión de potenciales competidores, la colusión entre proveedores y la pérdida de competencia inter marca.
\item
  \textbf{Distribución exclusiva}: es la práctica por la cual el proveedor acepta vender sus productos exclusivamente a un distribuidor para su reventa en un territorio determinado. Los riesgos de esta conducta son la exclusión de competidores potenciales, la colusión entre proveedores y la pérdida de competencia intra marca.
\item
  \textbf{Asignación de cliente exclusivo}: cuando un proveedor acuerda vender sus productos solamente a un distribuidor para la reventa a un grupo de clientes. En general, es eficiente si existen inversiones específicas, aunque puede aparejar riesgos de exclusión de competidores potenciales o la colusión entre proveedores.
\item
  \textbf{Distribución selectiva}: restringe el número de distribuidores autorizados y las posibilidades de reventa. A diferencia de la distribución exclusiva, la restricción en el número de distribuidores depende de criterios de selección relacionada con la naturaleza del producto. La práctica implica riesgos de exclusión de competidores potenciales, de colusión entre proveedores y de pérdida de competencia intra marca.
\item
  \textbf{Franquicia}: son acuerdos que contienen licencias de derechos de propiedad intelectual relativos a marcas o signos registrados, y conocimientos técnicos para el uso y la distribución de bienes y servicios. Sirve para transferir conocimientos, y se requiere para proteger este activo. Los pagos de acceso inicial son cánones fijos que los proveedores pagan a los distribuidores en el marco de una relación vertical al principio de un período, para obtener acceso a su red de distribución y remunerar servicios proporcionados a los proveedores por los minoristas. Incluye las tasas por asignación de espacio, las tasas de mantenimiento, los pagos para tener acceso a las campañas de promoción de un distribuidor, entre otros. El riesgo que se enfrenta con esta práctica es la exclusión de competidores potenciales y además facilita la colusión.
\item
  \textbf{Mantenimiento de precio de reventa (RPM)}: son acuerdos o prácticas concertadas ---nótese la similitud del lenguaje con la colusión--- cuyo objeto directo o indirecto es el establecimiento de un precio de reventa fijo o mínimo o un nivel de precio fijo o mínimo al que debe ajustarse el comprador. Para la Unión Europea este tipo de conducta es especialmente grave y no tiene exoneración alguna, siendo la práctica a la que más espacio dedican los lineamientos de la Comisión. La lista de potenciales riesgos incluye la colusión (de proveedores o de distribuidores), la reducción de la competencia entre productores o minoristas, el incremento de los precios de los distribui dores y del margen del productor, la exclusión de productores pequeños, así como la reducción en la innovación en distribución.
\end{enumerate}

\hypertarget{fusiones-y-adquisiciones}{%
\section{Fusiones y Adquisiciones}\label{fusiones-y-adquisiciones}}

El tema más controvertido en defensa de la competencia refiere al control de las estructuras de mercado. En el comienzo de las actividades de defensa de la competencia, las legislaciones estudiaban el abuso de posición dominante o los carteles, mientras que el control de las fusiones se introdujo con posterioridad. En EUA, mientras el Acta Sherman es de 1890, la primera norma de control de estructuras de mercado, la \href{https://wps.prenhall.com/wps/media/objects/751/769950/Documents_Library/clayton.htm}{Ley Clayton}, data de 1914. Por su parte, en Europa hay un movimiento similar ya que si bien las normas de control de las conductas de las empresas se establecen con el Acuerdo Constitutivo de la Comunidad Económica Europea en 1962, el control de las fusiones comienza en el 2004 con la \href{https://eur-lex.europa.eu/legal-content/ES/TXT/PDF/?uri=CELEX:32004R0139\&from=ES}{Regulación No.~139/2004}. Es decir, en general los países comenzaron con normas que regulan conductas de los agentes y con posterioridad pasaron a aplicar normas que regulan la estructura del mercado.

Sin embargo, este control varía entre países. Por ejemplo, mientras que las legislaciones de competencia en el mundo regulan la fusión de una empresa, en EUA se han dado casos en los cuales el Departamento de Justicia ordena la separación de empresas, lo que resulta impensable en otros países.

El principal problema del análisis de fusiones es que se realiza antes de que la fusión se concrete e implica proyectar una situación futura del mercado, lo que resulta complejo y no está exento de controversias. Una vez consumada o denegada una fusión, múltiples elementos inciden en su resultado, el que no puede ser atribuido completamente al accionar ni de las empresas ni de los organismos de aplicación de las normas de competencia. Realizar un análisis prospectivo de la situación de mercado, en caso de que una fusión se materialice, es costoso
y complejo. Corresponde señalar también que los casos de denegatorias de fusión son la excepción más que la regla, aunque muchas veces las autoridades de competencia condicionan la misma a que las empresas desinviertan activos.

Las fusiones pueden ser de tres tipos. Las fusiones son \textbf{horizontales}, cuando las partes son empresas competidoras. Esta fusión transforma a dos empresas competidoras en una única empresa. Ello tiene impactos positivos sobre la eficiencia toda vez que permita reducir los costos de producción (ya sea mejor utilización de activos fijos o economías de escala), pero también puede facilitar la colusión o prácticas explotativas.

Estas fusiones no son idénticas a una colusión. Cuando dos empresas se transforman en una, pueden racionalizar el uso del personal y de las instalaciones, lo que no sucede cuando son unidades separadas. Por tanto, las fusiones horizontales pueden generar mejoras de eficiencia. Estas eficiencias generan la necesidad de balancear los intereses de productores y consumidores.

\citet{Williamson1968} plantea esta dicotomía de intereses a través de un simple ejemplo: una fusión permite reducir los costos de las empresas, lo que aumenta el bienestar social, pero a la vez incrementa el precio a los consumidores, lo que disminuye el bienestar social. Si, como resultado, el excedente total se incrementa, esta medida va en sentido contrario al excedente del consumidor que disminuye, producto del aumento de precio. Por tanto, si se adopta el excedente del consumidor como medida del bienestar, no se puede aceptar ninguna fusión que incremente precios, independientemente de las reducciones de costos; mientras que si se adopta el excedente total, fusiones que aumenten los precios pueden ser aprobadas. Es decir, la fusión produce un problema distributivo.

En general los órganos de defensa de la competencia concentran su interés en determinar si el resultado final de la fusión será un incremento de los precios, mientras que las eficiencias de costo son de segundo orden. Si producto de una fusión se puede predecir un aumento de precios, entonces aquella nunca será aprobada. Esta vinculación entre precios y eficiencias también se obtiene en los principales modelos económicos que analizan las fusiones. Estos demuestran que las eficiencias de costo ---reducción de costos variables--- tienen que ser muy importantes para que se trasladen a precio. Ello, en general, perjudica a las empresas que no integran la fusión, por lo que buscarán oponerse a la misma.

Por otra parte, la fusión de empresas puede impactar también sobre la posibilidad de que las empresas formen un cartel ex post. Una fusión ---con eficiencias--- tiene un impacto a priori ambiguo sobre la probabilidad de que las empresas coludan después de que la fusión ocurra. Por un lado, facilita la colusión dado que hay menos empresas en el mercado y ello aumenta la posibilidad de que alcancen un acuerdo. Por otro, si las empresas fusionadas ganan en eficiencia, aumenta la asimetría entre las empresas del mercado lo que dificulta la colusión. Se puede demostrar que la probabilidad de colusión aumenta si las eficiencias no son muy importantes, es decir, si las asimetrías no son grandes.

En resumen, si las ganancias de eficiencia entre las empresas fusionadas no son importantes, aumenta la probabilidad tanto de que la nueva empresa ejerza comportamientos explotativos ---prácticas unilaterales--- como que se conformen carteles en el mercado ---prácticas concertadas.

Las fusiones \textbf{verticales} involucran empresas en distintos eslabones de la cadena de valor. Son una alternativa a las restricciones, por tanto tienen los mismos riesgos y beneficios. Las fusiones verticales, a diferencia de las horizontales, buscan en general resolver problemas de eficiencia como los mencionados en la sección \ref{eficiencia}. Los problemas de doble margen, los incentivos a la inversión y el problema de parasitismo se resuelven si las empresas se integran.

Pero estas fusiones pueden servir también a fines anticompetitivos, si tienen como objetivo resolver el problema de compromiso o llevan a la exclusión de competidores. El problema de compromiso surge cuando la empresa no puede comprometerse a no arbitrar a sus distribuidores, lo que lleva a que la competencia entre ellos reduzca los beneficios extra normales que puede obtener. Si el productor se fusiona resuelve en forma sencilla este problema, ya que ahora hay un único distribuidor y puede fijar el precio que crea pertinente al consumidor. La exclusión de competidores se da cuando la fusión entre una empresa productora y su proveedor obliga a un competidor de la empresa productora a obtener sus insumos con proveedores más ineficientes, lo que incrementa sus costos y la hace menos competitiva.

Por último, las fusiones de \textbf{conglomerado} involucran empresas que participan en mercados que no están vinculados entre sí, como puede ser un productor de alimentos y otro de limpieza. Estas fusiones permiten mejorar el poder de negociación de las empresas, ya que incorpora productos a su portafolio. Pueden utilizarse para atar productos o cerrar el mercado, como cuando los productores arman canastas de productos y mejoran su posición negociadora con los distribuidores. En particular, se verifican los problemas asociados a las canastas: (i) extender el monopolio a otros mercados; (ii) cerrar el mercado a rivales; e (iii) inducir exclusividades o servir para mejorar descuentos.

Para analizar las fusiones es clave conocer el mercado relevante. En la medida en que este análisis prevé predecir el escenario competitivo con posterioridad a la fusión, hay que entender claramente el mercado y los agentes que participan en él. Excepcionalmente, a veces no es necesario un análisis detallado del mercado si se cuenta con información y un análisis estadístico adecuado para evaluar el impacto de la fusión ex ante.

Un caso paradigmático fue el análisis de la fusión entre dos distribuidores minoristas de insumos para oficina Staples y su rival Home Depot, realizado por el Departamento de Justicia y la Comisión Federal de Comercio (FTC) de EUA (\citet{Baker1999}). La FTC observó la fusión en base a un estudio detallado en el que demostró que Staples cobraba precios 9\% mayores en aquellas ciudades donde no competía con Home Depot. Basándose en ese análisis, las empresas abandonaron el intento de fusión, aunque en 2015 estaban intentando fusionarse nuevamente. Lo interesante de este caso es que no fue necesaria una definición explícita del mercado relevante, debido a que el efecto de la fusión pudo predecirse.
\footnote{Aun así, el mercado relevante fue definido como el de venta de insumos de oficina a través de grandes tiendas de artículos de oficina. El mercado relevante geográfico fue definido como el área metropolitana de la ciudad.}

Lo que sí resulta claro es que el análisis de las fusiones es complejo y costoso para las empresas y para las agencias de competencia. Por ello, parece prudente afinar el interés en aquellas fusiones que pueden tener un impacto sobre el mercado, es decir, las que involucran a empresas dominantes.

En muchos países, como en Uruguay, existe un doble umbral para el análisis de fusiones: por monto y por porcentaje de mercado. Si la fusión involucra a dos empresas cuya facturación excede un determinado umbral, entonces deben solicitar autorización previa. Lo mismo ocurre si las empresas pasan un determinado umbral de mercado relevante. Este último indicador, sin embargo, se ha comenzado a abandonar en la medida en que no resulta objetivo para las empresas. En efecto, deben esperar a que el órgano de aplicación decida cuál es el mercado relevante para conocer si pasan o no el umbral establecido.

Por otra parte, los umbrales de facturación son objetivos para las empresas, aunque pueden permitir que se fusionen empresas de un tamaño pequeño respecto a la economía, pero relevante en su mercado. Algunos países han redactado lineamientos para las fusiones, entre los que destacan las \href{https://www.ftc.gov/sites/default/files/attachments/merger-review/100819hmg.pdf}{\emph{Horizontal Merger Guidelines (2010)}} de EUA, las \href{https://eur-lex.europa.eu/LexUriServ/LexUriServ.do?uri=OJ:C:2008:265:0006:0025:en:PDF}{\emph{Non Horizontal Merger Guidelines (2008)}} y las \href{https://eur-lex.europa.eu/legal-content/EN/TXT/PDF/?uri=CELEX:52004XC0205(02)\&from=EN}{\emph{Horizontal Merger Guidelines (2004)}} de la Unión Europea.

Las \emph{Merger Guidelines} en EUA establecen umbrales seguros por debajo de los cuales la fusión no será investigada, utilizando para ello el \emph{HHI}. En particular, si las empresas cumplen los siguientes requisitos la fusión se aprueba, en caso contrario se analiza: (i) \(HHI \leq 1.000\); (ii) \(1.000 < HHI \leq 1.800\) y \(\bigtriangleup HHI \leq 100\); y (iii) \(HHI > 1.800\) y \(\bigtriangleup HHI \leq 50\).

Un régimen de análisis particular tienen las fusiones donde alguna de las empresas involucradas se encuentra en quiebra. Las normas de competencia establecen algunas reglas para este tipo de fusión. La comparación relevante en este caso es entre la fusión de las empresas y la situación en la que la empresa sale del mercado. En general, las reglas que se siguen para que se acepte una fusión de este tipo son: (i) la empresa en quiebra no puede cumplir sus próximas obligaciones; (ii) no puede reorganizarse en forma exitosa bajo las normas de bancarrota; (iii) no existen compradores alternativos que mantengan la empresa en funcionamiento; y (iv) si la fusión no se produce, la empresa sale del mercado. Si estos cuatro pasos se cumplen, entonces podría aprobarse una fusión que involucre a una empresa competidora en quiebra.

Quizá el punto más controvertido en el análisis de fusiones es el de las eficiencias de las empresas fusionadas. Si existen importantes ganancias de eficiencia, entonces será beneficioso desde el punto de vista social que se fusionen dos empresas competidoras. Las eficiencias de este tipo de fusiones pueden involucrar economías de escala o de alcance, sinergias en I+D, racionalización de la distribución y el marketing, entre otros.

Sin embargo, las fusiones horizontales que reducen costos fijos, si bien mejoran el bienestar social ---eficiencia productiva--- no se traducen en reducciones de precio, ya que en su determinación sólo se toman en cuenta los costos variables. Por tanto, una reducción de costos fijos, aunque beneficioso para las empresas, no servirá para que una fusión sea aprobada.

El resultado del análisis de las fusiones no tiene porqué implicar aceptación o rechazo. Algunas veces los órganos de competencia establecen condiciones para aprobar las fusiones, si se detectan efectos sobre el bienestar. Estas condiciones son de dos tipos. Por un lado, se pueden establecer condiciones estructurales que establecen que las empresas deben desinvertir determinados activos, como ser marcas, productos, plantas; o deben permitir el acceso a cierta infraestructura. El objetivo es permitir que competidores nuevos o existentes puedan utilizar esos activos para competir de mejor forma con la empresa fusionada.

Sin embargo, estas condiciones enfrentan dos tipos de problemas. Primero, la empresa vendedora tiene incentivos a engañar al comprador respecto a la importancia relativa del activo, o a venderlos a empresas fantasmas u otras que no sean competidoras directas. Segundo, si los activos son transferidos a empresas competidoras, ello restablece la simetría en el mercado y aumenta la posibilidad de colusión ex post de la fusión. Por otro lado, se pueden establecer condiciones al comportamiento de las empresas fusionadas que restringen los derechos de propiedad sobre el uso de los activos, como las patentes.
Estas condiciones también pueden restringir la publicidad de los productos, limitar la posibilidad de realizar canastas entre productos, o limitar las restricciones de no competencia a los funcionarios desplazados de las empresas. 25
\footnote{Cuando ocurren fusiones y algunos funcionarios de alto rango son cesados, las compensaciones que reciben incluyen como contrapartida cláusulas que limitan la posibilidad de trabajar para empresas competidoras o crear otras que compitan con la empresa que los despide.}

En otros casos, se le puede requerir a la empresa fusionada que reporte sus precios de venta por un tiempo, de forma de monitorear la evolución del mercado. En cualquier caso, las soluciones de comportamiento trasforman a las autoridades de competencia en un regulador, tarea que es difícil y compleja de llevar a cabo.

En conclusión, el análisis de fusiones es un trabajo complejo. Requiere un análisis detallado del mercado y su competencia. Hay que predecir el comportamiento futuro de los precios, si las fusiones son horizontales; que se produzca un cierre del mercado a competidores, si son verticales; o que se generen portafolios que puedan ser utilizados para bloquear el acceso a empresas rivales, si son de conglomerado. También hay que ponderar adecuadamente las posibles eficiencias que la fusión genere. Por último, si la fusión tiene efectos positivos y negativos sobre el mercado, deben diseñarse condicionamientos que mitiguen los efectos negativos, a la vez que permitan aprovechar los positivos.

\hypertarget{reflexiones-finales}{%
\section{Reflexiones Finales}\label{reflexiones-finales}}

La esencia de una economía de mercado es que las empresas compitan entre sí por los clientes. Cuando ello ocurre, los productos mejoran, se hacen más accesibles y las empresas introducen nuevos bienes y servicios que permiten mejorar el bienestar de los consumidores. En este marco, la recompensa es la fidelidad de los clientes, que se traduce en la rentabilidad y tamaño relativo otras estrategias que les permite consolidar o mantener un poder de mercado del que no gozarían en otro caso. Asimismo, estas conductas no reportan bene ficios a los consumidores, que pierden variedad, nuevos productos o poder de compra.

El análisis de defensa de la competencia es una herramienta útil para entender la competencia en el mercado. No se requiere que exista una norma de competencia para su aplicación, ya que es una metodología general que sirve para entender dónde se encuentran los principales cuellos de botella competitivos. En particular, el instrumento del mercado relevante es una herramienta interesante para describir los mercados y comprender su funcionamiento. Muchas veces, sirve para poner el foco no sólo en los comportamientos de las empresas, sino también en las regulaciones de los mercados.

Las empresas desarrollan comportamientos que son consistentes con las reglas de juego que reciben, en particular, con las regulaciones existentes o inexistentes. Contar con una metodología para analizar casos de defensa de la competencia, permite focalizar rápidamente una investigación, sirve para concentrar recursos y, en última instancia, establece prioridades.

¿Cómo llevar a cabo una investigación por temas de competencia? Cuatro pasos eliminatorios que ayudan a guiar una investigación de defensa de la competencia. En primer lugar, ¿la conducta denunciada tiene sentido? No siempre las denuncias refieren a problemas de competencia y es importante descartarlos rápidamente. Detenerse a analizarla y comprender la racionalidad económica que explica las conductas analizadas es un primer paso importante que resuelve muchos problemas jurídicos posteriores.

En segundo lugar, ¿el infractor puede o podría tener posición dominante? La respuesta en este caso no debería ser ``claramente sí'', sino ``claramente no''. A priori puede no ser claro si una empresa o un conjunto de empresas tienen posición dominante, pero si puede ser obvio si no la detentan. Si las empresas claramente no tienen posición dominante, entonces el caso debería ser archivado pues continuar con él representa un costo en términos de recursos, pero también de posteriores situaciones incómodas para los órganos de aplicación.

En tercer lugar, ¿qué prueba es necesaria para probar o descartar una conducta anticompetitiva? Recolectar y procesar información es costoso en términos de recursos y tiempo. Un primer paso es buscar antecedentes de otros países que sirven para ahorrar el proceso de comprender el mercado y entender sus particularidades. También es importante poder contar con informantes calificados que aporten información cualitativa sobre el mercado. Si bien estos informantes no tendrán información detallada, son útiles para entender los detalles del de las empresas. Sin embargo, no siempre las empresas tienen los incentivos correctos a competir por sus clientes. Otras veces, buscan confabularse en su contra, desalientan a potenciales rivales de entrar al mercado, o llevan a cabo mercado, cómo operan las empresas y cuáles son los elementos en los que hay que centrar el análisis. Asimismo, estos informantes permiten saber dónde está la información que se requiere para sancionar o no una conducta, o para determinar la posición de la presunta infractora.

Por último, no puede dejar de señalarse que los casos de defensa de la competencia son siempre motivo de escrutinio y presiones políticas. Las empresas involucradas son siempre actores relevantes en sus mercados y tienen a su disposición recursos que a veces el sector público no dispone. Asimismo, pueden ejercer presiones políticas o movilizar a la prensa. Esas son las reglas de juego. Por tanto, cualquier decisión que se tome debe ser considerada como un antecedente de posibles futuros casos. Sean los antecedentes fuente o no de jurisprudencia, mantener una línea coherente ayuda a percibir la fortaleza de los órganos de aplicación.

\hypertarget{instituciones-reg}{%
\chapter{Instituciones regulatorias para economías en desarrollo}\label{instituciones-reg}}

Jorge Ponce

En el pasado reciente diversos países de América Latina, así como también en otros continentes, han llevado adelante diversos procesos de reforma en los mercados de servicios públicos tales como telecomunicaciones, energía eléctrica, agua y saneamiento. El éxito de estos procesos ha sido variado. Diversas razones que hacen a las diferencias en el entorno institucional, los procesos políticos e incluso la historia concreta, los diferentes puntos de partida y el estadio de desarrollo de los países, podrían determinar el diferencial en los resultados.

Por ejemplo, \citet{Bergara2005} argumentan que la política relativa a servicios públicos en Argentina ha sido altamente volátil y llevada adelante sin el desarrollo de una institucionalidad regulatoria adecuada. Como resultado, identifican situaciones extremas que van desde la captura de los reguladores por parte de la industria a la expropiación de las firmas privadas por parte del gobierno. Esto, en un entorno institucional signado por la casi imposibilidad de recurrir al sistema judicial para resolver controversias y la rápida limitación de los poderes de las agencias regulatorias por parte del sistema político, por ejemplo restringiendo sus cometidos, capacidad técnica y recursos financieros. \citet{Spiller2003} también argumentan que los mecanismos para hacer cumplir los compromisos han sido extremadamente débiles y que los ``servidores públicos'' no visualizaban objetivos de largo plazo, alentando el incumplimiento de las normas y la corrupción.

En tanto, las reformas llevadas adelante en Chile obtuvieron resultados más estables al desarrollarse en un marco de consistencia institucional. La relativa fortaleza institucional posibilitó el establecimiento de sistemas de regulación por precio límite, fijado para cumplir con una meta de rentabilidad sobre costos y capital de una empresa modelo eficiente. Este tipo de regulación determinala provisión de incentivos de alto poder a las firmas y ha operado en un marco de estrecha coordinación para mantener la eficiencia en el uso de los fondos públicos. Además, la relativa fortaleza institucional de los reguladores se ha basado en su capacidad técnica, generada por contar con recursos humanos y económicos considerablemente mayores que en el resto de la región.

Estos ejemplos muestran que los países en desarrollo presentan particularidades que, de una manera u otra, determinan la posibilidad de realizar determinado tipo de reformas, así como también sus resultados, una vez que las mismas son llevadas adelante. La práctica ha demostrado entonces que las teorías regulatorias tradicionales, elaboradas para ser aplicadas en países desarrollados, son de una aplicabilidad mucho más limitada en países menos desarrollados e incluso pueden aparejar graves consecuencias en términos de bienestar social. De esta forma, entender el contexto institucional y sus implicaciones resulta de crucial importancia para el diseño de las instituciones regulatorias en economías en desarrollo.

Este capítulo considera algunas de las principales limitaciones institucionales que ocurren con frecuencia en economías en desarrollo y analiza las dificultades que las mismas imponen para el diseño de la regulación, así como las posibles soluciones para paliar esas dificultades. A los efectos de tener un punto de referencia a partir del cual comparar los hallazgos, se introduce (en la sección \ref{modelo}) un modelo simple y comúnmente utilizado para analizar la regulación de actividades económicas en economías desarrolladas. A partir del mismo, se extiende la contribución seminal de \citet{Laffont2005} para analizar las implicaciones de considerar aspectos distintivos de las economías menos desarrolladas. En particular, la sección \ref{desafios} estará dedicada a explorar cómo el modelo puede ser adaptado para analizar algunas limitaciones institucionales comunes a estas economías: baja capacidad técnica para la regulación y la supervisión, baja rendición de cuentas, compromiso limitado y falta de credibilidad, baja eficiencia fiscal, corrupción, baja capacidad de hacer cumplir las normas y alta posibilidad de captura del regulador.

El capítulo provee entonces un panorama general de los principales aspectos a considerar a la hora de diseñar instituciones regulatorias en economías en desarrollo y de los posibles caminos a seguir para obtener resultados que beneficien a la sociedad en su conjunto. En los comentarios finales, ofrecidos en la sección \ref{coment-fin}, se resumen las principales características que la regulación eficiente debería mostrar según los diferentes estadios de desarrollo de las economías.

Antes de entrar de lleno al análisis del diseño de las instituciones regulatorias para economías en desarrollo se provee, en la siguiente sección, una breve introducción a las razones, los objetivos y las dificultades de la regulación de la actividad económica.

\hypertarget{regulaciuxf3n-razones-objetivos-y-dificultades}{%
\section{Regulación: razones, objetivos y dificultades}\label{regulaciuxf3n-razones-objetivos-y-dificultades}}

¿Qué es regulación? Regulación generalmente refiere a un conjunto específico de reglas aplicadas a determinado conjunto de agentes con un propósito. Más en general, de acuerdo a \citet{Baldwin2011}, \textbf{regulación} es cualquier influencia deliberada por parte del Estado, como por ejemplo a través de los sistemas impositivos, de franquicias, etc. Es interesante distinguir el concepto de regulación del concepto íntimamente relacionado de supervisión. Mientras que regulación refiere al conjunto específico de reglas y normas que restringen el accionar y proporcionan incentivos (guían) a los agentes en determinados mercados, la \textbf{supervisión} refiere a la observación del cumplimiento de la regulación. Por tanto, la supervisión representa un importante esfuerzo para recolectar información y para hacer cumplir las normas.

Pero, ¿por qué es necesaria la regulación y la supervisión de determinadas actividades económicas? ¿Cuál es su objetivo? En términos generales, la regulación y supervisión se justifican por la existencia de fallas de mercado. Esto es, situaciones en las que el funcionamiento libre de los mercados no es suficiente para obtener los resultados o comportamientos buscados por el interés público.

En otros términos, si el funcionamiento del mercado proveyera resultados adecuados desde el punto de vista del bienestar social no sería necesario introducir regulación alguna. A modo de ejemplo, varios procesos de reestructura en la provisión de servicios públicos tales como electricidad, gas o agua potable generalmente determinan la existencia de un proveedor monopólico operando en el sector privado.

En ausencia de intervención pública, bajo la forma de regulación y supervisión del mercado, el monopolista restringirá la cantidad (y posiblemente la calidad) provista y cargará un precio excesivamente alto a los consumidores. Incluso sus decisiones de inversión podrían determinar que importantes sectores de la población, por ejemplo aquellas áreas de menor densidad poblacional, no puedan acceder a los servicios aún si estuvieran dispuestos a pagar el precio solicitado. Entonces, algún tipo de regulación se hace necesaria para balancear el bienestar de los consumidores y del monopolista, a los efectos de alcanzar un nivel superior de bienestar social.

Existe un acuerdo generalizado, tanto en la industria como en la academia, en cuanto a que el objetivo de la regulación de servicios públicos es promover la eficiencia en la provisión de los bienes y servicios regulados, tanto desde un punto de vista estático como dinámico (esto es promoviendo los procesos de inversión que posibiliten reducir costos en el futuro). Además, al tiempo de promover la eficiencia, la regulación debe proteger a los consumidores del potencial abuso de la posición dominante de la firma monopólica, así como a los inversores y a los operadores de los servicios de la potencial influencia y oportunismo del sistema político.

Pero, ¿por qué los consumidores no son capaces de implementar y hacer cumplir algún tipo de regulación directamente? Además de una razón evidente que es el diferente poder de negociación de consumidores y firma monopólica (muchas veces explicada a su vez por la diferente información a la que unos y otros acceden), existen otras razones tales como lo que en la jerga económica se conoce como problema de polizón en la provisión de bienes públicos. Esto es, el costo privado de cada consumidor de llevar adelante la regulación superará ampliamente el beneficio del cual el consumidor se podrá apropiar debido a que los otros consumidores, aun no habiendo incurrido el costo de regular, disfrutarán de una parte de los beneficios generados por la regulación. De esta manera, la regulación sólo podrá ser llevada adelante si alguna institución regulatoria, actuando en nombre de los consumidores, colectiviza tanto los costos como los beneficios generados por la introducción de la regulación.

Entonces, tan importantes como las reglas son las estructuras y los procesos en los que se sustenta la regulación. En otros términos, tan importante como el qué regulación introducir son el quién la llevará adelante y el cómo se desarrollará el proceso regulatorio. A modo de ejemplo, las estructuras que podrían afectar el resultado de la regulación incluyen la distribución de los poderes entre diferentes agencias para desarrollar la regulación, los objetivos de estas agencias y los procedimientos de votación en ellas utilizados.

En cuanto a los procesos, aspectos tales como el momento en el que se produce la intervención pública, el poder del regulador, las dimensiones sobre las que puede actuar y los canales de comunicación al interior de la jerarquía regulatoria, cobran particular importancia. En general, reglas, estructuras y procesos definen las instituciones regulatorias. Del diseño de estas instituciones para la regulación dependerá el resultado final de la misma. A su vez, este diseño deberá contemplar los diferentes costos de transacción que las características del entorno contractual imponen.

En particular, es posible que una regulación que funciona adecuadamente en un país no brinde los resultados esperados en otro. Al análisis de estas diferencias, en función del grado de desarrollo de los países y de sus implicaciones para el diseño de instituciones regulatorias, estará dedicado el resto de este capítulo. Antes de proseguir con la introducción del modelo básico de regulación que servirá de guía para el análisis subsiguiente, brindaremos una exposición conceptual de las principales dificultades que, en general, el diseño de las instituciones regulatorias enfrenta.

Una vez que los consumidores delegan en una agencia regulatoria el control de los aspectos regulatorios sobre una determinada industria aparecen una serie de dificultades. En primer lugar, el control que un regulador puede ejercer sobre una firma determinada es sólo imperfecto por al menos dos razones. Primero, debido a la asimetría en la información que maneja la firma con respecto al regulador. A menudo el regulador desconoce la tecnología exacta de producción, sus costos o la elasticidad de la demanda que la firma enfrenta y tiene escaso conocimiento de la estructura de incentivos interna de la firma o de los contratos que pueda tener con sus proveedores.

Segundo, si bien a menudo los reguladores tiene pocos cometidos a cumplir, también tienen relativamente pocos instrumentos para llevar adelante sus funciones. Más imperfecciones al proceso de control se agregan una vez que se tiene en cuenta que los reguladores no rinden cuenta directamente a los consumidores sino al sistema político (ya sea a través de la ramas Ejecutiva o Legislativa). Generalmente, los políticos están en desventaja informacional con respecto a los reguladores, no sólo por su acceso diferencial a la información relevante, sino también por su relativa inexperiencia a la hora de considerar aspectos que pueden ser extremadamente técnicos. Además, los propios políticos difícilmente reciban los incentivos correctos de los consumidores para actuar en pos del interés social, debido al mismo problema del polizón que impide a los consumidores controlar directamente a la firma.

En cambio, los consumidores utilizan el sistema de votación, que resulta ser imperfecto, a la hora de garantizar la
rendición de cuentas de los políticos. De esta manera, cada una de las capas que conforman la estructura regulatoria puede ser vista como una relación de principal-agente con sus propios problemas de información asimétrica. En este sentido, el regulador (el principal) posee menos información que la firma (su agente), el sistema político posee menos información que el regulador que le debe rendir cuentas y, finalmente, los consumidores sólo pueden observar imperfectamente el esfuerzo y la eficiencia de los políticos que han votado.

El hecho que el principal tenga menos información que su agente determina una serie de implicaciones, dentro de las cuales la más importante es que el principal requiere encontrar una solución de compromiso entre la eficiencia, en términos económicos, y la extracción de rentas por parte de la firma. Por ejemplo, un regulador puede no observar cuál es el verdadero costo de producción de la firma (un problema conocido en la literatura económica como de \textbf{selección adversa}) o, una vez que la firma está operando, el nivel de esfuerzo realizado por el empresario puede no ser verificable por parte del gobierno (un problema de \textbf{riesgo moral}) con lo cual se verá obligado a brindarle alguna pago para que revele su información (una \textbf{renta informacional}) y pueda lograrse una asignación eficiente de recursos.

Por lo general, este compromiso entre rentas de la información y eficiencia en la producción se logra a través de mecanismos de \textbf{incentivos} que se encuentran implícitos en \textbf{contratos} entre el principal y el agente. En el caso de la relación entre el regulador y la firma, estos contratos toman la forma de regulaciones específicas. En lo que sigue se analizará el diseño de estos contratos.

\hypertarget{modelo}{%
\section{Un modelo básico de regulación}\label{modelo}}

En esta sección se introduce un modelo canónico, muy similar al utilizado por \citet{Laffont2005} para el análisis de la regulación en economías desarrolladas. La metodología y los resultados de este modelo servirán de punto de referencia para, en la siguiente sección, analizar los desafíos que las características de las economías en desarrollo imponen para el diseño de la regulación y las posibles soluciones sugeridas por la teoría.

En la presentación utilizaremos expresiones matemáticas comúnmente utilizadas en la literatura sobre el tema. Si bien a los lectores más familiarizados con este tipo de modelos estas expresiones les facilitarán la lectura, a aquellos lectores menos familiarizados puede dificultársela. Hemos realizado un esfuerzo para que los conceptos puedan ser igualmente comprendidos sin reparar en tales expresiones matemáticas, por lo cual sugerimos a aquellos lectores menos familiarizados pasarlas rápidamente en una primera lectura.

\hypertarget{principales-elementos-del-modelo}{%
\subsection{Principales elementos del modelo}\label{principales-elementos-del-modelo}}

El modelo considera: (i) consumidores; (ii) una firma, monopolio natural, a cargo de la provisión de un servicio público tal como agua, electricidad, telecomunicaciones o transporte; (iii) un gobierno benevolente; y (iv) un regulador y supervisor del mercado.

\begin{enumerate}
\def\labelenumi{\arabic{enumi}.}
\item
  \textbf{Consumidores}. Los consumidores obtienen utilidad al consumir el producto ofrecido por la firma. Cuando consumen una cantidad q de la producción del monopolio, el nivel de utilidad obtenido por los consumidores estará definido por \(S(q)\) (creciente pero con utilidad marginal decreciente: \(S′ > 0\) y \(S′′ < 0\)). La demanda está representada por \(P(q)\) (esto es, una relación decreciente entre el precio y la cantidad demandada), de tal forma que el bienestar neto de los consumidores por el consumo del producto ofrecido por el monopolio es: \(S(q) – qP(q)\), donde \(qP(q)\) es el monto pagado a la firma por una cantidad \(q\).
  Los consumidores, además, pagan impuestos al gobierno. La recolección de impuestos está sujeta a ciertas ineficiencias y costos de transacción que determinan que la recolección de una cantidad \(t\) de impuestos cueste a los consumidores \((1 + λ)t\), donde \(λ > 0\) es el costo de oportunidad de los fondos públicos. Entonces, el bienestar neto de los consumidores por todas sus actividades será:
  \[V = S(q) – qP(q) – (1 + λ)t\]
\item
  \textbf{Firma e información}. La producción tiene costos que están representados por la siguiente función: \[C(q) = (β – e)q + K\]
  El costo fijo de producción es K , mientras que el costo marginal es \(c = β – e\). La asimetría de información entre la firma y los consumidores (así como también con el gobierno que los representa) puede ser fácilmente mode lada a través de este último término. En particular, mientras que el costo fijo es conocido por todas las partes y el costo marginal \((c = β – e)\) puede ser observado ex post por el gobierno, los componentes del costo marginal son información privada de la firma.
  Más precisamente, \(β\) representa las características específicas de la firma que están fuera del control directo del empresario: la firma puede ser eficiente en su producción \((β = \underline β\)) o in eficiente \((β = \overline{β})\). Dado que el gobierno no observa \(β\), enfrenta un problema de \textbf{selección adversa}. La firma conoce el valor de \(β\) , pero el gobierno sólo conoce la probabilidad con la cual una firma es eficiente \(ν = Pr (β = \underline β\)) .
  Entonces, una firma eficiente puede querer hacerse pasar por una ineficiente para conseguir mayores transferencias por parte del gobierno. En tanto, e es el término del costo marginal que es directamente controlable por el empresario, representa su nivel de esfuerzo. Para mantener la simplicidad se asume que este esfuerzo reduce uno a uno el costo marginal de la firma.
  Esforzarse implica una desutilidad para el empresario \(ψ(e)\), la que es creciente \((ψ′ > 0)\) y con tasa marginal de desutilidad también creciente \((ψ′′ > 0)\). Dado que el gobierno no puede observar el nivel de esfuerzo del empresario, este parámetro \(e\) representa el \textbf{riesgo moral}: luego de obtenida cierta transferencia por parte del gobierno el empresario podría preferir no esforzarse.
  La firma recibe además una transferencia t del gobierno, la que es financiada con impuestos a los consumidores. De esta forma, el beneficio neto de la firma es:
  \[U = qP(q) – (β – e)q – K – ψ(e) + t\]
  En este punto realizaremos dos supuestos. Primero, que la firma puede decidir no producir una vez que ha conocido su tipo (esto es, el componente \(β\) de su costo marginal). En otras palabras, esto impone una restricción de participación para la firma tal que \(\overline U ≥ 0\) y \(\underline U ≥ 0\), donde \(\overline U\) es la utilidad de la firma cuando \(\beta = \overline \beta\) y \(\underline U\) es la utilidad de la firma cuando \(U = \underline U\).
  \footnote{A lo largo del texto las variables con una línea superpuesta corresponderán a las de una firma ineficiente, de costo \(β = \overline β\), mientras que las variables con una línea subyacente corresponderán a una eficiente, de costo \(β = \underline β\).}
  De esta manera, el gobierno estará restringido en sus instrumentos para motivar a la firma ya que no podrá ofrecer contratos que le brinden una utilidad negativa.
  Segundo, asumiremos que el gobierno desea que tanto la firma del tipo eficiente como la del tipo ineficiente participen. Recuerde que la firma es en realidad una sola --un monopolio-- pero que su tipo o costo puede ser alto (\(\beta = \overline \beta\), firma ineficiente) o bajo (\(β = \underline β\), firma eficiente). Este supuesto implica que la provisión, aún ineficiente, del servicio público es preferida a su no provisión.
\item
  \textbf{Gobierno}. El gobierno es benevolente y desea maximizar el bienestar agregado de toda la sociedad,
  \footnote{Este supuesto de gobierno benevolente implica dejar de lado, por el momento, los posibles problemas de agencia entre consumidores y sistema político. Más precisamente, bajo este supuesto los objetivos del gobierno están perfectamente alineados con los de los consumidores. En este sentido, no hay problemas de principal-agente entre consumidores ya que los políticos representan fielmente sus intereses.}
  esto es de consumidores y la firma monopólica:
  \[W = U + V = S(q) – (β – e)q – K – ψ(e) – λt\]
  Para ello el gobierno utiliza contratos con la firma que estipulan el costo marginal \((c)\), la cantidad del producto \((q)\) y la transferencia que recibirá la firma \((t)\). Más precisamente, el gobierno puede utilizar un menú de contratos diseñados para cada tipo de firma: \((\underline c, \underline q, \underline t )\) para la firma eficiente y para \((\overline c, \overline q, \overline t)\) para la firma ineficiente.
\item
  \textbf{Regulador/Supervisor}. El gobierno utiliza un regulador y supervisor del mercado para reducir su asimetría de información con respecto a la firma. El supervisor puede hacer conocer al gobierno el verdadero valor del parámetro de costo de \(β\) de la firma. Más precisamente, el supervisor tiene acceso a una tecnología que le permite obtener una señal \(r\) que puede ser informativa sobre el parámetro de costo de la firma: con probabilidad \(ξ\) el supervisor conoce el verdadero costo de la firma \((r = β)\) y con probabilidad \(1 – ξ\) no recibe información alguna.
  El gobierno enfrenta un problema de asimetría de información con el supervisor similar al que tiene con la firma: el supervisor puede preferir ocultar la información que ha obtenido y necesita recibir incentivos para revelarla. Entonces, el gobierno tiene que elaborar un contrato con el supervisor estipulando la transferencia a ser pagada al supervisor como una función de la información por este proporcionada. Esta transferencia es financiada con impuestos de forma que una transferencia de un importe \(s\) cuesta a los consumidores \((1 + λ)s\). Entonces, cuando se incorpora el ingreso del supervisor al bienestar general el costo social de la transferencia al supervisor λs tiene que ser considerado:
  \[W = S(q) – (β – e)q – K – ψ(e) – λt – λs\]
  Que el supervisor pueda ocultar la información y que la firma observe tanto la señal que ha obtenido el supervisor como la información que este suministra al gobierno, abren la posibilidad de que la firma intente capturar al supervisor para que este mantenga desinformado al gobierno.
  En la práctica, son comunes las situaciones en las cuales puede ser de interés de una firma capturar a su supervisor. Por ejemplo, un artículo que apareciera en la prensa financiera daba cuenta que al recibir nuevos poderes para regular el mercado de productos derivativos en los Estados Unidos de América,
  \footnote{Fuente: Bloomberg, 13 de octubre de 2010: \href{http://www.bloomberg.com/news/articles/2010-10-14/wall-street-lobbyists-besiege-cftc-to-influence-regulations-on-derivatives}{``Wall Street Lobbyists Besiege CFTC to Shape Derivatives Rules''}.}
  la \emph{Commodity Futures Trading Commission}, que hasta entonces era una ``pequeña y dormida agencia'' (tal como su entonces Presidente, Mary Schapiro, la nombrara en los años noventa) se convirtió en uno de los puntos más concurridos por los cabilderos de la industria.
  Los mecanismos a través de los cuales una firma puede intentar capturar al supervisor pueden ser tan variados como las razones para la captura: desde ``simples regalos'', promesas de futuro empleo en la industria, sobornos, presiones políticas y hasta amenazas de ser conducido a juicio por las acciones que pudieran perjudicar los intereses de la firma.
  Dado que es imposible introducir en nuestro modelo esta variedad de mecanismos, realizaremos el supuesto que el resultado de los mismos puede ser representado por el valor de una transferencia monetaria. Por ejemplo, la promesa de un empleo en la firma puede ser representada por el valor presente de las compensaciones a recibir, mientras
  que la amenaza de ser llevado a juicio puede ser representada por el valor presente de los costos judiciales que el supervisor evitaría al ser indulgente con la firma. Dadas las características de las transferencias en las cuales se resume la captura del supervisor por parte de la firma (por ejemplo por su baja transparencia e incluso potencial ilegalidad), asumiremos que el supervisor sólo accede a una fracción \(k ε (0,1)\) de la misma.
\end{enumerate}

\hypertarget{puntos-de-referencia-bajo-un-marco-institucional-completo}{%
\subsection{Puntos de referencia bajo un marco institucional completo}\label{puntos-de-referencia-bajo-un-marco-institucional-completo}}

En la siguiente sección se discutirán los desafíos que la presencia de un marco institucional más débil en economías en desarrollo impone para el diseño de la regulación. En esta sección, se resolverá el modelo sin considerar tales debilidades, a los efectos de proporcionar un punto de referencia con respecto al cual evaluar los resultados de la próxima sección.

Consideraremos diferentes situaciones: (a) cuando el supervisor informa el valor de \(β\); (b) cuando el supervisor no reporta el valor de \(β\); (c) cuándo tiene incentivo el supervisor para reportar la información; (d) cómo puede el gobierno minimizar el costo de proveer incentivos; y (e) cómo se define el poder de los incentivos.

\hypertarget{cuando-el-supervisor-informa-el-valor-de-beta}{%
\subsubsection{\texorpdfstring{Cuando el supervisor informa el valor de \(\beta\)}{Cuando el supervisor informa el valor de \textbackslash beta}}\label{cuando-el-supervisor-informa-el-valor-de-beta}}

En este caso desaparece la asimetría de información entre la firma y el gobierno. Entonces, para maximizar el bienestar social el gobierno debería fijar una transferencia a la firma (contingente en su tipo ya que es conocido por el gobierno, esto es en el valor de \(β\) de tal forma que no queden rentas excedentarias para la misma \((\underline U = \overline U=0)\). Además, en este caso el nivel de esfuerzo de la firma es eficiente. Esto es: \(ψ′(e) = q\), la utilidad marginal del esfuerzo es igual al costo marginal de esforzarse. Una vez establecidos el valor de la transferencia, \(t\), y el costo marginal de la firma, \(c = β – e\), solo resta la fijación de la cantidad, \(q\), o equivalentemente la fijación del precio \(p\).

La única diferencia entre el problema que enfrenta el gobierno en este caso y el problema de fijación de precio de monopolio, comúnmente estudiado en los cursos introductorios de economía, es que las transferencias son financiadas con impuestos que introducen distorsiones y costos a la sociedad. Por lo tanto, el resultado clásico de fijación de un precio por encima del costo marginal (un \emph{mark up}) inversamente proporcional a la reacción de la demanda ante variaciones en el precio (a la elasticidad de la demanda) se mantiene en este caso, pero el \emph{mark up} toma en cuenta la distorsión introducida por los impuestos (representada por el parámetro \(λ\)). En su expresión matemática, el precio se fija de forma tal que:
\[\frac{p-\left(\beta-e\right)}{p}=\frac{\lambda}{1+\lambda}\times\frac{1}{\eta}\]
donde \(η\) es la elasticidad de la demanda.

\hypertarget{cuando-el-supervisor-no-reporta-el-valor-de-ux3b2}{%
\subsubsection{\texorpdfstring{Cuando el supervisor no reporta el valor de \(β\)}{Cuando el supervisor no reporta el valor de β}}\label{cuando-el-supervisor-no-reporta-el-valor-de-ux3b2}}

En este caso el gobierno desconoce el tipo de la firma. Det odas maneras, el gobierno puede utilizar un menú de contratos diseñados para cada tipo de firmas. Esto es, un mecanismo de revelación directo tal que, dado un mensaje de la firma, se estipula la transferencia, el costo marginal y la cantidad correspondientes: \({(\underline c, \underline q, \underline t), (\overline c, \overline q, \overline t)}\).
\footnote{El principio de revelación garantiza que no hay pérdida de generalidad por considerar mecanismos de revelación directos. Ver \citet{Laffont1993} para una descripción detallada
  de este principio.}
Para que estos contratos sean aceptados por la firma tienen que cumplir con la restricción de participación: \(\overline U \geq 0\) y \(\underline U \geq 0\).

Además, para que cada tipo de firma encuentre óptimo revelar su tipo en forma verdadera, el contrato diseñado para ese tipo tiene que ser al menos tan bueno como el diseñado para el otro tipo de firma. Esto introduce dos restricciones adicionales conocidas como restricciones de compatibilidad de incentivos. Más precisamente, la firma eficiente \((β = \underline β)\) deberá encontrar óptimo revelar su tipo y obtener el contrato \({(\underline c, \underline q, \underline t)}\) que hacerse pasar por una firma ineficiente y obtener el contrato correspondiente \({(\overline c, \overline q, \overline t)}\):

\[
\underline{U}=\underline{q}P\left(q\right)-\left(\underline{\beta}-\underline{e}\right)\underline{q}-K-\psi\left(\underline{e}\right)\geq\overline{q}P\left(\overline{q}\right)-\left(\overline{\beta}-\overline{e}\right)\overline{q}-K-\psi\left(\overline{e}\right)+\overline{t}
\]

o lo que es lo mismo:
\[\underline U \geq \overline U + \triangle \beta \overline q\]
donde \(\triangle \beta = \overline \beta - \underline \beta\)

De la misma manera, la firma ineficiente no debería tener incentivos a hacerse pasar por una firma eficiente:

\[
\overline U \geq \underline U = \triangle \beta \underline q
\]

Un resultado estándar de la teoría de incentivos bajo información asimétrica es que las restricciones que se cumplirán con igualdad son la de participación de la firma ineficiente y la de compatibilidad de incentivos de la firma eficiente. Entonces, la firma ineficiente, al igual que en el caso anterior, no recibirá renta excedentaria alguna: \(U= 0\). En cambio, la asimetría de información implica que la firma eficiente, debido a que puede hacerse pasar por la ineficiente, deba obtener una renta estrictamente positiva para revelar su verdadero tipo: \(U = Δ\beta \overline q\). Esta renta es comúnmente conocida como \textbf{renta de la información}, ya que deriva del hecho que la firma eficiente tiene más información que el gobierno. Es preciso notar que, al ser financiada con impuestos, esta renta de la información tiene un costo \(\lambda \triangle \beta \overline q\) para la sociedad.

\hypertarget{cuuxe1ndo-tiene-el-supervisor-incentivo-a-reportar-la-informaciuxf3n}{%
\subsubsection{¿Cuándo tiene el supervisor incentivo a reportar la información?}\label{cuuxe1ndo-tiene-el-supervisor-incentivo-a-reportar-la-informaciuxf3n}}

La asimetría de información también puede implicar que el supervisor, aun habiendo obtenido la señal de que la firma es eficiente, pueda preferir no reportar esta información al gobierno. Cuando el supervisor oculta la información, la firma eficiente recibe una renta \(U = Δ\beta \overline q\). En cambio, recibiría una renta nula si el supervisor reporta su tipo al gobierno. Por tanto, la firma eficiente estará dispuesta a utilizar hasta la suma de \(Δ\beta \overline q\) para que este no revele su tipo al gobierno. Dadas las características de las transacciones que determinan la captura por parte de la firma, el supervisor enfrentará sólo \(k Δ\beta \overline q\) (recuerde que \(k ε (0,1)\)) del monto utilizado por la firma.

Un resultado directo de la observación anterior es que si el gobierno desea prevenir la captura del supervisor por parte de la firma deberá ofrecer al primero una compensación de por lo menos \(s = k Δ\beta \overline q\) cada vez que este último reporte que la firma es del tipo eficiente \((r = \underline β)\). Es fácil apreciar que el gobierno siempre encontrará óptimo ofrecer esta compensación al supervisor ya que de otra forma el costo social será superior debido a que será necesario otorgar una renta de información a la firma eficiente igual a \(\underline U = Δβ \overline q\). Esta renta es mayor que la compensación requerida por el supervisor para reportar su señal debido a la ineficiencia generada por las transacciones tendientes a lograr la captura: \(\underline U = Δβ \overline q \geq kΔβ \overline q = s\) ya que \(k ε (0,1)\).

Cuando el supervisor obtiene la información que la firma es ineficiente y reporta \(r = \overline β\), la firma no tiene incentivos para que esta información sea ocultada ya que de todas formas recibirá una renta nula. Por tanto, tampoco tendrá incentivos para capturar al supervisor y el gobierno no necesitará compensar a este último por este reporte.

\hypertarget{cuxf3mo-puede-el-gobierno-minimizar-el-costo-de-proveer-incentivos}{%
\subsubsection{¿Cómo puede el gobierno minimizar el costo de proveer incentivos?}\label{cuxf3mo-puede-el-gobierno-minimizar-el-costo-de-proveer-incentivos}}

Cuando el gobierno ignora que la firma es eficiente, la renta que esta obtiene es socialmente costosa debido a que las transferencias son financiadas con impuestos que generan distorsiones. Además, la compensación requerida para que el supervisor reporte su señal al gobierno aumenta con el tamaño de la renta, lo que implica costos sociales adicionales. Por estas razones, el gobierno preferirá reducir la renta de la firma eficiente: \(\underline U = Δβ \overline q\).

Para ello, el gobierno debe hacer menos atractivo para la firma eficiente el nivel de producción de la firma ineficiente (esto es, bajar \(\overline q\)) de forma que se reduzcan los incentivos de la primera para hacerse pasar por la segunda. Esto genera distorsiones para la firma ineficiente ya que estará autorizada a producir menos que el nivel eficiente, con lo cual también se reducirá su nivel de esfuerzo.

\hypertarget{incentivos-de-bajo-y-alto-poder}{%
\subsubsection{Incentivos de bajo y alto poder}\label{incentivos-de-bajo-y-alto-poder}}

Hasta ahora hemos supuesto que el gobierno es capaz de controlar la cantidad ofrecida por la firma (indistintamente el precio), la transferencia y, debido a que puede observar \(c = β – e\), también el costo marginal. De todas maneras, el gobierno podría implementar los contratos óptimos \({(\underline c, \underline q, \underline t), (\overline c, \overline q, \overline t)}\) a través de un mecanismo de incentivos donde el gobierno determine la cantidad y ofrezca a la firma una regla de reembolso de una parte de los costos a través de transferencias. En este caso, el nivel de esfuerzo de la firma (y por tanto su costo marginal) quedaría determinado endógenamente por la misma a los efectos de maximizar su utilidad.

El mecanismo de incentivos propuesto por el gobierno puede ser definido como de \textbf{bajo poder} o de \textbf{alto poder según su grado de impacto}. Un mecanismo será de alto poder si brinda a la firma un fuerte incentivo a reducir sus costos (por ejemplo a esforzarse más) al permitirle la apropiación de los beneficios derivados del aumento en su eficiencia. Por ejemplo, una \textbf{regulación de precio límite}, donde el gobierno fija el precio máximo de venta y una transferencia independiente del costo marginal, permite a la firma apropiarse completamente de cualquier reducción en los costos, con lo cual
la firma realizará un nivel eficiente de esfuerzo. Por el contrario, en un mecanismo de incentivos de bajo poder la firma tendrá menos predisposición a reducir sus costos debido a que ello aparejaría relativamente menores transferencias.

\hypertarget{desafios}{%
\section{Regulación en economías en desarrollo: desafíos y soluciones}\label{desafios}}

En esta sección se discutirán algunas características de las economías en desarrollo y se analizarán las implicaciones para el diseño de instituciones regulatorias. El análisis estará basado en el modelo presentado en la sección anterior.
\footnote{El lector interesado en profundizar en el análisis de los temas contenidos en esta sección puede referirse a los trabajos de \citet{Estache2009} y a la contribución seminal de \citet{Laffont2005}.}

\hypertarget{baja-capacidad-tuxe9cnica-para-la-supervisiuxf3n}{%
\subsection{Baja capacidad técnica para la supervisión}\label{baja-capacidad-tuxe9cnica-para-la-supervisiuxf3n}}

Los sistemas contables y de auditoría de algunos países en desarrollo pueden adolecer de carencias que hace más difícil para el supervisor obtener la información sobre el verdadero tipo de la firma. Además, muchas veces el propio supervisor se ve impedido de penalizar a la firma por no cumplimiento. En otras ocasiones su personal no es suficiente, o no posee la experiencia y el conocimiento requerido, con lo cual la brecha de información entre el supervisor y la firma se ensancha. En términos del modelo anterior, estas características del entorno institucional se traducirían en una menor probabilidad de obtener una señal sobre el tipo de firma: un menor valor de \(ξ\).

Cuando el gobierno está informado con menos frecuencia, la firma eficiente deberá recibir una renta con más frecuencia para revelar su verdadero tipo. Entonces, en términos esperados la firma recibirá mayores rentas. Esto es consistente con la evidencia empírica que indica que en países en desarrollo las firmas reguladas reciben retornos que exceden aquellos que le compensan por los mayores riesgos asumidos.
\footnote{Ver, por ejemplo, \citet{Sirtaine2005} y \citet{Guasch2007}.}
Además, las mayores transferencias son financiadas con impuestos que generan distorsiones, con lo cual el bienestar social será menor.

Llevando la incapacidad técnica al extremo, por ejemplo si los sistemas contables o de auditoría son inexistentes, es necesario alejarse del supuesto bajo el cual hemos venido trabajando de que el costo marginal \(c = β – e\) es observable por parte del supervisor. En este caso el gobierno pierde la posibilidad de con trolar el costo marginal y el esfuerzo de la firma ineficiente para así reducir la renta informacional de la empresa eficiente, como se analizara en la sección anterior. El gobierno sólo puede controlar la cantidad (equivalentemente el precio) y la transferencia. Entonces, podrá usar un mayor \emph{mark up} para reducir la cantidad producida por la firma ineficiente, así reducir su atractivo para la firma eficiente y, por tanto, su renta debido a la información asimétrica. El incrementar el \emph{mark up} de la firma ineficiente es una herramienta menos potente que el control directo del costo marginal y, entonces, también reduce el bienestar social.

Volviendo al caso más común de una capacidad limitada de supervisión, el gobierno podría aumentar el poder de los incentivos para mitigar los problemas introducidos por la menor probabilidad de obtener información de la firma. Debido a que es menos frecuente que el gobierno conozca el verdadero costo de la firma, las compensaciones para que el supervisor revele la información también serán menos frecuentes. Estas compensaciones eran una razón para distorsionar el nivel de producción y el esfuerzo de la firma ineficiente a la baja (recuerde que la compensación al supervisor es una fracción de la renta de información de la firma eficiente, la que a su vez es una función creciente del
nivel de producción de la firma ineficiente).

Como estas compensaciones son ahora menos preocupantes, existen razones para que el nivel de esfuerzo y de producción de la firma ineficiente aumente. Esto puede lograrse aumentando el poder de los incentivos para la firma eficiente de forma que pueda apropiarse de una mayor parte del beneficio ocasionado por la reducción de costos.

Otras medidas propuestas por \citet{Laffont2005} consisten en aumentar la competencia en el mercado y en rediseñar las instituciones de supervisión. Una mayor competencia podría ayudar al supervisor a obtener más información y constituye una presión adicional para mantener bajo el precio de la firma. De todas formas, la regulación de mercados parcialmente competitivos no está exenta de problemas. El supervisor debería ahora reducir el poder de mercado y regular los precios de acceso (esto es, por ejemplo, los precios para acceder a una red de transmisión eléctrica).

Por otra parte, la escasez de recursos técnicos calificados puede ser paliada mediante el rediseño de las instituciones de supervisión. Por ejemplo, a través de la creación de agencias regulatorias unificadas que permitan un mejor aprovechamiento de los recursos humanos calificados, hagan más eficiente el procesamiento de información y más ágil el
intercambio de conocimiento e información.

Además de la escasa capacidad técnica para la supervisión, que como fuera analizado anteriormente aumenta los problemas de información asimétrica, es común que en los países en desarrollo la capacidad técnica también sea reducida en aspectos regulatorios tales como el conocimiento de los instrumentos y las formas de implementar la regulación. Este tipo de incapacidad cognitiva y falta de experiencia de los reguladores también tiene importantes implicaciones. Por ejemplo, es más probable que los contratos diseñados por reguladores con limitada capacidad técnica en países en desarrollo sean incapaces de prever muchas de las contingencias que podrían ocurrir (la magnitud del problema es aún mayor si se tiene en cuenta que en estos países, generalmente, ocurren
contingencias más extremas que en países desarrollados, tales como grandes
devaluaciones o cambios políticos).

En otros términos, es probable que los contratos sean más incompletos, en el sentido que no serán contingentes en todas las posibilidades en las que sería óptimo que lo fueran. Como evidencia de esto puede citarse el hecho que las disputas por ``falta de claridad'' en los contratos son más frecuentes en las economías en desarrollo.

Al ser los contratos más incompletos a causa de la baja capacidad regulatoria y, por tanto, no estar previsto un importante número de eventos que podrían ocurrir, la renegociación de los contratos se vuelve más probable.
\footnote{En la sección \ref{credibilidad} se discutirá en profundidad las implicaciones de la renegociación causada por la falta de compromiso del gobierno y la pérdida de credibilidad en la regulación.}
El proceso de renegociación \emph{per se} introduce ineficiencias y el resultado final dependerá, en gran medida, del poder de negociación de las partes. Una consecuencia directa de esto es que la firma postergue inversiones, anticipando un proceso incierto de renegociación en el futuro que podría dejarle con menores retornos que los suficientes para hacer rentable la inversión.

Otra consecuencia, es que es más probable que la firma obtenga mayores rentas en términos esperados. Por un lado, el poder de negociación del gobierno es más probable que sea menor, una vez que los contratos con una firma ya están en marcha, con lo cual la firma obtendría una mejor posición frente a la renegociación del contrato. Por ejemplo, una vez que un proveedor ya está instalado y ha realizado sus inversiones, el
reemplazarlo podría ser costoso para el gobierno debido a la fuerte señal negativa que enviará a otros inversores. Por otro lado, los contratos son incompletos pero podrían estar sesgados a favor de la firma, debido a que esta posee mejor conocimiento técnico para incluir en los contratos contingencias que podrían afectarla en el futuro y, de esta forma, obtener mayor poder de negociación.

La consideración de contratos incompletos posee implicaciones sobre el poder de los incentivos y sobre el diseño de otras instituciones regulatorias. Para visualizar los efectos sobre el poder de los incentivos piense que existen contingencias que pueden afectar el costo marginal, que no han sido consideradas en los contratos. Si el gobierno ha ofrecido incentivos de alto poder a la firma a través, por ejemplo, de una regulación de precio límite o máximo, difícilmente obtenga buena información sobre el costo marginal.

En cambio, si ha seguido una política de reembolso de una proporción de los costos de la firma debería tener un buen conocimiento del costo marginal, con lo cual estará en una mejor posición de negociación cuando alguna contingencia no prevista suceda. Entonces, la imposibilidad de prever todas las contingencias, por ejemplo debido a la escasa capacidad técnica en países en desarrollo, brinda un argumento para el diseño de esquemas de incentivos de más bajo poder.

Por su parte, también brinda un argumento para el diseño de las instituciones regulatorias. Desde que los problemas introducidos por los contratos incompletos están determinados por lo que sucede en la etapa de renegociación, sería eficiente que las partes pudieran fijar su poder de negociación ex ante, así como también sus posiciones por defecto. Algunas formas de implementar esto es a través de un proceso de arbitraje, donde el poder de negociación sea decisión de un panel de expertos o mediante el uso de organizaciones internacionales que contribuyan a fijar posiciones por defecto, a través del uso de sistemas de penalidades y garantías.

A modo de resumen, la menor capacidad técnica para la supervisión en países en desarrollo reduce la información disponible para el gobierno. Mecanismos de incentivos de mayor poder, mayor competencia y la unificación de agencias regulatorias son respuestas para mitigar los efectos negativos de este problema.

La menor capacidad técnica también podría hacer los contratos más incompletos, determinando una reducción de la inversión y un aumento de la renta de la firma. En este caso, los efectos negativos pueden ser mitigados mediante la implementación de medidas para hacer el proceso de renegociación relativamente justo y eficiente. A diferencia del caso anterior, donde la menor capacidad técnica aumentaba la asimetría de información, en el caso donde el entorno contractual es más incompleto los esquemas de incentivos de menor poder son los que ayudan a mitigar los efectos negativos.

\hypertarget{rendicion}{%
\subsection{Baja rendición de cuentas}\label{rendicion}}

La relativa debilidad institucional de algunos países en desarrollo determina que los mecanismos de rendición de cuentas sean menos potentes. En este caso, la captura del supervisor por parte de la firma puede resultar más fácil debido a que, por ejemplo, el supervisor no debe rendir cuentas al gobierno de manera habitual.

En términos del modelo, una baja rendición de cuentas que facilite la captura por la firma del supervisor puede ser modelada por un aumento del parámetro \(k\), ya sea porque es relativamente más fácil para la firma mantener sus acciones ocultas o porque las sanciones en caso de ser descubierta son menores. La predicción inmediata del modelo es que, para evitar la colusión entre firma y supervisor, la compensación a este último por reportar que la firma es eficiente, \(s = kΔβ \overline q\), debería ser mayor y, por tanto, el bienestar social sería menor.

Comorespuesta a este problema, el gobierno podría modificar los contratos ofrecidos de forma que los incentivos contenidos en ellos sean de más bajo poder. Al bajar el poder de los incentivos, el gobierno reduce la tentación a capturar al supervisor ya que las rentas de información hacia la firma eficiente serán menores.

Hay que tener en cuenta que el resultado anterior asume que los costos de la firma pueden ser observados. Pero si la baja rendición de cuentas también afecta los procesos contables y de auditoría de tal forma que los costos de la firma puedan ser exagerados, entonces la respuesta óptima del gobierno sería reducir el uso de contratos basados en los costos, como por ejemplo aquellos que permitan el reembolso de una parte de los mismos, generando un sesgo hacia la utilización de esquemas de incentivos de más alto poder tales como contratos de renta fija o precio máximo para la firma. De esta manera, el efecto neto de una más baja rendición de cuentas sobre el poder de los incentivos es ambiguo y depende de qué proceso es más afectado por la baja rendición de cuentas.

De todas formas, la predicción del modelo no parece ser validada por la evidencia observada en países en desarrollo. En general, los supervisores en estos países no reciben mayores compensaciones que los supervisores en países con mayor rendición de cuentas. En cambio, es más frecuente observar casos de captura en países menos desarrollados que en países con un entorno institucional más fuerte. Esta brecha entre la realidad y la predicción del modelo se debe a que en este último la posibilidad de captura es completamente eliminada, mediante una compensación suficientemente elevada para el supervisor.

Para reconciliar las predicciones del modelo con la evidencia empírica consideremos la siguiente extensión del modelo, donde con alguna probabilidad positiva el supervisor puede ser benevolente (en el sentido que soportará las presiones y nunca aceptará ser capturado) o no benevolente (interesado solamente en su utilidad privada como hasta ahora).

Aunque el gobierno no conozca el tipo de supervisor, con esta sencilla extensión las predicciones del modelo se acercan más a lo observado en países en desarrollo. En particular, el gobierno ya no estará siempre interesado en prevenir la captura del supervisor ya que si lo intenta (mediante el pago de una compensación suficientemente alta) y este últimos es benevolente, la compensación no alterará en nada el resultado (esto es, no captura) pero será socialmente costosa debido a la ineficiencia en el financiamiento a través de impuestos.

Entonces, en economías en desarrollo donde \(k\) es grande pero hay posibilidades que el supervisor sea benevolente, el gobierno podría encontrar óptimo permitir cierto grado de captura del supervisor por parte de la firma. Si bien esto reduce el bienestar de la sociedad, ya que el supervisor informará al gobierno con menos frecuencia, la reducción de bienestar debido al costo de los impuestos necesarios para prevenir la captura podría ser mayor.

Hasta ahora hemos analizado la situación donde un gobierno benevolente (potencialmente) utiliza un supervisor interesado en su utilidad privada (esto es no benevolente) para estar informado. En la práctica, sin embargo, la baja rendición de cuentas en países en desarrollo alcanza también al gobierno y entonces la preocupación es que el propio gobierno sea capturado directamente por grupos de interés y, en términos del modelo, favorezca a la firma por sobre los consumidores.

En este caso, el gobierno estará menos preocupado por la distorsión introducida por los impuestos para financiar la renta que tiene que recibir la firma. Por lo tanto, el gobierno ofrecerá incentivos de mayor poder a la firma debido a que estará menos preocupado en reducir su renta.

En resumen, la existencia de mecanismos más débiles para la rendición de cuentas en las economías en desarrollo facilita la captura, disminuye el flujo de información hacia el gobierno y reduce el bienestar social. En términos generales, una solución pasa por reducir la importancia que posee la información para la captura de rentas por las diversas partes involucradas. En particular, una forma de reducir la dependencia en la información privada de la firma es bajando el poder de los incentivos para hacer menos atractiva la colusión. Esto también puede ayudar a mitigar las distorsiones introducidas por un gobierno no benevolente.

Sin embargo, hay que tener en cuenta que la baja rendición de cuentas también puede determinar la existencia de sistemas contables y de auditoría ineficientes donde los costos puedan ser fácilmente exagerados, en cuyo caso un gobierno benevolente preferiría la introducción de esquemas de incentivos de más alto poder. El resultado neto dependerá, entonces, de cuál problema es el predominante.

De todas formas, es necesario tener en cuenta que en algunos países en desarrollo, donde la institucionalidad para la rendición de cuentas es muy ineficiente, puede ser óptimo permitir algún grado de captura del supervisor, ya que evitar completamente la misma podría resultar prohibitivamente costoso desde el punto de vista social.

\hypertarget{credibilidad}{%
\subsection{Compromiso limitado y falta de credibilidad}\label{credibilidad}}

Una forma de compromiso limitado que puede afectar el bienestar social está dada por la posibilidad de renegociación. Si una vez revelada cierta información, como por ejemplo el grado de eficiencia de la firma, las partes encuentran óptimo renegociar, entonces el contrato original no será creíble. Para devolver credibilidad, el gobierno deberá restringirse y sólo podrá usar contratos que sean eficientes ex post (o sea que no sea atractiva su renegociación), con lo cual el bienestar social será menor.

Para fijar ideas piense que la firma ineficiente ha revelado su tipo al seleccionar el contrato diseñado para ella. Conocida esta información, tanto el gobierno como la firma encontrarán óptimo no distorsionar a la baja el nivel de producción, de forma que la firma ineficiente realice el nivel óptimo de esfuerzo. En otras palabras, una vez conocido que la firma es ineficiente, la provisión de incentivos de bajo poder no será óptima y, por tanto, el cumplimiento del contrato original no será creíble. Si este incumplimiento del contrato es anticipado, la firma eficiente tendrá mayores incentivos a hacerse pasar por una firma ineficiente. Entonces, para eliminar esta posibilidad, que se origina en la eventualidad de renegociación, el contrato ofrecido a la firma eficiente debería incluir una renta de información mayor. De esta manera, el bienestar social sería menor a causa de la posibilidad de renegociación.

Dentro de las razones por las que el gobierno puede preferir la renegociación se encuentra la insatisfacción por el nivel de rentas ofrecidas a las empresas. Por ejemplo, el gobierno puede desear que las empresas no obtengan más que determinada renta. En este caso, si la renta requerida por la firma eficiente para no hacerse pasar por la firma ineficiente excede ese umbral, entonces la falta de compromiso del gobierno favorecería la imposición de un sistema de incentivos de menor poder.

Al reducir el poder de los incentivos, el gobierno restringe la posibilidad de la firma de obtener rentas mayores, haciendo menos atractivo para la firma eficiente el hacerse pasar por una firma ineficiente y, por tanto, mitiga los efectos negativos de la falta de compromiso. De esta forma, esquemas de incentivos de menor poder, como por ejemplo aquellos que fijan una \textbf{tasa de retorno} para la firma, serían más convenientes que esquemas de alto poder, tales como un \textbf{precio límite}, en economías donde la falta de compromiso es un problema.

La evidencia empírica aporta elementos a favor de este resultado. Por ejemplo, \citet{Guasch2007} y \citet{Guasch2008} encuentran que en América Latina los regímenes de precio límite generalmente han causado más renegociaciones que esquemas de incentivos de menor poder.

Otras potenciales soluciones al problema de falta de compromiso podrían venir por el lado del financiamiento de la firma. Por ejemplo, una mayor participación del gobierno en la propiedad de la misma o una mayor proporción de la deuda, por sobre los aportes de capital privado, reducirían las rentas hacia la firma y podrían mitigar los problemas de falta de compromiso.

Mientras que la posibilidad de renegociación es importante, la posibilidad de que el gobierno sea incapaz de comprometerse en absoluto y pueda romper el contrato en el futuro, aún sin que la firma quisiera renegociar, es también importante en economías en desarrollo. Si este fuera el caso, la falta de compromiso y credibilidad en los contratos determinaría una elevada incertidumbre para la firma, en cuanto a los retornos futuros de los que se podría apropiar.

A su vez, esta incertidumbre implicaría que importantes y necesarias inversiones, por ejemplo inversiones en infraestructuras que requieren largos períodos de tiempo para su recupero e involucran activos específicos y no transferibles, no sean llevadas adelante, determinando una situación de subinversión en países en desarrollo. Esto es consistente con la evidencia empírica que documenta que una mayor estabilidad del gobierno, que a su vez está correlacionada con la capacidad de compromiso, está relacionada con mayores niveles de inversión \citep{Banerjee2006}.

La imposibilidad de comprometerse en forma creíble del gobierno también tiene implicaciones dinámicas. Una vez que el gobierno toma conocimiento que la firma es eficiente se verá tentado a utilizar esta información para extraer las rentas de la firma. Al anticipar este comportamiento oportunista por parte del gobierno, la firma eficiente tendrá menos incentivos a revelar su tipo, lo que será costoso para la sociedad debido a las rentas de información que le deben ser transferidas. Si en cambio el gobierno tuviera la capacidad de comprometerse a no utilizar la información para volverse más demandante con la firma, el bienestar general podría ser mejorado.

Mientras que los problemas analizados hasta aquí tienen lugar una vez que la firma ha contratado con el gobierno, la falta de compromiso y la probabilidad de renegociación también podrían afectar las etapas previas, tales como los procesos licitatorios, e incluso pueden comprometer la viabilidad de ciertas reformas. Por ejemplo, la incertidumbre sobre el respeto futuro de los contratos y la posibilidad que el gobierno pueda modificarlos unilateralmente podrían ser un motivo para que las empresas reduzcan el valor de sus ofertas en un proceso licitatorio. Además, es altamente posible que la firma adjudicataria no sea en realidad la firma más eficiente, sino la firma que considera que tendrá una buena posición en caso de renegociación o, en otros términos, que tiene una baja chance de ser expropiada.

La baja credibilidad puede también dar razones a los ciudadanos para oponerse a procesos de reforma. Por ejemplo, algunos procesos de reforma comienzan con un período de aumento de precios que se espera posibilite a la firma realizar inversiones para mejorar la eficiencia, que luego permitirán una reducción permanente de precios en beneficio de los consumidores. En este caso, si el compromiso para cumplir con la baja de los precios futuros es bajo, los consumidores podrían preferir oponerse a la reforma ya que anticipan que terminarán pagando precios más altos.

Una posible solución para mitigar los problemas introducidos por la posibilidad de renegociación viene dada por el diseño de las instituciones regulatorias. En particular, mediante la creación de un regulador (esto es, un agente que diseñe los contratos a ser ofrecidos a la firma) independiente del gobierno se podría incrementar la capacidad de compromiso y la credibilidad de las regulaciones.
\footnote{En general el diseño de reglas y procesos para la etapa de renegociación pueden mejorar la capacidad de compromiso. En particular, la creación de más de un regulador independiente puede, bajo ciertas condiciones, ser una mejor opción a la unificación bajo un solo regulador. En la Sección \ref{captura} se brinda una discusión de los beneficios de la separación con respecto a la unificación que puede ser complementada por el análisis en \citet{Estache1999}.}

La razón para ello es simple: un regulador independiente podrá tener una función objetivo diferente de la del gobierno y, por tanto, una capacidad diferente de compromiso. Si, por ejemplo, el regulador pone un mayor peso que el gobierno en el bienestar de la firma, entonces el primero tendrá menos incentivos a renegociar los contratos que el segundo, ya que estará menos interesado en reducir las rentas obtenidas por la firma. Al igual que en el caso de renegociación, en el caso de subinversión por falta de compromiso del gobierno la creación de un regulador independiente ayudaría a mitigar los problemas. El regulador permitiría la aplicación de incentivos de más alto poder, con los cuales la firma se vería impulsada a invertir.

La evidencia empírica ha encontrado que la existencia de un regulador independiente no solamente reduce la ocurrencia de renegociación, sino que también está relacionada a un mayor tamaño de la red, en el caso de inversiones en el sector de las telecomunicaciones. Otros aspectos, tales como la reputación, podrían determinar una mayor capacidad de compromiso del regulador con respecto al gobierno.

A modo de resumen, la capacidad de compromiso limitado del gobierno y la falta de credibilidad en la regulación pueden manifestarse de varias maneras y, por tanto, las soluciones pueden variar dependiendo del problema predominante. En términos generales, las soluciones pueden transitar desde el diseño de esquemas de incentivos de menor poder (para mitigar los problemas originados por la renegociación de los contratos impulsada desde el gobierno) a la creación de un regulador independiente y sesgado a favorecer a la firma (para mitigar los problemas originados tanto por la posibilidad de renegociación como por la falta de compromiso y comportamiento oportunista del gobierno).

\hypertarget{baja-eficiencia-fiscal}{%
\subsection{Baja eficiencia fiscal}\label{baja-eficiencia-fiscal}}

La baja eficiencia de los sistemas impositivos en los países en desarrollo hace aún más acuciantes algunos de los problemas analizados previamente. Por ejemplo, el potencial de generar capacidad técnica para la supervisión y de retener capital humano calificado, así como la posibilidad de generar instituciones capaces de mejorar el compromiso y la credibilidad de las regulaciones dependen crucialmente de la eficiencia fiscal. Además, una baja eficiencia fiscal también potencia los problemas producidos por una ineficiente rendición de cuentas, al hacer la provisión de incentivos al supervisor más costoso para la sociedad.

Una implicación directa de una más baja eficiencia fiscal (o lo que es lo mismo de un mayor costo de oportunidad de los fondos públicos), que está representada en el modelo por un mayor parámetro \(λ\), es que el precio que enfrentarán los consumidores será mayor.
\footnote{Recuerde que en términos del modelo presentado en la Sección \ref{modelo} la fijación de precios se realiza de acuerdo a la siguiente expresión: \(\frac{p-\left(\beta-e\right)}{p}=\frac{\lambda}{1+\lambda}\times\frac{1}{\eta}\), con lo cual un mayor costo de oportunidad de los fondos públicos determina una brecha mayor entre el precio y el costo marginal de producción.}
Una razón para ello está en que, dada la ineficiencia general en la recaudación de impuestos, cobrar impuestos a la firma es relativamente más eficiente para el gobierno. De esta manera, es posible que firmas en sectores que son subsidiados en países más avanzados deban pagar importantes impuestos en países menos desarrollados. Además,
la baja eficiencia fiscal podría brindar una razón para aumentar los beneficios de la firma, ya que estos sirven de base para la recaudación impositiva. De esta manera, es de esperar mayores transferencias y más monopolios públicos en los países en desarrollo.

Otra implicación de una más baja eficiencia fiscal es que el gobierno preferirá utilizar esquemas de incentivos de menor poder. Esto es, en los hechos, una respuesta para mitigar los problemas introducidos por la ineficiencia en la recaudación de impuestos y el uso de los fondos públicos. Debido a que el costo de oportunidad de estos fondos es elevado en economías en desarrollo, el costo de transferir una renta de información a la firma eficiente, así como el costo de compensar al supervisor por sus reportes, también serán mayores. Dado que estos costos están asociados al nivel de producción de la firma ineficiente, el gobierno podrá reducir los mismos, reduciendo la producción de esta firma. Para ello, podrá bajar el poder de los incentivos de forma que el nivel de es-
fuerzo será menor, el costo marginal mayor y, por tanto, la producción de la
firma ineficiente menor.

Otro aspecto crucial para las economías en desarrollo, sobre el que tiene implicancia directa la ineficiencia impositiva, está dado por el escaso alcance, cobertura y desarrollo de las redes de servicios públicos. Los fondos públicos, además de ser utilizados para realizar transferencias a la firma, compensar las agencias de supervisión y generar instituciones regulatorias, son utilizados para desarrollar infraestructuras a través de la inversión pública. En los países en desarrollo la expansión de las redes de servicios públicos tales como electricidad, telefonía y agua potable es sumamente necesaria y, posiblemente, más costosa que en otras economías debido a características geográficas y urbanísticas.

Pero es justamente en estas economías donde la capacidad de extender los servicios mediante el uso de fondos públicos está más limitada por la ineficiencia fiscal. A su vez, esto determina redes de servicios públicos de menor escala que el mínimo que sería necesario para atraer inversores extranjeros (los que también podrían no encontrar atractivo el mercado debido a los problemas de compromiso analizados en la sección previa). Como resultado, algunos sectores de la sociedad podrían ver restringido su acceso a los servicios públicos.

Además, el acceso a los servicios públicos también se podría ver restringido debido a efectos indeseados de algunas políticas públicas impulsadas con el objetivo de hacer abordable los servicios para los consumidores. Por ejemplo, una política que prohíbe la discriminación de precios por parte de la firma puede tener el loable objetivo de posibilitar que consumidores en zonas alejadas (como zonas rurales) compren el servicio al mismo precio que consumidores en zonas más densamente pobladas (por ejemplo, en las ciudades).

Pero, por lo general, el costo de suministrar el servicio en zonas alejadas es mayor, con lo cual la firma podrá preferir no suministrar el servicio en dichas zonas. Existe entonces una tensión entre hacer abordable el servicio y la posibilidad que los consumidores puedan acceder al mismo. ¿Cómo es posible mitigar estos problemas de acceso? Una posible solución es permitir a la firma la utilización de subsidios cruzados, donde la actividad en las zonas menos costosa subsidie la actividad en las zonas más costosas.

Si la ineficiencia fiscal es tal que el costo de los fondos públicos es extremadamente elevado, el gobierno podría igualmente garantizar el acceso a toda la población mediante la imposición de un requerimiento de cobertura universal, donde la firma deba satisfacer la demanda de toda la población, al tiempo que podría reducir el costo de las transferencias necesarias, permitiendo que la propia firma obtenga rentas extraordinarias en las zonas menos costosas, de forma de subsidiar internamente la actividad en las zonas más costosas.

A modo de resumen, la mayor ineficiencia fiscal en los países en desarrollo potencia algunos de los problemas analizados previamente y, además, puede determinar la existencia de mayores precios para los consumidores y mayores impuestos para la firma que en economías con mayor eficiencia en la recaudación y uso de fondos públicos. Una respuesta para mitigar estos problemas consiste en reducir el poder de los incentivos, para así reducir el costo fiscal de proveer los mismos.

Otra importante implicación está dada por la existencia de redes de servicios públicos más pequeñas y fallas en la cobertura. Una respuesta a este problema es la utilización de subsidios cruzados, para cuyo diseño es necesario considerar la tensión entre la posibilidad de pagar por el servicio de los consumidores y el deseo de la firma de brindar acceso a los mismos.

\hypertarget{corrupciuxf3n}{%
\subsection{Corrupción}\label{corrupciuxf3n}}

En la sección \ref{rendicion} se analizaron los efectos producidos por el hecho de que en países en desarrollo la rendición de cuentas es más débil. En particular, se concluyó que el atractivo para que la firma capture al supervisor se reduciría si los esquemas de incentivos fueran de menor poder. El modelo presentado también puede ser utilizado para el análisis de otros tipos de corrupción que comúnmente son cometidos por otros agentes, posiblemente más relevantes que el supervisor, en economías en desarrollo,
\footnote{A modo de ejemplo, sobre finales del año 2014 tomó conocimiento público un esquema de sobornos por un monto cercano a los cuatro mil millones de dólares en la petrolera brasilera Petrobras.}
así como también para entender el resultado de ciertos procesos de reforma en estos países.

Generalmente, los países en desarrollo han recibido de asesores internacionales la recomendación de descentralizar la supervisión. Esto es, por ejemplo, separar de una agencia central o nacional ciertas actividades que serán llevadas adelante por agencias locales. Este tipo de estructura puede ser fácilmente representada en términos del modelo de la sección \ref{modelo}. En esa sección se analizó la jerarquía formada por el gobierno, el supervisor y la firma. El análisis de la descentralización focaliza en una jerarquía formada por una agencia central, una agencia local y la firma. Entonces, en esta jerarquía la firma tendrá incentivos para capturar a la agencia local y, como resultado, será más probable que la agencia local oculte información a la agencia central.

Además de los motivos contemplados en el modelo básico (donde la captura es modelada como una transferencia), en la práctica la agencia local puede tener otras razones para coludir con la firma. Por ejemplo, si esta agencia es un gobierno local sus intereses para mantener una firma (aún ineficiente) operando en su territorio pueden estar sustentados por el mantenimiento de puestos de trabajo o la recaudación de impuestos locales.

Este tipo de actividades, a su vez, pueden facilitar el logro de los objetivos de los políticos locales tales como ser promovidos o reelectos. En China, por ejemplo, los gobiernos locales son reconocidos por su colusión con pequeñas e ineficientes firmas de carbón para que estas no sean cerradas por el gobierno central.

Más recientemente, el vínculo de los gobiernos locales ha sido muy fuerte con las empresas constructoras y una parte importante de sus ingresos proviene de actividades relacionadas con el sector inmobiliario (aproximadamente un 40\% según algunas estimaciones). En estos casos, el modelo indica que la agencia central deberá compensar a la agencia local para que esta reporte, en forma verdadera, la información que posee y, de esta forma, la corrupción y sus efectos negativos puedan ser mitigados.

La propia composición de la agencia supervisora puede afectar el grado de corrupción ya que diferentes objetivos de los supervisores pueden determinar diferentes incentivos. Por ejemplo, un supervisor puede estar interesado en su carrera en la industria luego de abandonar la agencia supervisora, mientras que otro puede ser un profesional retirado de la industria que, para mantener influencia en el proceso regulatorio, se une a las tareas del supervisor y un tercero puede ser un político para el cual la agencia supervisora es un primer escalón en su carrera política. Claramente los dos primeros grupos de supervisores darán un valor de continuación más alto al hecho de coludir con la firma.

Esto sugiere que darle más peso a los políticos dentro de la agencia supervisora sería beneficioso para mitigar los efectos de la corrupción. De todas maneras, esta recomendación no es consistente con la sugerencia tradicional de utilizar supervisores técnicos por sobre políticos. El modelo es también de utilidad para analizar la corrupción por parte del propio gobierno así como para comprender por qué algunas reformas, que podrían mejorar el bienestar general en países en desarrollo, no obtienen el apoyo de la ciudadanía en general.

Por ejemplo, Bergara y Pereyra (2005) documentan que los distintos intentos de privatización y apertura en Uruguay fueron rechazados mediante la utilización de mecanismos de democracia directa, en tanto \citet{Martimort2009} encuentran que varias encuestas, en países de América Latina, indican que el descontento de la ciudadanía con los procesos de privatización está correlacionado con la percepción de una creciente corrupción.

Lo anterior puede ser racionalizado en términos del modelo al considerar una jerarquía compuesta por la ciudadanía, el gobierno y la firma en un entorno de baja rendición de cuentas del gobierno a su electorado (la ciudadanía). Cuando la firma es regulada por el gobierno, la corrupción toma la forma de una mayor transferencia de fondos públicos hacia la firma. Cuando la actividad es privatizada, los efectos de la corrupción pasarán a los consumidores (la ciudadanía) directamente a través de mayores precios. Además, si el costo de los fondos públicos es relativamente elevado, debido a la baja eficiencia de los sistemas impositivos en países en desarrollo, que el costo de aumentar los precios, entonces la corrupción será menos costosa y más frecuente, en el caso que la actividad sea privatizada.

Además, generalmente la ineficiencia fiscal está asociada a sistemas de impuestos y subsidios relativamente opacos para los ciudadanos, en tanto la obtención de rentas extraordinarias a través de un aumento de los precios del producto es más susceptible de ser identificada por los ciudadanos.

En términos generales, el modelo de la sección \ref{modelo} es lo suficientemente versátil tanto para sugerir caminos que mitiguen los efectos negativos de la baja rendición de cuentas y la corrupción en países en desarrollo, como para analizar los riesgos de descentralización y privatización. Una enseñanza de estos análisis es que identificar el tipo de corrupción predominante en cada situación es vital para proponer la solución que mejor mitigue el problema.

\hypertarget{baja-capacidad-de-hacer-cumplir-las-normas}{%
\subsection{Baja capacidad de hacer cumplir las normas}\label{baja-capacidad-de-hacer-cumplir-las-normas}}

En la sección \ref{credibilidad} se analizaron las implicaciones de una baja capacidad de compromiso por parte del gobierno. En esta sección se discutirá, brevemente, qué sucede cuando hay una baja capacidad de hacer cumplir las normas y, por tanto, la firma puede salirse de lo estipulado en el contrato, incluso cuando esta acción sea desventajosa para el gobierno.

Dada la fortaleza de las instituciones en los países desarrollados es poco probable que una firma pueda desviarse si esto no está en el interés del gobierno. En cambio, \citet{Estache2009} presentan ejemplos de países menos desarrollados tales como Ghana, donde el monopolista de la telefonía fija entró en el segmento de la telefonía móvil a pesar de tenerlo expresamente prohibido, o de Tanzania donde el supervisor intentó restringir la operación de la compañía proveedora de telefonía móvil pero esta no estuvo de acuerdo y se expandió nacionalmente.

Una posible explicación para esta relación entre el nivel de desarrollo y la capacidad para hacer cumplir las normas es que a medida que el nivel de desarrollo aumenta la eficiencia del sistema impositivo aumenta y, entonces, los gastos en mejorar la eficiencia para hacer cumplir las normas también aumentan, ya que los fondos públicos son menos costosos.

En términos del modelo presentado no es difícil mostrar que una firma ineficiente puede tener incentivos a renegociar el contrato cada vez que la utilidad que obtiene, luego del proceso de renegociación, es mayor que la utilidad especificada en el contrato.
\footnote{\citet{Guasch2006} documentan que el 41\% de los contratos de concesión de infraestructura otorgados en América Latina entre 1990 y 2000 fueron renegociados a instancias de las firmas concesionarias.}
En particular, Laffont (2005) muestra que esto sucede si la firma ineficiente sólo descubre su nivel de eficiencia luego de haber aceptado el contrato ya que, en este caso, el contrato óptimo ofrecido por el gobierno implicará un pago negativo para la misma. Desde que los procesos de renegociación son ineficientes y generan costos, por ejemplo porque insumen tiempo, una baja capacidad de hacer cumplir los contratos redundará en una caída del bienestar social. Entonces, puede resultar valioso generar algún tipo de institución para mejorar el cumplimiento de las normas. En particular, este tipo de institución será más necesaria cuando la eficiencia de los procesos de renegociación es más baja.

En la sección \ref{credibilidad} se discutió como un regulador independiente y sesgado hacia la firma podía mitigar los problemas de baja inversión ocasionados por la falta de compromiso del gobierno. De manera similar, en el caso que los problemas sean a consecuencia de una baja capacidad de hacer cumplir los contratos, un regulador independiente, pero sesgado hacia los consumidores, podría ser beneficioso.

\hypertarget{captura}{%
\subsection{Alta posibilidad de captura del regulador}\label{captura}}

A lo largo de las secciones previas se ha abordado repetidamente el problema que el supervisor puede ser capturado por la firma para no reportar al gobierno la información sobre su eficiencia. En esta sección se profundizará el análisis de las causas y consecuencias de una más alta posibilidad de captura en economías en desarrollo y se discutirán posibles soluciones. Se focalizará en el diseño de las instituciones para la supervisión y, en particular, en la conveniencia y la posibilidad de utilizar más de un supervisor en estas economías, a los efectos de mitigar la posibilidad y el costo de la captura regulatoria.

En el contexto del modelo de la sección \ref{modelo}, la separación de la supervisión en dos agencias independientes puede ser visualizada como un mecanismo para mitigar los efectos de la captura. Esto es posible debido a que el gobierno podrá utilizar la información proporcionada por uno de los supervisores para evaluar la información brindada por el otro (un mecanismo conocido en la literatura como \emph{yardstick competition}). Al utilizar la correlación entre las señales que cada supervisor recibe, el gobierno es capaz de extraer de ellos sus rentas en forma gratuita.
\footnote{De todas formas existirá un costo asociado a la duplicación de los supervisores.}

En lo que sigue se discutirá la intuición detrás de los resultados teóricos de los principales trabajos que han analizado la conveniencia de separar la supervisión para mitigar los efectos de la captura por parte de la industria.
\footnote{Esta sección está basada en los trabajos de \citet{Laffont1999} y \citet{Laffont2005}. \citet{Boyer2012} presentan una aplicación para el caso de la industria bancaria y la separación entre supervisión micro y macro-prudencial.}
A modo de adelanto, la conclusión es que la separación de poderes es más valiosa en países en desarrollo que en países más desarrollados.
\footnote{Otras razones que pueden determinar que la separación de poderes sea deseable tienen que ver con que la separación puede ser beneficiosa cuando la capacidad de compromiso intertemporal es limitada, para mitigar la ocurrencia de potenciales errores e incluso para proveer poderosos incentivos reputacionales.}
Pero, desafortunadamente, las mismas razones que determinan este resultado también implican que sea más difícil implementar la separación en estos países y garantizar un adecuado nivel de eficiencia.

Considere una simple extensión del modelo antes presentado donde en lugar de haber una señal sobre el nivel de eficiencia de la firma hay dos señales, cada una de las cuales proviene de la aplicación de una tecnología de supervisión diferente. El caso en el cual un único supervisor maneja ambas tecnologías es análogo al discutido en la sección \ref{rendicion} y, entonces, la respuesta óptima a la amenaza de captura del supervisor por parte del gobierno es bajar el poder de los incentivos. Entonces, los esquemas de incentivos deberían ser de más bajo poder en los países en desarrollo.

Adicionalmente, el gobierno también puede modificar el diseño institucional y disponer que cada tecnología de supervisión sea responsabilidad de un super visor independiente del otro. En este caso, es posible probar que la separación de poderes permite un ahorro en el pago de compensaciones a los supervisores para que estos revelen sus señales y, por tanto, también permite la implementación de mecanismos de incentivos de mayor poder. Una razón detrás de este resultado es que el gobierno podrá utilizar la señal proporcionada por un supervisor para compensar (o penalizar) al otro supervisor.

Para fijar ideas, asuma que la correlación entre las señales de los supervisores es perfecta. En este caso, si el gobierno recibe información de un supervisor pero no del otro, el gobierno sabrá con certeza que este último está ocultando su señal y podrá penalizarle. Como los supervisores conocen que este será el caso si ocultan información, en equilibrio ambos reportarán sus señales al gobierno aún sin la necesidad de recibir una compensación a cambio.

En este caso particular, el gobierno estará plenamente informado sin costo alguno para proveer incentivos a los supervisores y la regulación óptima, con un gobierno benevolente, podrá ser implementada. Otra razón por la cual se puede ahorrar la compensación necesaria para que cada supervisor revele su información está dada por la relativamente más desventajosa posición de cada uno de ellos vis-à-vis la firma. Bajo separación cada supervisor estará sólo parcialmente informado y, por tanto, la firma tendrá menos incentivos para capturar a un supervisor que es relativamente más débil, con lo cual la compensación necesaria para lograr que estos revelen su señales también será menor.
\footnote{Laffont y Martimort (1999) muestran que los beneficios de la separación también existen si un supervisor observa el reporte del otro supervisor antes de decidir si ceder a las presiones de la firma, tal como ocurre frecuentemente en la práctica.}

No es difícil mostrar que la ganancia de bienestar asociada a la separación de poderes aumenta con el parámetro \(k\) (con una más baja rendición de cuentas) y con el parámetro \(λ\) (con una menor eficiencia en el uso de los fondos públicos).

En cambio, la ganancia se reduce con una más baja capacidad técnica para la supervisión. Entonces, el valor de la separación de supervisores, para combatir el problema de captura por parte de la industria, es mayor en países menos desarrollados cuando estos poseen tecnologías de supervisión similares a las de los países avanzados. Sin embargo, el resultado se vuelve ambiguo cuando se tiene en cuenta que los países en desarrollo también poseen tecnologías de supervisión menos eficientes.

Además, las propias características de los países en desarrollo hacen más costosa la implementación de las reformas estructurales necesarias para garantizar una separación efectiva. En particular, es sumamente probable que cuanto más baja sea la capacidad de hacer que los supervisores rindan cuentas, más alta será la posibilidad que los propios supervisores coludan entre sí y no reporten la información al gobierno. Esta más alta probabilidad de que la separación de poderes sea puenteada por la colusión entre supervisores se suma a los mayores costos de compensarles (debido a la mayor ineficiencia impositiva) y a que, generalmente, los costos de transacción son mayores en las economías en desarrollo.

A modo de resumen, la mejora en las instituciones mediante la separación de supervisores es más valiosa en las economías en desarrollo que en las más desarrolladas, para mitigar los efectos de la captura por parte de la industria y el comportamiento oportunista de los supervisores. Esto es así debido a los mayores costos de los fondos públicos, a la más baja rendición de cuentas y a la existencia de tecnologías de supervisión menos eficientes.

Desafortunadamente, estas mismas razones determinan que las reformas estructurales necesarias sean más difíciles y costosas de implementar y que, de implementarse, muestren resultados menos eficientes. De esta manera, la recomendación de política con respecto a la separación de poderes será ambigua, a menos que se atiendan en forma simultánea varios de los problemas de las economías en desarrollo previamente analizados.

\hypertarget{coment-fin}{%
\section{Comentarios}\label{coment-fin}}

Este capítulo ha analizado los distintos problemas a los que se enfrenta el diseño de instituciones regulatorias en países en desarrollo. En términos generales, la existencia de asimetría de información entre las partes (firmas, reguladores, gobierno, ciudadanía, etc.) permite a algunos de estos agentes obtener rentas que no existirían si todas ellas tuvieran la misma información. Estas rentas de la información pueden ser mitigadas, aunque no eliminadas, a través del correcto diseño de las regulaciones y de las instituciones para la supervisión.

El diseño de las agencias reguladoras y los instrumentos disponibles dependerán de cuáles sean los principales cuellos de botella y restricciones que enfrente el gobierno. Esto determina una fuente importante de diferencias entre las regulaciones necesarias para economía desarrolladas y economías en desarrollo. De esta forma, es crucial mantener un adecuado balance entre las reglas de juego que permitan la provisión de bienes y servicios de calidad a un precio razonable y las intervenciones diseñadas para mitigar los problemas de asimetría de información entre las partes.

La experiencia en países en desarrollo ha mostrado que los efectos de un marco institucional más débil, sobre los resultados de las regulaciones, han sido importantes. El análisis contenido en las secciones previas sirve de guía para ver cómo la consideración explícita de las características distintivas de los países en desarrollo necesariamente implica que la regulación óptima en estos países tenga que apartarse, en aspectos clave, de la regulación óptima para países desarrollados.

Una síntesis de potenciales problemas y posibles soluciones es la siguiente:
* Baja capacidad técnica:
* Implementar esquemas de incentivos de más alto poder (si es posible diseñar contratos completos).
* Unificar agencias supervisoras.
* Fomentar la competencia del mercado.
* En un entorno de contratos incompletos, limitar el poder de los esquemas de incentivos e implementar mecanismos de arbitraje para fijar \emph{ex ante} el poder de negociación de las partes.

\begin{itemize}
\tightlist
\item
  Baja rendición de cuentas:

  \begin{itemize}
  \tightlist
  \item
    Implementar esquemas de incentivos de más bajo poder.
  \item
    Balancear la baja de poder de los esquemas de incentivos si la escasa rendición de cuentas también afecta la observación de costos.
  \end{itemize}
\item
  Compromiso limitado y falta de credibilidad:

  \begin{itemize}
  \tightlist
  \item
    Implementar esquemas de incentivos de más bajo poder.
  \item
    Dotar de mayor independencia al regulador.
  \item
    Propiciar un mayor peso de la deuda en el financiamiento de la firma.
  \end{itemize}
\item
  Baja eficiencia fiscal:

  \begin{itemize}
  \tightlist
  \item
    Implementar esquemas de incentivos de más bajo poder.
  \item
    Si se generan deficiencias en la cobertura, permitir el subsidio cruzado.
  \end{itemize}
\item
  Corrupción:

  \begin{itemize}
  \tightlist
  \item
    Reformas que potencialmente incrementan el bienestar social podrían carecer de apoyo de los ciudadanos.
  \end{itemize}
\item
  Baja capacidad de hacer cumplir las normas:

  \begin{itemize}
  \tightlist
  \item
    Dotar de mayor independencia al regulador.
  \end{itemize}
\item
  Alta posibilidad de captura del regulador:

  \begin{itemize}
  \tightlist
  \item
    Implementar esquemas de incentivos de más bajo poder.
  \item
    Separar las tareas de supervisión en varios supervisores independientes.
  \end{itemize}
\end{itemize}

Algunas de las recomendaciones sugeridas por la teoría económica de la regulación, que fueran analizadas en las secciones previas, para diferentes problemas en países en desarrollo tienen su correlato con lo efectivamente observado en estos países a través de sus diferentes etapas de desarrollo. \citet{Laffont2005} argumenta, por ejemplo, que las diferentes etapas de la regulación del mercado eléctrico en Europa del Este, los procesos de nacionalización en el resto de Europa o la regulación de firmas privadas en los Estados Unidos pueden ser identificadas con los resultados previos.

En una etapa de menor desarrollo (etapa 1) la falta de capacidad técnica es tal que los sistemas contables y de auditoría son inexistentes. En este caso el gobierno no posee muchas más opciones que proponer esquemas de incentivos de alto poder, tales como contratos de renta fija o de precio máximo, que dejan a la firma la posibilidad de apropiarse de cualquier reducción de los costos. Estos contratos poseen un alto costo social, en términos de rentas de información hacia la firma, el que puede ser mitigado mediante la calibración adecuada de los parámetros de los contratos, tal como se analizara previamente.

A medida que los sistemas contables y de auditoría se desarrollan y se ponen en funcionamiento (etapa 2), el gobierno estará mejor informado y tiene la oportunidad de utilizar otros esquemas de incentivos de menor poder tales como el reembolso parcial de los costos de la firma. De todas formas, el grado de desarrollo todavía no es suficientemente alto y otros de los problemas analizados previamente, tales como baja rendición de cuentas e ineficiencia de los sistemas impositivos, están aún presente. Entonces, a medida que los países acceden a etapas superiores de desarrollo, la regulación se debería mover, quizás en forma discreta, hacia esquemas de incentivos de menor poder.

Finalmente, a medida que se alcanzan etapas superiores de desarrollo (etapa 3), la eficiencia en el uso de los fondos públicos mejora paulatinamente. Esto posibilita un retorno gradual a la utilización de esquemas de incentivos de más alto poder.

Otra implicación del análisis previo es que cierta independencia de los reguladores y separación de los supervisores puede ser conveniente para mitigar los problemas ocasionados por la baja capacidad de compromiso y de rendición de cuentas en economías en desarrollo.

\hypertarget{reg-eepp}{%
\chapter[Las empresas públicas y su regulación ]{\texorpdfstring{Las empresas públicas y su regulación \footnote{Basado en el marco del acuerdo de cooperación entre la Facultad de Ciencias Sociales de la UdelaR y la Unidad Reguladora de los Servicios de Energía y Agua (URSEA) y contó con financiamiento de la Corporación Andina de Fomento (CAF).}}{Las empresas públicas y su regulación }}\label{reg-eepp}}

Rosario Domingo
Leandro Zipitría

Este capítulo tiene como objetivo la revisión de las características fundamentales que definen a las empresas públicas y las diferencian de las privadas. El énfasis está en discutir y comparar cómo afecta la propiedad de las empresas la eficacia de los instrumentos de regulación tradicionales utilizados para empresas privadas.

Se analiza las características de las empresas públicas y la regulación de monopolios naturales puesto que muchas empresas públicas actúan en sectores con estas características. Asimismo, considerando como principio general que las empresas deben competir, independientemente de la propiedad de las mismas, se presenta una breve discusión de la regulación en aquellos sectores donde operan empresas públicas y la competencia es factible.

\hypertarget{empresas-puxfablicas-su-origen}{%
\section{Empresas públicas: su origen}\label{empresas-puxfablicas-su-origen}}

La existencia de empresas públicas parece ser un fenómeno habitual y económicamente significativo, pudiendo encontrarse empresas públicas en diversos sectores relevantes de la actividad económica, en la mayoría de los países. \citet{Kowalski2013} calculan que entre las 2.000 empresas más importantes del mundo, en el período 2010-2011, 204 de ellas cuentan con participación del Estado en su capital. Estas 204 empresas pertenecen a 37 países diferentes y sus ventas alcanzan al 10\% de las ventas totales de las mayores 2.000 empresas de Forbes. Este monto de ventas resulta mayor que el producto bruto interno (PBI) de países como Reino Unido, Francia o Alemania y representa alrededor del 6\% del PBI mundial.

Según el \citet{Robinett2006}, las empresas públicas fueron responsables por el 20\% de la inversión y 5\% del empleo mundial, teniendo mayor presencia sobre todo en países de ingreso medio y bajo, en economías en transición y en los países emergentes. Esto implica que a pesar de la ola de privatizaciones de los 90 las empresas públicas siguen siendo importantes en varias regiones. En África son responsables del 15\% del PBI, en Asia del 8\% y en América Latina del 6\%. Asimismo, entre las mayores empresas de diferentes países las empresas públicas tienen un peso importante, como se aprecia en el gráfico \ref{fig:fig1}.

\begin{figure}
\centering
\includegraphics{Regulacion_files/figure-latex/fig1-1.pdf}
\caption{\label{fig:fig1}Participación pública en las 10 principales empresas por país, años 2010-2011}
\end{figure}

Fuente: \citet{Kowalski2013}.

Por su parte \citet{Jones1982} consideran que el porcentaje de participación de las empresas públicas en el PBI de los países en desarrollo se mueve en un rango del 7\% al 15\%, independientemente de la forma de organización que
adopte la economía en cuestión. Las situaciones que observan fuera de este rango se deben fundamentalmente a una dotación de recursos diferentes, tal el caso de los países con fuertes recursos en la minería donde la participación pública suele ser alta, debido a las fuertes inversiones que se requieren en ese sector que, en general, se considera estratégico.

En aquellos sectores de la economía donde la tecnología de producción de bienes y servicios se caracteriza por la existencia de importantes inversiones fijas y hundidas,
\footnote{Las inversiones hundidas son aquellas que una vez realizadas no pueden asignarse a otros usos más allá de los previstos originalmente.}
la provisión de estos servicios públicos resulta más eficiente si la realiza un número limitado de oferentes y, en algunos casos, sólo si lo hace una única empresa.
\footnote{A vía de ejemplo, la inversión prevista por ANTEL para instalar fibra óptica al hogar para la transmisión de datos alcanza a U\$S 550 millones, aproximadamente el 50\% de la facturación de la empresa en 2012.}
La necesidad de contar con activos fijos que tienen un costo muy importante y que en sí mismos permiten atender a toda la demanda, lleva a que la duplicación de esta infraestructura sea ineficiente desde el punto de
vista económico. Entre estos sectores se encuentran la transmisión de energía, agua potable, saneamiento, transporte aéreo y terrestre, minería o transmisión de datos. En varios países desarrollados y en desarrollo estas actividades las realizan empresas públicas, como puede verse en el gráfico \ref{fig:fig2} que presenta el porcentaje de empresas públicas por sector en las mayores 2.000 empresas.

\begin{figure}
\centering
\includegraphics{Regulacion_files/figure-latex/fig2-1.pdf}
\caption{\label{fig:fig2}Participación pública por sector económico en las 2.000 mayores empresas, años 2010-2011}
\end{figure}

Fuente: \citet{Kowalski2013}.

La discusión acerca de la propiedad de las empresas en estos sectores donde opera un monopolio natural adquiere relevancia debido a que, en general, los gobiernos no pueden suscribir contratos completos con las firmas privadas. \citet{Laffont1993} al analizar las diferencias entre una empresa pública y una privada regulada señalan que si los contratos fueran completos entonces la propiedad de las empresas sería irrelevante. En este caso el gobierno siempre podría diseñar un contrato y hacerlo cumplir de forma de que la empresa privada lleve a cabo los objetivos que se le impone; a su vez, la empresa privada podría realizar esos objetivos sin temer expropiación alguna por parte del gobierno.

La imposibilidad de suscribir contratos completos responde a distintos factores. Puede ser imposible prever todas las contingencias futuras o resultar costoso redactarlas, por lo cual, cualquier contrato entre las partes presentará lagunas o aspectos no resueltos. Aún si todas las contingencias se incluyen en el contrato, puede ser imposible verificar alguna de las variables relevantes. Por tanto, los contratos incompletos impiden regular efectivamente a las empresas, en la medida en que alguna de las variables relevantes no puede ser contratada \emph{ex ante}. En estos casos, el derecho de propiedad residual sobre los activos permite resolver los problemas contractuales (\citet{Hart1995}; \citet{Perotti2004}).

La literatura económica, por su parte, aborda extensamente la temática de las empresas públicas, principalmente, en países de menor desarrollo relativo.
\footnote{Véase \citet{Jones1982}, \citet{WorldBank1995} y, para un análisis de las empresas en los países socialistas, \citet{Roland2000}.}
Sin embargo la discusión de fondo sobre la propiedad de las empresas es un tema que se encuentra lejos de estar resuelto \citep{Hart2003}. Debe aclararse que la discusión relevante sobre la propiedad de las empresas, para este trabajo, es sobre aquellas que tienen características de monopolios naturales en todo o parte de sus segmentos.

Esta discusión tuvo su momento de mayor desarrollo en los trabajos de \citet{Lange1936}, \citet{Lange1937} y \citet{Lerner1944} sobre socialismo de mercado (ver \citet{Coloma2004}). Los argumentos económicos sobre los que se fundamenta la existencia de empresas públicas se basan en diversas fallas de mercado, como el control del poder de mercado de las empresas, el bajo desarrollo de los mercados de capitales, así como la necesidad de distribuir el riesgo de esas actividades entre el conjunto de la sociedad \citep{Sappington1987}. Asimismo, se basan en el supuesto de que el gobierno actúa en forma benevolente y que tiene como objetivo maximizar el bienestar de largo plazo de la sociedad, entre otros (\citet{Dixit1997}; \citet{Martimort2006}).

Teorías más modernas consideran las complejidades que tienen los gobiernos para llevar adelante las políticas. Desde la posibilidad de la captura por parte de grupos de interés, hasta la visión de la política como un proceso influido por distintos actores y grupos de interés, la visión de un gobierno imparcial y benevolente se ha transformado para introducir la idea de que las decisiones políticas tienen lógica propia \citep{Dixit1997}. En este marco, los costos de implementar una determinada política por parte del Estado pueden ser mayores a los beneficios que se buscan con ella.

En el capítulo @ref(\#reg-ec) se analizó la teoría de Shleifer (2005) que señala que la propiedad pública de empresas es una solución por defecto en aquellos casos donde el gobierno no logra alinear el comportamiento de las empresas con el interés de la sociedad mediante otras políticas públicas. En este marco, los gobiernos
tienen un continuo de instrumentos a su disposición para llevar a cabo sus actividades y balancean dos pérdidas sociales opuestas: la expropiación por parte de privados (desorden) y la expropiación por parte del Estado (dictadura).

Todas las estrategias de control social de los negocios son imperfectas, y el diseño institucional concreto implica elegir entre alternativas imperfectas. El ordenamiento privado aparece en un extremo, donde existe un costo importante en términos de pérdidas asociadas a la expropiación por parte de privados, pero baja por parte del Estado. El continuo pasa por jueces independientes y organismos reguladores, hasta llegar a la propiedad pública de las empresas en el otro extremo, donde los costos de expropiación privada son bajos, pero altos los de expropiación por parte del Estado.

\citet{Perotti2004} señala que coexisten dos tipos de problemas que pueden desembocar en la necesidad de que la propiedad de las empresas sea pública. Por un lado, el que denomina problema de compromiso público e implica la imposibilidad de algunos gobiernos de no renegociar sus políticas hacia el sector privado, lo que resulta en el desaliento de la inversión privada. \citet{Bergara2003} desarrolla esta visión enfatizando en el diseño institucional de las políticas públicas. Por otro, los que denomina problemas de compromiso privado, que hacen referencia a las dificultades del regulador para controlar en forma efectiva las decisiones de empresas privadas. En cualquiera de los dos escenarios,

la empresa pública aparece como un sustituto a la privada, debido a problemas de compromiso de distinta naturaleza.
Otras veces los problemas están vinculados al monto de los recursos necesarios para el desarrollo de estas actividades. Si los mercados de capitales están poco desarrollados y la economía no resulta atractiva para los inversores externos, entonces sólo los Estados pueden movilizar recursos para llevar a cabo determinadas actividades económicas (\citet{Mintz1982}; \citet{LaPorta1998}).

En el caso de Uruguay, este fue uno de los principales problemas que determinaron la conformación de las empresas públicas. \citet{Jones1982} señalan otros factores, políticos y sociales, que determinan el surgimiento de las empresas públicas, y los agrupan en cuatro:

\begin{enumerate}
\def\labelenumi{\arabic{enumi}.}
\item
  \textbf{Predilección ideológica}: es la creencia a priori de que ciertas formas de propiedad son preferibles. No necesariamente es una decisión irracional, a vía de ejemplo, señalan que con el fin de mejorar la distribución del ingreso, el medio utilizado por los países escandinavos ha sido los impuestos, mientras que en los países de Europa del Este ha sido la producción pública.
\item
  \textbf{Adquisición o consolidación de poder político o económico}: la propiedad y el control de unidades económicas son instrumentos para que los intereses de ciertos grupos avancen y otros se frustren.
\item
  \textbf{Herencia histórica}: independientemente de cuál haya sido el origen de las empresas públicas, una vez que se constituyen y permanecen durante un período significativo de tiempo, llevan a una fuerte inercia que hace facti-
  ble su permanencia más allá de aspectos ideológicos o de desempeño.
\item
  \textbf{Pragmatismo político}: la empresa pública es simplemente una herramienta, entre varias, que tiene el gobierno a su disposición para modificar el comportamiento de una economía mixta.
\end{enumerate}

Del punto de vista de las sociedades, alcanzar un determinado tipo de arreglo institucional (privado-público) implica valoraciones diferentes respecto al peso que se otorga a los distintos costos económicos ---eficiencia productiva y asignativa, accesos a financiamiento--- o políticos ---electorales, desorden social, entre otros---. En general, las empresas públicas operan en sectores de actividad que se caracterizan por ser monopolios naturales donde, como se analizó anteriormente, los productos son ampliamente consumidos; existen importantes economías de escala o de ámbito; y, las inversiones, en gran parte, son hundidas. Por lo que, en estos mercados, resulta imposible desentrañar lo económico de lo político (\citet{Bergara2003}; \citet{Spiller2013})

\hypertarget{caracteruxedsticas-de-las-empresas-puxfablicas}{%
\section{Características de las Empresas Públicas}\label{caracteruxedsticas-de-las-empresas-puxfablicas}}

Las empresas públicas son organizaciones híbridas \citep{Jones1982}. Por un lado, tiene características similares a cualquier empresa privada (vende un producto y realiza las funciones de producción, publicidad y marketing de sus productos) y enfrenta las presiones del mercado. Por otro, como organización pública controlada por el gobierno, enfrenta presiones directas o indirectas de la burocracia, los políticos, y el público en general. Sin embargo, esta mezcla de características hace que las empresas públicas sean unidades diferentes a los gobiernos y a las empresas privadas.

La principal característica del gobierno de las empresas públicas implica que éstas se enfrenten a múltiples principales (\citet{Dixit1998} y \citet{Dixit1997}; \citet{Martimort1996}; \citet{Tirole1994}). La teoría del principal-agente plantea que existe una relación de agencia cuando una parte (el principal) delega en otra la realización de una tarea (el agente). Las acciones del agente tienen impacto sobre el bienestar del principal y existe, además, asimetría de información entre las partes respecto a alguna de las variables relevantes que determinan el desempeño del agente. Cuando varios principales delegan en un mismo agente la realización de una tarea, cada principal procurará que éste realice la tarea según sus propios intereses u objetivos por sobre los de los demás.

Como consecuencia, las empresas deben perseguir múltiples objetivos, algunos de los cuales pueden llegar a ser antagónicos entre sí \citep{Tirole1994}. Estos abarcan tanto objetivos comerciales como no comerciales: beneficios, redistribución de ingresos, subsidios a regiones o sectores específicos, generación de empleo, obtención de divisas, aumentar la probabilidad de reelección del partido en el poder, entre otros. Esta multiplicidad de objetivos tiene impactos sobre la eficiencia productiva de las empresas públicas.

\citet{Miltnisky2001} señala que las empresas públicas juegan, en términos institucionales, dos juegos que tienen reglas distintas: el político y el económico.
\footnote{\citet{Dixit1997} señala que estas organizaciones operan directamente en un marco político.}
En este marco, las instituciones establecen lo que los agentes pueden o no hacer, para fomentar la cooperación entre ellos. Señala la imposibilidad de jugar los dos juegos a la vez, dado que los objetivos de un juego se contradicen con los del otro. El juego económico es evaluar la eficiencia en la provisión de los productos, mientras que el juego político busca captar o fidelizar votantes. El juego económico requiere normas flexibles de contratación y compra, el juego político impone normas rígidas similares a las del propio Estado. El juego económico requiere autonomía para el cumplimiento de los objetivos, el juego político establece restricciones a la forma de llevar a cabo el negocio. El juego económico requiere gerentes y funcionarios elegidos por sus capacidades, el juego político los elige según sus voluntades.

\citet{Martimort1996} destaca otra dimensión asociado a la naturaleza de múltiples principales en que se encuentran las empresas públicas. En este marco, las asimetrías de información provocan un efecto de free rider entre reguladores que reduce el control que cada uno ejerce sobre la empresa. Aun cuando las tareas de los reguladores sean sustitutas entre sí, el nivel de control que estos ejercen es menor al que ejercería un único principal.

\citet{Miltnisky2001} y \citet{Jones1982} enfatizan la idea de que es complejo medir los resultados de las empresas públicas debido a sus múltiples objetivos o reglas de juego. Alcanzar un objetivo -universalizar un servicio, como llevar energía eléctrica a todas las escuelas- puede ir en detrimento de otro -maximización de beneficios-. En otros términos, en la ecuación objetivo de quienes conducen empresas privadas entra el beneficio, mientras que en las empresas públicas el beneficio tiene un peso mucho menor dado que el principal objetivo pasa a ser obtener apoyo político para la gestión. En última instancia, los directores de empresas públicas son actores políticos que se mueven según sus propias reglas de juego.

Por último, los incentivos en las empresas públicas son más débiles que en las privadas \citep{Dixit1997}. La naturaleza multidimensional de las tareas que realizan estas empresas implica que incentivar un objetivo puede llevar a que otro de los objetivos no sea alcanzado. Por ello, los principales no pueden establecer mecanismos para incentivar al agente: el incentivo de uno será un desincentivo para otro. Si un ministerio quiere establecer premios a la eficiencia, pero otro busca premiar la cobertura del servicio, entonces los incentivos se cancelan entre sí y no hay cambios en el comportamiento del agente. En el caso extremo, si el número de principales es importante, se puede llegar a la inacción de la empresa.

\hypertarget{empresa-puxfablica-y-su-eficiencia}{%
\section{Empresa pública y su eficiencia}\label{empresa-puxfablica-y-su-eficiencia}}

A las empresas públicas, en la mayoría de los países, se las considera ineficientes desde el punto de vista productivo. Según \citet{Jones1982}, esto no es sorprendente debido a su doble característica de empresa y de organización pública que sufre presiones diversas, por un lado las del mercado y por el otro las del poder político. En esta dualidad de objetivos el resultado de la firma suele ser sub-óptimo para los estándares del mercado y/o del poder político. En consecuencia, la existencia de múltiples principales es, en sí mismo, una fuente importante de ineficiencia para las empresas públicas.

Perseguir objetivos diversos y muchas veces contradictorios, la mayor laxitud en los controles por parte de principales que hacen free riding entre ellos y la dificultad para establecer incentivos fuertes al desempeño se asocian para impedir que el resultado de las empresas públicas sea igual de eficiente al de las privadas.

La imposición de fines políticos a las empresas públicas \citep{Shleifer1994} tiene su contrapartida en controles diferenciados respecto a las privadas. En efecto, como la empresa pública está sujeta a controles políticos, los procesos de contratación de funcionarios, de compra e inversiones tienden a ser más rígidos que en una empresa privada. Además, las autorizaciones que requieren a los efectos de las inversiones, no necesariamente tienen que ver con la eficiencia sino que pueden atender a otros fines, como evitar que crezca el endeudamiento del país.

El control político de las empresas facilita su utilización como un instrumento más de la política, ya sea partidaria o de gobierno \citep{Miltnisky2001}, por lo que se requieren salvaguardas para impedir que las empresas sean vaciadas y quiebren en los casos en que no existan restricciones legales al respecto. A estos problemas se agregan los controles que enfrentan estas empresas por su naturaleza política. Las empresas públicas no pueden tener, por definición, las mayores libertades y flexibilidades de las privadas para contratar, licitar y comprar. Asimismo, muchas veces asumen en sus costos -en el marco de sus objetivos políticos implícitos- acciones que no realizan las empresas privadas, por las cuales no obtienen compensación del principal.

Por otra parte, la literatura abunda sobre la ineficiencia productiva de las empresas públicas y, como lo señalan \citet{Jones1982}, entre las principales causas de este comportamiento se encuentra que, en general, estas empresas:

\begin{itemize}
\item
  operan bajo sistemas burocráticos que controlan procesos antes que resultados, a través de procedimientos que demandan múltiples aprobaciones ministeriales y que generan demoras que pueden resultar muy costosas cuando las empresas operan en mercados dinámicos;
\item
  son utilizadas para transferir ingresos del sector público a diferentes grupos de interés, mediante mecanismos poco transparentes frente a quienes pagan el costo en términos de mayores impuestos, inflación y menores gastos gubernamentales en salud, educación y bienestar; y
\item
  el desempeño gerencial no puede evaluarse ya que resulta imposible distinguir un desempeño bueno de uno malo, puesto que cualquier efecto de las decisiones de los gerentes sobre los beneficios de la empresa pueden esconderse, imputando el mismo a políticas gubernamentales sobre precios que afectan sus productos y sus insumos. Así los gerentes tienen pocos incentivos a controlar los costos, porque pueden esconder las ineficiencias y, además, porque, en general, la estructura salarial pública no permite otorgar bonificaciones por los resultados obtenidos.
\end{itemize}

En conjunto, estos elementos ayudan a explicar por qué es más probable que las empresas públicas enfrenten un escenario de menor eficiencia productiva que las empresas privadas.
\^{}{[}Este resultado ha sido recogido en el informe del año 1995 del Banco Mundial \citep{WorldBank1995}.
El problema de ineficiencia de las empresas públicas es un tema ampliamente discutido en la literatura, en particular aquella que estudia la caída del sistema socialista \citep{Roland2000}. Esta literatura agrega una dimensión de compromiso dinámico adicional que enfrentan los gobiernos que tienen empresas a su cargo.

En general, la evidencia ha demostrado que muchas veces las empresas públicas actúan bajo un marco de restricciones presupuestales blandas. Este efecto, identificado inicialmente por \citet{Kornai1980} refiere a que los gobiernos no pueden comprometerse en forma creíble a no refinanciar a estas empresas cuando presentan pérdidas. Ello relaja ex ante los incentivos de las empresas a ser eficientes. El problema es de compromiso dinámico. El gobierno quiere comprometerse a no relajar la restricción de la empresa, pero como tiene activos hundidos en ellas, a la hora de ejecutar su amenaza la misma no es creíble.
\footnote{La referencia clásica es \citet{Dewatripont1995}. Véase también \citet{Bergara2011} para un análisis de cómo el gobierno puede imponer una restricción presupuestal blanda a las empresas.}
Sabiendo eso, la empresa tiene menores incentivos a esforzarse al inicio dado que, cualquiera sea el resultado, será refinanciada.

Debe señalarse que el problema de las restricciones presupuestales blandas no es un problema excluyente de las empresas públicas. En Uruguay muchas empresas privadas se han movido bajo esa lógica \citep{Vaz1993}. Sin embargo, en las empresas públicas uruguayas, hay un consenso implícito de que estas no quiebran.
{[}La Administración de Ferrocarriles del Estado (AFE) en Uruguay, recibe subsidios desde hace años y no ha cerrado aún.{]}
Ello determina que tarde o temprano los precios van a acomodar las cuentas de las empresas, o en algún momento algún director impulsará la gestión de la misma. En otros términos, las empresas responden en forma menos intensa a los incentivos de precio y, por tanto, este no sirve como el único instrumento regulatorio.

Un último elemento que señalan \citet{Sappington1987} refiere a que la propiedad pública de las empresas reduce los costos de transacción de renegociar los objetivos, frente a hacerlo con empresas privadas. Por ello, el gobierno no puede comprometerse a no renegociar sus objetivos cuando la empresa es pública, a menos que las privatice.

La discusión anterior presenta una división entre las posibles fuentes de eficiencia relativa de las empresas públicas. Por un lado, la propiedad de las mismas puede no generar los incentivos suficientes para una adecuada gestión, debido a los diversos objetivos y múltiples principales. Por otra parte, el entorno en el que operan las empresas públicas, donde muchas veces la competencia es limitada o existen restricciones presupuestales blandas no les permite ser eficientes.

\citet{Bartel2005} analizan estas fuentes de posibles ineficiencias y encuentran que ambas tienen un rol en la explicación de la eficiencia relativa de las empresas. Encuentran que las empresas públicas son menos eficientes que las privadas, dado un mismo entorno competitivo y financiamiento del gobierno. Sin embargo, el entorno en el que operan es relevante dado que sólo las empresas que reciben financiamiento del gobierno o están aisladas de la competencia tienen un peor desempeño que las empresas privadas.

En este sentido, sostienen que la privatización de las empresas y la reforma del entorno en el que operan son políticas sustitutas. Es decir, un incremento de la competencia o la reducción de subsidios permitirían alcanzar resultados similares a la privatización en términos de mejora de eficiencia. Este elemento, que muchas veces se soslaya, pone el énfasis en la necesidad de que las empresas públicas se muevan en un entorno competitivo y sin subsidios, siempre que ello sea posible.

En el cuadro \ref{tab:cuadro5} se resumen las principales diferencias entre empresas públicas y privadas, consideradas anteriormente.

\begin{table}

\caption{\label{tab:cuadro5}Diferencias entre empresas públicas y privadas}
\centering
\begin{tabular}[t]{l|l|l}
\hline
Dimensión & Empresa privada & Empresa pública\\
\hline
Principal & Único & Múltiplos\\
\hline
Objetivos & Maximización de beneficios & Múltiples, a veces difusos\\
\hline
Potencia de incentivos & Fuerte & Baja\\
\hline
Restricción presupuestal & Dura & Blanda\\
\hline
Ajuste ante imprevistos & Renegociación costosa & Flexible\\
\hline
\end{tabular}
\end{table}

Fuente: elaboración propia.

Por su parte, en la sección siguiente se desarrollará la idea de lograr un entorno competitivo en el que las empresas públicas operen, en el marco de los instrumentos regulatorios para dichas empresas.

\hypertarget{la-regulaciuxf3n-de-empresas-puxfablicas}{%
\section{La regulación de empresas públicas}\label{la-regulaciuxf3n-de-empresas-puxfablicas}}

\hypertarget{regulaciuxf3n-de-actividades-en-ruxe9gimen-monopolio}{%
\subsection{Regulación de actividades en régimen monopolio}\label{regulaciuxf3n-de-actividades-en-ruxe9gimen-monopolio}}

Para aquellos que ven a la producción de bienes y servicios por parte del Estado como una forma alternativa de resguardar el poder de mercado de las empresas, la propiedad pública de las empresas es una alternativa a la regulación (\citet{Shleifer2005}; \citet{Viscusi2005}). En consecuencia, en esta situación la regulación de las empresas públicas no tendría sentido ya que el principal problema que se considera en el monopolio privado -el poder de mercado- queda resuelto a través de la propiedad estatal.

Sin embargo, cuando operan empresas de propiedad estatal surgen problemas distintos al poder de mercado. En este caso, el principal problema a considerar, como lo muestra la evidencia empírica de los países del bloque socialista, es que las empresas públicas operan, muchas veces, en forma ineficiente y sus directores no son penalizados por ello.

Si bien la eficiencia es un problema al que también se enfrentan las empresas monopólicas privadas, en la medida en que no hay presión competitiva en esos mercados, esto resulta de segundo orden en relación con el problema principal del monopolio, el poder de mercado. En el caso de la empresa pública el orden de los problemas se invierte: la eficiencia productiva pasa a ser el problema fundamental mientras que el poder de mercado pasa a un segundo plano (\citet{Roland2000}, \citet{WorldBank1995}).

Por tanto, la regulación y la propiedad pública de las empresas no son conceptos sustitutos, sino que resultan complementarios. La regulación, cuando refiere a empresas públicas, persigue objetivos distintos a la regulación de empresas privadas. La regulación de empresas públicas debería tener como principal objetivo alcanzar la eficiencia económica, en particular la eficiencia productiva. La regulación debe inducir a las empresas públicas a ser costo eficiente, sustituyendo al mercado para generar esos incentivos. Ello sin descuidar la sustentabilidad de largo plazo que permita mantener los servicios en el futuro.

La naturaleza económica y política de las empresas públicas hace que su regulación sea más compleja que la de las empresas privadas. Para estas empresas la maximización de beneficio es uno de los múltiples objetivos que guían su accionar. Asimismo, dentro de cierto margen, operan bajo un régimen implícito de restricciones presupuestales blandas. Ambas características atentan contra la minimización de los costos.

\citet{Laffont1993} diferencian dos tipos de controles que puede establecer el gobierno. Por un lado, el control externo, que consideran las variables que vinculan a la empresa con los agentes externos a ella: consumidores (regulación de precios, calidad); competidores (regulación de entrada, tarifa de acceso); contribuyentes (auditoría de costo). Por otro, el control interno que implica el control de los insumos y del proceso de minimización de costos, a través de esquemas de incentivos, o la intervención sobre decisiones que involucran el empleo, la localización, el tipo de inversión o el endeudamiento.

En la regulación de estas empresas, el instrumento clave utilizado para incidir en el comportamiento de las empresas privadas, la fijación de la tarifa, pierde fuerza ya que por sí sola no puede resolver los problemas de eficiencia de las empresas públicas. Como estas empresas están sometidas a un régimen implícito de restricciones blandas, entonces la fijación de tarifas como instrumento regulatorio no es suficiente para transmitir los incentivos necesarios para la adopción de la tecnología eficiente. Si el regulador conoce el costo de la tecnología eficiente, y la eficiencia fuera el único objetivo, podría fijar el precio de forma de que sea igual a este costo.

Sin embargo, si este precio implica pérdidas para la empresa, y esta sabe que no quebrará, entonces este instrumento no genera los incentivos para la adopción de una mejor tecnología. En la terminología de \citet{Laffont1993} se requiere influir sobre las variables de control externo, pero también sobre las de control interno para alcanzar la eficiencia. Por tanto, la regulación debe combinar la regulación de precio con la regulación del gobierno corporativo, incentivando las buenas prácticas empresariales \citep{Berg2013}.

La regulación de precio debe complementarse con otras medidas de gestión (gobierno corporativo) que atiendan a la naturaleza híbrida de las empresas. \citet{Andres2011} presentan una revisión del estado del gobierno corporativo de las empresas públicas en América Latina, y encuentran una gran dispersión en la incorporación de mejores prácticas, con una mayor incorporación de las mismas en las empresas de energía eléctrica con relación a las de agua.

Las buenas prácticas de gobierno corporativo de la OECD incluyen una serie de criterios generales para el manejo de las empresas públicas. Transparencia y divulgación; explicitación clara de los objetivos; separación de los roles productivos y reguladores; responsabilidad, competencia y objetividad de los directores de las empresas, son conceptos que apuntan a una adecuada y eficiente gestión de las empresas, independientemente de sus objetivos \citep{OECD2011}.

En este marco, el regulador, como parte del Estado pero también como órgano técnico independiente, tiene que participar en la determinación técnica de las tarifas de las empresas públicas, aun cuando este instrumento tenga un menor efecto como incentivo para inducir el comportamiento de estas empresas que en el caso de las empresas privadas, ya que se requiere un control técnico que sirva como contralor y balance de la ineficiencia relativa de las primeras.

El problema real que enfrenta la regulación de empresas públicas, es definir el papel legítimo de los diferentes órganos estatales y las dimensiones del control social que se requiere en el funcionamiento de estas empresas. En la literatura de la administración pública, el problema es el balance entre la autonomía suficiente para desarrollar las actividades del negocio sin las dificultades que implican las rigideces inherentes a las estructuras burocráticas del gobierno, y la rendición de cuentas al gobierno y al Parlamento, organismos que dictan los objetivos de estas empresas.

\citet{Ramamurti1991} establece que la autonomía gerencial es el método que asegura que estas empresas realicen en forma eficiente sus actividades, siempre y cuando se combinen con una rendición de cuentas transparente y a tiempo. Este autor señala que el control de la actividad de las empresas tiene dos enfoques posibles. En primer lugar, el control puede ser cualitativo y enfocarse sobre los resultados de la empresa. En este caso, la atención se centra en establecer los objetivos y otorgar autonomía a la empresa para elegir la tecnología para alcanzarlos.

En segundo término, el enfoque puede estar en el control cuantitativo de los procesos de producción o, de otra forma, estudiando la forma en la que los insumos se convierten en productos. Ello se traduce en controles constantes de forma de verificar el proceso de producción.

En el primer caso, existe una importante autonomía de gestión u operativa, mientras que en el segundo existe una gran autonomía para fijar los objetivos estratégicos de la empresa. En términos generales, los gerentes deberían tener baja autonomía en la definición de la estrategia y alta autonomía en materia operacional, mientras que los gobiernos deberían ser activos en la fijación de objetivos, estrategias y políticas, otorgando a los gerentes amplia libertad en la implementación de los programas aprobados.

Sin embargo, en el marco del gobierno corporativo, los gerentes de las empresas públicas tratan de influenciar el proceso de especificación de metas y resistir la imposición de ciertos objetivos, al tiempo que reclaman el derecho de participar en su formulación. Asimismo, según \citet{Jones1991} muchos gobiernos no están organizados para realizar un control por resultado eficiente y, en consecuencia, tratan de controlar una variedad de procesos internos, resultando en excesivo control de baja calidad.

Por su parte, \citet{Aharoni1982} sostiene que en muchos países las estructuras de control no están bien definidas y el ejecutivo (consejo, directorio) no puede disciplinar o reemplazar al gerente, al menos sin una larga negociación con el ministro y otros agentes gubernamentales. Asimismo, considera que la experiencia muestra que en las firmas más grandes, más independientes del gobierno (genera sus propios recursos) y que utilizan información más técnica para su operación, mayor es el poder que se concentra en la gerencia y, por lo tanto, los gerentes tienen más grados de libertad para fijar las metas a alcanzar por la empresa. En estos casos, el regulador es el único que puede ejercer el balance en el control de las decisiones técnicas de las empresas públicas.

En otros términos, el regulador es el órgano técnico que puede asesorar a la toma de decisiones cuando existe asimetría de información entre los gerentes de las empresas y el sector político que define los lineamientos. En este marco, en Uruguay es claro que hay que repensar el rol que cumplen los órganos reguladores sectoriales. Hay que incorporar nuevas funciones a su menú de instrumentos y balancear sus roles de regulador y asesor en la fijación de las distintas variables, tanto internas como externas a la empresa.

\citet{Ramamurti1991}, por su parte, señala una serie de barreras que enfrentan los gobiernos a la hora de incrementar la calidad de los controles sobre las empresas públicas. Estas incluyen:

\begin{enumerate}
\def\labelenumi{\arabic{enumi}.}
\tightlist
\item
  barreras técnicas: definir el criterio más adecuado para medir la actuación de las empresas, discernir los resultados asociados a las decisiones gerenciales, balancear las metas de corto y largo plazo a través de planificación estratégica; y
\item
  barreras organizacionales: asimetría en la experiencia e información, conflicto entre los múltiples objetivos, dificultades de coordinación en el ámbito de una organización imprecisa como el Estado.
\end{enumerate}

Sólo organismos reguladores independientes y eficientes pueden levantar las barreras mencionadas y colaborar en la valoración de los objetivos no económicos, bajo el entendido que cualquier objetivo de estas empresas se desarrolla a través de instrumentos económicos y, por tanto, son siempre cuantificables ex post.

\citet{Berg2013}, por su parte, señala que la tendencia reciente en materia regulatoria lleva a establecer agencias reguladoras relativamente independientes con el objetivo de reducir el poder de los ministerios gubernamentales responsables de la política sectorial sobre el funcionamiento de las empresas públicas. Por otra parte, la creación de las agencias reguladoras, en última instancia, implica poner un principal más a la empresa pública, que ya tiene diversos principales que fijan sus objetivos. Estas agencias deben jugar un papel en supervisar al sector en que operan las empresas públicas, promover la transparencia, y establecer incentivos para mejorar los resultados, y no debería fijar objetivos nuevos para la misma.

A su vez, la existencia de múltiples objetivos en las empresas públicas, introduce un problema adicional para la regulación. Esta realidad determina que es poco factible contratar objetivos con las empresas, dado que el sistema político tiene incentivos a revisar y renegociar estos objetivos. Por un lado, algunos objetivos pueden entrar en conflicto entre sí. Pero por otro, si los objetivos de los múltiples principales cambian en forma muy volátil, no es posible determinar si es la dirección de la empresa o los cambios los que provocan el resultado observado. Esta diversidad y cambio reducen los incentivos que los propios contratos de desempeño pretenden incentivar. Por tanto, no se puede evaluar contractualmente el desempeño de los directores o gerentes (\citet{Martimort1996}; \citet{Dixit1997}).

La regulación de las empresas públicas debe, por tanto, incluir diversas dimensiones. En primer lugar, el regulador debe fijar el precio del servicio, no dejarlo librado a la empresa dado que ello agudiza los problemas de ineficiencia productiva. En este marco, un criterio creíble sería un mecanismo de determinación de precios techo o incrementos máximos para la empresa, determinados para un horizonte de mediano plazo. Ello da previsibilidad a la empresa para planificar su proceso productivo, por un lado, y, por otro, otorga credibilidad al regulador al no someter la política regulatoria a renegociación constante.

En segundo lugar, las empresas públicas deben adoptar una serie de principios que atiendan a las mejores prácticas de gobierno corporativo (transparencia, auditoría, etc.). Estas deben incluir, también, la separación contable de las actividades y transparentar los subsidios cruzados entre ellas, así como el análisis costo beneficio de los objetivos perseguidos independientemente de su naturaleza.

En tercer lugar, se debe transitar a un esquema donde los principales establecen los objetivos o resultados esperados por la empresa de la forma más objetiva y cuantificable posible. La empresa es la que debe llevar esos objetivos a la práctica decidiendo el proceso productivo más eficiente en el marco de la autonomía técnica que posee.

En cuarto lugar, existe un rol para que los reguladores asesoren a los organismos involucrados en el control de gestión de la empresa, en particular a aquel que tiene injerencia sobre el presupuesto y las inversiones. Por otra parte, las empresas públicas están sometidas a fuertes controles presupuestales y de inversión. Es sobre estos controles que el órgano regulador puede aportar la visión técnica del mercado.

En última instancia, el regulador es el órgano con las capacidades técnicas para evaluar la coherencia del conjunto de objetivos, procesos y tarifas propuestos, así como el presupuesto definido para alcanzarlos. Algunos de estos cometidos pueden encontrarse dispersos entre distintos organismos, mientras que otros no son controlados.

\hypertarget{regulaciuxf3n-de-actividades-en-competencia}{%
\subsection{Regulación de actividades en competencia}\label{regulaciuxf3n-de-actividades-en-competencia}}

El problema regulatorio es diferente cuando las empresas públicas compiten con otras empresas privadas en un sector de actividad. Aquí el principio general debería ser el de no discriminación e iguales reglas de juego para todos los actores, en particular la competencia entre todos los agentes. En estos mercados la competencia puede adoptar las dos dimensiones: competencia en el mercado, o competencia por el mercado. Aislar a las empresas públicas de la competencia, al igual que a las empresas privadas, sólo lleva al agravamiento de la ineficiencia productiva.

Hay un elemento común que caracteriza a estos mercados: las empresas públicas tienen algún monopolio en algún eslabón de la cadena de valor del sector. Por ejemplo, la generación de energía eléctrica es un mercado en competencia, pero la transmisión opera en un segmento monopólico; la telefonía fija opera en un mercado monopólico, mientras la telefonía móvil es un mercado competitivo; el refinamiento e importación de combustibles se realiza en régimen de monopolio, mientras que la distribución opera en competencia.

El regulador debe necesariamente intervenir en la interfase entre los segmentos monopólicos y competitivos de forma de evitar que las empresas públicas puedan tomar acciones que las beneficien a sí mismas, o incidan limitando la competencia en los segmentos competitivos. No se requiere que la empresa pública tenga intereses directos sobre ese segmento para que se produzcan distorsiones. A vía de ejemplo, en Uruguay la empresa pública de combustibles (ANCAP) tiene contratos con las estaciones que dispensan gasolina, los que tienen impactos sobre la competencia en ese mercado.

En resumen, el objetivo de la regulación en los mercados donde la competencia es posible y operan empresas públicas debe ser fomentar la competencia y fijar reglas iguales para todos los actores. Para ello, debe instrumentarse la separación contable de las actividades de estas empresas y hacer explícitos los subsidios cruzados entre sus actividades, si alguna de ellas fuera monopólica.

\hypertarget{conclusiones}{%
\section{Conclusiones}\label{conclusiones}}

El análisis realizado permite comprender el funcionamiento y las características de las empresas públicas y su vinculación con los gobiernos o sea sus propietarios.

Si existe la posibilidad de suscribir contratos completos entre el gobierno y las empresas que llevan adelante la producción, entonces la propiedad de las empresas resulta irrelevante. En la medida en que existen costos de transacción, los contratos entre las partes serán incompletos y la propiedad pasa a ser relevante. En estos casos, la propiedad pública de empresas es una solución por defecto, en los casos donde el gobierno no logra alinear el comportamiento de las empresas con el interés de la sociedad mediante otras políticas públicas.

Empresas públicas y privadas difieren en dos aspectos. En primer lugar, las primeras responden a múltiples principales. Ello se traduce en un conjunto de objetivos económicos, sociales y políticos, mientras que las privadas sólo se guían por objetivos económicos. Ello determina que las empresas públicas jueguen, en términos institucionales, dos juegos con reglas distintas: el político y el económico. En segundo término, las empresas públicas operan bajo un régimen implícito de restricciones blandas. Si a esto se suma que tienden a actuar en mercados monopólicos, entonces el resultado es potencialmente peligroso para la eficiencia de estas empresas.

En este marco, la regulación es un instrumento complementario de la propiedad pública. A diferencia de las empresas privadas que requieren ser reguladas para evitar el abuso de posición dominante, las empresas públicas requieren de la regulación para inducir un comportamiento eficiente. Los múltiples objetivos y las restricciones blandas llevan a que la eficiencia pueda ser un objetivo inferior con relación a otros fines de la empresa. Por otra parte, los distintos ministerios sectoriales que tienen injerencia sobre estas empresas no tienen como objetivo de política lograr la eficiencia de las mismas, y si este fuera uno de sus objetivos no podrían controlarlo debido a que se mueven con asimetría de información respecto de las mismas.

El problema que enfrenta la regulación de empresas públicas es definir el papel legítimo de los diferentes órganos estatales y las dimensiones del control social que se requiere en el funcionamiento de estas empresas. En términos de administración pública, alcanzar el balance adecuado entre autonomía suficiente para desarrollar las actividades del negocio sin las dificultades que generan las estructuras burocráticas del gobierno, y la rendición de cuentas transparente y en tiempo a los organismos que dictan los objetivos de estas empresas (gobierno y Parlamento).

La existencia de organismos reguladores independientes y eficientes permite levantar las barreras técnicas y organizacionales que enfrentan los gobiernos a la hora de incrementar la calidad de los controles sobre las empresas públicas.

La regulación de empresas públicas debería comprender las siguientes dimensiones: el regulador debería fijar el precio techo o incrementos máximos para la empresa para un horizonte de mediano plazo; las empresas deberían adoptar principios que atiendan a las mejores prácticas de gobierno corporativo; los principales deberían establecer los objetivos o resultados esperados de forma objetiva y cuantificable; los reguladores deberían asesorar a los organismos involucrados en el control de gestión de la empresa (presupuesto e inversiones).

Cuando las empresas públicas participan en mercados en competencia, el regulador debe intervenir en la interfase entre los segmentos monopólicos y competitivos de forma de evitar que las empresas públicas realicen acciones que las beneficien o limiten la competencia en los segmentos competitivos. En definitiva, el regulador debe fomentar la competencia y fijar reglas iguales para todos los actores.

\hypertarget{eepp-uy}{%
\chapter[Empresas públicas: el caso uruguayo ]{\texorpdfstring{Empresas públicas: el caso uruguayo \footnote{Basado en el marco del acuerdo de cooperación entre la Facultad de Ciencias Sociales de la UdelaR y la Unidad Reguladora de los Servicios de Energía y Agua (URSEA) y contó con financiamiento de la Corporación Andina de Fomento (CAF).}}{Empresas públicas: el caso uruguayo }}\label{eepp-uy}}

Rosario Domingo
Leandro Zipitría

Como se discutió en el capítulo \ref{reg-eepp}, la existencia de empresas públicas es un fenómeno habitual y económicamente significativo, pudiendo encontrarse empresas públicas en diversos sectores relevantes de la actividad económica en la mayoría de los países. En el caso de la economía uruguaya, en particular, las empresas públicas también son actores relevantes. Los servicios que prestan estas empresas alcanzan al 12\% de la canasta del Índice de Precios al Consumo (IPC),
\footnote{Se consideraron los rubros agua, saneamiento, electricidad, gas por red, supergas, nafta, gas oil, telefonía fija y celular.}
los gastos en que incurren representan un 10\% del Producto Bruto Interno (PBI),
\footnote{\citet{WorldBank2014}}
y algunas de ellas representan una fuente de ingresos para el gobierno central.
\footnote{\citet{WorldBank1995} e información del MEF tomada del siguiente \href{http://www.mef.gub.uy/indicadores.php}{sitio}.}

La literatura económica, por su parte, aborda extensamente la temática de las empresas públicas, principalmente, en países en vías de desarrollo.
\footnote{En el capítulo @ref(\#reg-eepp) se desarrolla esta temática. Una bibliografía de referencia es \citet{Jones1982}, \citet{WorldBank1995}, y para un análisis de las empresas en los países socialistas, \citet{Roland2000}.}
En este marco, este trabajo analiza la existencia de empresas públicas en América Latina, el proceso de creación y la situación actual de las empresas públicas en Uruguay, así como de las unidades reguladoras correspondientes. La primera sección presenta brevemente la evolución de las empresas públicas en la región. La segunda presenta la génesis y creación de las empresas públicas en Uruguay. En la tercera se analiza la situación de los principales mercados en que operan las empresas públicas uruguayas, mientras que en la cuarta y en la quinta se consideran los mecanismos de regulación establecidos para estas empresas en América Latina y en Uruguay, respectivamente. Finalmente, se caracterizar el marco jurídico institucional en el que operan empresas y reguladores en Uruguay.

\hypertarget{empresas-puxfablicas-en-amuxe9rica-latina}{%
\section{Empresas Públicas en América Latina}\label{empresas-puxfablicas-en-amuxe9rica-latina}}

Desde fines del siglo XIX se pueden encontrar en muchos países latinoamericanos, empresas de propiedad estatal actuando en sectores claves de la economía. A fines de ese siglo y comienzos del siglo XX, estos países se encontraban en una fase primaria de su desarrollo económico. Los problemas técnicos que enfrentaba el Estado para poder regular los sectores de infraestructura de transporte y comunicación, necesarios para el desarrollo de la actividad exportadora, y las fallas en los mercados de capitales fomentó el surgimiento de empresas estatales en estos mercados. En general, la presencia del Estado en estas actividades respondía a la sensibilidad política de las mismas (suministro de agua potable, transporte, etc.) al hacerse cargo de servicios que dejaban de atender las empresas extranjeras.

A partir de los años treinta del siglo pasado y como consecuencia de la crisis mundial que generó dificultades para adquirir productos manufacturados, los países de la región se plantean el desarrollo de una industria liviana orientada al mercado interno. Este proceso fue promovido por nuevos actores políticos, económicos y sociales que demandaban un papel más activo del Estado en las áreas que se consideran estratégicas para el crecimiento económico. En las décadas del 30 y del 40 puede observarse una creciente participación del Estado en los sectores de energía e infraestructura.

En este período se producen, también, una serie de nacionalizaciones de empresas extranjeras que operaban en algunos servicios públicos y en sectores extractivos, donde se concentraban las riquezas básicas de muchas de estas economías. Estas nacionalizaciones se debieron tanto a motivos políticos, como al cese de las concesiones otorgadas a las empresas extranjeras y a la necesidad de rescatar aquellas empresas que presentaban muy baja rentabilidad.

A mediados del siglo XX, como consecuencia del agotamiento del modelo exportador primario y de las propuestas de desarrollo surgidas desde la CEPAL, la mayoría de los países de la región iniciaron un proceso de sustitución de importaciones que implicó desarrollar una industria pesada, para lo cual fue necesario el apoyo de los estados nacionales. Este apoyo no sólo incluyó el diseño de diversas políticas económicas -inversión pública, cambiaria, arancelaria, fiscal, etc.- sino que también implicó que el Estado actuara como empresario, con el objetivo de suministrar bienes y servicios necesarios para el proceso de industrialización.

En el período 1950-1970, las empresas de propiedad estatal se multiplicaron y se expandieron a diversos sectores: producción de insumos, infraestructura industrial y servicios públicos que utilizaba el sector industrial privado -generalmente a tarifas y precios subsidiados-, así como producción de bienes de consumo final. En este proceso, como el desarrollo de los grandes proyectos industriales implicaban recursos que el capital privado nacional no disponía, la opción del Estado empresario fue, en muchos casos, preferida a la inversión extranjera. Por tanto la participación del sector público en la actividad económica creció sustancialmente.

Si bien el origen de las empresas públicas en América Latina es heterogéneo, según \citet{Morales1990}, los principales motivos para su creación fueron:
- suministrar servicios públicos tradicionales (electricidad, agua, transporte);
- sustituir viejos monopolios coloniales en manos de empresas extranjeras, a través de la nacionalización de estas empresas;
- cubrir la falta de inversión privada, sobre todo en sectores de baja rentabilidad o de alto riesgo;
- apoyar la ejecución de políticas o planes económicos mediante la realización de actividades estratégicas o inexistentes pero necesarias para el desarrollo económico del país;
- comprar empresas privadas en quiebra con el objetivo de mantener el empleo y la producción;
- evitar la penetración extranjera en actividades tecnológicas de punta; y
- cubrir necesidades sociales.

En los setenta, como consecuencia de la crisis internacional y el agotamiento del modelo de sustitución de importaciones como motor del crecimiento, en muchos países de la región las empresas públicas fueron el principal inversor mediante la obtención de créditos internacionales. Esto implicó una expansión de la participación de las empresas de propiedad estatal en la generación del producto y un incremento del endeudamiento externo de los países. La excepción fue Chile, donde entre 1974 y 1978 se produjeron las primeras privatizaciones de empresas públicas (se vendieron 550 empresas) aunque se excluyeron aquellas vinculadas a los servicios públicos \citep{Estache2004}.

La crisis de la deuda determinó que en muchos países latinoamericanos se replanteara el papel del Estado como empresario. Influyeron, además, las recomendaciones que los organismos internacionales de crédito realizaron con el objetivo de que estos países pudieran superar los acuciantes problemas económicos que enfrentaban (inflación, alto endeudamiento externo, crisis de balanza de pagos, falta de inversión, etc.)

En los ochenta, evaluaciones realizadas sobre el desempeño del Estado empresario en América Latina señalan una serie de problemas. Entre ellos cabe mencionar: (i) la falta de claridad y jerarquización de los objetivos de las empresas; (ii) deficiencia en la gestión empresarial y falta de un cuerpo profesional de administradores; (iii) menosprecio de la eficiencia y la racionalidad económica; y (iv) excesiva politización del manejo de estas empresas (\citet{Morales1990} a partir de las conclusiones del seminario regional sobre reestructura económica en América Latina organizado por CLAD-ILPES-INAP en 1988).

Esta situación la ilustra citando la definición de una empresa hipotética realizada por \citet{Kelly1985}: ``Se crea una empresa pública grande, dotada de recursos que superan con creces los de la gran mayoría de las principales empresas del país. Su presidente necesariamente debe ser amigo del partido de gobierno y es de libre remoción. Podrá nombrar gerentes que a él mejor le parezca, pero los salarios que les ofrece alcanzan menos de la mitad de lo que ganan sus colegas en la empresa privada. La gerencia no puede fijar los precios de venta unilateralmente; no puede despedir a los trabajadores fácilmente; no puede cambiar ciertos proveedores; tiene que dar créditos que nunca se cobran a ciertos clientes (\ldots)''.

Es así que a fines de los ochenta y fundamentalmente en los noventa se produce, en la mayoría de los países de la región, un proceso de privatización de empresas públicas con diferente grado de profundización según los países. En 1996, Argentina, Bolivia y Perú habían privatizado más de la mitad de los activos de las empresas propiedad del Estado, México cerca de un cuarto, Brasil un 12\%, y Chile y Venezuela un 7\% \citep{Ramamurti1999}. \citet{Estache2004} señalan que este proceso implicó que 1.500 empresas públicas fueran privatizadas o cerradas, generando un flujo de recursos importante para los gobiernos.

Las privatizaciones no tuvieron las mismas características e impacto en los diferentes países y sectores. Por otra parte, no lograron la profundidad y celeridad que algunos organismos internacionales, como el Banco Mundial, proponían en sus recomendaciones de política a la salida de la crisis de la deuda. El propio Banco Mundial señala que si bien las privatizaciones son buenas del punto de vista económico, en raras ocasiones también son buenas desde una óptica política \citep{WorldBank1995}.

\hypertarget{empresas-puxfablicas-en-uruguay}{%
\section{Empresas Públicas en Uruguay}\label{empresas-puxfablicas-en-uruguay}}

Los estudios económicos sobre empresas públicas son escasos en Uruguay. En general, se centran en el análisis de temas específicos como demanda, precios, tarifas, eficiencia o productividad. Asimismo, la temática de las empresas públicas también se aborda en el marco de estudios sobre cambios institucionales, proceso de diseño e implementación de políticas y reforma del Estado.

Por otra parte, en el marco de la discusión sobre la privatización de las empresas del Estado desarrollada a comienzos de los 90, se encuentran algunos estudios con un perfil de historia económica y social que buscaron analizar la génesis de estas empresas y el papel de las mismas en la historia económica y social del Uruguay moderno.

Tanto \citet{Nahum1993} como \citet{Solari1983} datan en las últimas dos décadas del siglo XIX la génesis de las empresas públicas en Uruguay. En este período opera un incremento de la acción del Estado en la economía como parte de un proceso de modernización y consolidación de lo que \citet{Nahum1993} denomina ``la élite profesional del gobierno''. Es en esos años que se aprueban leyes sobre los ferrocarriles (1884 y 1887), el Estado construye el puerto de Montevideo (1901) y administra, en forma provisoria, los servicios de energía eléctrica (1987-1906).

A partir de comienzos del siglo XX, en el período denominado ``primer batllismo'', se incrementa sustancialmente la acción del Estado en la esfera económica-productiva del país. Según \citet{Nahum1993}, este proceso se da bajo el influjo de una élite de políticos profesionales que ``veían en el poder público su instrumento y su medio de vida'', y por la falta de un empresariado nacional con iniciativa y capital para desarrollar alguno de los emprendimientos necesarios para la modernización del país. En este período, se estatizan los bancos (Banco de la República Oriental del Uruguay-BROU y Banco Hipotecario del Uruguay-BHU), se nacionalizan los seguros, se establece el monopolio de la producción de energía eléctrica y de los servicios portuarios, y se aprueban una serie de leyes sociales que caracterizan el desarrollo del Uruguay moderno.

\citet{Nahum1993} señala diversas razones que fueron fundamentales para que en el primer batllismo se concentrara la creación de empresas públicas o la estatización de empresas que, hasta esa fecha, eran propiedad tanto de capitales extranjeros como de capitales nacionales. Entre ellas pueden sintetizarse las siguientes:

\begin{enumerate}
\def\labelenumi{\arabic{enumi}.}
\item
  \textbf{Económicas}: bajar el precio de los servicios; mejorar la calidad los servicios; contribuir a las necesidades fiscales y sociales del Estado (bajar impuestos y sustituir impuestos indirectos a los efectos que la carga fiscal no lleve a una mayor desigualdad en la distribución de la riqueza); impedir el drenaje de oro al exterior, a través de las ganancias que remitían las empresas extranjeras; y consolidar la ``soberanía nacional'' y el ``desarrollo''.
\item
  \textbf{Sociales}: solidaridad, como fin de las acciones del Estado; extensión de los servicios; y proporcionar servicios de interés general.
\item
  \textbf{Políticas}: el Estado como representante de los intereses superiores de la sociedad, por encima de las clases sociales, y como impulsor del progreso mediante el crecimiento sostenido de la economía, lo que le otorgaba derecho para ``invadir'' el campo de la actividad económica privada dado que el ``interés general'' es superior al particular de las empresas; defensa de la soberanía económica, fundamentalmente a través del rechazo al poderío que ejercían las empresas, principalmente las extranjeras; y defensa del Estado como buen administrador aduciendo que se podían formar organismos públicos completamente independientes de la política y sin los defectos de la burocracia.
\end{enumerate}

De la revisión de los motivos que llevaron a la creación de las empresas públicas en Uruguay, se puede observar la conjunción de los siguientes fenómenos: (i) problemas de fallas de mercado que impidieron el desarrollo de los servicios, como ser mercados de capitales poco desarrollados para movilizar los recursos o empresarios adversos al riesgo para llevarlos a cabo; (ii) una importante debilidad institucional del Estado para poder controlar la forma en la que las empresas privadas desarrollaban sus funciones, que lo obligó a asumir un rol productivo;
\footnote{Ello forma parte de la falta de compromisos privados que señala \citet{Perotti2004} y que refiere a las limitaciones que enfrentan los Estados para poder controlar y regular en forma efectiva decisiones de los agentes privados.}
(iii) este último problema llevó no sólo a la creación de empresas públicas, sino también a otorgarles el monopolio para el desarrollo de las actividades y, en algunos casos, como en la telefonía, el rol regulador del propio sector.

El marco en el cual las empresas públicas uruguayas operaron, hasta hace un par de décadas, era reflejo de las debilidades técnicas, institucionales y políticas que enfrentaba el Estado para poder regular en forma efectiva a estos sectores. Muchos de los diversos controles que deben enfrentar las empresas públicas están diseñados en la Constitución de la República, sin considerar las capacidades de los diversos organismos técnicos para realizar un control adecuado de las mismas.

Según \citet{Nahum1993} las empresas públicas fueron creadas, en su mayoría, para atender finalidades sociales, sobre todo a través de la consecución de objetivos económicos específicos. Al inicio de su actividad ampliaron y abarataron los servicios de interés general; contribuyeron a Rentas Generales permitiendo suplir la recaudación de algunos impuestos; redujeron el déficit externo al disminuir las transferencias al exterior por ganancias o beneficios; contribuyeron a afirmar el papel asistencial del Estado; y dieron espacio y poder al Estado para incidir con fuerza en la vida económica.

Sin embargo, a partir de 1930 se produce un proceso de reducción de la eficiencia y de la buena administración de alguna de estas empresas, fundamentalmente por el incremento del número de funcionarios contratados. Esta situación se agrava a partir de la década del 50, debido al estancamiento económico que sufrió Uruguay y que llevó a que el Estado absorbiera mano de obra que no necesitaba, a los efectos de evitar las altas tasas de desempleo. Lo anterior, unido al clientelismo político, puso a la gestión de las empresas del Estado al servicio exclusivo del sistema político lo que provocó, entre otros males, un deterioro del resultado de estas empresas llegando a cuestionarse su viabilidad económica. \citet{Solari1983} señalan que a partir de la segunda posguerra se observa un proceso en el cual se hizo uso del aparato estatal para lograr mecanismos de redistribución del ingreso, como instrumento para asegurar la subsistencia del sistema político.

Este cambio es consistente con el gradual proceso de mutación de los objetivos de las empresas públicas. Al ser empresas controladas por el sistema político, siguen los avatares de la economía política. A principios del siglo XX la economía uruguaya era sólida respecto a la economía internacional, lo que se mantuvo hasta el final de la década del 30. Ello permitió que el Estado aumentara su papel en la economía.

A partir de la década del 50, en el marco de una economía estancada, las empresas públicas pasan a ser un instrumento más de la política económica, es decir, cambia la ponderación relativa de sus diversos objetivos. \citet{Rama1990} señala un crecimiento importante en el número de funcionarios de la administración pública en general, y de los entes públicos en particular, entre los años 1951 y 1957, como consecuencia de un shock positivo -pero transitorio- que enfrenta el país. Las cifras que presentan \citet{Bertino2012} y que se reproducen en el siguiente gráfico, son elocuentes respecto al incremento sistemático en el número de trabajadores de las empresas públicas, hasta la década de los setenta.

\begin{figure}
\centering
\includegraphics{imagenes/fig5.jpg}
\caption{Evolución del número de funcionarios en cinco empresas públicas (1912-2010).}
\end{figure}

Fuente: \citet{Bertino2012}, gráfico 4.

En los 90, predominaban en la región y en el resto del mundo las propuestas de reformas orientadas al mercado, con el objetivo de promover la eficiencia económica y el crecimiento. \citet{Forteza2003} mencionan las políticas desarrolladas en esos años, con el objetivo de promover la competencia, la apertura comercial, la privatización y el desmantelamiento de los monopolios de las empresas públicas. Estas medidas se planteaban bajo el supuesto que las mismas permitirían generar mayores niveles de competencia, promover la inversión privada y el progreso técnico y, en consecuencia, una asignación de recursos más eficiente.

Asimismo, sostienen que los cambios tecnológicos, observados hacia fines del siglo pasado, habían modificado la vieja concepción de que los servicios públicos constituían monopolios naturales y, por tanto, surgían propuestas de incentivar la competencia en dichos mercados con el objetivo de reducir los costos para el consumidor final y, fundamentalmente, para las empresas, lo que mejoraría la competitividad de la economía.

Bajo estos supuestos, en la mayoría de los países de la región se produjeron procesos o intentos de privatización de empresas estatales que operaban en diferentes mercados, entre ellas las empresas de servicios públicos. Según \citet{Bergara2005} los intentos de privatización en la región tuvieron resultados parciales, observándose una mayor aceptación en aquellos países donde la provisión estatal de servicios públicos era sumamente deficiente.

Si bien en Uruguay las empresas públicas tenían algunas características que las diferenciaban de lo que sucedía en otros países de la región (no tenían grandes déficits y realizaban una cobertura del servicio bastante amplia) las mismas presentaban deficiencias en lo relativo a la calidad del servicio que prestaban \citep{Forteza2003}.

En este marco el gobierno de la época intenta aplicar reformas orientadas al mercado, principalmente, en los sectores de energía eléctrica, comunicaciones y combustibles. La finalidad de las reformas era incrementar la competencia en los mercados de los servicios públicos y privatizar, parcialmente, a las empresas públicas. Para ello se propusieron leyes que modificaban el alcance de los monopolios de las empresas públicas y posibilitaban la incorporación de capitales privados \citep{Bergara2005}.

La \href{http://www.parlamento.gub.uy/leyes/ley16211.htm}{Ley No.~16.211} de Empresas Públicas, aprobada por el Parlamento en 1992, establecía el cierre de algunas empresas públicas (ILPE - pesca), la privatización de otras (PLUNA - aviación) y la venta parcial de la Administración Nacional de Telecomunicaciones (ANTEL), y representó el primer intento por realizar reformas orientadas al mercado. Este intento fracasa ante el referéndum que provocó la derogación de algunos artículos de esta ley (principalmente los referidos a ANTEL). Ello llevó a profundizar la política de mejora y búsqueda de eficiencia de las empresas públicas, en el marco de cierto consenso sobre la utilización de los beneficios de estas empresas como fuente adicional de ingresos fiscales \citep{Bergara2005}.

\citet{Forteza2003}, a su vez, señalan que una vez cerrada la posibilidad de las privatizaciones, la agenda de la reforma de los servicios públicos se reorientó hacia la remoción de los monopolios a través de la desregulación, la competencia y la asociación con privados en nuevos emprendimientos. La \href{https://www.impo.com.uy/bases/leyes/16832-1997}{Ley No.~16.832} de Marco Regulatorio del Sector Eléctrico (1997) reiteraba la posibilidad de competencia en la generación de energía tal como lo establecía la Ley Nacional de Electricidad del año 1977 (\href{https://www.impo.com.uy/bases/decretos-ley/14694-1977}{Decreto Ley No.~14.694}), mientras mantiene a la Administración Nacional de Usinas y Transmisiones Eléctricas (UTE) como empresa monopólica estatal en la transmisión y distribución de la energía eléctrica.

Con la regulación de esta ley se establece, en el año 2000, la Unidad Reguladora de la Energía Eléctrica (UREE) la que comienza a funcionar en el año 2001 cuando se le asigna presupuesto específico. Poco después, en el año 2002 se modifican sus cometidos, creándose la Unidad Reguladora de los Servicios de Energía y Agua (URSEA).

En el sector de comunicaciones el proceso de reformas se inicia cuando se aprueba la nueva carta orgánica de ANTEL, que la modifica sustancialmente, y se crea la Unidad Reguladora de los Servicios de Comunicación (URSEC), en el año 2001. Esta legislación permitía la privatización de la división de telefonía celular de ANTEL, y abría la competencia en varios segmentos. Sin embargo, en 2002 los artículos fundamentales de esta ley se derogan, permitiendo en el breve período de aplicación la incorporación de competencia en el mercado de telefonía celular y de llamadas internacionales.

El último intento, en este proceso de apertura al mercado, fue la autorización a la Administración Nacional de Combustibles, Alcohol y Portland (ANCAP) para asociarse con privados en la refinación de petróleo y venta de productos refinados por un período de 30 años, y la liberalización de la importación de petróleo a partir de 2006. Esta norma también fue derogada por plebiscito en 2003.

Según \citet{Bergara2005} las reformas orientadas al mercado fueron de alcance limitado y variaron considerablemente en función del sector de actividad. Aquellas reformas que requerían cambios en el marco legal tuvieron problemas serios. En algunos casos los plebiscitos impidieron su aplicación (comunicaciones, combustibles), mientras que en otros -energía- se demoró su puesta en funcionamiento. Por su parte, en las reformas que solo requerían la implementación de medidas administrativas, la participación de los privados se promovió a través de la concesión de los monopolios locales, tal el caso de las concesiones en los servicios de agua y saneamiento (en escala reducida) así como en la infraestructura de carreteras, puertos y aeropuertos, y en el mercado de los seguros y el correo.

Por su parte, \citet{Forteza2003} señalan que como resultado del plebiscito de 1992 sobre las empresas públicas, los sectores reformistas cambiaron sus propuestas hacia el fortalecimiento de estas empresas. En este marco se observa, a vía de ejemplo: (i) la reestructura, entre 1995 y 2000, de UTE a través de cuantiosas inversiones; (ii) la mejora de ANCAP, invirtiendo en la ampliación de la capacidad de su refinería, la compra de una compañía distribuidora en Argentina, y la transferencia de la distribución de productos derivados del petróleo en Uruguay a una empresa privada (de su propiedad); (iii) la expansión de las actividades de ANTEL hacia la telefonía celular.

A comienzos del siglo XXI y como consecuencia de la mejora en la gestión de las empresas estatales en energía y telecomunicaciones, tanto UTE como ANTEL presentaban indicadores considerablemente mejores a los de las empresas estatales en países en desarrollo, y habían aumentado considerablemente las utilidades vertidas al fisco de forma permanente \citep{Bergara2005}.

Estos cambios fueron acompañados por transformaciones en la gestión de estas empresas. \citet{Oria2008} señala que las empresas pasaron de centrar su atención en los problemas internos a centrarla en el cliente y su posicionamiento en el mercado. Es decir, el paradigma volvió a centrar a las empresas públicas como empresas que intervienen en mercados con clientes y, a veces, competidores.

Luego de una década de diversos intentos de reformas de los servicios públicos, \citet{Bergara2005} sostienen que en Uruguay este proceso ha sido relativamente volátil ya que las reformas que pasaron por el Parlamento fueron, en su mayoría, derogadas por la vía de los referéndum, mientras que las que no requerían de ley -como el sector eléctrico- sufrieron retrasos importantes en su implementación. Atribuyen este resultado a diversos factores: (i) preferencia de lo público frente a lo privado por parte de la mayoría de la población, lo que determina la alta valoración que esta tiene sobre las empresas públicas; (ii) preferencias políticas no muy divergentes con relación a la propiedad de las empresas de servicios públicos, donde la privatización masiva de las empresas públicas no era una propuesta predominante; y (iii) razones pragmáticas de quienes gobiernan que requieren de las ganancias de las empresas para lograr cerrar la brecha fiscal.

Asimismo, señalan que la consolidación de las empresas públicas como financiador de las cuentas públicas ha sido un factor determinante en el desarrollo de su reforma estructural. Ante situaciones problemáticas de déficit fiscal recurrir a recursos provenientes de estas empresas tiene ventajas frente a otras formas de financiamiento (impuestos, endeudamiento), ya que no requiere aprobación parlamentaria. Estos factores jugaron en contra del interés de promover mayor competencia en los sectores de servicios públicos.

El debate en torno a la reforma de los servicios públicos tuvo amplia difusión y participación, fundamentalmente en el ámbito político. El documento elaborado en el marco del proyecto Agenda Uruguay y publicado en 2001 ``Servicios públicos: aportes hacia una política de Estado'' \citep{CIIIPUPAZ2001} trata de recoger diferentes concepciones respecto a esta temática. Los artículos señalan la necesidad de una política de Estado con el objetivo de mejorar la calidad de los servicios públicos, reducir las tarifas que pagan los contribuyentes y lograr el mejor desarrollo económico y social para Uruguay. La mayoría de las propuestas combinan la introducción de una mayor competencia en estos mercados con la incorporación de la regulación de los mismos.

A ese respecto \citet{Bergara2001} sostiene que ``(\ldots) en los albores del nuevo milenio, el Estado del bienestar debe seguir existiendo, pero ha cambiado las formas con las que pretende cumplir sus fines. Este nuevo Estado del bienestar distingue de manera sana su rol de regulador, de proveedor de servicios y de compensador de desigualdades, buscando ser el garante de los intereses de los ciudadanos. Los nuevos códigos son mercados competitivos, regulación fuerte que evite abusos monopólicos y promueva la competencia, participación más directa de los usuarios y subsidios explícitos que amplíen el acceso a los servicios básicos de las capas sociales más excluidas''. \citet{Mederos2001} propone transitar hacia una ``competencia regulada'' reconociendo que existen fallas de mercado que impiden la libre competencia y por ende la maximización de los beneficios sociales.

Finalmente, mientras se procesaba este debate en diferentes foros, se aprobó la modificación constitucional que establece que el agua es ``parte del dominio público'' y por ende su provisión privada resulta ilegal. La disposición constitucional, aprobada en 2004, llevó a la nacionalización de la provisión de agua y saneamiento en aquellas localizaciones donde el servicio se proveía por parte de firmas privadas, con anterioridad a la existencia de la empresa estatal Obras Sanitarias del Estado (OSE), o donde el mismo se había privatizado o entregado en concesión en los 90.

\hypertarget{panorama-actual-de-las-empresas-puxfablicas-en-uruguay}{%
\section{Panorama Actual de las Empresas Públicas en Uruguay}\label{panorama-actual-de-las-empresas-puxfablicas-en-uruguay}}

Las actuales empresas públicas difieren según el sector de actividad en el que operan. En este trabajo, se consideran las empresas públicas de combustibles (ANCAP), electricidad (UTE), telefonía (ANTEL) y agua (OSE). Las restantes empresas públicas no se analizan debido a que la lógica de sus mercados y su consiguiente regulación es diferente (Banco de Seguros del Estado (BSE), BROU, sector financiero), porque tienen una baja actividad empresarial (AFE, ferrocarriles), o porque la prestación del servicio está dispersa entre diversos operadores (saneamiento).

En términos generales, se puede señalar que las tarifas que han cobrado históricamente estas empresas por sus servicios han estado por debajo de la inflación, con excepción de los combustibles líquidos (gasolinas) y la energía eléctrica. En el gráfico \ref{fig:fig3} se puede observar que las tarifas de gas, teléfono y agua evolucionan por debajo de la inflación promedio anual, mientras que la energía eléctrica lo hace fundamentalmente por encima.

\begin{figure}
\centering
\includegraphics{Regulacion_files/figure-latex/fig3-1.pdf}
\caption{\label{fig:fig3}Evolución real de tarifas públicas, 1968-2000 (Base 1968=100)}
\end{figure}

Fuente: elaboración propia con base en datos del INE.

Por su parte, las tarifas de las gasolinas se disparan a partir del año 1973, con motivo de la crisis del petróleo, y convergen nuevamente a la inflación a partir de la década de los noventa (gráfico \ref{fig:fig4}).\\

\begin{figure}
\centering
\includegraphics{Regulacion_files/figure-latex/fig4-1.pdf}
\caption{\label{fig:fig4}Evolución real del precio de las gasolinas, 1968-2019 (Base 1968=100)}
\end{figure}

Fuente: elaboración propia con base en datos del INE y \href{https://www.ancap.com.uy/innovaportal/v/6088/1/innova.front/historico-precio-combustibles.html}{ANCAP}.

\hypertarget{el-mercado-de-las-telecomunicaciones}{%
\subsection{El mercado de las telecomunicaciones}\label{el-mercado-de-las-telecomunicaciones}}

En el mercado de la telefonía, ANTEL tiene el monopolio de la telefonía fija, compite con dos empresas privadas en telefonía celular, y existe un monopolio de hecho en la transmisión de datos. Uruguay tuvo, históricamente las tarifas de telefonía fija más altas de la región, al tiempo que la conectividad a internet estuvo limitada durante muchos años, lo que impactaba en la competitividad de las empresas que utilizan este medio como insumo para sus servicios.

Desde el año 2014 ANTEL comenzó la instalación de fibra óptica al hogar lo que ha permitido mejorar la conectividad a internet, y también permitiría suplantar la tecnología de telefonía básica ya que ambas utilizan la misma plataforma. Si bien aún conviven ambas plataformas, este cambio tecnológico hizo crecer a Uruguay en los rankings internacionales. En el cuadro \ref{tab:cuadro6} se presentan diversos indicadores del sector de telecomunicaciones a nivel internacional.

\begin{table}

\caption{\label{tab:cuadro6}Indicadores del sector telecomunicaciones para países seleccionados de América}
\centering
\begin{tabular}[t]{l|r|r|r|r|r}
\hline
País & Teléfonos c/100 habitantes & Celulares c/100 habitantes & Hogares con acceso a internet (\%) & Precio banda ancha (U\$S PPC) & Precio banda ancha (U\$S PPC\\
\hline
Argentina & 22 & 142 & 81 & NA & NA\\
\hline
Bolivia & 8 & 99 & 32 & 51 & 21\\
\hline
Brasil & 20 & 113 & 61 & 25 & 16\\
\hline
Chile & 18 & 128 & 89 & 46 & 18\\
\hline
Colombia & 14 & 127 & 50 & 42 & 28\\
\hline
Ecuador & 15 & 84 & 37 & 39 & 25\\
\hline
México & 16 & 89 & 51 & 10 & 12\\
\hline
Paraguay & 4 & 110 & 20 & 50 & 13\\
\hline
Perú & 10 & 121 & 28 & 28 & 20\\
\hline
Uruguay & 33 & 148 & 64 & 0 & 22\\
\hline
Venezuela & 19 & 77 & 34 & NA & NA\\
\hline
EUA & 37 & 123 & 87 & 50 & 44\\
\hline
Canadá & 40 & 86 & 91 & 25 & 25\\
\hline
\end{tabular}
\end{table}

Fuente: elaboración propia con base en Measuring the Information Society de la International Telecomumunication Union 2018 \href{https://www.itu.int/en/ITU-D/Statistics/Documents/publications/misr2018/MISR-2018-Vol-1-E.pdf}{vol.~1} datos de precios (tablas 4.1, 4.6 y ) y \href{https://www.itu.int/en/ITU-D/Statistics/Documents/publications/misr2018/MISR-2018-Vol-2-E.pdf}{vol.~2}, datos de acceso.

Los datos destacan lo bien posicionado que está el país en telecomunicaciones respecto a sus pares en América Latina, e inclusive tomando como referencia Canadá y Estados Unidos (EUA). Ello es el resultado de la fuerte competencia a la que está sometida la empresa pública en los distintos mercados (telefonía celular y transmisión de datos), lo que la ha llevado a mejorar notoriamente los indicadores de desempeño. En lo que refiere a banda ancha, donde ANTEL está desarrollando el plan de instalación de fibra óptica al hogar, se apreciaba inicialmente una fuerte competencia por el mercado.

Es de hacer notar que en esta competencia la empresa pública ha recibido ayuda del Estado, ya sea por acción u omisión, dado que los competidores privados no han podido obtener licencias para instalar tendido de fibra óptica. Sin embargo, para que ANTEL pueda sostener estos buenos resultados, es necesario retomar la competencia en este sector que en la actualidad se encuentra acotada. Cuando la competencia \emph{por el mercado} se alcance, habrá que retomar la competencia \emph{en el mercado} para que la eficiencia de la empresa no se deteriore. En el cuadro \ref{tab:cuadro7} se presenta un panorama del sector de las telecomunicaciones en Uruguay.

\begin{table}

\caption{\label{tab:cuadro7}Indicadores seleccionados del mercado de telecomunicaciones en Uruguay (2008-2018)}
\centering
\begin{tabular}[t]{r|r|r|r|r|r}
\hline
Año & Numero teléfonos celulares (miles) & Numero teléfonos fijos (residenciales, miles) & Minutos tráfico celular (millones) & Cómputos telefonía fija (millones) & Servicios banda ancha (miles)\\
\hline
2008 & 3508 & NA & 2669 & 3401 & 244\\
\hline
2009 & 4112 & NA & 3818 & 3106 & 317\\
\hline
2010 & 4437 & 775 & 4812 & 2804 & 383\\
\hline
2011 & 4575 & 796 & 5276 & 2519 & 473\\
\hline
2012 & 4995 & 821 & 5886 & 2355 & 581\\
\hline
2013 & 5268 & 856 & 6181 & 2157 & 737\\
\hline
2014 & 5497 & 886 & 6184 & 1979 & 840\\
\hline
2015 & 5495 & 909 & 6281 & 1757 & 901\\
\hline
2016 & 5420 & 918 & 5846 & 1559 & 922\\
\hline
2017 & 5391 & 934 & 5278 & 1304 & 950\\
\hline
2018 & 5437 & 951 & 4683 & 1094 & 977\\
\hline
\end{tabular}
\end{table}

Fuente: elaboración propia con base en datos \href{https://www.gub.uy/unidad-reguladora-servicios-comunicaciones/sites/unidad-reguladora-servicios-comunicaciones/files/2019-10/Informe\%20Telecomunicaciones\%20a\%20diciembre\%20de\%202018\%20corregido.pdf}{URSEC}.

Los resultados del sector de telefonía celular demuestran también como la competencia funciona como incentivo a la búsqueda de mejores resultados (tarifas adecuadas y expansión del servicio). Este proceso comenzó con el ingreso de América Móvil en 2004, lo que provocó una fuerte expansión en el mercado. Ese incremento también se observa en el uso de celulares, con una duplicación del número de minutos entre 2008 y 2013. En el año 2015 se observa un máximo y luego una fuerte caída en el número de minutos.

Por su parte, la telefonía fija presenta un aumento sostenido en el número de lineas, quizá explicado por el incremento en el número de servicios de banda ancha fija, que necesita disponer de un teléfono fijo para obtener el servicio. Esto se verifica al observar el incremento en el número de servicios de banda ancha que se cuadriplica entre 2008 y 2018, y la sostenida caída en el número de cómputos de telefonía fija que determina, que en 2018, este sea menos de un tercio del valor de 2008. La telefonía fija es un mercado que tiende a ser sustituido por la telefonía celular y por la transmisión de datos.

Por último, debe señalarse que ANTEL opera fundamentalmente bajo la figura de empresa pública, pero posee cuatro empresas de propiedad pública que operan en el ámbito del derecho privado, con el objetivo de dar funciones de apoyo no sustantivas.

Estas empresas son:
- ITC S.A. que tiene por objeto brindar servicios de asesoramiento y asistencia en el área de telecomunicaciones, tecnología de la información y de la gestión tanto en el país como en el exterior, de la cual ANTEL es propietaria del 99,924\% de su capital;
- HG S.A. que se dedica a la realización de proyectos de integración tecnológica y de servicios, asociados al desarrollo y operación de sitios web y portales siendo propiedad de ANTEL el 99,8026\% de su capital;
- ACCESA S.A. cuyo objeto principal es brindar servicios de Call Center y Contact Center, procesamiento de información, datos y contenidos mediante sistemas de telecomunicaciones y tecnología de la información, siendo ANTEL propietaria del 95,74\% del capital; y
- ANTEL USA Inc.~con sede en Estados Unidos cuyo cometido es proveer servicios de interconexión de datos (IP) desde ese país a compañías de telecomunicaciones en América Latina, siendo ANTEL propietaria del 100\% del capital.

\hypertarget{mercado-de-combustibles-glp-y-gas}{%
\subsection{Mercado de combustibles, GLP y gas}\label{mercado-de-combustibles-glp-y-gas}}

ANCAP tiene el monopolio de la importación y refinamiento de combustibles desde su creación en el año 1931. Los combustibles se producen en una única refinaría con petróleo importado. En el mercado de distribución de combustibles operan tres empresas: DUCSA, 99\% propiedad de ANCAP; ESSO y Petrobras. Asimismo, las estaciones que venden el combustible al público son privadas y operan bajo la marca de algunos de los distribuidores.
\footnote{Existen unas 448 estaciones de servicio en el país, el 60\% con bandera ANCAP \citep{URSEA2013}.}

Por su parte el suministro de combustibles a las aeronaves, en la terminal del aeropuerto, lo realiza Talobras empresa en la que ANCAP, Orodone S.A. y Petrobras comparten el capital. A los efectos de observar la evolución del precio de la gasolina y compararlo con otros países, en el gráfico \ref{fig:fig6} se presenta el precio de la gasolina para los países del MERCOSUR, NAFTA y otros de Sudamérica, para los años en que la información estaba disponible.

\begin{figure}
\centering
\includegraphics{Regulacion_files/figure-latex/fig6-1.pdf}
\caption{\label{fig:fig6}Precio de la gasolina para países de América Latina (en dólares)}
\end{figure}

Fuente: World Development Indicators, Banco Mundial.

Como se puede observar, el precio de la gasolina en Uruguay está entre los más caros de la región. Este resultado, a diferencia del anterior referido a las telecomunicaciones, tiene diversas consideraciones. En efecto, Uruguay no es un país que tenga petróleo, como lo es Venezuela o Ecuador, y por tanto debe importar el crudo que refina. Por otra parte, el monopolio que rige en Uruguay, sometido a una regulación laxa -la propia empresa fija sus precios-, influye en su eficiencia.

El GLP se obtiene del refinamiento de petróleo en la planta de La Teja de ANCAP. Una vez producido se transfiere a la planta de despacho de La Tablada y de allí a las dos plantas de envasado: la de GASUR, que envasa para las empresas Riogas y Acodike, y la planta de Megal. En la distribución minorista operan cuatro empresas: Acodike, Riogas, Megal y DUCSA (sociedad anónima propiedad de ANCAP en 99\% del capital). El precio máximo al consumidor final lo fija el Poder Ejecutivo, a iniciativa de ANCAP.

ANCAP es accionista en más de una docena de sociedades anónimas a través de las cuales diversifica sus negocios. Estas empresas se concentran en cemento, gas natural, alcoholes y bebidas alcohólicas, agroindustrias, exploración y producción de petróleo, negocios en Argentina y asistencia técnica.

La comercialización de los cementos ANCAP se realiza a través de la firma Cementos del Plata S.A. en la cual ANCAP posee el 99,25\% del paquete accionario. Vinculado al negocio de cementos, ANCAP también es propietaria de PAMACOR S.A. empresa dedicada a la prospección, exploración, explotación, comercialización, importación y exportación de recursos minerales y productos derivados.

En el sector de gas natural ANCAP, a través de Gaseoducto Cruz del Sur S.A., donde participa con el 20\% del capital accionario junto a firmas extranjeras, opera en el transporte de este combustible desde Argentina. Por su parte, mediante la empresa CONECTA, propiedad de ANCAP (45\% del capital accionario) y Petrobras Uruguay, participa en la distribución interna de gas por cañería (excepto Montevideo); mientras que Petrobras Uruguay tiene la distribución en Montevideo. Asimismo, ANCAP participa junto a UTE en el desarrollo de la Planta Regasificadora para recepción, almacenamiento y regasificación de gas natural licuado, mediante la empresa conjunta Gas Sayago S.A.

En GLP, ANCAP (40\%) participa con empresas privadas (Acodike y Riogas) en Gasur Uruguay S.A. para la venta y distribución, en sus diferentes modalidades, de propano industrial a granel y gases canalizados, a clientes que cuenten con instalaciones adecuadas para su consumo.

En negocios agroindustriales, ALUR S.A., 97\% propiedad de ANCAP, es el principal productor de biocombustibles, azúcar y harinas proteicas, así como de alcoholes. En el sector de alcoholes también opera la Compañía ANCAP de Bebidas y Alcoholes, S.A. (CABA) que produce, compra, comercializa y distribuye alcoholes y bebidas alcohólicas, entre otros.

Por último, ANCAP maneja sus negocios en Argentina a través de dos subsidiarias. Petrouruguay S.A., que desarrolla actividades de exploración, producción y comercialización de gas y petróleo, en cuyo capital ANCAP participa con el 99,84\% directamente y a través de ANCSOL S.A. (una SAFI 100\% propiedad de ANCAP). Por su parte, Carboclor S.A. propiedad de ANCSOL S.A. y de Petrouruguay S.A., se dedica principalmente a la producción de solventes y alcoholes a partir de corrientes de refinerías de petróleo.

En los últimos años, ANCAP se embarcó en una serie de inversiones para sus diferentes divisiones, muchas de las cuales se han visto cuestionadas por el sobre costo en la ejecución de las obras. Asimismo, los resultados económicos de la empresa han devenido negativos, lo que ha determinado la necesidad de una sustantiva capitalización de la empresa.

\hypertarget{energuxeda-eluxe9ctrica}{%
\subsection{Energía eléctrica}\label{energuxeda-eluxe9ctrica}}

El sector eléctrico es el más complejos de los sectores objeto de estudio. Tiene tres componentes: generación, transmisión y distribución. En la generación operan distintos agentes que producen energía eléctrica con base en distintas fuentes (hidroeléctrica, solar, biomasa, eólica, etc.). En 2019, el 55\% de la generación eléctrica correspondía a fuentes hidráulica, 34\% a eólica, 6\% a biomasa, 3\% a fotovoltaica y 2\% térmica.

En la transmisión eléctrica, típico segmento de monopolio natural, UTE tiene el monopolio legal de la actividad al igual que en la distribución. Lo interesantes es que esta empresa pública compite en los mercados internacionales, en la medida en que vende energía eléctrica a los países vecinos, en particular a Argentina.

La eficiencia en el sector depende de la posibilidad de utilizar la fuente más barata para la generación de energía, y esta es la hidroeléctrica. Sin embargo, ello depende de factores no controlables por la empresa como el clima. Asimismo, el país ha tomado importantes decisiones para la instalación de fuentes alternativas de energía (parques eólicos principalmente), los que se encuentran en etapa de desarrollo.

Si bien UTE es el principal agente de suministro de energía eléctrica, es posible para los grandes consumidores contratar directamente con las empresas generadoras y utilizar la red de transmisión de UTE para que lleve la energía eléctrica contratada. El marco jurídico que permite la efectivización de estos contratos se encuentra reglamentado y operativo, sin embargo, a la fecha, no se han registrado contratos entre particulares y todos los grandes consumidores continúan contratando directamente con UTE.

En este sector resulta difícil establecer un indicador que permita comparar la eficiencia relativa de la empresa con otros actores de la región. Los datos disponibles, sistematizados a nivel internacional, refieren principalmente a la cobertura de energía eléctrica a los hogares. Este indicador puede ser considerado un resultado de eficacia (acceso universal a la red de energía eléctrica) desde el punto de vista social, sin embargo nada dice de la eficiencia de la empresa. Otro indicador, como el precio del servicio, depende de factores externos a la empresa, como el clima, cuando la generación se basa en energía hidroeléctrica y tampoco nos permite comparar eficiencia relativa. Solo en los casos en que la mayor parte de la generación se base en fuentes alternativas a la hidroeléctrica, las mismas serían controlables por la empresa y su operativa puede tener impactos sobre los costos de generación, haciendo que el precio pueda servir como variable de comparación.

\hypertarget{agua-potable}{%
\subsection{Agua potable}\label{agua-potable}}

Como fuera señalado, desde el año 2004 la producción y distribución de agua potable se encuentra bajo monopolio del Estado por norma constitucional. Por tanto, OSE actúa como monopolista en este mercado. El hecho de que con anterioridad a esa fecha existieran otras empresas prestadoras del servicio de agua potable no implicaba que existiera competencia entre ellas, ya que dado un mercado geográfico, el servicio de agua potable es un monopolio natural.

Al igual que en los casos anteriores, existe una variedad de dimensiones sobre las cuales se puede comparar el desempeño de la empresa, por ejemplo, agua facturada en el total de agua producida, número de roturas en los caños, horas sin servicio de agua potable, etc. Sin embargo, para tener información comparable, se toma aquella sistematizada por el Banco Mundial sobre conexión a la red, tanto en las ciudades como en el medio rural de algunos países de América, la que se presenta en el cuadro 3.

De manera general, puede observarse que el desempeño de OSE, medido en términos de cobertura, ha mejorado sustancialmente entre 1990 y 2012 al igual que en el resto de los países de la región. Asimismo, \citet{Borraz2013} demuestran que la estatización del servicio, producto de la reforma constitucional, se tradujo en un incremento en el acceso a la red de agua potable y en la mejora en la calidad del agua.

\begin{table}

\caption{\label{tab:cuadro8}Tasa de cobertura del sistema de agua potable. Países seleccionados. (en porcentaje}
\centering
\begin{tabular}[t]{c|c|c|c|c}
\hline
País & Rural, año 1990 & Rural, año 2018 & Urbana, año 1990 & Urbana, año 2018\\
\hline
Argentina & 69 & 93 & 97 & 99\\
\hline
Bolivia & 41 & 78 & 91 & 99\\
\hline
Brasil & 68 & 90 & 96 & 99\\
\hline
Chile & 48 & 99 & 99 & 99\\
\hline
Colombia & 69 & 86 & 98 & 99\\
\hline
Ecuador & 61 & 83 & 84 & 99\\
\hline
Paraguay & 24 & 99 & 83 & 99\\
\hline
Perú & 44 & 76 & 88 & 96\\
\hline
Uruguay & 75 & 95 & 98 & 99\\
\hline
Venezuela & 71 & NA & 93 & NA\\
\hline
EUA & 94 & 97 & 100 & 99\\
\hline
Canadá & 99 & 99 & 100 & 100\\
\hline
\end{tabular}
\end{table}

Fuente: World Development Indicators, Banco Mundial.

OSE tiene participación en tres empresas privadas:
- Aguasur (Manantial Dorado S.A.), en la que posee el 95\% de las acciones, cuyo cometido es la construcción, remodelación, y/o mantenimiento de soluciones estructurales para el tratamiento de líquidos residuales, el abastecimiento de agua potable y/o actividades vinculadas a la misma.
- Aguas de la Costa S.A., donde OSE posee el 60\% del capital accionario y cuya actividad principal es dar cumplimiento al contrato de obra pública para el suministro de agua potable a una zona del Departamento de Maldonado.
- Consorcio Canario Ciudad de la Costa S.A. cuyo cometido es la administración de las contrataciones y gestión de las actividades relativas al Programa Integrado de Saneamiento, Drenaje Pluvial y Vialidad en Ciudad de la Costa, Departamento de Canelones, siendo OSE y la Intendencia de Canelones titulares de las acciones por partes iguales.

\hypertarget{el-marco-institucional-de-las-empresas-puxfablicas-en-uruguay}{%
\section{El Marco Institucional de las Empresas Públicas en Uruguay}\label{el-marco-institucional-de-las-empresas-puxfablicas-en-uruguay}}

La importancia que la sociedad -o el sistema político- le ha asignado tradicionalmente a las empresas públicas en Uruguay, se manifiesta en el hecho de que las actividades industriales y comerciales del Estado tienen una figura específica en la Constitución de la República.

Las empresas propiedad del Estado, antes de 1934, operaban bajo el mismo régimen jurídico que las empresas privadas. La Constitución de ese año crea una nueva figura jurídica para ``los diversos servicios que constituyen el dominio industrial y comercial del Estado (\ldots)'', la que se fue modificando con las siguientes reformas constitucionales.

A partir de la Constitución de 1934,
\footnote{Véase la \href{https://legislativo.parlamento.gub.uy/temporales/2680947.HTML}{Constitución de 1934} artículo 181.}
esta figura jurídica adopta dos formas posibles: Entes Autónomos y Servicios Descentralizados. Su objetivo fue permitir dotar a las empresas públicas de mayor grado de autonomía que a otras instituciones de la administración pública.

Esta movida estratégica, en los hechos, impuso un límite a la discrecionalidad en la que operan las empresas públicas en Uruguay. En efecto, la forma jurídica que les hubiera permitido a estas empresas alcanzar la mayor independencia posible era que se hubiera mantenido bajo el régimen de sociedades anónimas y se guiaran, en ese momento, por el Código de Comercio, o por la Ley de Sociedades Comerciales, en la actualidad. La Constitución de 1918, sacó a las empresas de la órbita privada y las puso bajo controles similares al resto de los organismos del Estado.

En la actualidad, la sección XI ``De los Entes Autónomos y de los Servicios Descentralizados'' (artículos 185 a 201) de la Constitución, constituye el marco legal general que ampara la creación y funcionamiento de los Entes Industriales o Comerciales del Estado. En esta sección de la Constitución se establecen las condiciones para la creación o supresión de Entes Autónomos (artículo 189), así como el marco de actuación de las empresas públicas:
- Designación de autoridades y las características de las mismas en cuanto a duración en sus funciones, cese o inhibiciones (artículos 185 y 187).
- Condiciones relativas a posible participación de capitales privados, limitando su participación y representación a un porcentaje menor al del Estado (artículo 188).
- Restricción de actividades a las establecidas por ley (artículo 190).
- Obligación de presentar estados financieros con el visto bueno del Tribunal de Cuentas (artículo 191).

Por otra parte, en la Constitución también se establecen otras disposiciones que afectan el funcionamiento de las empresas públicas. Entre ellos merecen señalarse, los siguientes:

\begin{enumerate}
\def\labelenumi{\arabic{enumi}.}
\item
  El establecimiento de monopolios (a favor del Estado) que requiere la mayoría absoluta de los votos del total de componentes de la Asamblea General (artículo 85, inciso 17).
\item
  Las disposiciones sobre los regímenes relativos a los funcionarios de estas empresas, que se regirán por un Estatuto que, proyectado por las mismas, deberá ser aprobado por el Poder Ejecutivo, en el caso de los Entes Autónomos (artículo 63). Por su parte, si la empresa pública tiene el carácter de servicio descentralizado (OSE, ANTEL) el Estatuto debe ser aprobado por ley (artículo 59). Asimismo, establece que la ley deberá crear el Servicio Civil de la Administración Central, Entes Autónomos y Servicios Descentralizados que tendrá injerencia en esta materia sobre todos los órganos del Estado.
\item
  Intervención preventiva de los gastos y pagos, y en todo lo referente a su gestión financiera, por parte del Tribunal de Cuentas de la República (artículo 211), el que ejercerá además la superintendencia sobre las oficinas de contabilidad, recaudación y pagos de las empresas públicas (artículo 212).
\item
  Las disposiciones relativas a la Hacienda Pública donde se indican los mecanismos referidos a la elaboración de los presupuestos de los Entes Industriales o Comerciales del Estado (artículo 220), su aprobación y su control (artículo 221). En este proceso participan preceptivamente tanto el Tribunal de Cuentas como la Oficina de Planeamiento y Presupuesto (OPP).
\end{enumerate}

El carácter de Ente Autónomo o Servicio Descentralizado define principalmente el grado de autonomía y descentralización con que pueden operar estas empresas. En general, los segundos están sometidos a un control más intenso por parte de los poderes del Estado y su autonomía de administración está limitada a la que le confiere la Ley. Otras diferencias tienen que ver con:

\begin{itemize}
\tightlist
\item
  Las mayorías parlamentarias necesarias para su creación: en el caso de un Ente Autónomo se necesitan dos tercios del total de componentes de cada cámara, mientas que para el caso de un Servicio Descentralizado se requiere mayoría absoluta.
\item
  Los procedimientos para recurrir los actos de los mismos: en el caso de los Entes Autónomos la vía administrativa se agota en el recurso de revocación ante el Directorio, mientras que en los Servicios Descentralizados opera el recurso de anulación ante el Poder Ejecutivo (artículo 317).
\item
  La conformación de los órganos que rigen a estas empresas: podrá ser unipersonal en el caso de los Servicios Descentralizados, pero en el caso de los Entes Autónomos deberá ser colectivo (artículo 185).
\end{itemize}

Entre las principales empresas públicas que fueron creadas como Entes Autónomos se encuentran: UTE, ANCAP y la Administración de Ferrocarriles del Estado (AFE). Mientras que se establecen como Servicios Descentralizados ANTEL, OSE, la Administración Nacional de Correos (ANC) y la Administración Nacional de Puertos (ANP).

El régimen jurídico nacional tiene una fuerte impronta del Poder Ejecutivo \citep{Bergara2005} en lo que refiere, a vía de ejemplo, a la iniciativa que se requiere para la aprobación de determinadas leyes, o la integración de los directorios de los entes autónomos y servicios descentralizados. En esta línea, si bien los entes autónomos y servicios descentralizados tienen un importante grado de autonomía, el mecanismo constitucional determina que no tengan libertad para fijar su presupuesto fuera de los límites que le establece el Poder Ejecutivo.

Por su parte diversas leyes y decretos determinan otras dimensiones de la actuación de estas empresas. En primer término se encuentran las leyes orgánicas que definen con precisión los cometidos, competencias y el funcionamiento de cada una de ellas, y, en particular, establecen los mecanismos de fijación de tarifas y elaboración de presupuestos.

Por otra parte, diversas leyes les otorgan a los Directorios de los Entes Autónomos y Servicios Descentralizados la potestad de proponer al Poder Ejecutivo las tarifas a cobrar por los servicios que prestan: ANCAP (literal F del artículo 3 de la Ley 8.764, año 1931) modificado por la Ley 15.312 de 1982); ANTEL (artículo 12 de la Ley 14.235, año 1974); OSE (artículo 11 de la Ley 11.907, año 1952); y UTE (artículo 14 de la Ley 15.031, año 1980).

Otros aspectos a destacar con relación al régimen en que operan estas empresas, refieren a: (i) sus posibilidades de endeudamiento que requiere aprobación del Poder Ejecutivo si la deuda supera los \$U 85 millones (artículo 337 de la Ley 18.996 del 2012); (ii) necesidad de informe previo de la Oficina Nacional de Servicio Civil (ONSC) y de la OPP para realizar contratos de obra (artículo 47 de la Ley 18.719 del 2011); (iii) la obligatoriedad de realizar sus depósitos en los bancos estatales, en particular en el BROU (artículo 19 del Decreto Ley 15.322 de 1982 y artículo 80 de la Ley 17.555 de 2002); y (iv) obligación de contratar con el BSE exclusivamente, en el caso de los seguros de accidentes de trabajo y enfermedades profesionales.

De este marco surge un control mucho mayor sobre las empresas públicas del que originalmente estaba previsto en la Constitución. Es decir, la figura jurídica que adoptan las empresas públicas determina que se requiera aprobación del Poder Ejecutivo para determinar las principales variables (presupuesto, precios, inversiones, endeudamiento) y, por tanto, este tiene poder de veto sobre las decisiones de las empresas. A través de la OPP, al principio de cada período de gobierno, se fijan los lineamientos que deben cumplir las empresas públicas para la presentación de sus presupuestos.

Por su parte, las inversiones de estas empresas se definen siguiendo los criterios que establece la OPP en el marco del Sistema Nacional de Inversión Pública (SNIP). Por último, las empresas definen la actualización de la tarifa sobre la base de paramétricas que ellas elaboran, pero que en última instancia requieren del visto bueno del Ministerio de Economía y Finanzas (MEF) para su aprobación por parte del Poder Ejecutivo.

Por otra parte, la figura jurídica adoptada en el derecho uruguayo es particular en cuanto a que somete a las empresas públicas a controles similares a los del propio sector público. Ello tiene como resultado restringir las acciones de las empresas, aun cuando las mismas cuenten con autonomía relativa. Por tanto, existen mecanismos muy fuertes de control sobre las empresas públicas donde el Poder Ejecutivo tiene poder de veto sobre decisiones económicas fundamentales.

Las empresas públicas uruguayas han mostrado adecuados niveles de cobertura de los servicios, al menos en las últimas décadas, y no han recibido subsidios del gobierno central para operar, como si lo hacen en otros países.
\footnote{Sin embargo, esta situación se ha visto cuestionada en los últimos años en el caso de ANCAP, empresa que ha requerido una fuerte capitalización por parte del Ministerio de Economía y Finanzas.}
Por tanto, este mecanismo regulador no parece haber fallado, al menos en grandes trazos. La única excepción ha sido ANCAP, sobre la que pesa una investigación aún no concluida para determinar las responsabilidades en su delicada situación patrimonial.

Como señala \citet{Tirole1994}, el hecho de que las empresas públicas estén sometidos al control de un ministerio sectorial que establece los lineamientos de actuación, y al de economía o finanzas, que tiene la misión de controlar los gastos, es un mecanismo eficiente que se transforma en un mecanismo creíble de control sobre las empresas públicas. Si el gasto crece mucho, entonces existe la amenaza de que el principal cambie del ministerio sectorial al de economía que será mucho más duro para autorizar los gastos o cambios de precio.

\hypertarget{el-marco-regulatorio-en-amuxe9rica-latina}{%
\section{El Marco Regulatorio en América Latina}\label{el-marco-regulatorio-en-amuxe9rica-latina}}

En América Latina la regulación de los servicios públicos se planteó, en la década del 90, vinculada a la supervisión y fiscalización de la conducta de los prestadores privados, a través de mecanismos de incentivos económicos y financieros, en el marco de un proceso amplio de privatizaciones. La posterior salida del mercado de muchos de estos prestadores privados, en varios países de la región, determinó que el marco legal, diseñado en su origen para regular a empresas privadas, se aplicara a las empresas públicas.

Las dificultades de los ministerios para realizar la gestión de la regulación llevaron, en muchos casos, a definir organismos reguladores independientes como parte de su diseño institucional. Sin embargo, no todos los países de la región cuentan con un marco jurídico que defina los lineamientos generales de funcionamiento de estas organizaciones en forma independiente del Poder Ejecutivo, existiendo vacíos en la normativa que define sus funciones y problemas en el diseño institucional \citep{Rozas2013}. Algunas de las debilidades de estas agencias son producto de que regulan a diversos sectores, tienen responsabilidades acotadas o tienen interferencia de los departamentos ministeriales en el desarrollo de sus actividades, y se les presentan dificultades a la hora de contratar recursos humanos capacitados para el desempeño de estas funciones.

En el diseño de las agencias reguladoras se tuvieron en cuenta criterios que buscaban alcanzar su independencia del gobierno por la vía de la composición, selección y remoción de sus directivos, su financiamiento, y potestades. Sin embargo, estas medidas no sirven para mitigar el riesgo de la captura regulatoria por parte de la empresa regulada. Asimismo, como demuestra la experiencia uruguaya, la supuesta autarquía política no necesariamente es sostenible en el tiempo. El problema es, también, cómo se sostiene en el tiempo esa independencia.

En la mayoría de los países de la región, y a diferencia de lo que sucede con las empresas privadas, no hay contratos que encuadren la relación regulador--empresa pública. En muchos casos ello se debe a la costumbre, sobre todo cuando el prestador del servicio ha sido históricamente la misma empresa pública que ha operado sin que existiera un regulador independiente. En otros, se vincula a la debilidad de los organismos reguladores o a la renuncia de las propias autoridades a realizar un mayor control de la empresa.

\citet{Rozas2013} señalan algunas características de los mecanismos de regulación en los servicios de agua potable y saneamiento en la región. Las metas de gestión se establecen, en general, a partir de propuestas de las entidades prestadoras. Sin embargo, el control o supervisión se realiza por más de un agente: (i) la contraparte en las metas de gestión de la empresa es el Ministerio de Economía o Finanzas; (ii) la entidad que aprueba los planes de la empresa es el organismo regulador; y (iii) en algunos casos hay también un control ex post del Tribunal de Cuentas.

Por su parte, la verificación del cumplimiento de las metas se realiza con base en la información que proporciona la propia empresa pública sin que exista una auditoría para validar la información presentada, lo que genera una baja confiabilidad de la misma. También señalan que muchos países encuentran dificultades de coordinación entre las instituciones estatales reguladoras lo que produce sobrecostos debido a que varias agencias controlan y regulan lo mismo, con dos excepciones Colombia (específicamente en Medellín) y Uruguay.

Asimismo, en lo referente a la forma institucional que adopta la regulación se observan diferentes modelos. Por un lado, están los países que han creado una autoridad regulatoria federal que revisa y aprueba tarifas, controla niveles de calidad del servicio e impone medidas sancionadoras a los operadores cuando se producen incumplimientos (Colombia, Chile, Bolivia, Perú y Honduras). Por otro, los que transfieren la responsabilidad regulatoria a las administraciones estaduales o provinciales (Argentina y Brasil).

\hypertarget{el-marco-regulatorio-en-uruguay}{%
\section{El Marco Regulatorio en Uruguay}\label{el-marco-regulatorio-en-uruguay}}

A lo largo del tiempo, se han formulado diversas propuestas sobre la necesidad de regular los mercados de servicios públicos en Uruguay. \citet{Bergara2001} considera que la provisión de servicios públicos es un terreno donde es necesario un ``balance entre una mayor liberalización y una justa intervención del Estado'', en función de que estos servicios, generalmente, se asocian a monopolios naturales. En la misma línea agrega que ``si el Estado tiene objetivos claros en aspectos de eficiencia e inversión y además posee preocupaciones distributivas, surge la necesidad de controlar el mercado monopólico'' \citep[p.~38]{Bergara2001}, reconociendo que esta intervención regulatoria opera tanto si la empresa que ofrece el servicio público es privada, pública u opera en régimen de concesión. La regulación debe incorporar, también, la promoción de la competencia en todos los ámbitos donde esta sea posible.

Asimismo, plantea que para lograr los objetivos de la regulación es fundamental el marco institucional y la calidad de las instituciones en que este proceso se desarrolla. Este marco institucional debe hacer creíble el proceso regulatorio y tanto el sistema político como el judicial debe impedir la captura del regulador por parte de la empresa regulada.

Para ello resulta importante tomar como unidad de análisis a los servicios públicos y no a las empresas públicas. Ello permite delimitar el objetivo del diseño institucional distinguiendo los diferentes roles del Estado, como regulador y generador de marcos competitivos, como proveedor directo de los servicios, y como compensador de desigualdades sociales \citep{Bergara2001}.

\hypertarget{los-uxf3rganos-reguladores-en-uruguay}{%
\subsection{Los órganos reguladores en Uruguay}\label{los-uxf3rganos-reguladores-en-uruguay}}

Los órganos reguladores surgen en la década del 2000 y se insertan en el marco institucional existente que regulaba a las empresas públicas. Tanto la Unidad Reguladora de los Servicios de Energía y Agua (URSEA) como la Unidad Reguladora de los Servicios de Comunicaciones (URSEC), son organismos técnicos que se han establecido con el objetivo de aportar criterios técnicos para velar por el funcionamiento de los mercados bajo su órbita de actuación.

Ello tiene lógica en sectores que se han abierto a la participación de empresas privadas, o a la competencia. En efecto, si bien algunos sectores continúan siendo monopólicos por parte de empresas del Estado -y requieren regulación- otros han visto la introducción de competencia, como ser la generación de energía eléctrica, la telefonía celular, la larga distancia internacional y la transmisión de datos, la distribución de combustibles o el GLP.

Tanto URSEA como URSEC son órganos desconcentrados del Poder Ejecutivo pertenecientes a la Presidencia de la República. Por tanto, sus decisiones son revisadas por el Presidente. Tienen directorios integrados por tres miembros designados también por el Presidente de la República en Consejo de Ministros, con mandatos de seis años renovables. Sus miembros no podrán ser candidatos a ningún cargo electivo por un período de gobierno.

La designación de los miembros de los Directorios ha seguido la misma lógica que la de las empresas públicas, esto es directores técnicos pero con vinculación a alguno de los partidos políticos, principalmente al partido de gobierno. Ello ha sido explícito en el último período de gobierno y en el fondo transmite la imagen de estas agencias como un órgano más cercano a lo político que a lo técnico en la toma de decisiones.

Si bien duran seis años en sus funciones, lo que permite traspasar el período de gobierno que en Uruguay es de cinco años, no tienen previsto un mecanismo rotativo para el cambio de los directores. Estos órganos gozan de independencia técnica relativa, pero están sujetos al control del Presidente quien, en última instancia, puede decidir sobre sus fallos.

Existe un relativo balance entre la capacidad técnica que se solicita a los directores y la independencia relativa del poder político. En la medida en que la designación es potestad del Presidente de la República, el control seguirá siendo del poder político de turno. Una estructura similar a la de las empresas públicas, es decir la figura de ente autónomo, implicaría que las decisiones no serían revisadas por el ejecutivo sino directamente por el Tribunal de lo Contencioso Administrativo. Sin embargo, ello podría acentuar el carácter político de sus directores, ante la reticencia del Poder Ejecutivo y Legislativo de perder el control sobre las decisiones de estos organismos.

Con la creación de la agencia reguladora del sector eléctrico (Unidad Reguladora de la Energía Eléctrica - UREE), en al año 2000, se desarrollaron propuestas tendientes al incremento de la competencia en este sector, las que generaron un conflicto lógico entre el regulador y la empresa UTE \citep{Bergara2005}. Este conflicto fue una señal de alerta para el sistema político en cuanto al riesgo de perder el control de la operación de estos mercados, a través del control de las empresas públicas, cuando la creación de órganos reguladores autónomos quita discrecionalidad al sistema político y le da poder de decisión y autonomía a la burocracia.

En este marco, la autonomía de la UREE era relativa ya que podía proponer los reglamentos vinculados al funcionamiento del mercado eléctrico pero la aprobación de los mismos correspondía al Poder Ejecutivo. En el año 2002 se modifica la estructura y los cometidos de esta unidad reguladora, la que pasó a denominarse URSEA, asumió la regulación de otros mercados (agua, combustible y gas) y vio menoscabada su autonomía, a través de la reducción de su capacidad económica y por lo tanto técnica (Bergara y Pereyra, 2005).

En 2001 se crea la URSEC que, al igual que la URSEA, tiene ciertas independencia institucional pero, en el momento, algunas restricciones. Por ejemplo, no regulan la fijación de tarifas en los sectores monopólicos (es potestad del Poder Ejecutivo) y su posibilidad de evitar situaciones anticompetitivas también es limitada.

En relación al proceso que llevó a la creación de los órganos reguladores, \citet{Bergara2005} concluyen que si bien ha sido el avance institucional más importante hacia la introducción de cierto nivel de competencia en los servicios públicos, su incidencia en las decisiones de política es sustancialmente menor que la de las propias empresas públicas, las que al momento de la creación de los órganos reguladores contaban con una burocracia mucho más capacitada que aquella que los entes reguladores podían contratar con los escasos recursos que les fueran asignados.

En los últimos años se observa un cambio de paradigma de las empresas públicas en Uruguay, donde las mismas son vistas como motor de desarrollo y, por tanto, los órganos reguladores se transforman en obstáculos para el cumplimiento de los objetivos políticos. Esta visión representa un gran retroceso no sólo institucional, sino también para las propias empresas del sector. En la medida en que gran parte de ellas actúan en mercados monopólicos, los órganos reguladores son los únicos que pueden garantizar que estas empresas alcancen algún grado de eficiencia técnica en el uso de sus recursos.

En este marco, en 2011 se aprueba una fuerte revisión de los cometidos de URSEA que elimina los objetivos más controvertidos del órgano regulador y lo deja, casi exclusivamente, como un organismo de control de calidad y recepción de reclamos técnicos. En particular, se eliminan los objetivos de fomento del nivel óptimo de inversión; la libre elección de los usuarios de los prestadores; y la aplicación de tarifas que reflejen los costos económicos (literales B, H e I del artículo 2 de la Ley 17.598).

En relación a los cometidos, también se aprecia una importante reducción respecto a los considerados originalmente. Desaparecen los cometidos de: (i) establecer los requisitos que deben cumplir los agentes regulados; (ii) dictaminar preceptivamente sobre los procedimientos de selección de concesionarios y autorizados en su órbita; (iii) preparar pliegos de bases y condiciones para la habilitación de servicios en su órbita; y (iv) formular las determinaciones técnicas y recomendaciones para la fijación de tarifas, si bien puede examinar las mismas.

En el caso de la URSEC, si bien no ha habido un cambio legal que revise sus cometidos, se observa la captura del regulador o una limitación en sus acciones sobre la empresa estatal. Ello se puede constatar en la cadencia con la que se han desarrollado las investigaciones por prácticas anticompetitivas por parte de la empresa estatal, cuando ataba productos monopólicos a los competitivos, así como en la negativa a autorizar el servicio de transmisión de datos por fibra óptica por parte de las empresas privadas, violando el principio de neutralidad tecnológica.

Este retroceso se observa también en relación a la separación de actividad de la empresa pública. Antes de 2005, ANTEL era un grupo de tres empresas: ANTEL en telefonía fija, ANCEL en telefonía celular y ANTELDATA en trasmisión de datos. Ese esquema fue abandonado por la anterior administración que reunió a las empresas bajo un mismo nombre: ANTEL. Asimismo, los esfuerzos de la URSEC de intentar la separación contable de las actividades de la empresa pública, que tiene al menos un segmento monopólico, no han prosperado.

\hypertarget{evaluaciuxf3n-del-marco-regulatorio-en-uruguay}{%
\subsection{Evaluación del marco regulatorio en Uruguay}\label{evaluaciuxf3n-del-marco-regulatorio-en-uruguay}}

Como ya se mencionara, la regulación implica suplantar al mercado en la determinación de las variables relevantes (precio, calidad, ingreso y egreso del mercado) por la determinación administrativa de las mismas. Ello requiere de conocimientos especializados del funcionamiento del mercado, tecnología que es muy demandante de recursos humanos calificados.

La actuación del regulador se realiza en la intersección entre la arena económica y política, que es la que define sus roles y cometidos. Distintas variables económicas presentan intereses diversos para el sector político. En el cuadro \ref{tab:cuadro9} se presenta una serie de variables económicas a las que se les asignan importancia relativa tanto desde el punto de vista económico como para el sistema político, y se señala el papel que juega el organismo regulador en Uruguay, en la definición de cada una de ellas.

\begin{table}

\caption{\label{tab:cuadro9}Variables económicas y rol del regulador}
\centering
\begin{tabular}[t]{l|l|l|l}
\hline
Variables & Importancia económica & Importancia política & Intervención regulatoria en Uruguay\\
\hline
Precio & ++ & ++ & No\\
\hline
Directorio & ++ & ++ & No\\
\hline
Acceso a mercado & ++ & ++ & Limitada\\
\hline
Acceso a infraestructura & ++ & + & Limitada\\
\hline
Inversión & ++ & + & No\\
\hline
Endeudamiento & ++ & + & No\\
\hline
RRHH & + & + & No requerida\\
\hline
Presupuesto & + & + & No requerida\\
\hline
Calidad producto & + & 0 & Si\\
\hline
Calidad proceso & + & 0 & Si\\
\hline
\end{tabular}
\end{table}

Fuente: elaboración propia.

Las variables precio, integración de directorios y acceso a mercado están establecidas a nivel constitucional o legal y son, por tanto, de la mayor importancia política. Es en estas variables donde se observa ninguna o limitada intervención por parte de los órganos reguladores, siendo otros organismos del Estado los que ejercen el control.

El acceso a infraestructura, inversión y endeudamiento si bien son relevantes en términos económicos, resultan menos importantes para el sistema político. En los hechos, estas variables las controlan tanto la OPP como el MEF, ya que resultan relevantes para estos organismos. En este caso, el sistema político se involucra sólo si las variables toman valores excesivos.

Donde se observa un mayor control por parte de los organismos regulatorios son sobre variables que no resultan de mayor relevancia económica y tienen nula importancia, desde el punto de vista político, como es el caso del control de calidad o calidad de procesos. Por tanto, desde el punto de vista institucional, en Uruguay los órganos reguladores tienen mayor injerencia sobre los temas que son políticamente menos sensibles.

En términos generales, el regulador tiene un rol clave para comprender el funcionamiento del mercado que no tiene ninguno de los demás organismos involucrados en el control de las empresas públicas. Es, por su naturaleza, el que puede evaluar la forma en la que estas empresas se desempeñan, si los niveles de precio son adecuados o no, y qué inversiones son necesarias para el desarrollo de largo plazo de las mismas. O sea, en última instancia considerar la eficiencia en el funcionamiento de las empresas públicas. Esta tarea es la única que es distintiva de los órganos reguladores y sobre la que no existen duplicaciones con otros órganos del Estado.

En este marco, existe un margen claro para que las agencias reguladoras asesoren a los tomadores de decisión respecto a las particularidades técnicas de los diversos procesos de decisión, ya sea fijación de tarifas, aprobación del presupuesto o de los planes de inversión. En los hechos, debería ser la agencia reguladora la que desarrolle las capacidades para comprender las características técnicas y económicas de cada mercado y la que pueda analizar la coherencia general de las propuestas que las empresas realizan.

Por otra parte, las empresas públicas se enfrentan a objetivos y restricciones cambiantes, ya que deben cumplir con distintos principales (ministerio sectorial, MEF, Presidencia, etc.). En un marco de múltiples principales, el regulador será uno más cuya opinión debería ser tomada en cuenta por los demás organismos involucrados.

Uno de los principales problemas del marco regulatorio ad-hoc que opera en Uruguay, es que la opción de veto a las decisiones de las empresas está instalada en otros agentes distintos al regulador. Sin embargo, ninguno de los actores con poder de veto tiene los conocimientos específicos para comprender el funcionamiento de los mercados donde estas empresas operan. El MEF tiene otras prioridades, asociadas al cuidado de las variables macroeconómicas; la OPP, si bien controla e interactúa con las empresas, no tiene los funcionarios calificados que se requiere para analizar sectores de alta complejidad técnica; y los ministerios sectoriales se dedican a establecer las políticas generales de desarrollo en sus ámbitos de aplicación considerando exclusivamente su eficacia pero no la eficiencia con que se llevan a cabo.

Un elemento importante para poder regular los mercados de servicios públicos es que los órganos reguladores cuenten con personal altamente calificado y, por tanto, bien remunerado. Aunque parezca obvio, \citet{DalBo2013} demostraron que mejores salarios permite atraer a funcionarios mejor capacitados, motivados y con mejor personalidad. Por tanto, reclutar a una plantilla técnicamente calificada requiere de salarios competitivos con las empresas a las que se regula.

En el caso de Uruguay, las empresas públicas cuentan con funcionarios altamente calificados y relativamente bien remunerados, en relación al resto de la administración pública \citep{Bergara2005}. Ello determina que hayan logrado captar y retener una masa importante de capital humano que les permite alcanzar los objetivos que se proponen. Asimismo, los miembros de las unidades reguladoras deben enfrentar, en el desarrollo de su actividad, a funcionarios de empresas que tienen salarios muy superiores y que están muy bien capacitados \citep{Bergara2005}.

Es así, que las capacidades que han desarrollado las empresas públicas, ponen en aprietos a la administración a la hora de ejercer el control sobre las mismas, debido a que los diversos organismos involucrados en este control no disponen de capacidades técnicas adecuadas para esta función. Ello se traduce en que los principales establezcan reglas generales que se aplican a toda la administración, o particulares para todas las empresas públicas. Estas reglas no toman en consideración elementos particulares de cada mercado, más allá de que, en las negociaciones entre las partes en temas específicos, puedan considerarse.

Por otra parte, en lo que refiere al acceso al mercado, los reguladores deberían actuar considerando el tipo de mercado de referencia. El ingreso al mercado en sectores oligopólicos sólo puede ser atendido analizando la propia lógica competitiva en el mercado. Salvo contadas excepciones, entre las que están los monopolios naturales, la competencia es el instrumento regulatorio más efectivo para incentivar a las empresas. Este debe ser uno de los principios que guie la actuación de los órganos reguladores.

La regulación en la interacción entre segmentos monopólicos y competitivos, o la relación entre empresas públicas y privadas, en situaciones donde existe competencia pero la empresa pública tiene el monopolio de determinado segmento del mercado, es un cometido importante de los órganos reguladores.

O sea, la regulación del acceso a los segmentos monopólicos cuando aparecen empresas privadas en sectores verticalmente relacionados con sectores donde operan empresas públicas (ello acontece tanto en energía eléctrica, como en telefonía y transmisión de datos). Los cometidos de los reguladores, en este caso, incluyen distintas dimensiones, como autorizar el acceso al mercado, permitir el acceso a los segmentos monopólicos y el control de los contratos entre las empresas estatales monopólicas que actúan en el ámbito del derecho privado y los agentes privados.

Establecer una política de acceso a los segmentos monopólicos clara, transparente y equilibrada es clave para evitar distorsiones en los mercados. La negativa de acceso tiene impacto sobre los mercados donde las empresas dependen de obtener este acceso para brindar otros servicios. De no obtenerse el acceso pueden darse dos situaciones: o bien el mercado no se desarrolla, como podría ser el caso del mercado mayorista de energía eléctrica, o bien los privados desarrollan sus propias redes lo que resulta ineficiente de existir economías de escala, como en el caso de la fibra óptica.

En Uruguay, con anterioridad a la existencia de organismos reguladores, muchos cometidos de la regulación estaban asignados a las propias empresas públicas o no existía con claridad un organismo que se hiciera cargo de ellos. Por ejemplo, en el sector telecomunicaciones las autorizaciones para operar las otorgaba la propia ANTEL. En la actualidad, son las unidades reguladoras las que otorgan los permisos y las autorizaciones para operar.

Otro elemento importante en el funcionamiento de las empresas públicas en Uruguay, y que está menos estudiado que los anteriores, refiere al uso de estas empresas como instrumento del sistema político para relajar la restricción fiscal. A diferencia de otros países, algunas las empresas públicas en Uruguay han sido fuente neta de recursos para los gobiernos.
\footnote{Véase la tabla A7 en \citet{WorldBank1995}.}

Tres son las empresas que aportan a rentas generales: UTE, ANTEL y la Administración Nacional de Puertos (ANP). La empresa que más aporta es ANTEL, seguida por UTE y en tercer lugar por la ANP. Por su parte, AFE es la única empresa receptora neta de recursos -subsidios- por parte del gobierno central.
\footnote{partir del año 2008 se creó la Agencia Nacional de Vivienda, que también recibe
  subsidios por parte del gobierno central.}
Si bien ANCAP no aporta utilidades a rentas generales, al estar gravados sus productos con el Impuesto Específico Interno (IMESI), los cambios en las tarifas de la empresa implican una mayor recaudación a través de este impuesto. Ello genera incentivos a que el Poder Ejecutivo utilice la tarifa para recaudar impuestos. A vía de ejemplo, la recaudación por IMESI de ANCAP alcanzó en 2012 a U\$S 450 millones, mientras que las transferencias de ANTEL, UTE y OSE sumaron U\$S 105 millones.

En cuanto a los mecanismos de gobernanza de las empresas públicas, \citet{Vagliasindi2008} revisa aquellos que operan en los países de la OCDE y señala que existen tres tipos de organización. El primero es el modelo descentralizado, en donde el control y la propiedad están dispersos entre diversos ministerios. El segundo es el modelo dual, donde existe un ministerio sectorial y un ministerio común -el MEF en el caso de Uruguay- que ejercen las responsabilidades de control sobre las empresas públicas. El tercero es el modelo centralizado, en el cual las empresas públicas dependen de un único ministerio o agencia.

En el contexto de Uruguay, tales figuras son difíciles de aplicar por diversas razones. La primera es formal, dado que la Constitución de la República estableció la independencia de facto de las empresas públicas al crear las figuras de entes autónomos y servicios descentralizados. En segundo, porque los controles que leyes específicas establecieron determinan cometidos específicos para cada uno de los principales de las empresas (OPP tiene el cometido de aprobar presupuesto, gastos e inversiones; ministerios sectoriales y MEF deben aprobar tarifas o endeudamiento). Es interesante notar que el control que los distintos principales ejercen sobre las empresas públicas -que se realiza en el marco de sus cometidos legales- implica una coordinación de hecho o sobre la marcha entre ellos y, por tanto, de los distintos objetivos que cada uno representa.

En este marco, parece posible aplicar un mecanismo como el descentralizado que establezca un Comité entre los principales de las empresas (OPP, MEF y ministerio sectoriales) lo que permitiría determinar ex ante las grandes líneas para el funcionamiento de estas empresas. Desde el punto de vista institucional es claro que se requiere una mínima coordinación ex ante entre los distintos interesados.

En los hechos, actualmente esta coordinación se logra ex post, cuando alguno de los principales toma la posición de veto sobre las decisiones adoptadas por las empresas públicas. Por tanto, el mecanismo de trasladar la negociación entre los principales al momento inicial permitirá una mejor planificación de las acciones que debe desarrollar la empresa para cumplir con sus cometidos. Asimismo, permite transparentar los intereses de los distintos principales y, si debieran establecerse rectificaciones en el rumbo, permite determinar claramente el porqué de los desvíos y los mecanismos de compensación para la empresa.

En última instancia, es dejar la gestión de la empresa a su directorio y que la institución coordinadora defina los objetivos que deben cumplir las empresas, tal como establece \citet{Ramamurti1991}. Ello también permite un mejor control del desempeño de los directores, aunque este aspecto siempre estará afectado por los vaivenes que pueda sufrir los cambios en las ponderaciones de los diferentes principales, producto de los cambios en otros sectores de la economía y de la política.

\hypertarget{conclusiones}{%
\section{Conclusiones}\label{conclusiones}}

En América Latina las empresas propiedad del Estado en sectores claves de la economía es un fenómeno que se manifiesta desde fines del siglo XIX. Por su parte, la crisis de los ochenta deja de manifiesto una serie de problemas presentes en la mayoría de las empresas públicas de la región: falta de claridad y jerarquización de objetivos; deficiencia en la gestión empresarial; menosprecio por la eficiencia y la racionalidad económica; y excesiva politización en su manejo. Esta situación deriva en los noventa en un proceso de privatización de empresas públicas en la mayoría de los países, con diferente grado de profundización.

En Uruguay el proceso de creación de empresas públicas fue similar al de otros países de América Latina. Las fallas de mercado, como mercados de capitales poco desarrollados o empresarios adversos al riesgo que impedían el desarrollo de algunos servicios públicos, fueron, junto con la debilidad institucional del Estado para poder controlar a las empresas privadas, los principales motivos que determinaron no solo la creación de las empresas públicas en el país, sino que también se les otorgara el monopolio para el desarrollo de sus actividades y, en el caso de la telefonía, el rol regulador del propio sector.

El proceso de estancamiento de la economía uruguaya observado a partir de la segunda posguerra repercutió en el desempeño de las empresas públicas. La necesidad de evitar altas tasas de desempleo, llevó a que estas empresas vieran incrementada sustancialmente su plantilla lo que, unido al clientelismo político, puso en tela de juicio la gestión de las mismas, llegando a cuestionarse su viabilidad económica.

En los noventa predominaron en la región y el resto del mundo propuestas de reformas orientadas al mercado con el objeto de promover la eficiencia económica y el crecimiento. En este marco, el gobierno de la época trató de aplicar estas reformas en los sectores de energía eléctrica, comunicaciones y combustibles. El objetivo fue incrementar la competencia en estos mercados y privatizar parcialmente a las empresas públicas.

En Uruguay, los controles constitucionales y legales previstos para las empresas públicas tienen como objetivo implícito controlar los incentivos que estas pueden tener en descuidar la eficiencia. Estos controles están definidos con antelación a la creación de los órganos reguladores y han tenido una eficacia relativa: se encuentran empresas que presentan excedentes, así como otras con déficits crónicos. Estas últimas permanecen operativas (no cierran) lo que indica que tienen restricciones blandas de facto. Estos controles no están diseñados para atender las particularidades de cada sector y se rigen en base a criterios generales que se aplican tanto a todo el sector público como a todas las empresas públicas.

A diferencia de otros países, en Uruguay existen fuertes, diversos y dispares mecanismos de control de las empresas públicas, y la fijación del precio o la tarifa no es un tema excluyente del regulador, ya que hay otros actores que intervienen en mayor medida en su determinación. En la misma línea, los reguladores no tienen intervención directa en los temas políticamente más sensibles (la competencia en los mercados o la determinación del precio de los servicios). Sin embargo, existe un margen para que las unidades reguladoras participen en la discusión general de las principales variables económicas: precio, inversión, tarifa y condiciones de acceso.

En los últimos años con el cambio de paradigma de las empresas públicas, en Uruguay, se observa un retroceso en la institucionalidad de los órganos reguladores. Este se verifica en una mayor libertad de estas empresas para actuar en sus mercados con prescindencia del entorno. Sin embargo, aislar a las empresas de la competencia, o de guías técnicas a la eficiencia resulta peligroso aún para las propias empresas, que ponen en juego su propia eficiencia que sólo el regulador puede controlar.

Por el contrario, los reguladores deberían tener una mayor injerencia en los mercados donde las empresas públicas están sometidas a competencia del sector privado, o interactúan con él. Este rol es único del regulador y debería potenciar su actuación en estos segmentos. El marco de referencia debería ser potenciar la competencia en aquellos mercados donde sea posible y mantener iguales reglas de juego para los agentes públicos y privados.

Los principios que deben guiar el accionar del regulador, para que la interacción entre las empresas públicas y privadas incentive a las primeras a mantener la eficiencia en el uso de sus recursos, deberían ser los siguientes: (i) separación contable de las actividades; (ii) transparencia de los subsidios cruzados que existan entre actividades monopólicas y competitivas; y (iii) acceso a las plataformas monopólicas creando las mismas reglas que se instrumentarían si la empresa fuera privada en vez de pública.

Uruguay necesita normas más agresivas para el acceso a los sectores monopólicos. Ello es fundamental para evitar la duplicación de recursos, por un lado, pero también para someter a la presión competitiva a las empresas. Más allá de las diferencias entre empresas públicas y privadas, lo cierto es que los trabajos empíricos no arrojan evidencia definitiva sobre la superioridad de una sobre la otra. El problema parece radicar en el entorno en el que operan, más que en el tipo de propiedad. Por tanto, excluir a las empresas públicas de la competencia es condenar, lentamente, su futuro. Bajo esta premisa, el regulador debe tener un rol más activo para otorgar acceso a segmentos monopólicos fijando un peaje acorde.

También sería deseable una mayor distancia de los reguladores del poder político, aunque sometidos a los controles jurisdiccionales correspondientes. Pensar que los miembros de los órganos reguladores son cargos políticos, que entran o salen según los avatares electorales, no sirve para la construcción de institucionalidad, sino para políticas de gobierno. Los miembros del directorio de las unidades reguladoras deberían tener mandatos claros y su cadencia de ingreso y salida debería estar desvinculada del ciclo político.

  \bibliography{regulacion.bib}

\end{document}
